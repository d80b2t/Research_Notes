\documentclass[11pt,a4paper]{article}
%
\usepackage[toc,page]{appendix}
\usepackage{amsmath, amssymb}
\usepackage{bm}% bold math
\usepackage{cancel, caption}
\usepackage{dcolumn}% Align table columns on decimal point
\usepackage{epsfig, epsf}
\usepackage{graphicx,fancyhdr,natbib,subfigure}
\usepackage{lscape, longtable}
\usepackage{hyperref,ifthen}
\usepackage{verbatim}
\usepackage{color}
\usepackage[usenames,dvipsnames]{xcolor}
\usepackage{listings}
%% http://en.wikibooks.org/wiki/LaTeX/Colors



%%%%%%%%%%%%%%%%%%%%%%%%%%%%%%%%%%%%%%%%%%%
%       define Journal abbreviations      %
%%%%%%%%%%%%%%%%%%%%%%%%%%%%%%%%%%%%%%%%%%%
\def\nat{Nat} \def\apjl{ApJ~Lett.} \def\apj{ApJ}
\def\apjs{ApJS} \def\aj{AJ} \def\mnras{MNRAS}
\def\prd{Phys.~Rev.~D} \def\prl{Phys.~Rev.~Lett.}
\def\plb{Phys.~Lett.~B} \def\jhep{JHEP} \def\nar{NewAR}
\def\npbps{NUC.~Phys.~B~Proc.~Suppl.} \def\prep{Phys.~Rep.}
\def\pasp{PASP} \def\aap{Astron.~\&~Astrophys.} \def\araa{ARA\&A}
\def\jcap{\ref@jnl{J. Cosmology Astropart. Phys.}}%
\def\physrep{Phys.~Rep.}

\newcommand{\preep}[1]{{\tt #1} }

%%%%%%%%%%%%%%%%%%%%%%%%%%%%%%%%%%%%%%%%%%%%%%%%%%%%%
%              define symbols                       %
%%%%%%%%%%%%%%%%%%%%%%%%%%%%%%%%%%%%%%%%%%%%%%%%%%%%%
\def \Mpc {~{\rm Mpc} }
\def \Om {\Omega_0}
\def \Omb {\Omega_{\rm b}}
\def \Omcdm {\Omega_{\rm CDM}}
\def \Omlam {\Omega_{\Lambda}}
\def \Omm {\Omega_{\rm m}}
\def \ho {H_0}
\def \qo {q_0}
\def \lo {\lambda_0}
\def \kms {{\rm ~km~s}^{-1}}
\def \kmsmpc {{\rm ~km~s}^{-1}~{\rm Mpc}^{-1}}
\def \hmpc{~\;h^{-1}~{\rm Mpc}} 
\def \hkpc{\;h^{-1}{\rm kpc}} 
\def \hmpcb{h^{-1}{\rm Mpc}}
\def \dif {{\rm d}}
\def \mlim {m_{\rm l}}
\def \bj {b_{\rm J}}
\def \mb {M_{\rm b_{\rm J}}}
\def \mg {M_{\rm g}}
\def \qso {_{\rm QSO}}
\def \lrg {_{\rm LRG}}
\def \gal {_{\rm gal}}
\def \xibar {\bar{\xi}}
\def \xis{\xi(s)}
\def \xisp{\xi(\sigma, \pi)}
\def \Xisig{\Xi(\sigma)}
\def \xir{\xi(r)}
\def \max {_{\rm max}}
\def \gsim { \lower .75ex \hbox{$\sim$} \llap{\raise .27ex \hbox{$>$}} }
\def \lsim { \lower .75ex \hbox{$\sim$} \llap{\raise .27ex \hbox{$<$}} }
\def \deg {^{\circ}}
%\def \sqdeg {\rm deg^{-2}}
\def \deltac {\delta_{\rm c}}
\def \mmin {M_{\rm min}}
\def \mbh  {M_{\rm BH}}
\def \mdh  {M_{\rm DH}}
\def \msun {M_{\odot}}
\def \z {_{\rm z}}
\def \edd {_{\rm Edd}}
\def \lin {_{\rm lin}}
\def \nonlin {_{\rm non-lin}}
\def \wrms {\langle w_{\rm z}^2\rangle^{1/2}}
\def \dc {\delta_{\rm c}}
\def \wp {w_{p}(\sigma)}
\def \PwrSp {\mathcal{P}(k)}
\def \DelSq {$\Delta^{2}(k)$}
\def \WMAP {{\it WMAP \,}}
\def \cobe {{\it COBE }}
\def \COBE {{\it COBE \;}}
\def \HST  {{\it HST \,\,}}
\def \Spitzer  {{\it Spitzer \,}}
\def \ATLAS {VST-AA$\Omega$ {\it ATLAS} }
\def \BEST   {{\tt best} }
\def \TARGET {{\tt target} }
\def \TQSO   {{\tt TARGET\_QSO}}
\def \HIZ    {{\tt TARGET\_HIZ}}
\def \FIRST  {{\tt TARGET\_FIRST}}
\def \zc {z_{\rm c}}
\def \zcz {z_{\rm c,0}}

\newcommand{\ltsim}{\raisebox{-0.6ex}{$\,\stackrel
        {\raisebox{-.2ex}{$\textstyle <$}}{\sim}\,$}}
\newcommand{\gtsim}{\raisebox{-0.6ex}{$\,\stackrel
        {\raisebox{-.2ex}{$\textstyle >$}}{\sim}\,$}}
\newcommand{\simlt}{\raisebox{-0.6ex}{$\,\stackrel
        {\raisebox{-.2ex}{$\textstyle <$}}{\sim}\,$}}
\newcommand{\simgt}{\raisebox{-0.6ex}{$\,\stackrel
        {\raisebox{-.2ex}{$\textstyle >$}}{\sim}\,$}}

\newcommand{\Msun}{M_\odot}
\newcommand{\Lsun}{L_\odot}
\newcommand{\lsun}{L_\odot}
\newcommand{\Mdot}{\dot M}

\newcommand{\sqdeg}{deg$^{-2}$}
\newcommand{\lya}{Ly$\alpha$\ }
%\newcommand{\lya}{Ly\,$\alpha$\ }
\newcommand{\lyaf}{Ly\,$\alpha$\ forest}
%\newcommand{\eg}{e.g.~}
%\newcommand{\etal}{et~al.~}
\newcommand{\lyb}{Ly$\beta$\ }
\newcommand{\cii}{C\,{\sc ii}\ }
\newcommand{\ciii}{C\,{\sc iii}]\ }
\newcommand{\civ}{C\,{\sc iv}\ }
\newcommand{\SiIV}{Si\,{\sc iv}\ }
\newcommand{\mgii}{Mg\,{\sc ii}\ }
\newcommand{\feii}{Fe\,{\sc ii}\ }
\newcommand{\feiii}{Fe\,{\sc iii}\ }
\newcommand{\caii}{Ca\,{\sc ii}\ }
\newcommand{\halpha}{H\,$\alpha$\ }
\newcommand{\hbeta}{H\,$\beta$\ }
\newcommand{\hgamma}{H\,$\gamma$\ }
\newcommand{\hdelta}{H\,$\delta$\ }
\newcommand{\oi}{[O\,{\sc i}]\ }
\newcommand{\oii}{[O\,{\sc ii}]\ }
\newcommand{\oiii}{[O\,{\sc iii}]\ }
\newcommand{\heii}{[He\,{\sc ii}]\ }
\newcommand{\nv}{N\,{\sc v}\ }
\newcommand{\nev}{Ne\,{\sc v}\ }
\newcommand{\neiii}{[Ne\,{\sc iii}]\ }
\newcommand{\aliii}{Al\,{\sc iii}\ }
\newcommand{\siiii}{Si\,{\sc iii}]\ }


%%%%%%%%%%%%%%%%%%%%%%%%%%%%%%%%%%%%%%%%%%%%%%%%%%%%%
%              define Listings                       %
%%%%%%%%%%%%%%%%%%%%%%%%%%%%%%%%%%%%%%%%%%%%%%%%%%%%%
\definecolor{dkgreen}{rgb}{0,0.6,0}
\definecolor{gray}{rgb}{0.5,0.5,0.5}
\definecolor{mauve}{rgb}{0.58,0,0.82}

\lstset{frame=tb,
  language=Python,
  aboveskip=3mm,
  belowskip=3mm,
  showstringspaces=false,
  columns=flexible,
  basicstyle={\small\ttfamily},
  numbers=none,
  numberstyle=\tiny\color{gray},
  keywordstyle=\color{blue},
  commentstyle=\color{dkgreen},
  stringstyle=\color{mauve},
  breaklines=true,
  breakatwhitespace=true,
  tabsize=3
}

\usepackage{multirow}
\usepackage{graphicx,fancyhdr,natbib,subfigure}
\usepackage{epsfig, epsf}
\usepackage{amsmath, cancel, amssymb}
\usepackage{lscape, longtable, caption}
\usepackage{dcolumn}% Align table columns on decimal point
\usepackage{bm}% bold math
\usepackage{hyperref,ifthen}
\usepackage{verbatim}
\usepackage{color}
\usepackage[usenames,dvipsnames]{xcolor}
\usepackage{listings}


\usepackage[toc,page]{appendix}
\usepackage{amsmath, amssymb}
\usepackage{bm}% bold math
\usepackage{cancel, caption}
\usepackage{dcolumn}% Align table columns on decimal point
\usepackage{epsfig, epsf}
\usepackage{graphicx,fancyhdr,natbib,subfigure}
\usepackage{lscape, longtable}
\usepackage{hyperref,ifthen}
\usepackage{verbatim}
\usepackage{color}
\usepackage[usenames,dvipsnames]{xcolor}
\usepackage{listings}
%% http://en.wikibooks.org/wiki/LaTeX/Colors



%%%%%%%%%%%%%%%%%%%%%%%%%%%%%%%%%%%%%%%%%%%
%       define Journal abbreviations      %
%%%%%%%%%%%%%%%%%%%%%%%%%%%%%%%%%%%%%%%%%%%
\def\nat{Nat} \def\apjl{ApJ~Lett.} \def\apj{ApJ}
\def\apjs{ApJS} \def\aj{AJ} \def\mnras{MNRAS}
\def\prd{Phys.~Rev.~D} \def\prl{Phys.~Rev.~Lett.}
\def\plb{Phys.~Lett.~B} \def\jhep{JHEP} \def\nar{NewAR}
\def\npbps{NUC.~Phys.~B~Proc.~Suppl.} \def\prep{Phys.~Rep.}
\def\pasp{PASP} \def\aap{Astron.~\&~Astrophys.} \def\araa{ARA\&A}
\def\jcap{\ref@jnl{J. Cosmology Astropart. Phys.}}%
\def\physrep{Phys.~Rep.}

\newcommand{\preep}[1]{{\tt #1} }

%%%%%%%%%%%%%%%%%%%%%%%%%%%%%%%%%%%%%%%%%%%%%%%%%%%%%
%              define symbols                       %
%%%%%%%%%%%%%%%%%%%%%%%%%%%%%%%%%%%%%%%%%%%%%%%%%%%%%
\def \Mpc {~{\rm Mpc} }
\def \Om {\Omega_0}
\def \Omb {\Omega_{\rm b}}
\def \Omcdm {\Omega_{\rm CDM}}
\def \Omlam {\Omega_{\Lambda}}
\def \Omm {\Omega_{\rm m}}
\def \ho {H_0}
\def \qo {q_0}
\def \lo {\lambda_0}
\def \kms {{\rm ~km~s}^{-1}}
\def \kmsmpc {{\rm ~km~s}^{-1}~{\rm Mpc}^{-1}}
\def \hmpc{~\;h^{-1}~{\rm Mpc}} 
\def \hkpc{\;h^{-1}{\rm kpc}} 
\def \hmpcb{h^{-1}{\rm Mpc}}
\def \dif {{\rm d}}
\def \mlim {m_{\rm l}}
\def \bj {b_{\rm J}}
\def \mb {M_{\rm b_{\rm J}}}
\def \mg {M_{\rm g}}
\def \qso {_{\rm QSO}}
\def \lrg {_{\rm LRG}}
\def \gal {_{\rm gal}}
\def \xibar {\bar{\xi}}
\def \xis{\xi(s)}
\def \xisp{\xi(\sigma, \pi)}
\def \Xisig{\Xi(\sigma)}
\def \xir{\xi(r)}
\def \max {_{\rm max}}
\def \gsim { \lower .75ex \hbox{$\sim$} \llap{\raise .27ex \hbox{$>$}} }
\def \lsim { \lower .75ex \hbox{$\sim$} \llap{\raise .27ex \hbox{$<$}} }
\def \deg {^{\circ}}
%\def \sqdeg {\rm deg^{-2}}
\def \deltac {\delta_{\rm c}}
\def \mmin {M_{\rm min}}
\def \mbh  {M_{\rm BH}}
\def \mdh  {M_{\rm DH}}
\def \msun {M_{\odot}}
\def \z {_{\rm z}}
\def \edd {_{\rm Edd}}
\def \lin {_{\rm lin}}
\def \nonlin {_{\rm non-lin}}
\def \wrms {\langle w_{\rm z}^2\rangle^{1/2}}
\def \dc {\delta_{\rm c}}
\def \wp {w_{p}(\sigma)}
\def \PwrSp {\mathcal{P}(k)}
\def \DelSq {$\Delta^{2}(k)$}
\def \WMAP {{\it WMAP \,}}
\def \cobe {{\it COBE }}
\def \COBE {{\it COBE \;}}
\def \HST  {{\it HST \,\,}}
\def \Spitzer  {{\it Spitzer \,}}
\def \ATLAS {VST-AA$\Omega$ {\it ATLAS} }
\def \BEST   {{\tt best} }
\def \TARGET {{\tt target} }
\def \TQSO   {{\tt TARGET\_QSO}}
\def \HIZ    {{\tt TARGET\_HIZ}}
\def \FIRST  {{\tt TARGET\_FIRST}}
\def \zc {z_{\rm c}}
\def \zcz {z_{\rm c,0}}

\newcommand{\ltsim}{\raisebox{-0.6ex}{$\,\stackrel
        {\raisebox{-.2ex}{$\textstyle <$}}{\sim}\,$}}
\newcommand{\gtsim}{\raisebox{-0.6ex}{$\,\stackrel
        {\raisebox{-.2ex}{$\textstyle >$}}{\sim}\,$}}
\newcommand{\simlt}{\raisebox{-0.6ex}{$\,\stackrel
        {\raisebox{-.2ex}{$\textstyle <$}}{\sim}\,$}}
\newcommand{\simgt}{\raisebox{-0.6ex}{$\,\stackrel
        {\raisebox{-.2ex}{$\textstyle >$}}{\sim}\,$}}

\newcommand{\Msun}{M_\odot}
\newcommand{\Lsun}{L_\odot}
\newcommand{\lsun}{L_\odot}
\newcommand{\Mdot}{\dot M}

\newcommand{\sqdeg}{deg$^{-2}$}
\newcommand{\lya}{Ly$\alpha$\ }
%\newcommand{\lya}{Ly\,$\alpha$\ }
\newcommand{\lyaf}{Ly\,$\alpha$\ forest}
%\newcommand{\eg}{e.g.~}
%\newcommand{\etal}{et~al.~}
\newcommand{\lyb}{Ly$\beta$\ }
\newcommand{\cii}{C\,{\sc ii}\ }
\newcommand{\ciii}{C\,{\sc iii}]\ }
\newcommand{\civ}{C\,{\sc iv}\ }
\newcommand{\SiIV}{Si\,{\sc iv}\ }
\newcommand{\mgii}{Mg\,{\sc ii}\ }
\newcommand{\feii}{Fe\,{\sc ii}\ }
\newcommand{\feiii}{Fe\,{\sc iii}\ }
\newcommand{\caii}{Ca\,{\sc ii}\ }
\newcommand{\halpha}{H\,$\alpha$\ }
\newcommand{\hbeta}{H\,$\beta$\ }
\newcommand{\hgamma}{H\,$\gamma$\ }
\newcommand{\hdelta}{H\,$\delta$\ }
\newcommand{\oi}{[O\,{\sc i}]\ }
\newcommand{\oii}{[O\,{\sc ii}]\ }
\newcommand{\oiii}{[O\,{\sc iii}]\ }
\newcommand{\heii}{[He\,{\sc ii}]\ }
\newcommand{\nv}{N\,{\sc v}\ }
\newcommand{\nev}{Ne\,{\sc v}\ }
\newcommand{\neiii}{[Ne\,{\sc iii}]\ }
\newcommand{\aliii}{Al\,{\sc iii}\ }
\newcommand{\siiii}{Si\,{\sc iii}]\ }


%%%%%%%%%%%%%%%%%%%%%%%%%%%%%%%%%%%%%%%%%%%%%%%%%%%%%
%              define Listings                       %
%%%%%%%%%%%%%%%%%%%%%%%%%%%%%%%%%%%%%%%%%%%%%%%%%%%%%
\definecolor{dkgreen}{rgb}{0,0.6,0}
\definecolor{gray}{rgb}{0.5,0.5,0.5}
\definecolor{mauve}{rgb}{0.58,0,0.82}

\lstset{frame=tb,
  language=Python,
  aboveskip=3mm,
  belowskip=3mm,
  showstringspaces=false,
  columns=flexible,
  basicstyle={\small\ttfamily},
  numbers=none,
  numberstyle=\tiny\color{gray},
  keywordstyle=\color{blue},
  commentstyle=\color{dkgreen},
  stringstyle=\color{mauve},
  breaklines=true,
  breakatwhitespace=true,
  tabsize=3
}

\begin{document}

\section{Scope}
``Uniform Volumes with Baryonic Physics'' \\
e.g., ILLUSTRIS, EAGLE,  MAGNETICUM, + MUFASA, + HORIZON AGN, + MASSIVE BLACK.
From \citet{Fabian2012}:: 
\begin{eqnarray}
    E_{\rm gal} & \approx & M_{\rm gal} \sigma^2 \\
  M_{\rm BH} & \approx & 1.4 \times 10^{-3} M_{\rm gal}\\
\end{eqnarray}

\citep[From][]{Somerville_Dave2015} A simple calculation indicates
that the amount of energy that must have been released in growing
these BHs must exceed the binding energy of the host galaxy,
suggesting that it could have a very significant effect on galaxy
formation \citet{Silk_Rees1998};
\begin{eqnarray}
   E_{\rm BH}  & \approx & 0.1 M_{\rm BH} c^{2} \\ 
                  &  \approx & 0.1\times 10^{8} M_{\odot} c^{2} \\
                  &  \approx & 0.1\times 10^{8} \cdot 2\times10^{30} c^{2} \\
                  &  \approx & 10^{7} \cdot 2\times10^{30} \cdot 9\times10^{16} \\
                  &  \approx & 1.8\times10^{54} {\rm Joules}\\
                  &  \approx & 1.8\times10^{61} {\rm erg}\\
\end{eqnarray}
(Joules in kg $\cdot$ m$^{2}$ $\cdot$ s$^{-2}$ ;-) 

\begin{eqnarray}
   E_{\rm BE}  & = &  \frac {3GM^{2}} {5R} \\
                 & =  &  \frac{(3 \cdot 6.674\times10^{-11} \cdot (10^{11} \cdot 2\times10^{30})^{2})} 
                              {5 \cdot 3.086\times10^{19}} \;  {\rm Joules} \\ 
               & \approx  & 5.19 \times 10^{52} \; {\rm Joules}\\ 
               & \approx  &5 \times 10^{59} \; {\rm ergs} \\ 
\end{eqnarray}
for a $M_{\rm Sph} = 10^{11} M_{\odot}$ and 1 kpc bulge. 
Half this for a 2 kpc bulge etc. 

\noindent
$\Rightarrow$\\
$E_{\rm BH} \approx 35 E_{\rm BE}$ for a $M_{\rm BH} = 10^{8}
M_{\odot}$ in a $10^{11} M_{\odot}$ host galaxy with a 1 kpc bulge. \\

However, it is still uncertain how efficiently this energy can couple
to the gas in and around galaxies. Observational signatures of
feedback associated with active galactic nuclei (AGNs) include
high-velocity winds, which may be ejecting the cold ISM from galaxies,
and hot bubbles apparently generated by giant radio jets, which may be
heating the hot halo gas \citep[for recent reviews see][]{Fabian2012,
Heckman_Best2014}. AGN feedback is treated using subgrid recipes
in current cosmological simulations.

In cosmological simulations, the usual approach is to place seed BHs
by hand in halos above a critical mass ($M_{\rm halo} \gtrsim 10^{10}
- 10^{11} M_{\odot}$). In some cases, seeds of a fixed mass are used;
in others, the seed mass is chosen in order to place the BH on the
local $M_{\rm BH} − \sigma$ relation. The results that we discuss here
are generally insensitive to the details of the seeding procedure.
One must then calculate how rapidly these seed BHs accrete gas and
grow in mass. The currently predominant model relies on the idea that
BH growth is limited by Bondi accretion of mass within the sphere of
influence (Bondi 1952), given by
\begin{equation}
M _{\rm Bondi} = \alpha_{\rm boost} \frac{4\pi  G^2 M_{\rm BH}^{2} \rho}{ (c_{s}^2 + v^{2})^{3/2}}
\end{equation}
where $M_{\rm BH}$ is the mass of the BH, $c_{s}$ is the sound speed
of the gas, $v$ is the bulk velocity of the BH relative to the gas,
$\rho$ is the density of the gas, and $\alpha_{\rm boost}$ is a boost
parameter included because models typically lack the spatial
resolution to resolve the Bondi radius \citep{Booth_Schaye2009, 
Johansson2009a}.

Early models took $\alpha_{\rm boost}$ to be constant (typically
$\sim$100), but some simulators make $\alpha_{\rm boost}$ a function
of density \citep[e.g.,][]{Booth_Schaye2009} and some recent
simulations resolve the Bondi radius and therefore adopt $\alpha_{\rm
boost} = 1$. Typically, the accretion rate is capped at the Eddington
rate. As galaxies merge, their BHs are assumed to merge when they come
within some distance of each other, typically a softening length
(thereby ignoring GR timescales for BH inspiral).


Assuming Bondi accretion requires the accompaniment of strong feedback
to obtain BHs that follow the MBH − σ relation, as this simple
argument demonstrates \citep{Angles-Alcazar2013}. Consider two
BHs of mass $M_{\rm a}$ and $M_{\rm b}$. If they grow according to the general
prescription $\dot{M}_{\rm BH} = D(t) \, M^{p}_{\rm BH}$ then
\begin{equation}
\frac{d}{dt}  \left ( \frac{M_{a}}{M_{b}} \right ) = D(t) \frac{M_{a}^{p}}{M^{p}_{b}} \left [ 1 -  \left ( \frac{M_{a}}{M_{b}}  \right )^{1-p} \right ].
\end{equation}
It is ``easy to show'' that the two masses will diverge if $p > 1$,
and they will converge if $p < 1$. For Bondi accretion, $p = 2$; hence
for BHs to converge onto an $M_{\rm BH} − \sigma$ relation, some
strongly self-regulating feedback process must counteract Bondi
accretion and make $p$ effectively less than unity. We discuss possible
feedback processes in Section 3.3.3, but in general such tuned
self-regulation is not so straightforward to arrange, for the usual
reason that outward energetic processes tend to escape through paths
of least resistance, whereas inflows typically arrive through the
dense, harder-to-disrupt gas.

It is worth emphasizing that the widely used Bondi model implicitly
assumes that the accreting gas has negligible angular momentum, which
is unlikely to be a good assumption in general.  Recently, the problem
of dissipating angular momentum to enable BH accretion has received
more attention in the cosmological
milieu. \citet{Hopkins_Quataert2010, Hopkins_Quataert2011} studied
angular momentum transport in disks with nonaxisymmetric perturbations
both analytically and in sim- ulations, showing that such secular
processes can significantly fuel BH growth, as also suggested by
\citet{Bournaud2011} and \citet{Gabor_Bournaud2013}. Implementing this
analytic work into zooms and cosmological simulations,
\citet{Angles-Alcazar2013, Angles-Alcazar2014} showed that this
torque- limited accretion behaves qualitatively differently than Bondi
accretion, because in the \citet{Hopkins_Quataert2011} model, the
exponent of BH growth is $p = 1/6$ . Although this model also must
incorporate feedback, such feedback does not have to strongly couple
to the inflow to achieve self-regulation.



\begin{table}[]
  \centering
  \begin{tabular}{c cl}
    \hline
    \hline
     & \\
    ``Transition''                              &  ``Maintenance''  \\
     & \\
    \hline
     & \\
    Radiative mode                           & Jet mode \\
    Wind mode                                  & Kinetic mode \\
    Quasar mode, high $\dot{m}$     &  Radio mode, low $\dot{m}$ \\
    Moves mass from Blue to Red      &  Keeps things Red\\
    Rapid, ($\sim10^7$ years)           & Long-lived ($\sim$Hubble time)\\
    BH accreting efficiently                & accretion rate low cf. Eddington rate \\
    hot, nuclear wind feedback          & energy injected by relativistic jets \\ 
    pushes cold gas about                  & jets in hot halo \\
    Small(er) scales ($\sim$pc-kpc)  & Large (halo) scales \\
    Gas-rich/Dissipational Mergers   & Hot Haloes \& Dry Mergers \\
    Regulates BLACK HOLE mass        & Regulates GALAXY mass \\
     & \\
    \hline
    \hline
\end{tabular}
    \caption{Modified from Hopkins talk: www.astro.caltech.edu/q50/Program.html}
    \label{tab:modes}
\end{table}

\subsection{The Radiative (or Wind, or Quasar) Mode} 
\citep[from ][]{Fabian2012}; \\

\begin{equation}
M_{BH} \sim \frac{f \sigma^5 \sigma_T}{4 \pi G^2 m_p c}
\end{equation}
where $\sigma_{\rm T}$ is the Thomson cross section for electron
scattering and $f$ is the fraction of the galaxy massingas.The galaxy
is assumed to be isothermal with radius $r$, so that its mass is $Mgal
=2 \sigma^2 r / G$. The maximum collapse rate, $\sim 2f sigma 3/G$, is
equivalent to the gas content, $f Mgal$, collapsing on a free-fall
time, $r/sigma $, requiring a power of $\sim f \sigma 5/G$ to balance
it, which is limited by the Eddington luminosity, $LEdd = 4 \pi G
M_{BH} m_p c / \sigma_{\rm T}$. The argument is based on energy that
is necessary but may not be sufficient for ejecting matter (the rocket
equation, for example, is based on momentum).

\begin{equation}
\frac{4 \pi G M_{\rm BH} m_p}{\sigma_{\rm T}} = \frac{L_{\rm Edd}}{c} 
                                                                    = \frac{G M_{\rm gal} M_{\rm gas}}{r^2} 
                                                                    = \frac{f G M^{2}_{\rm gal}}{r^2} 
                                                                     = \frac{fG}{r} \left ( \frac{2\sigma^{2} r}{G}  \right )^2
\end{equation}
i.e. 

\begin{equation}
\frac{4 \pi G M_{\rm BH} m_p}{\sigma_{\rm T}} = \frac{f4\sigma^{4}}{G}
\end{equation}

%% From Annalisa's talk
Cosmological (M)HD Simulations \\
e.g. AREPO (Springel 2010): moving-mesh code  \\
Illustris  \\
IllustrisTNG \\
Auriga \\
FIRE-2  \\
MUFASA \\

or SPH/Particle Codes \\
(e.g. GADGET-n, GASOLINE, P-SPH) \\
for EAGLE  \\
Massive-Black  \\
Magneticum  \\
ERIS  \\
NIHAO  \\
FIRE \\

or \\
GRID/AMR Codes \\
(e.g. RAMSES,  \\
for the HORIZON-AGN)\\


\begin{landscape}
  \begin{table}[]
    \centering
%    \begin{tabular}{p{35mm}  p{35mm}  p{35mm}  p{35mm}   p{35mm}  p{35mm} }  %% 3 pretty-side columns
 \begin{tabular}{l  lllll}
      \hline
      \hline
      &                    &  &&& \\
      & ILLUSTRIS    & TNG  & EAGLE & FIRE & MUFASA\\
      &                     &  &&& \\
      \hline
      &                     &  &&&\\
      Codes                                                                                                 & AREPO           & AREPO               & GADGET-3 (``{\sc anarchy}'')             & ?           &  AREPO\\
      $(\Omega_{\rm m},  \Omega_{\lambda}, \Omega_{\rm b}, h, \sigma_{8})$ &                      &                         &   (0.307, 0.693, 0.04825, 0.6777, 0.8288) & & \\ 

      comoving box size                &                     &   & 100 &&\\
      $N$ DM particles                  &                      &   &  1504$^3$ &&\\
     initial baryonic particle mass &                     &    & 1.81$\times10^{6}$  &&\\
     DM particle mass                    &                     &    & 9.70$\times10^{6}$  &&\\
    $\epsilon_{\rm com}$                 &                      &   & 2.66 &&\\
    $\epsilon_{\rm prop}$                 &                     &    & 0.70 &&\\
    \hline
    \hline
\end{tabular}
    \caption{
      {\tt Ref-L100N1504} model from EAGLE.
BH Feedback: BH seeding and accretion.  
      BHs are usually placed by hand as``sink particles'':
      they can grow in mass by `accreting' material from the surroundings
      (Pillepich Edinburgh talk, 20171011).  
      Comoving, Plummer-equivalent gravitational softening length ($\epsilon_{\rm com}$ ) 
      and maximum proper softening length ($\epsilon_{\rm prop}$. 
    }
    \label{tab:previous_surveys}
\end{table}
\end{landscape}





\newpage
\section{TNG}
Key references:: \citet{Weinberger2017, Weinberger2017b, Pillepich2017, Pillepich2018} \\

\begin{equation}
  \dot{M}_{\rm BH} = \frac{4 \pi \alpha G^2 M_{\rm BH}^2 \rho}{(c^2_s + v^2)^{3/2}} 
\end{equation}
\begin{equation}
  \dot{M}_{\rm Edd} = \frac{4 \pi \alpha \, G \, M_{\rm BH} \, m_{p}}{\epsilon_{\rm r} \sigma_{\rm T} c  }
\end{equation}




\newpage
\section{EAGLE}
Key references::\citet{Schaye2015, Crain2015} \\

\subsection{From \citet{Schaye2015}}
We choose to implement only a single mode of AGN feedback with a fixed
efficiency. {\it The energy is injected thermally at the location of the BH
at a rate that is proportional to the gas accretion rate.}
\begin{equation}
E_{\rm inj, thermal} \propto \dot{m}_{\rm gas}
\end{equation}
This implementation may therefore be closest to the process referred to as quasar-mode feedback.

EAGLE implementation consists of two parts: {\it (i)} prescriptions for seeding low-mass galaxies with central BHs and for their growth via gas accretion and merging (we neglect any growth by accretion of stars and dark matter) and {\it (ii)} a prescription for the injection of feedback energy. Our method for the growth of BHs is based on the one introduced by \citet{Springel2005} and modified by \citet{Booth_Schaye2009} and \citet{Rosas-Guevara2015}, while our method for AGN feedback is close to the one described in \citet{Booth_Schaye2009}. 




\subsection{From \citet{Bower2017}}
3.2 Black holes and galaxies in the EAGLE simulations\\
In the EAGLE simulations, the accretion on to a black hole is
detrmined by a subgrid model which accounts for the mean density,
effective sound speed and relative motion and angular momentum of the
surrounding gas as detailed in \citet[][, with the exception that, in
the EAGLE simulations, we do not in- crease the accretion rate to
account for an unresolved clumping factor]{Rosas-Guevara2015}. The
model assumes that once gas reaches high densities on sub-kpc scales
around the black hole, it will be accreted on to the black hole at the
Bondi rate unless its angular momentum is sufficiently high to prevent
this. Our simulations allow us to understand how the density on
sub-kpc scales around the black hole is determined by the interaction
of star formation, feedback and gas accretion (on scales of 1 kpc and
greater) without making the simplifications adopted in the analytic
model. Importantly, the simulation does not impose a galaxy transition
mass scale by varying the black hole feedback efficiency as a function
of halo mass or accretion rate. We always assume that the energy
generated by feedback is 1.5 per cent of the rest mass energy of the
accreted material.




\subsection{From \citet{McAlpine2017}}

{\it BH seeding} follows the prescription first introduced by
\citet{Springel2005}, whereby BHs are introduced as collisionless sink
particles placed in the centres of dark matter haloes more massive
than $1.475\times10^{10}$ M$_{\odot}$, which do not already contain
one. BHs enter the simulation with a seed mass $m_{\rm seed} =
1.475\times 10^{5}$ M$_{\odot}$ and subsequently grow via accretion of
surrounding gas or mergers with other BHs.

{\it BHs grow via accretion} of nearby material at a rate estimated
from the modified Bondi–Hoyle formalism introduced in \citet{Rosas-Guevara2015}. 
In short, the model is an extension of the spherically 
symmetric case of \citet{Bondi_Hoyle1944} accounting now for the
circularization velocity of the surrounding gas, capped at the
Eddington limit. Contrary to \citet{Rosas-Guevara2015}, we do not
use an additional boost factor ($\alpha$).

{\it AGN feedback} is implemented as a single mode, where it is
injected thermally and stochastically into the surrounding inter-
stellar medium (ISM) as per \citet{Booth_Schaye2009}. Feedback is
performed assuming a single efficiency, independent of halo mass and
accretion rate.

\citet{McAlpine2017} report that throughout their investigation, they
have consistently found no evi- dence supporting a simple underlying
relationship between the rate of a galaxy’s star formation and the
accretion rate of its central BH.




\section{FIRE-2}

\section{{\sc mufasa}}

\section{Illustris}
Key references:: \citet{Sijacki2015} \\
``We find that black holes and galaxies co-evolve at the massive end,
but for low mass, blue and star-forming galaxies there is no tight
relation with either their central black hole masses or the nuclear
AGN activity.''  \citep{Sijacki2015}. 

\noindent
From \citep{Sijacki2015}:: \\ 
2.4.1 Black hole accretion\\
In the Illustris simulations collisionless black hole particles with a
seed mass of $1.42\times10^{5}$ M$_{\odot}$ $(10^{5} h^{-1}$ M$_{\rm
odot}$) are placed with the aid of the on-the-fly friends-of-friends
(FOF) algorithm in all haloes more massive than 7.1$\times10^{10}$
M$_{\odot}$ that do not contain a black hole particle
already. Thereafter, the black hole seeds can grow in mass either
through gas accretion, which we parametrize in terms of Ed- dington
limited Bondi–Hoyle–Lyttleton-like accretion \citep[for further
details see][]{DiMatteo2005, Springel2005}, or via mergers with other
black holes. At $z = 4$ our high-resolution Illustris simulation
already tracks 9414 black holes, at $z = 2$ this number more than
doubles leading to 24 878 black holes in total, while at $z = 0$ there
are 32 542 black holes in total with 3965 black holes more massive
than 10$^7$ M$_{\odot}$ black hole accretion.





\begin{landscape}
  \begin{table}[]
    \centering
    \label{my-label}
    \begin{tabular}{  p{65mm}   p{70mm}  p{70mm} }  %% 3 pretty-side columns
%    \begin{tabular}{lrr}
      \hline
      \hline
      &                    &  \\
      & ILLUSTRIS    & TNG \\
                  &                     &  \\
      \hline
                  &                     &  \\
      BH Seed Mass                             &  $1\times10^{5} h^{-1}$ M$_{\odot}$ &   $8\times10^{5} h^{-1}$ M$_{\odot}$\\
      FoF Halo Mass for BH seeding    & $5\times10^{10} h^{-1}$ M$_{\odot}$ &   $5\times10^{10} h^{-1}$ M$_{\odot}$\\
      BH accretion & $\alpha=100$ Boosted Bondi-Holye   & Unboosted Bondi-Hoyle $({\rm w} / v_{A})$  \\
      BH accretion & parent gas cell, Eddington limited & nearby cells, Eddington limited  \\
      BH positioning & fixed to halo potential minimum & fixed to halo potential minimum \\
    \hline
    \hline
\end{tabular}
    \caption{What are the similiarties and differences between 
      Proto-stellar and AGN accretion disks?
    }
    \label{BH Feedback: BH seeding and accretion.  
BHs are usually placed by hand as``sink particles'':
they can grow in mass by `accreting' material from the surroundings
(Pillepich Edinburgh talk, 20171011).   }
\end{table}
\end{landscape}



\section{Conroy \& White}


\section{Some general and left-field thoughts}
How {\it exactly} does the energy output from the central engine shutdown the SF?? \\
(Do we observe?) Ripples/concentric rings of shut-down star-formation??\\

\section{Glossary}
AREPO:: -- moving-mesh code \citep{Springel2010}.\\
GADGET:: -- SPH/Particle  code \\ 
GASOLINE:: --SPH/Particle  code \\
P-SPH:: --SPH/Particle  code \\
 
\subsection{Radiative Feedback}
From Ricci (2017) Nature paper references::\\
Fabian, A. C., Celotti, A. \& Erlund, M. C. Radiative pressure feedback by a quasar in a galactic bulge. Mon. Not. R. Astron. Soc. 373, L16–L20 (2006). \\
Menci, N., Fiore, F., Puccetti, S. \& Cavaliere, A. The blast wave model for AGN feedback: e ects on AGN obscuration. Astrophys. J. 686, 219–229 (2008). \\
Fabian, A. C., Vasudevan, R. V. \& Gandhi, P. The e ect of radiation pressure on dusty absorbing gas around active galactic nuclei. Mon. Not. R. Astron. Soc. 385, L43–L47 (2008).\\
Fabian, A. C., Vasudevan, R. V., Mushotzky, R. F., Winter, L. M. \& Reynolds, C. S. Radiation pressure and absorption in AGN: results from a complete unbiased sample from Swift. Mon. Not. R. Astron. Soc. 394, L89–L92 (2009). \\
Wada, K. Obscuring fraction of active galactic nuclei: implications from radiation-driven fountain models. Astrophys. J. 812, 82 (2015).

\section{Resources}
Evidence For Feedback: A highly biased review

\bibliographystyle{mn2e}
\bibliography{/cos_pc19a_npr/LaTeX/tester_mnras}

\end{document}

