
e.g.
Hopkins talk at:
http://www.astro.caltech.edu/q50/Program.html

So, Nic's claim is:

Radiative Mode = ``Transition'' Mode = ``Quasar'' mode. 
   cf. 
Jet Mode = ``Radio'' Mode = ``Maintenance'' mode...


L_mech is telling us about the Jets/jet modes...


From Hopkins talk:
   ``Transition''                    vs.           ``Maintenance'' 
- Quasar mode, high m_dot                        - Radio mode, low m_dot
- Moves mass from Blue to Red (?)                - Keeps things Red
- Rapid, (~10^7 years)                           - Long-lived (~Hubble time)
- Small(er) scales (~pc-kpc)                     - Large (~halo) scales
- Gas-rich/Dissipational Mergers                 - Hot Haloes & Dry Mergers
- Regulates BLACK HOLE mass                      - Regulates GALAXY mass


{\bf KEY POINTS!!!!!}
In ``jet-mode'' a distinct mode of accretion onto the SMBH exists that is apparently associated with low accretion rates and which is radiatively inefficient. The geometrically-thin accretion disk is either absent, or is truncated in the inner regions, and is replaced by a geometrically-thick structure in which the inflow time is much shorter than the radiative cooling time (e.g. Narayan \& Yi 1994, 1996, Quataert 2001, Narayan 2005, Ho 2008)

CRITICAL PAPER(S) TO READ:  
Blandford, R. D.; Payne, D. G.
 1982MNRAS.199..883B
Blandford, R. D.; Znajek, R. L.
 1977MNRAS.179..433B
Begelman, Mitchell C.; Blandford, Roger D.; Rees, Martin J
 1984RvMP...56..255B


Radiative Mode = ``Transition'' Mode = ``Quasar'' mode, 
is connected with ``quenching'' 
e.g. Halo quenching, pre-heating, quasar mode feedback, stellar feedback, morphological quenching. 

Jet Mode = ``Radio'' Mode = ``Maintenance'' mode...
is connected with ``maintenance''. 
e.g. Gravitational heating, Thermal conduction and Diffusion, Radio Mode Feedback, AGB heating

Sources: 
Van den Bosch and Schaye talks at:
https://www.youtube.com/channel/UCyLJaGie72EJnmBeyB9AauA



