\documentclass[11pt,a4paper]{article}

\usepackage{amsmath, amssymb}
\usepackage{caption, cancel, color}
\usepackage{graphicx, fancyhdr, natbib, subfigure}
\usepackage{epsfig, epsf}
\usepackage{lscape, longtable,listings}
\usepackage{dcolumn}% Align table columns on decimal point
\usepackage{bm}% bold math
\usepackage{hyperref,ifthen}
\usepackage{multirow}
\usepackage{verbatim}
\usepackage[usenames,dvipsnames]{xcolor}



\usepackage[toc,page]{appendix}
\usepackage{amsmath, amssymb}
\usepackage{bm}% bold math
\usepackage{cancel, caption}
\usepackage{dcolumn}% Align table columns on decimal point
\usepackage{epsfig, epsf}
\usepackage{graphicx,fancyhdr,natbib,subfigure}
\usepackage{lscape, longtable}
\usepackage{hyperref,ifthen}
\usepackage{verbatim}
\usepackage{color}
\usepackage[usenames,dvipsnames]{xcolor}
\usepackage{listings}
%% http://en.wikibooks.org/wiki/LaTeX/Colors



%%%%%%%%%%%%%%%%%%%%%%%%%%%%%%%%%%%%%%%%%%%
%       define Journal abbreviations      %
%%%%%%%%%%%%%%%%%%%%%%%%%%%%%%%%%%%%%%%%%%%
\def\nat{Nat} \def\apjl{ApJ~Lett.} \def\apj{ApJ}
\def\apjs{ApJS} \def\aj{AJ} \def\mnras{MNRAS}
\def\prd{Phys.~Rev.~D} \def\prl{Phys.~Rev.~Lett.}
\def\plb{Phys.~Lett.~B} \def\jhep{JHEP} \def\nar{NewAR}
\def\npbps{NUC.~Phys.~B~Proc.~Suppl.} \def\prep{Phys.~Rep.}
\def\pasp{PASP} \def\aap{Astron.~\&~Astrophys.} \def\araa{ARA\&A}
\def\jcap{\ref@jnl{J. Cosmology Astropart. Phys.}}%
\def\physrep{Phys.~Rep.}

\newcommand{\preep}[1]{{\tt #1} }

%%%%%%%%%%%%%%%%%%%%%%%%%%%%%%%%%%%%%%%%%%%%%%%%%%%%%
%              define symbols                       %
%%%%%%%%%%%%%%%%%%%%%%%%%%%%%%%%%%%%%%%%%%%%%%%%%%%%%
\def \Mpc {~{\rm Mpc} }
\def \Om {\Omega_0}
\def \Omb {\Omega_{\rm b}}
\def \Omcdm {\Omega_{\rm CDM}}
\def \Omlam {\Omega_{\Lambda}}
\def \Omm {\Omega_{\rm m}}
\def \ho {H_0}
\def \qo {q_0}
\def \lo {\lambda_0}
\def \kms {{\rm ~km~s}^{-1}}
\def \kmsmpc {{\rm ~km~s}^{-1}~{\rm Mpc}^{-1}}
\def \hmpc{~\;h^{-1}~{\rm Mpc}} 
\def \hkpc{\;h^{-1}{\rm kpc}} 
\def \hmpcb{h^{-1}{\rm Mpc}}
\def \dif {{\rm d}}
\def \mlim {m_{\rm l}}
\def \bj {b_{\rm J}}
\def \mb {M_{\rm b_{\rm J}}}
\def \mg {M_{\rm g}}
\def \qso {_{\rm QSO}}
\def \lrg {_{\rm LRG}}
\def \gal {_{\rm gal}}
\def \xibar {\bar{\xi}}
\def \xis{\xi(s)}
\def \xisp{\xi(\sigma, \pi)}
\def \Xisig{\Xi(\sigma)}
\def \xir{\xi(r)}
\def \max {_{\rm max}}
\def \gsim { \lower .75ex \hbox{$\sim$} \llap{\raise .27ex \hbox{$>$}} }
\def \lsim { \lower .75ex \hbox{$\sim$} \llap{\raise .27ex \hbox{$<$}} }
\def \deg {^{\circ}}
%\def \sqdeg {\rm deg^{-2}}
\def \deltac {\delta_{\rm c}}
\def \mmin {M_{\rm min}}
\def \mbh  {M_{\rm BH}}
\def \mdh  {M_{\rm DH}}
\def \msun {M_{\odot}}
\def \z {_{\rm z}}
\def \edd {_{\rm Edd}}
\def \lin {_{\rm lin}}
\def \nonlin {_{\rm non-lin}}
\def \wrms {\langle w_{\rm z}^2\rangle^{1/2}}
\def \dc {\delta_{\rm c}}
\def \wp {w_{p}(\sigma)}
\def \PwrSp {\mathcal{P}(k)}
\def \DelSq {$\Delta^{2}(k)$}
\def \WMAP {{\it WMAP \,}}
\def \cobe {{\it COBE }}
\def \COBE {{\it COBE \;}}
\def \HST  {{\it HST \,\,}}
\def \Spitzer  {{\it Spitzer \,}}
\def \ATLAS {VST-AA$\Omega$ {\it ATLAS} }
\def \BEST   {{\tt best} }
\def \TARGET {{\tt target} }
\def \TQSO   {{\tt TARGET\_QSO}}
\def \HIZ    {{\tt TARGET\_HIZ}}
\def \FIRST  {{\tt TARGET\_FIRST}}
\def \zc {z_{\rm c}}
\def \zcz {z_{\rm c,0}}

\newcommand{\ltsim}{\raisebox{-0.6ex}{$\,\stackrel
        {\raisebox{-.2ex}{$\textstyle <$}}{\sim}\,$}}
\newcommand{\gtsim}{\raisebox{-0.6ex}{$\,\stackrel
        {\raisebox{-.2ex}{$\textstyle >$}}{\sim}\,$}}
\newcommand{\simlt}{\raisebox{-0.6ex}{$\,\stackrel
        {\raisebox{-.2ex}{$\textstyle <$}}{\sim}\,$}}
\newcommand{\simgt}{\raisebox{-0.6ex}{$\,\stackrel
        {\raisebox{-.2ex}{$\textstyle >$}}{\sim}\,$}}

\newcommand{\Msun}{M_\odot}
\newcommand{\Lsun}{L_\odot}
\newcommand{\lsun}{L_\odot}
\newcommand{\Mdot}{\dot M}

\newcommand{\sqdeg}{deg$^{-2}$}
\newcommand{\lya}{Ly$\alpha$\ }
%\newcommand{\lya}{Ly\,$\alpha$\ }
\newcommand{\lyaf}{Ly\,$\alpha$\ forest}
%\newcommand{\eg}{e.g.~}
%\newcommand{\etal}{et~al.~}
\newcommand{\lyb}{Ly$\beta$\ }
\newcommand{\cii}{C\,{\sc ii}\ }
\newcommand{\ciii}{C\,{\sc iii}]\ }
\newcommand{\civ}{C\,{\sc iv}\ }
\newcommand{\SiIV}{Si\,{\sc iv}\ }
\newcommand{\mgii}{Mg\,{\sc ii}\ }
\newcommand{\feii}{Fe\,{\sc ii}\ }
\newcommand{\feiii}{Fe\,{\sc iii}\ }
\newcommand{\caii}{Ca\,{\sc ii}\ }
\newcommand{\halpha}{H\,$\alpha$\ }
\newcommand{\hbeta}{H\,$\beta$\ }
\newcommand{\hgamma}{H\,$\gamma$\ }
\newcommand{\hdelta}{H\,$\delta$\ }
\newcommand{\oi}{[O\,{\sc i}]\ }
\newcommand{\oii}{[O\,{\sc ii}]\ }
\newcommand{\oiii}{[O\,{\sc iii}]\ }
\newcommand{\heii}{[He\,{\sc ii}]\ }
\newcommand{\nv}{N\,{\sc v}\ }
\newcommand{\nev}{Ne\,{\sc v}\ }
\newcommand{\neiii}{[Ne\,{\sc iii}]\ }
\newcommand{\aliii}{Al\,{\sc iii}\ }
\newcommand{\siiii}{Si\,{\sc iii}]\ }


%%%%%%%%%%%%%%%%%%%%%%%%%%%%%%%%%%%%%%%%%%%%%%%%%%%%%
%              define Listings                       %
%%%%%%%%%%%%%%%%%%%%%%%%%%%%%%%%%%%%%%%%%%%%%%%%%%%%%
\definecolor{dkgreen}{rgb}{0,0.6,0}
\definecolor{gray}{rgb}{0.5,0.5,0.5}
\definecolor{mauve}{rgb}{0.58,0,0.82}

\lstset{frame=tb,
  language=Python,
  aboveskip=3mm,
  belowskip=3mm,
  showstringspaces=false,
  columns=flexible,
  basicstyle={\small\ttfamily},
  numbers=none,
  numberstyle=\tiny\color{gray},
  keywordstyle=\color{blue},
  commentstyle=\color{dkgreen},
  stringstyle=\color{mauve},
  breaklines=true,
  breakatwhitespace=true,
  tabsize=3
}

\begin{document}


\section{Bondi Hoyle}
\citet{Bondi_Hoyle1944};
The rate of accretion can be give as::
\begin{equation}
\dot{M} = \frac{4 \pi G^2 M^2 \rho_{\infty}}{v^3}
\end{equation}


\section{System physics and model properties} \label{sec:phys}
{\bf STRAIGHT FROM Camilo Fontecilla, Zoltán Haiman, Jorge Cuadra, https://arxiv.org/abs/1810.02857v1}. 
In this section, we first explain the relevant physics in our study,
then we discuss the simplifications needed to implement a 1D
simulation, and finally give details of our numerical code.

We model the inner-- and circumbinary discs assuming their scale
height $h$ is much smaller than the distance to the centre $r$, so the
system behaves as a standard, two-dimensional thin disc. Furthermore,
we consider the primary black hole to be at the center of mass of the
system, ignore the individual disc around the secondary, and assume
that the secondary and both accretion discs follow circular, co-planar
and prograde paths around the primary black hole. These assumptions
allow us to model the system in one dimension, with all disc
properties functions of the radial coordinate $r$ only.

\subsection{Viscosity, thickness and energetics} \label{sec:visc}

Viscosity acts as an angular momentum transport mechanism, which
produces a torque between contiguous rings of the disc. The standard
$\alpha$-disc model \citep{shakura73} parametrizes the turbulent
kinematic viscosity as a function of the sound speed $c_s$ and the
thickness of the disc $h$. While $c_s$ depends on the total pressure,
in this study we implement an alternative viscosity prescription
called a $\beta$-disc model, which ensures thermal and viscous
stability:
%
\begin{equation}
\nu = \alpha c_s h \beta, \label{eq:nu}.
\end{equation}
%
This allows us to put all the uncertainties in the constant $\alpha
\leq 1$. Here, $\beta$ is the ratio between the gas pressure $p_{\rm
gas}$ and total pressure $p_{\rm tot} = p_{\rm gas} + p_{\rm rad}$
(with $p_{\rm rad}$ the radiation pressure). For simplicity, we
consider the pressures to be determined by the mid-plane temperature
of the disc $T_c$: $p_{\rm rad} = 4 \sigma T_c^4 / (3 c)$ and $p_{\rm
gas} = \rho k T_c / (\mu m_p)$, with $k$ the Boltzmann constant,
$\sigma$ the Stefan-Boltzmann constant, and $\mu = 0.615$ the mean
particle mass in units of the proton mass $m_p$ for a plasma of solar
metallicity.

The sound speed in the disc is given by $c_s^2 = p_{\rm tot} / \rho$
with $\rho = \Sigma / (2 h)$ the volumetric density and $\Sigma$ the
surface density of the discs. Assuming hydrostatic equilibrium in the
vertical direction, we can express the sound speed as a function of
thickness $h$ and angular velocity: $c_s = \Omega h$, with $\Omega =
(GM(1+q)/r^3)^{1/2}$ the angular velocity of the material.

To close this system of equations, we relate the central temperature
$T_c$ to the surface density $\Sigma$, assuming photons are
transported to the disc surface by vertical diffusion: $F = 8 \sigma
T_c^4 / (3 \Sigma \kappa)$ (where $\kappa$ is the opacity due to
electron scattering), and that viscous and tidal heating are
dissipated in the form of radiation \citep{lodato09, kocsis12a},
%
\begin{equation}
F = D_\nu + D_\Lambda = \frac{9}{8} \nu \Sigma \Omega^2 - \frac{1}{2} \Lambda \Sigma (\Omega - \Omega_s). \label{eq:F}
\end{equation}
%
The tidal term $D_\Lambda$, as explained in \citet{lodato09}, comes
from angular momentum conservation and from the assumption that orbits
remain circular.

Using the above formulation leads to a system of equations, uniquely
determining the disc properties at each radius and time:
%
\begin{equation} 
\begin{aligned}
T_c &= \Bigg[\dfrac{3 \kappa \Sigma^2 \big(9 \alpha c_s^2 \Omega \beta - 4 \Lambda(\Omega - \Omega_s)\big)}{64 \sigma}\Bigg]^{\frac{1}{4}},\\
\beta &= \left[1 + \frac{8 \sigma \mu m_p T_c^3 c_s}{3 c k \Sigma \Omega}\right]^{- 1},\\
c_s &= \frac{8 \sigma T_c^4}{3 c \Omega \Sigma (1 - \beta)}.
\end{aligned}\label{eq:solution}
\end{equation}
%
It can be shown that there is only one real and positive solution for
the central temperature. Solving the equations allows us to obtain the
sound speed $c_s$, the thickness $h$ of the disc, and finally the
viscosity itself \citep{fontecilla17}. Having the central temperature
at each radius and assuming that the discs emit as a multi-temperature
blackbody \citep{frank02}, we can obtain the spectral energy
distribution (SED) and the bolometric luminosity ($L_{\rm bol}$) of
the system as a function of time.

\subsection{Surface Density evolution}

At least two mechanisms will make the surface density of the discs
evolve over time: viscosity, already explained in the previous
subsection, and the tidal torque produced by the changing
gravitational potential of the rotating binary. The latter is a 2D
effect, and cannot be implemented directly in a 1D simulation. To
bypass this, we adopt a commonly used recipe that models its effect on
the accretion discs. Following the literature \citep{armitage02},
\footnote{See \citet{dong11a, dong11b, petrovich12, rafikov16} for
discussion about the tidal prescription.}  we define an orbit-averaged
torque,
%
\begin{align}
\Lambda = \begin{cases}
- \dfrac{f}{2} q^2 \Omega^2 r^2 \bigg(\dfrac{r}{\Delta}\bigg)^4, & \mbox{if } r < a \\[12pt]
\enskip \dfrac{f}{2} q^2 \Omega^2 r^2 \bigg(\dfrac{a}{\Delta}\bigg)^4, & \mbox{if } r \geq a \\
\end{cases}\label{eq:smooth}
\end{align}
%
whose functional form depends on whether we are in the inner ($r < a$)
or circumbinary ($r \geq a$) discs. Here $f = 0.01$ is a
dimensionless parameter that controls the strength of the torque,
$\Delta = \max \{R_h,~h,~|r - a|\}$ is a smoothing term, and $R_h = a
(q / 3)^{1 / 3}$ is the Hill radius of the secondary black hole. While
this model was originally proposed for $q \ll 1$, it has been widely
used for binaries up to $q = 0.3$ \citep{lodato09, chang10, kocsis12a,
kocsis12b, tazzari15}.

While viscosity always transfers the angular momentum of the disc
outwards, the tidal effect depends on the position. Inside the binary
orbit ($r < a$) the tidal torque $\Lambda$ adds angular momentum to
the binary, producing an inward acceleration on the elements of the
disc. On the other hand, for the circumbinary disc ($r \geq a$), this
effect will add angular momentum to the gas, preventing it to cross
the binary orbit. Here, the balance between the tidal effect and the
viscous mechanism will produce a low density region called cavity or
gap, which depends on the mass ratio of the binary.

As pointed out by \citet{tazzari15}, for $q \geq 0.1$,
eq. (\ref{eq:smooth}) gives an unphysically large tidal contribution
on the discs outside the Lindblad resonances, which will alter the
migration velocity of the binary and the surface density of the
discs. Following their work we added an exponential cutoff for the
tidal torque outside this region,
%
\begin{align}
\Lambda = \begin{cases}
- \dfrac{f}{2} q^2 \Omega^2 r^2 \bigg(\dfrac{r}{\Delta}\bigg)^4 \exp\left[- \left(\dfrac{r - r_{\rm IMLR}}{w_{\rm IMLR}}\right)^2\right] & \mbox{if } r \leq r_{\rm IMLR}, \\[18pt]
\enskip \dfrac{f}{2}q^{2}\Omega^{2}r^{2}\bigg(\dfrac{a}{\Delta}\bigg)^4 \exp\left[-\left(\dfrac{r-r_{\rm OMLR}}{w_{\rm OMLR}}\right)^2\right] & \mbox{if } r\geq r_{\rm OMLR}.
\raisetag{38pt}
\end{cases}\label{eq:smooth2}
\end{align}
%
Here, $r_{\rm IMLR} = 0.63 a$ and $r_{\rm OMLR} = 1.59 a$ are the radius of the innermost and outermost Lindblad resonances, while $w_{\rm IMLR} = 370 h$ and $w_{\rm OMLR} = 75 h$ are the widths of the Gaussian smoothing. These values were used by \citet{tazzari15} to reproduce the gap sizes of a $q = 0.11$ binary from \citet{artymowicz94}.

Following \citet{frank02}, considering the mass continuity and angular momentum conservation equations in the discs, adding the viscous and tidal torques (eq. \ref{eq:smooth}), the surface density evolution can be written as:
%
\begin{equation}
\frac{\partial \Sigma}{\partial t} = - \frac{1}{r}\frac{\partial}{\partial r}\left( - 3 r^{1 / 2}\frac{\partial}{\partial r}\left(\nu \Sigma r^{1 / 2}\right) + 2 \frac{\Sigma \Lambda}{\Omega}\right), \label{eq:sigma}
\end{equation}
%
where the first term at the right hand side is the effect of viscosity and the second one is produced by the tidal interaction between the binary and the discs. 



\section{What's the difference, (``If Any!!'') between Proto-stellar disks \& AGN disks ???!!!!}
\begin{landscape}
\begin{table}[]
  \centering
  \caption{What are the similarities and differences between 
    Proto-stellar and AGN accretion disks?
}
  \label{my-label}
  \begin{tabular}{  p{65mm}   p{70mm}  p{70mm} }
    \hline
    \hline
 &  &  \\
                              & Proto-stellar disks & AGN disks \\
 &  &  \\
        \hline
 &  &  \\
    $h/r$                              & $\sim0.1$    &   \\
 &  &  \\
    Adiabatic/isothermal?      &  Mainly adiabatic &  \\
 &  &  \\
    $B$-field strength             &  Interesting issue.  Thought to be sensitive to MRI at later stages at least.  
                                               However, there are some arguing that global magnetic fields may play a key in transporting angular momentum away.  &  \\
 &  &  \\
%    \multirow{2}{*}{Mechamism(s) for  turbulence generation}          &  &  \\
      Mechamism(s) for  turbulence generation           &  Self-gravity at early times,  MRI later &    \\
           &  &  \\
Dust chemistry                       &  Certainly many people working on chemistry in these discs
&  \\
 &  &  \\
Dust opacity                        &  Regarded as important for cooling&  \\
 &  &  \\
Iron present?                        & Yes, and regarded as having, initially at least, an ISM composition. &  \\
 &  &  \\
    \hline
    \hline
\end{tabular}
\end{table}
\end{landscape}




\section{ADAFs, RIAFs, etcs.}



\section{ADAFs, RIAFs, etcs.}
see reviews by Quataert [2001]; Narayan [2005]; early versions of RIAF models were called advection-dominated accretion flows [ADAFs] or ‘‘ion tori’’;


\clearpage
%\Huge %\huge \LARGE \Large 
%\large 
%\normalsize (default) \small \footnotesize \scriptsize \tiny
\begin{landscape}
  \begin{table}[]
\large
    \centering
    \caption{}
    \label{my-label}
    \begin{tabular}{  p{50mm}   p{60mm}   p{100mm}  }
      \hline
      \hline
      &  &  \\
      Temperature                         & \multirow{2}{*}{\sc Cold}                                                  &  \multirow{2}{*} {\sc Hot} \\
      (cf. Virial temperature)           & & \\
      &  &  \\
%      \hline
 %     &  &  \\
      geometry ($h/R$)                          & thin, $\lesssim 0.1$                                                     & thick, $\sim0.5$ \\
%      &  &  \\
      gas opacity                                                & optically thick                                  & optically thin ($\tau <1$) \\
 %     &  &  \\
     $\dot{M}$                                            & generally high                                  & low(er) \\
  %    &  &  \\
      radiation pressure                               & negligible                                          & non-negligible \\
%      && \\
      radiative cooling                         &  generally efficient                               & generally inefficient       \\
 %     && \\
%      viscosity                                        & $ \nu=\alpha \, c_{s}, h$, not well constrained                                               & higher       \\
  %    && \\
%      radial velocity                                        &                                                        & large (due to large $c_{s}$ and $h/R$)\\
   %   &  &  \\
      angular velocity                               & generally  Keplerian                                    & sub-Keplerian \\
      &  &  \\
      Outflows? Jets?                               & Yes? No.                                          & Yes, Yes \\
      Feedback mode                               & ``Radiative/Wind/Transition''                 & ``Jet/Kinetic/Maintenances'' \\
      & & \\
    Named examples                           &   ``Slim''                        & advection dominated accretion flow (ADAF)\\
                                                           &                                             & (adiabatic inflow-outflow solution; ADIOS) \\
                                                           &   Shakura-Sunyaev disk                & (convection-dominated accretion flow; CDAF) \\
                                                           &   (a.k.a ``thin'' + $\alpha$)                    & radiatively inefficient accretion flow (RIAF)\\
                                                           &                                                         & ``luminous hot''  (LHAF)\\
                                                          &                                                          & SLE (Shaprio, Lightman, Eardley, 1976, ApJ, {\bf 204}, {\it 187})   \\ 
      & & \\
  Type of objects            & quasars                                           & Low-luminosity AGN (LLAGN) \\
                                                               &                                                      & BHXBs in ``Hard and Quiescent'' state \\      
& & \\
    \hline
    \hline
\end{tabular}
\end{table}
\end{landscape}
\normalsize  
%% From   ASTR 610: Theory of Galaxy Formation
%% Frank van den Bosch, Yale
%% Lecture 14. 
\begin{equation}
T_{\rm vir} = \frac{\mu m_{p}}{2 k_{B}} V^{2}_{\rm vir} \simeq 3.6\times10^{5} \left ( \frac{V_{\rm vir}} {100\, {\rm km/s}} \right )^{2}
\end{equation}

\newpage

\begin{align}
T(R) &= {\left \{   \frac{3 GM \dot{M}} {8 \pi R^{3} \sigma} \left [  1 -\frac{R}{R_{*}} \right ]^{1/2}   \right \}}^{1/4} \\
   & = \left ( \frac{3 G M \dot{M}}{8 \pi \sigma R^{3}_{*}}  \right )^{1/4} \\
& \approx 5 \times 10^{5}    \left(   \frac{M}{10^{8} M_{\odot}}   \right  )^{1/4}  \left(  \frac{\dot{M}}{10^{23} {\rm kg\, s^{-1}}}  \right )^{1/4} \left ( \frac{r}{r_{g}} \right )^{-3/4}  {\rm K} \\
\end{align}
with $T = T_{*} (R/R_{*})^{-3/4}$ for $R \gg R_{*}$. 

\begin{align}
T(R) &= \left ( \frac{3 G M \dot{M}}{8 \pi \sigma R^{3}_{*}}  \right )^{1/4} \\
      & \approx 5 \times 10^{5}    \left(   \frac{M}{10^{8} M_{\odot}}   \right  )^{1/4}  \left(  \frac{\dot{M}}{10^{23} {\rm kg\, s^{-1}}}  \right )^{1/4} \left ( \frac{r}{r_{g}} \right )^{-3/4}  {\rm K} \\
\end{align}

\begin{equation}
T(R) = \left ( \frac{3 G M \dot{M}}{8 \pi \sigma R^{3}_{*}}  \right )^{1/4} 
\end{equation}

\begin{equation}
  \approx 5 \times 10^{5}    \left(   \frac{M}{10^{8} M_{\odot}}   \right  )^{1/4}  \left(  \frac{\dot{M}}{10^{23} {\rm kg\, s^{-1}}}  \right )^{1/4} \left ( \frac{r}{r_{g}} \right )^{-3/4}  {\rm K} \\
\end{equation}

\begin{equation}
  \approx 5 \times 10^{5}    \left(   \frac{M}{10^{8} M_{\odot}}   \right  )^{1/4}  \left(  \frac{\dot{M}}{0.6 M_{\odot} {\rm yr^{-1}}}  \right )^{1/4} \left ( \frac{r}{r_{g}} \right )^{-3/4}  {\rm K} \\
\end{equation}

\smallskip
\smallskip
\noindent
\begin{equation*}
  \sim 8 \times 10^{5}    \left(   \frac{M}{10^{8} M_{\odot}}   \right  )^{1/4}  \left(  \frac{\dot{M}}{M_{\odot} {\rm yr^{-1}}}  \right )^{1/4} \left ( \frac{r}{r_{g}} \right )^{-3/4}  {\rm K} \\
\end{equation*}

\smallskip
\smallskip
\noindent
\begin{equation*}
T(R) = \left ( \frac{3 G M \dot{M}}{8 \pi \sigma R^{3}_{*}}  \right )^{1/4}   
  \approx 5 \times 10^{5}    \left(   \frac{M}{10^{8} M_{\odot}}   \right  )^{1/4}  \left(  \frac{\dot{M}}{0.6 M_{\odot} {\rm yr^{-1}}}  \right )^{1/4} \left ( \frac{r}{r_{g}} \right )^{-3/4}  {\rm K} \\
\end{equation*}


\section{This, THIS, THIS!!!!}
\href{http://www.scholarpedia.org/article/Accretion\_discs}{http://www.scholarpedia.org/article/Accretion\_discs}


\newpage
\section{Astrophysics of accretion disks - Charles Gammie}

``Disk's are at the heart of many of the most interesting problems in
theoretical astrophysics.''
\href{https://www.youtube.com/watch?v=5qwyU3_JWE0}{YouTube video}.

\subsection{Examples of Astro-disks}
\begin{equation}
    h / R \ll 1 
\end{equation}
where $h$ is the scale-height and $R$ is the (local) radius.  This is
$\sim10^{-3}$ for NGC 4258.  This ratio is proportional to sound
speed in the disk / rotational velocity, which is $\sim 1 {\rm km
s}^{-1} / 1000 {\rm km s}^{-1}$.

$T$ is 1 billion degrees, and $n_{\rm e} \sim 10^6$ means m.f.p. for
Columbo scattering is very large.  Introduce the ``Knuddson number'':
\begin{equation} 
  \kappa_{\rm n} = \lambda_{\rm mfp}/R
\end{equation} where $R$ is the general 'scale of the system and
$\kappa_{\rm n}$ for Sgr A$^*$ is around $10^5$.  Evolution of
turbulence in collisionless plasmas.

Two more dimensionless numbers.  Magnetic Reynolds number, which
allows you to asses the non-ideal effects in the plasma around a
young star 
\begin{equation} 
  Re_{m} = \frac{ c_{s} h} {\eta}
\end{equation} $\eta$ is the electrical resistivity.

Toomre $Q$ parameter, measures the importance of self-gravity in the
disk.
\begin{equation} 
  Q = \frac{ c_{s} \Omega } {\pi G \Sigma}
\end{equation} where $\Sigma$ is the surface density of the disk.
When $Q\lesssim1$ then self-gravity becomes important in the disk.


\subsection{Disk Evolution}
All about ANGULAR MOMENTUM. 

First, disk Equilibrium.  
\begin{itemize}
  \item Dynamical equilibrium: 
\end{itemize}
\begin{equation}
  \Omega = (GM / R^3)^(1/2)  + O(H/R)^{2}          \;\;\;\;\;\; v_{z} = 0
\end{equation}
and $\Delta t \sim \Omega^{-1}$. 
%%
{\it Nota bene} the scale height is proportional to the sound speed divided 
by the rotational frequency: $H = c_{s} / \Omega$. 

In a PROTOPLANETARY disk, this is a GAS pressure dominated disk.
Whereas in an AGN accretion disk, the pressure is dominated by the
RADIATION pressure.

\begin{itemize}
  \item Thermal equilibrium:   $Q^{+}=Q^{-}$
\end{itemize}
 $Q^{+}$ is heating rate,  $Q^{-}$ is radiative cooling rate. 
\begin{equation}
  \Delta t \sim \Sigma c_{s}^{2} / Q^{+} \sim (\alpha \Omega)^{-1}
\end{equation}
The heating in some disks is dominated by internal friction related to
the dissipation of turbulence.  And in an optically thick disk, $Q^{+}
\sim \sigma T^{4}$ (just like a star).  $\alpha$ describes the
intensity of turbulence inside the disk.  So the $Q^{+} \propto$ local
dynamical time with constant of proportionality being $\alpha$.
Typically, $\alpha \sim 1/10 - 1/100$ which is substantially longer
than $t$ for dynamical equilibrium.

\begin{itemize}
  \item Viscous equilibrium:   $\dot{M} =$ const. 
\end{itemize}
Time for which {\it inflow} equilibrium is reached. 
\begin{equation}
  \Delta t \sim M_{\rm disk}/\dot{M}   \sim  (\alpha \Omega)^{-1} (R/H)^2. 
\end{equation}
e.g. protoplanatary disks, Viscous timescale maybe a factor 100 longer
than Thermal timescale. 

\subsection{Disk Evolution}



\bibliographystyle{mn2e}
\bibliography{/cos_pc19a_npr/LaTeX/tester_mnras}

\end{document}
