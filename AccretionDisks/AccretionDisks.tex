\documentclass[11pt,a4paper]{article}

\usepackage{multirow}
\usepackage{graphicx,fancyhdr,natbib,subfigure}
\usepackage{epsfig, epsf}
\usepackage{amsmath, cancel, amssymb}
\usepackage{lscape, longtable, caption}
\usepackage{dcolumn}% Align table columns on decimal point
\usepackage{bm}% bold math
\usepackage{hyperref,ifthen}
\usepackage{verbatim}
\usepackage{color}
\usepackage[usenames,dvipsnames]{xcolor}
\usepackage{listings}


\usepackage[toc,page]{appendix}
\usepackage{amsmath, amssymb}
\usepackage{bm}% bold math
\usepackage{cancel, caption}
\usepackage{dcolumn}% Align table columns on decimal point
\usepackage{epsfig, epsf}
\usepackage{graphicx,fancyhdr,natbib,subfigure}
\usepackage{lscape, longtable}
\usepackage{hyperref,ifthen}
\usepackage{verbatim}
\usepackage{color}
\usepackage[usenames,dvipsnames]{xcolor}
\usepackage{listings}
%% http://en.wikibooks.org/wiki/LaTeX/Colors



%%%%%%%%%%%%%%%%%%%%%%%%%%%%%%%%%%%%%%%%%%%
%       define Journal abbreviations      %
%%%%%%%%%%%%%%%%%%%%%%%%%%%%%%%%%%%%%%%%%%%
\def\nat{Nat} \def\apjl{ApJ~Lett.} \def\apj{ApJ}
\def\apjs{ApJS} \def\aj{AJ} \def\mnras{MNRAS}
\def\prd{Phys.~Rev.~D} \def\prl{Phys.~Rev.~Lett.}
\def\plb{Phys.~Lett.~B} \def\jhep{JHEP} \def\nar{NewAR}
\def\npbps{NUC.~Phys.~B~Proc.~Suppl.} \def\prep{Phys.~Rep.}
\def\pasp{PASP} \def\aap{Astron.~\&~Astrophys.} \def\araa{ARA\&A}
\def\jcap{\ref@jnl{J. Cosmology Astropart. Phys.}}%
\def\physrep{Phys.~Rep.}

\newcommand{\preep}[1]{{\tt #1} }

%%%%%%%%%%%%%%%%%%%%%%%%%%%%%%%%%%%%%%%%%%%%%%%%%%%%%
%              define symbols                       %
%%%%%%%%%%%%%%%%%%%%%%%%%%%%%%%%%%%%%%%%%%%%%%%%%%%%%
\def \Mpc {~{\rm Mpc} }
\def \Om {\Omega_0}
\def \Omb {\Omega_{\rm b}}
\def \Omcdm {\Omega_{\rm CDM}}
\def \Omlam {\Omega_{\Lambda}}
\def \Omm {\Omega_{\rm m}}
\def \ho {H_0}
\def \qo {q_0}
\def \lo {\lambda_0}
\def \kms {{\rm ~km~s}^{-1}}
\def \kmsmpc {{\rm ~km~s}^{-1}~{\rm Mpc}^{-1}}
\def \hmpc{~\;h^{-1}~{\rm Mpc}} 
\def \hkpc{\;h^{-1}{\rm kpc}} 
\def \hmpcb{h^{-1}{\rm Mpc}}
\def \dif {{\rm d}}
\def \mlim {m_{\rm l}}
\def \bj {b_{\rm J}}
\def \mb {M_{\rm b_{\rm J}}}
\def \mg {M_{\rm g}}
\def \qso {_{\rm QSO}}
\def \lrg {_{\rm LRG}}
\def \gal {_{\rm gal}}
\def \xibar {\bar{\xi}}
\def \xis{\xi(s)}
\def \xisp{\xi(\sigma, \pi)}
\def \Xisig{\Xi(\sigma)}
\def \xir{\xi(r)}
\def \max {_{\rm max}}
\def \gsim { \lower .75ex \hbox{$\sim$} \llap{\raise .27ex \hbox{$>$}} }
\def \lsim { \lower .75ex \hbox{$\sim$} \llap{\raise .27ex \hbox{$<$}} }
\def \deg {^{\circ}}
%\def \sqdeg {\rm deg^{-2}}
\def \deltac {\delta_{\rm c}}
\def \mmin {M_{\rm min}}
\def \mbh  {M_{\rm BH}}
\def \mdh  {M_{\rm DH}}
\def \msun {M_{\odot}}
\def \z {_{\rm z}}
\def \edd {_{\rm Edd}}
\def \lin {_{\rm lin}}
\def \nonlin {_{\rm non-lin}}
\def \wrms {\langle w_{\rm z}^2\rangle^{1/2}}
\def \dc {\delta_{\rm c}}
\def \wp {w_{p}(\sigma)}
\def \PwrSp {\mathcal{P}(k)}
\def \DelSq {$\Delta^{2}(k)$}
\def \WMAP {{\it WMAP \,}}
\def \cobe {{\it COBE }}
\def \COBE {{\it COBE \;}}
\def \HST  {{\it HST \,\,}}
\def \Spitzer  {{\it Spitzer \,}}
\def \ATLAS {VST-AA$\Omega$ {\it ATLAS} }
\def \BEST   {{\tt best} }
\def \TARGET {{\tt target} }
\def \TQSO   {{\tt TARGET\_QSO}}
\def \HIZ    {{\tt TARGET\_HIZ}}
\def \FIRST  {{\tt TARGET\_FIRST}}
\def \zc {z_{\rm c}}
\def \zcz {z_{\rm c,0}}

\newcommand{\ltsim}{\raisebox{-0.6ex}{$\,\stackrel
        {\raisebox{-.2ex}{$\textstyle <$}}{\sim}\,$}}
\newcommand{\gtsim}{\raisebox{-0.6ex}{$\,\stackrel
        {\raisebox{-.2ex}{$\textstyle >$}}{\sim}\,$}}
\newcommand{\simlt}{\raisebox{-0.6ex}{$\,\stackrel
        {\raisebox{-.2ex}{$\textstyle <$}}{\sim}\,$}}
\newcommand{\simgt}{\raisebox{-0.6ex}{$\,\stackrel
        {\raisebox{-.2ex}{$\textstyle >$}}{\sim}\,$}}

\newcommand{\Msun}{M_\odot}
\newcommand{\Lsun}{L_\odot}
\newcommand{\lsun}{L_\odot}
\newcommand{\Mdot}{\dot M}

\newcommand{\sqdeg}{deg$^{-2}$}
\newcommand{\lya}{Ly$\alpha$\ }
%\newcommand{\lya}{Ly\,$\alpha$\ }
\newcommand{\lyaf}{Ly\,$\alpha$\ forest}
%\newcommand{\eg}{e.g.~}
%\newcommand{\etal}{et~al.~}
\newcommand{\lyb}{Ly$\beta$\ }
\newcommand{\cii}{C\,{\sc ii}\ }
\newcommand{\ciii}{C\,{\sc iii}]\ }
\newcommand{\civ}{C\,{\sc iv}\ }
\newcommand{\SiIV}{Si\,{\sc iv}\ }
\newcommand{\mgii}{Mg\,{\sc ii}\ }
\newcommand{\feii}{Fe\,{\sc ii}\ }
\newcommand{\feiii}{Fe\,{\sc iii}\ }
\newcommand{\caii}{Ca\,{\sc ii}\ }
\newcommand{\halpha}{H\,$\alpha$\ }
\newcommand{\hbeta}{H\,$\beta$\ }
\newcommand{\hgamma}{H\,$\gamma$\ }
\newcommand{\hdelta}{H\,$\delta$\ }
\newcommand{\oi}{[O\,{\sc i}]\ }
\newcommand{\oii}{[O\,{\sc ii}]\ }
\newcommand{\oiii}{[O\,{\sc iii}]\ }
\newcommand{\heii}{[He\,{\sc ii}]\ }
\newcommand{\nv}{N\,{\sc v}\ }
\newcommand{\nev}{Ne\,{\sc v}\ }
\newcommand{\neiii}{[Ne\,{\sc iii}]\ }
\newcommand{\aliii}{Al\,{\sc iii}\ }
\newcommand{\siiii}{Si\,{\sc iii}]\ }


%%%%%%%%%%%%%%%%%%%%%%%%%%%%%%%%%%%%%%%%%%%%%%%%%%%%%
%              define Listings                       %
%%%%%%%%%%%%%%%%%%%%%%%%%%%%%%%%%%%%%%%%%%%%%%%%%%%%%
\definecolor{dkgreen}{rgb}{0,0.6,0}
\definecolor{gray}{rgb}{0.5,0.5,0.5}
\definecolor{mauve}{rgb}{0.58,0,0.82}

\lstset{frame=tb,
  language=Python,
  aboveskip=3mm,
  belowskip=3mm,
  showstringspaces=false,
  columns=flexible,
  basicstyle={\small\ttfamily},
  numbers=none,
  numberstyle=\tiny\color{gray},
  keywordstyle=\color{blue},
  commentstyle=\color{dkgreen},
  stringstyle=\color{mauve},
  breaklines=true,
  breakatwhitespace=true,
  tabsize=3
}

\begin{document}


\section{Bondi Hoyle}
\citet{Bondi_Hoyle1944};
The rate of accretion can be give as::
\begin{equation}
\dot{M} = \frac{4 \pi G^2 M^2 \rho_{\infty}}{v^3}
\end{equation}



\section{What's the difference, (``If Any!!'') between Proto-stellar disks \& AGN disks ???!!!!}
\begin{landscape}
\begin{table}[]
  \centering
  \caption{What are the similiarties and differences between 
    Proto-stellar and AGN accretion disks?
}
  \label{my-label}
  \begin{tabular}{  p{65mm}   p{70mm}  p{70mm} }
    \hline
    \hline
 &  &  \\
                              & Proto-stellar disks & AGN disks \\
 &  &  \\
        \hline
 &  &  \\
    $h/r$                              & $\sim0.1$    &   \\
 &  &  \\
    Adiabatic/isothermal?      &  Mainly adiabatic &  \\
 &  &  \\
    $B$-field strength             &  Interesting issue.  Thought to be sensitive to MRI at later stages at least.  
                                               However, there are some arguing that global magnetic fields may play a key in transporting angular momentum away.  &  \\
 &  &  \\
%    \multirow{2}{*}{Mechamism(s) for  turbulence generation}          &  &  \\
      Mechamism(s) for  turbulence generation           &  Self-gravity at early times,  MRI later &    \\
           &  &  \\
Dust chemistry                       &  Certainly many people working on chemistry in these discs
&  \\
 &  &  \\
Dust opacity                        &  Regarded as important for cooling&  \\
 &  &  \\
Iron present?                        & Yes, and regarded as having, initially at least, an ISM composition. &  \\
 &  &  \\
    \hline
    \hline
\end{tabular}
\end{table}
\end{landscape}




\section{ADAFs, RIAFs, etcs.}



\section{ADAFs, RIAFs, etcs.}
see reviews by Quataert [2001]; Narayan [2005]; early versions of RIAF models were called advection-dominated accretion flows [ADAFs] or ‘‘ion tori’’;


\clearpage
%\Huge %\huge \LARGE \Large 
%\large 
%\normalsize (default) \small \footnotesize \scriptsize \tiny
\begin{landscape}
  \begin{table}[]
\large
    \centering
    \caption{}
    \label{my-label}
    \begin{tabular}{  p{50mm}   p{60mm}   p{100mm}  }
      \hline
      \hline
      &  &  \\
      Temperature                         & \multirow{2}{*}{\sc Cold}                                                  &  \multirow{2}{*} {\sc Hot} \\
      (cf. Virial temperature)           & & \\
      &  &  \\
%      \hline
 %     &  &  \\
      geometry ($h/R$)                          & thin, $\lesssim 0.1$                                                     & thick, $\sim0.5$ \\
%      &  &  \\
      gas opacity                                                & optically thick                                  & optically thin ($\tau <1$) \\
 %     &  &  \\
     $\dot{M}$                                            & generally high                                  & low(er) \\
  %    &  &  \\
      radiation pressure                               & negligible                                          & non-negligible \\
%      && \\
      radiative cooling                         &  generally efficient                               & generally inefficient       \\
 %     && \\
%      viscosity                                        & $ \nu=\alpha \, c_{s}, h$, not well constrained                                               & higher       \\
  %    && \\
%      radial velocity                                        &                                                        & large (due to large $c_{s}$ and $h/R$)\\
   %   &  &  \\
      angular velocity                               & generally  Keplerian                                    & sub-Keplerian \\
      &  &  \\
      Outflows? Jets?                               & Yes? No.                                          & Yes, Yes \\
      Feedback mode                               & ``Radiative/Wind/Tranistionn''                 & ``Jet/Kinetic/Maintenances'' \\
      & & \\
    Named examples                           &   ``Slim''                        & advection dominated accretion flow (ADAF)\\
                                                           &                                             & (adiabatic inflow-outflow solution; ADIOS) \\
                                                           &   Shakura-Sunyaev disk                & (convection-dominated accretion flow; CDAF) \\
                                                           &   (a.k.a ``thin'' + $\alpha$)                    & radiatively inefficient accretion flow (RIAF)\\
                                                           &                                                         & ``luminous hot''  (LHAF)\\
                                                          &                                                          & SLE (Shaprio, Lightman, Eardley, 1976, ApJ, {\bf 204}, {\it 187})   \\ 
      & & \\
  Type of objects            & quasars                                           & Low-luminosity AGN (LLAGN) \\
                                                               &                                                      & BHXBs in ``Hard and Quiescent'' state \\      
& & \\
    \hline
    \hline
\end{tabular}
\end{table}
\end{landscape}
\normalsize  
%% From   ASTR 610: Theory of Galaxy Formation
%% Frank van den Bosch, Yale
%% Lecture 14. 
\begin{equation}
T_{\rm vir} = \frac{\mu m_{p}}{2 k_{B}} V^{2}_{\rm vir} \simeq 3.6\times10^{5} \left ( \frac{V_{\rm vir}} {100\, {\rm km/s}} \right )^{2}
\end{equation}

\newpage

\begin{align}
T(R) &= {\left \{   \frac{3 GM \dot{M}} {8 \pi R^{3} \sigma} \left [  1 -\frac{R}{R_{*}} \right ]^{1/2}   \right \}}^{1/4} \\
   & = \left ( \frac{3 G M \dot{M}}{8 \pi \sigma R^{3}_{*}}  \right )^{1/4} \\
& \approx 5 \times 10^{5}    \left(   \frac{M}{10^{8} M_{\odot}}   \right  )^{1/4}  \left(  \frac{\dot{M}}{10^{23} {\rm kg\, s^{-1}}}  \right )^{1/4} \left ( \frac{r}{r_{g}} \right )^{-3/4}  {\rm K} \\
\end{align}
with $T = T_{*} (R/R_{*})^{-3/4}$ for $R \gg R_{*}$. 

\begin{align}
T(R) &= \left ( \frac{3 G M \dot{M}}{8 \pi \sigma R^{3}_{*}}  \right )^{1/4} \\
      & \approx 5 \times 10^{5}    \left(   \frac{M}{10^{8} M_{\odot}}   \right  )^{1/4}  \left(  \frac{\dot{M}}{10^{23} {\rm kg\, s^{-1}}}  \right )^{1/4} \left ( \frac{r}{r_{g}} \right )^{-3/4}  {\rm K} \\
\end{align}

\begin{equation}
T(R) = \left ( \frac{3 G M \dot{M}}{8 \pi \sigma R^{3}_{*}}  \right )^{1/4} 
\end{equation}

\begin{equation}
  \approx 5 \times 10^{5}    \left(   \frac{M}{10^{8} M_{\odot}}   \right  )^{1/4}  \left(  \frac{\dot{M}}{10^{23} {\rm kg\, s^{-1}}}  \right )^{1/4} \left ( \frac{r}{r_{g}} \right )^{-3/4}  {\rm K} \\
\end{equation}

\begin{equation}
  \approx 5 \times 10^{5}    \left(   \frac{M}{10^{8} M_{\odot}}   \right  )^{1/4}  \left(  \frac{\dot{M}}{0.6 M_{\odot} {\rm yr^{-1}}}  \right )^{1/4} \left ( \frac{r}{r_{g}} \right )^{-3/4}  {\rm K} \\
\end{equation}

\smallskip
\smallskip
\noindent
\begin{equation*}
  \sim 8 \times 10^{5}    \left(   \frac{M}{10^{8} M_{\odot}}   \right  )^{1/4}  \left(  \frac{\dot{M}}{M_{\odot} {\rm yr^{-1}}}  \right )^{1/4} \left ( \frac{r}{r_{g}} \right )^{-3/4}  {\rm K} \\
\end{equation*}

\smallskip
\smallskip
\noindent
\begin{equation*}
T(R) = \left ( \frac{3 G M \dot{M}}{8 \pi \sigma R^{3}_{*}}  \right )^{1/4}   
  \approx 5 \times 10^{5}    \left(   \frac{M}{10^{8} M_{\odot}}   \right  )^{1/4}  \left(  \frac{\dot{M}}{0.6 M_{\odot} {\rm yr^{-1}}}  \right )^{1/4} \left ( \frac{r}{r_{g}} \right )^{-3/4}  {\rm K} \\
\end{equation*}


\section{This, THIS, THIS!!!!}
\href{http://www.scholarpedia.org/article/Accretion\_discs}{http://www.scholarpedia.org/article/Accretion\_discs}



\bibliographystyle{mn2e}
\bibliography{/cos_pc19a_npr/LaTeX/tester_mnras}

\end{document}
