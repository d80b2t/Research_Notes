\documentclass[11pt]{article}
%\setlength {\textwidth}{180mm} 
%\setlength {\textheight}{260mm}
%\topmargin=-35.00mm
%\oddsidemargin=-10.00mm
%\pagestyle{empty}



\usepackage[toc,page]{appendix}
\usepackage{amsmath, amssymb}
\usepackage{bm}% bold math
\usepackage{cancel, caption}
\usepackage{dcolumn}% Align table columns on decimal point
\usepackage{epsfig, epsf}
\usepackage{graphicx,fancyhdr,natbib,subfigure}
\usepackage{lscape, longtable}
\usepackage{hyperref,ifthen}
\usepackage{verbatim}
\usepackage{color}
\usepackage[usenames,dvipsnames]{xcolor}
\usepackage{listings}
%% http://en.wikibooks.org/wiki/LaTeX/Colors



%%%%%%%%%%%%%%%%%%%%%%%%%%%%%%%%%%%%%%%%%%%
%       define Journal abbreviations      %
%%%%%%%%%%%%%%%%%%%%%%%%%%%%%%%%%%%%%%%%%%%
\def\nat{Nat} \def\apjl{ApJ~Lett.} \def\apj{ApJ}
\def\apjs{ApJS} \def\aj{AJ} \def\mnras{MNRAS}
\def\prd{Phys.~Rev.~D} \def\prl{Phys.~Rev.~Lett.}
\def\plb{Phys.~Lett.~B} \def\jhep{JHEP} \def\nar{NewAR}
\def\npbps{NUC.~Phys.~B~Proc.~Suppl.} \def\prep{Phys.~Rep.}
\def\pasp{PASP} \def\aap{Astron.~\&~Astrophys.} \def\araa{ARA\&A}
\def\jcap{\ref@jnl{J. Cosmology Astropart. Phys.}}%
\def\physrep{Phys.~Rep.}

\newcommand{\preep}[1]{{\tt #1} }

%%%%%%%%%%%%%%%%%%%%%%%%%%%%%%%%%%%%%%%%%%%%%%%%%%%%%
%              define symbols                       %
%%%%%%%%%%%%%%%%%%%%%%%%%%%%%%%%%%%%%%%%%%%%%%%%%%%%%
\def \Mpc {~{\rm Mpc} }
\def \Om {\Omega_0}
\def \Omb {\Omega_{\rm b}}
\def \Omcdm {\Omega_{\rm CDM}}
\def \Omlam {\Omega_{\Lambda}}
\def \Omm {\Omega_{\rm m}}
\def \ho {H_0}
\def \qo {q_0}
\def \lo {\lambda_0}
\def \kms {{\rm ~km~s}^{-1}}
\def \kmsmpc {{\rm ~km~s}^{-1}~{\rm Mpc}^{-1}}
\def \hmpc{~\;h^{-1}~{\rm Mpc}} 
\def \hkpc{\;h^{-1}{\rm kpc}} 
\def \hmpcb{h^{-1}{\rm Mpc}}
\def \dif {{\rm d}}
\def \mlim {m_{\rm l}}
\def \bj {b_{\rm J}}
\def \mb {M_{\rm b_{\rm J}}}
\def \mg {M_{\rm g}}
\def \qso {_{\rm QSO}}
\def \lrg {_{\rm LRG}}
\def \gal {_{\rm gal}}
\def \xibar {\bar{\xi}}
\def \xis{\xi(s)}
\def \xisp{\xi(\sigma, \pi)}
\def \Xisig{\Xi(\sigma)}
\def \xir{\xi(r)}
\def \max {_{\rm max}}
\def \gsim { \lower .75ex \hbox{$\sim$} \llap{\raise .27ex \hbox{$>$}} }
\def \lsim { \lower .75ex \hbox{$\sim$} \llap{\raise .27ex \hbox{$<$}} }
\def \deg {^{\circ}}
%\def \sqdeg {\rm deg^{-2}}
\def \deltac {\delta_{\rm c}}
\def \mmin {M_{\rm min}}
\def \mbh  {M_{\rm BH}}
\def \mdh  {M_{\rm DH}}
\def \msun {M_{\odot}}
\def \z {_{\rm z}}
\def \edd {_{\rm Edd}}
\def \lin {_{\rm lin}}
\def \nonlin {_{\rm non-lin}}
\def \wrms {\langle w_{\rm z}^2\rangle^{1/2}}
\def \dc {\delta_{\rm c}}
\def \wp {w_{p}(\sigma)}
\def \PwrSp {\mathcal{P}(k)}
\def \DelSq {$\Delta^{2}(k)$}
\def \WMAP {{\it WMAP \,}}
\def \cobe {{\it COBE }}
\def \COBE {{\it COBE \;}}
\def \HST  {{\it HST \,\,}}
\def \Spitzer  {{\it Spitzer \,}}
\def \ATLAS {VST-AA$\Omega$ {\it ATLAS} }
\def \BEST   {{\tt best} }
\def \TARGET {{\tt target} }
\def \TQSO   {{\tt TARGET\_QSO}}
\def \HIZ    {{\tt TARGET\_HIZ}}
\def \FIRST  {{\tt TARGET\_FIRST}}
\def \zc {z_{\rm c}}
\def \zcz {z_{\rm c,0}}

\newcommand{\ltsim}{\raisebox{-0.6ex}{$\,\stackrel
        {\raisebox{-.2ex}{$\textstyle <$}}{\sim}\,$}}
\newcommand{\gtsim}{\raisebox{-0.6ex}{$\,\stackrel
        {\raisebox{-.2ex}{$\textstyle >$}}{\sim}\,$}}
\newcommand{\simlt}{\raisebox{-0.6ex}{$\,\stackrel
        {\raisebox{-.2ex}{$\textstyle <$}}{\sim}\,$}}
\newcommand{\simgt}{\raisebox{-0.6ex}{$\,\stackrel
        {\raisebox{-.2ex}{$\textstyle >$}}{\sim}\,$}}

\newcommand{\Msun}{M_\odot}
\newcommand{\Lsun}{L_\odot}
\newcommand{\lsun}{L_\odot}
\newcommand{\Mdot}{\dot M}

\newcommand{\sqdeg}{deg$^{-2}$}
\newcommand{\lya}{Ly$\alpha$\ }
%\newcommand{\lya}{Ly\,$\alpha$\ }
\newcommand{\lyaf}{Ly\,$\alpha$\ forest}
%\newcommand{\eg}{e.g.~}
%\newcommand{\etal}{et~al.~}
\newcommand{\lyb}{Ly$\beta$\ }
\newcommand{\cii}{C\,{\sc ii}\ }
\newcommand{\ciii}{C\,{\sc iii}]\ }
\newcommand{\civ}{C\,{\sc iv}\ }
\newcommand{\SiIV}{Si\,{\sc iv}\ }
\newcommand{\mgii}{Mg\,{\sc ii}\ }
\newcommand{\feii}{Fe\,{\sc ii}\ }
\newcommand{\feiii}{Fe\,{\sc iii}\ }
\newcommand{\caii}{Ca\,{\sc ii}\ }
\newcommand{\halpha}{H\,$\alpha$\ }
\newcommand{\hbeta}{H\,$\beta$\ }
\newcommand{\hgamma}{H\,$\gamma$\ }
\newcommand{\hdelta}{H\,$\delta$\ }
\newcommand{\oi}{[O\,{\sc i}]\ }
\newcommand{\oii}{[O\,{\sc ii}]\ }
\newcommand{\oiii}{[O\,{\sc iii}]\ }
\newcommand{\heii}{[He\,{\sc ii}]\ }
\newcommand{\nv}{N\,{\sc v}\ }
\newcommand{\nev}{Ne\,{\sc v}\ }
\newcommand{\neiii}{[Ne\,{\sc iii}]\ }
\newcommand{\aliii}{Al\,{\sc iii}\ }
\newcommand{\siiii}{Si\,{\sc iii}]\ }


%%%%%%%%%%%%%%%%%%%%%%%%%%%%%%%%%%%%%%%%%%%%%%%%%%%%%
%              define Listings                       %
%%%%%%%%%%%%%%%%%%%%%%%%%%%%%%%%%%%%%%%%%%%%%%%%%%%%%
\definecolor{dkgreen}{rgb}{0,0.6,0}
\definecolor{gray}{rgb}{0.5,0.5,0.5}
\definecolor{mauve}{rgb}{0.58,0,0.82}

\lstset{frame=tb,
  language=Python,
  aboveskip=3mm,
  belowskip=3mm,
  showstringspaces=false,
  columns=flexible,
  basicstyle={\small\ttfamily},
  numbers=none,
  numberstyle=\tiny\color{gray},
  keywordstyle=\color{blue},
  commentstyle=\color{dkgreen},
  stringstyle=\color{mauve},
  breaklines=true,
  breakatwhitespace=true,
  tabsize=3
}

\begin{document}

\title{A Guide to Bayes}
\author{Nicholas P. Ross}
\date{\today}
\maketitle


\begin{abstract}
This is a simple document that discusses the basis and basics of Bayes. 
\end{abstract}


\tableofcontents


\newpage
\section{Bayes' Theorem}

Bayes' theorem is:

\begin{equation} 
P(A|B) = \frac{ P(B|A)  \:   P(A)}  {P(B)}
\end{equation}
where\\

\noindent
$P(A)$ and $P(B)$ are the probabilities of observing A and B without regard to each other.\\

\noindent
$P(A | B)$, a conditional probability, {\it is the probability of observing event A given that B is true.}\\

\noindent
$P(B | A)$ is the probability {\it of observing event B given that A is true}.\\




\newpage
\section{Bayesian inference}

Bayesian inference derives the posterior probability as a consequence
of two antecedents, a prior probability and a ``likelihood function''
derived from a statistical model for the observed data. Bayesian
inference computes the posterior probability according to Bayes'
theorem:

\begin{equation} 
P(H|E) = \frac{ P(E|H)  \:   P(H)}  {P(E)}
\end{equation}
where:\\

\noindent
$|$ denotes a conditional probability; more specifically, it means
``given''.\\

\noindent
$H$ stands for any hypothesis whose probability may be affected by
data (called evidence below). Often there are competing hypotheses,
from which one chooses the most probable.\\

\noindent
the evidence $E$ corresponds to new data that were not used in
computing the prior probability.\\

\noindent
$P(H)$, the prior probability, is the probability of $H$ before $E$ is
observed. This indicates one's previous estimate of the probability
that a hypothesis is true, before gaining the current evidence.\\

\noindent
$P(H | E)$ the posterior probability, is the probability of $H$ given
$E$, i.e., after $E$ is observed. This tells us what we want to know:
the probability of a hypothesis given the observed evidence.\\

\noindent
$P (E | H)$ is the probability of observing $E$ given $H$. As a
function of $E$ with $H$ fixed, this is the likelihood. The likelihood
function should {\bf not} be confused with $P(H | E)$ as a function of 
$H$ rather than of $E$. It indicates the compatibility of the evidence with the given
hypothesis.\\

\noindent
$P(E)$ is sometimes termed the marginal likelihood or ``model
evidence''. This factor is the same for all possible hypotheses being
considered. (This can be seen by the fact that the hypothesis $H$ does
not appear anywhere in the symbol, unlike for all the other factors.)
This means that this factor does not enter into determining the
relative probabilities of different hypotheses. \\

%Note that, for different values of {\displaystyle \textstyle H} \textstyle H, only the factors {\displaystyle \textstyle P(H)} \textstyle P(H) and {\displaystyle \textstyle P(E\mid H)} \textstyle P(E\mid H) affect the value of {\displaystyle \textstyle P(H\mid E)} \textstyle P(H\mid E). As both of these factors appear in the numerator, the posterior probability is proportional to both. In words:

%(more precisely) The posterior probability of a hypothesis is determined by a combination of the inherent likeliness of a hypothesis (the prior) and the compatibility of the observed evidence with the hypothesis (the likelihood).

%(more concisely) Posterior is proportional to likelihood times prior.

\noindent
Note that Bayes' rule can also be written as follows:
\begin{equation}
P(H\mid E)={\frac {P(E\mid H)}{P(E)}}\cdot P(H)
\end{equation}
where the factor $\frac {P(E\mid H)}{P(E)}$ represents the impact of
$E$ on the probability of $H$.









\newpage
\section{References}
\href{https://en.wikipedia.org/wiki/Bayesian\_inference}{https://en.wikipedia.org/wiki/Bayesian\_inference}\\
\href{hrefhttps://en.wikipedia.org/wiki/Bayes\%27\_theorem}{https://en.wikipedia.org/wiki/Bayes\%27\_theorem}\\

\citet{Croom04}

\bibliographystyle{mn2e}
\bibliography{/cos_pc19a_npr/LaTeX/tester_mnras}

\end{document}

