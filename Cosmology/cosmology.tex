\documentclass[11pt]{article}


\usepackage{amsmath, amssymb}
\usepackage{bm, booktabs}
\usepackage{cancel, caption}
\usepackage{dcolumn}  % Align table columns on decimal point
\usepackage{epsfig, epsf, enumitem}
\usepackage{fancyhdr}
\usepackage[T1]{fontenc}
\usepackage{graphicx, geometry}
\usepackage{hyperref}
\usepackage{ifthen}
\usepackage[utf8]{inputenc}
\usepackage{lscape, longtable}
\usepackage{multirow}
\usepackage{natbib}
\usepackage{pifont}
\usepackage{ragged2e}
\usepackage{subfigure}
\usepackage{sectsty}
\usepackage{times, tabularx}
\usepackage{tcolorbox}
\usepackage{verbatim}
%\usepackage[usenames,dvipsnames,svgnames,table]{xcolor}



%%%%%%%%%%%%%%%%%%%%%%%%%%%%%%%%%%%%%%%%%%%
%       define Journal abbreviations      %
%%%%%%%%%%%%%%%%%%%%%%%%%%%%%%%%%%%%%%%%%%%
\def\nat{Nat} \def\apjl{ApJ~Lett.} \def\apj{ApJ}
\def\apjs{ApJS} \def\aj{AJ} \def\mnras{MNRAS}
\def\prd{Phys.~Rev.~D} \def\prl{Phys.~Rev.~Lett.}
\def\plb{Phys.~Lett.~B} \def\jhep{JHEP}
\def\npbps{NUC.~Phys.~B~Proc.~Suppl.} \def\prep{Phys.~Rep.}
\def\pasp{PASP} \def\aap{Astron.~\&~Astrophys.} \def\araa{ARA\&A}


%%%%%%%%%%%%%%%%%%%%%%%%%%%%%%%%%%%%%%%%%%%%%%%%%%%%%
%              define symbols                       %
%%%%%%%%%%%%%%%%%%%%%%%%%%%%%%%%%%%%%%%%%%%%%%%%%%%%%
\def \Mpc {~{\rm Mpc} }
\def \Om {\Omega_0}
\def \Omb {\Omega_{\rm b}}
\def \Omcdm {\Omega_{\rm CDM}}
\def \Omlam {\Omega_{\Lambda}}
\def \Omm {\Omega_{\rm m}}
\def \ho {H_0}
\def \qo {q_0}
\def \lo {\lambda_0}
\def \kms {{\rm ~km~s}^{-1}}
\def \kmsmpc {{\rm ~km~s}^{-1}~{\rm Mpc}^{-1}}
\def \hmpc{~\;h^{-1}~{\rm Mpc}} 
\def \hkpc{\;h^{-1}{\rm kpc}} 
\def \hmpcb{h^{-1}{\rm Mpc}}
\def \dif {{\rm d}}
\def \mlim {m_{\rm l}}
\def \bj {b_{\rm J}}
\def \mb {M_{\rm b_{\rm J}}}
\def \qso {_{\rm QSO}}
\def \lrg {_{\rm LRG}}
\def \gal {_{\rm gal}}
\def \xibar {\bar{\xi}}
\def \xis{\xi(s)}
\def \xisp{\xi(\sigma, \pi)}
\def \Xisig{\Xi(\sigma)}
\def \xir{\xi(r)}
\def \max {_{\rm max}}
\def \gsim { \lower .75ex \hbox{$\sim$} \llap{\raise .27ex \hbox{$>$}} }
\def \lsim { \lower .75ex \hbox{$\sim$} \llap{\raise .27ex \hbox{$<$}} }
\def \deg {^{\circ}}
\def \deltac {\delta_{\rm c}}
\def \mmin {M_{\rm min}}
\def \mbh  {M_{\rm BH}}
\def \mdh  {M_{\rm DH}}
\def \msun {M_{\odot}}
\def \z {_{\rm z}}
\def \edd {_{\rm Edd}}
\def \lin {_{\rm lin}}
\def \nonlin {_{\rm non-lin}}
\def \wrms {\langle w_{\rm z}^2\rangle^{1/2}}
\def \dc {\delta_{\rm c}}
\def \wp {w_{p}(\sigma)}
\def \PwrSp {\mathcal{P}(k)}
\def \DelSq {$\Delta^{2}(k)$}
\def \WMAP {{\it WMAP \,}}
\def \cobe {{\it COBE }}
\def \COBE {{\it COBE \;}}
\def \HST  {{\it HST \,\,}}
\def \Spitzer  {{\it Spitzer \,}}



\begin{document}

\title{NPR's Research Notes: \\ 
		Cosmology}
\author{Nicholas P. Ross}
\date{\today}
\maketitle

\begin{abstract}
Some of NPRs' notes on Cosmology. 
\end{abstract}

\section{Model, parameters, and methodology} \label{sec:model}

\begin{table*}[tmb] % table* does a two-column table
\begingroup % this + \endgroup at the end keep table things local
\newdimen\tblskip \tblskip=5pt
\caption{Cosmological parameters used in our analysis. For each, we give the symbol, prior range, value taken in the base \lcdm\ cosmology (where appropriate), and summary definition (see text for details).
The top block contains parameters with uniform priors that are varied in the
MCMC chains. The ranges of these priors are listed in square brackets.
The lower blocks define various derived parameters.}
\label{tab:params}
%\nointerlineskip
\vskip -5mm
\footnotesize % good font size for a table, but can be changed
\setbox\tablebox=\vbox{ %
\newdimen\digitwidth % see \S\,18.12 for the purpose of the next 10 lines
\setbox0=\hbox{\rm 0}
\digitwidth=\wd0
\catcode`*=\active
\def*{\kern\digitwidth}
%
\newdimen\signwidth
\setbox0=\hbox{+}
\signwidth=\wd0
\catcode`!=\active
\def!{\kern\signwidth}
%
\halign{\hbox to 2.7cm{#\leaderfil}\tabskip=0.4cm& \hfil#\hfil\tabskip=0.6cm&
 \hfil#\hfil\tabskip=0.6cm&  #\hfil\tabskip=0pt\cr
\noalign{\doubleline}
\omit\hfil Parameter\hfil&\omit\hfil Prior range\hfil&\omit\hfil Baseline\hfil&\omit\hfil Definition\hfil\cr
\noalign{\vskip 3pt\hrule\vskip 3pt}
%\halign{#\hfil\tabskip=0.2cm& #\hfil\tabskip=0.2cm&#\hfil\tabskip=0.2cm& #\hfil\tabskip=0pt\cr
% template goes here. See examples.
%\noalign{\doubleline}
%\noalign{\vskip -2pt}
%Parameter& Prior range& Baseline& Definitions\cr
%heading
%\noalign{\vskip 3pt\hrule\vskip 3pt}
$\omb \equiv \Omb h^2$& $[0.005, 0.1]$ & \dots& Baryon density today\cr
$\omc \equiv \Omc h^2$& $[0.001, 0.99]$& \dots& Cold dark matter density today\cr
$100\theta_{\mathrm{MC}}$ & $[0.5, 10.0]$ & \dots& $100\,{\times}$ approximation to $\rstar/D_{\rm A}$ (CosmoMC)\cr
$\tau                $&   $[0.01, 0.8]$ & \dots& Thomson scattering optical depth due to reionization\cr
$\Omk            $&  $[-0.3, 0.3]$ & 0& Curvature parameter today with $\Omtot= 1 - \Omk$\cr
$\mnu        $& $[0, 5]$ & $0.06$ & The sum of neutrino masses in eV\cr
$\mnusterile$&  $[0, 3]$ &0&Effective mass of sterile neutrino in eV\cr
$w_0                $& $[-3.0, -0.3]$ & $-1$& Dark energy equation of state$^{a}$, $w(a) = w_0 + (1-a) w_a$\cr
$w_a                 $& $[-2, 2]$ & 0&  As above (perturbations modelled using PPF)\cr
$\neff       $& $[0.05, 10.0]$ & 3.046& Effective number of neutrino-like relativistic degrees of freedom (see text)\cr
$\yhe                $& $[0.1, 0.5]$ & BBN& Fraction of baryonic mass in helium\cr
%$\alpha_{-1}        $& {\tt alpha1}& 0& Correlated isocurvature parameter (see text)\cr
%$\alpha_0            $& {\tt alpha0}& 0& Uncorrelated isocurvature parameter (see, e.g., Larson et al. 2011)\cr
%$\Delta \zre   $& $[0.1, 3.0]$ & 0.5& Width of reionization transition (see text)\cr
$\Alens        $& $[0,  10]$& 1& Amplitude of the lensing power relative to the physical value\cr
%$\Aphiphi           $& {\tt Aphiphi}& 1& Amplitude of the lensing power from the 4-point function relative to the physical value\cr
$\ns           $& $[0.9, 1.1]$ & \dots& Scalar spectrum power-law index ($k_0 = 0.05\Mpc^{-1}$)\cr
$\nt           $& $\nt = -r_{0.05}/8$ & Inflation& Tensor spectrum power-law index ($k_0 = 0.05\Mpc^{-1}$)\cr
$\nrun$&   $[-1, 1]$ & 0& Running of the spectral index\cr
$\ln(10^{10}\As) $& $[2.7, 4.0]$ & \dots& Log power of the primordial curvature perturbations ($k_0 = 0.05\,\Mpc^{-1}$)\cr
$\rpivot          $& $[0, 2]$ & 0& Ratio of tensor primordial power to curvature power at $k_0 = 0.05\,\Mpc^{-1}$\cr
\noalign{\vskip 3pt\hrule\vskip 3pt}
$\Oml      $&     & \dots& Dark energy density divided by the critical density today\cr
$t_0                 $&  & \dots& Age of the Universe today (in Gyr)\cr
$\Omm     $&  & \dots& Matter density (inc.\ massive neutrinos) today divided by the critical density\cr
$\sigma_8            $&   & \dots& RMS matter fluctuations today in linear theory\cr
$\zre                $&   & \dots& Redshift at which Universe is half reionized\cr
%$r_{10}             $&   & 0& tensor-scalar $C_\ell$ amplitude at $\ell=10$\cr
$H_0                 $&[20,100] & \dots& Current expansion rate in $\rm{km}\, \rm{s}^{-1}\Mpc^{-1}$\cr
$r_{0.002}           $&   & 0& Ratio of tensor primordial power to curvature power at $k_0 = 0.002\,\Mpc^{-1}$\cr
$10^9 \As      $&    & \dots& $10^9\,\times$
dimensionless curvature power spectrum at $k_0 = 0.05\,\Mpc^{-1}$\cr
$\omm\equiv\Omm h^2$&  & \dots& Total matter density today (inc.\ massive neutrinos)\cr
%$\Omm h^3  $&   & \dots& $h\times $total matter density today\cr
%$\yhe               $&   & bbn& Fraction of baryonic mass in Helium\cr
%clamp
\noalign{\vskip 3pt\hrule\vskip 3pt}
$\zstar            $&   & \dots& Redshift for which the optical depth equals unity (see text)\cr
$\rstar=\rs(\zstar) $&    & \dots& Comoving size of the sound horizon at $z = z_\ast$\cr
$100\theta_\ast         $&    & \dots& $100\,\times$ angular size of
sound horizon at $z=\zstar$ ($\rstar/D_{\rm A})$ \cr
%Angular size of the sound horizon at $z=\zstar$ (see text)\cr
$\zdrag       $&     & \dots& Redshift at which baryon-drag optical depth equals unity (see text)\cr
$\rdrag=\rs(\zdrag)$&   & \dots& Comoving size of the sound horizon at $z = \zdrag$\cr
$k_{\rm D}           $&   & \dots& Characteristic damping comoving wavenumber ($\Mpc^{-1}$)\cr
$100\theta_{\rm D}      $&   & \dots& $100\,\times$ angular extent of photon diffusion at last scattering (see text)\cr
$\zeq          $&   & \dots& Redshift of matter-radiation equality (massless neutrinos)\cr
$100\theta_{\rm eq}     $&  & \dots& $100\,\times$ angular size of the comoving horizon at matter-radiation equality\cr
$\rdrag/D_{\mathrm{V}}(0.57)$ &   &\dots& BAO distance
ratio at $z=0.57$ (see Sect.~\ref{sec:BAO})\cr
%$\Omb     $& & Density of baryons today divided by the critical density\cr
\noalign{\vskip 3pt\hrule\vskip 3pt}}}
\endPlancktablewide % this command is defined in Planck.tex.
\tablenote \textit{a} For dynamical dark energy models with constant equation of state, we denote the equation of state by $w$ and adopt the same prior as for $w_0$.\par
\endgroup
\end{table*}

\medskip



To make accurate predictions for the CMB power spectra, the background
ionization history has to be calculated to high accuracy.  Although
the main processes that lead to recombination at $z\approx 1090$ are well
understood, cosmological parameters from \planck\ can be sensitive to
sub-percent differences in the ionization fraction
$x_{\rm e}$ \citep{Hu:1995fqa,Lewis:2006ym,RubinoMartin:2009ry,Shaw:2011ez}.
The process of recombination takes the Universe from a state of fully
ionized hydrogen and helium in the early Universe, through to the
completion of recombination with residual fraction $x_{\rm e} \sim 10^{-4}$.
Sensitivity of the CMB power spectrum to $x_{\rm e}$ enters through changes
to the sound horizon at recombination, from changes in the timing of
recombination, and to the detailed shape of the recombination transition,
which affects the thickness of the last-scattering surface and hence the
amount of small-scale diffusion (Silk) damping, polarization, and line-of-sight
averaging of the perturbations.

Since the pioneering work of~\cite{Peebles:68} and \cite{Zeldovich:69},
which identified the main physical processes involved in recombination,
there has been significant progress in numerically modelling the many
relevant atomic transitions and processes that can affect the details
of the recombination process \citep{Hu:1995fqa,Seager:1999km,Wong:2007ym,Hirata:2007sp,
Switzer:2007sn,RubinoMartin:2009ry,Grin:2009ik,Chluba:2010ca,AliHaimoud:2010ym,AliHaimoud:2010dx}.
In recent years a consensus has emerged between the results of two
multi-level atom codes
\HYREC\footnote{\url{http://www.sns.ias.edu/~yacine/hyrec/hyrec.html}}~\citep{Switzer:2007sn,Hirata:2008ny,AliHaimoud:2010dx},
and
\COSMOREC\footnote{\url{http://www.chluba.de/CosmoRec/}}~\citep{Chluba:2010fy,Chluba:2010ca},
demonstrating agreement at a level better than that required for
\planck\ (differences less that $4\times 10^{-4}$ in the predicted
temperature power spectra on small scales).

These recombination codes are remarkably fast,  given the complexity of the calculation.
However, the recombination history can be computed even more rapidly by
using the simple effective three-level atom model developed by
\cite{Seager:1999km} and implemented in the \RECFAST\
code\footnote{\url{http://www.astro.ubc.ca/people/scott/recfast.html}},
with appropriately chosen small correction functions calibrated to the
full numerical results \citep{Wong:2007ym,RubinoMartin:2009ry,Shaw:2011ez}.
We use \RECFAST\ in our baseline parameter analysis, with correction functions
adjusted so that the predicted power spectra $C_\ell$ agree with those from
the latest versions of \HYREC\ (January 2012) and \COSMOREC\ (v2) to
better than $ 0.05\%$\footnote{The updated \RECFAST\ used here in the baseline model is
publicly available as version 1.5.2 and is the default in \CAMB\ as of
October 2012.}.
We have confirmed, using importance sampling, that cosmological parameter
constraints using \RECFAST\ are consistent with those using \COSMOREC\ at
the $0.05\,\sigma$ level.
Since the results of the \planck\ parameter analysis are crucially dependent
on the accuracy of the recombination history, we have also checked, following~\cite{Lewis:2006ym},
that there is no strong evidence for simple deviations from the assumed history. However, we note that any deviation from
the assumed history could significantly shift parameters compared to the results presented here and we have not
performed a detailed sensitivity analysis.

The background recombination model should accurately capture the
ionization history until the Universe is reionized at late times via
ultra-violet photons from stars and/or active galactic nuclei.  We
approximate reionization as being relatively sharp, with the mid-point
parameterized by a redshift $\zre$ (where $x_{\rm e}=f/2$) and width
parameter $\Delta z_{\rm re}=0.5$.  Hydrogen reionization and the
first reionization of helium are assumed to occur simultaneously, so
that when reionization is complete $x_{\rm e }= f \equiv
1+f_{\rm{He}}\approx 1.08$ \citep{Lewis:2008wr}, where $f_{\rm He}$ is the
helium-to-hydrogen ratio by number.  In this
parameterization, the optical depth is almost independent of $\Delta
z_{\rm re}$ and the only impact of the specific functional form on
cosmological parameters comes from very small changes to the shape of
the polarization power spectrum on large angular scales.  The second reionization of
helium (i.e., ${\rm He}^{+}\to{\rm He}^{++}$) produces
very small changes to the power spectra ($\Delta\tau \sim 0.001$,
where $\tau$ is the optical depth to Thomson scattering) and does not
need to be modelled in detail.  We include the second reionization of
helium at a fixed redshift of $z=3.5$ (consistent with observations of
Lyman-$\alpha$ forest lines in quasar spectra, e.g.,~\citealt{Becker:11}),
which is sufficiently accurate for the parameter analyses described in
this paper.

\subsubsection{Initial conditions}

In our baseline model we assume  purely adiabatic scalar perturbations
at very early times, with a (dimensionless) curvature power spectrum parameterized by
\begin{equation}
  \clp_\clr(k) = \As
    \left(\frac{k}{k_0}\right)^{\ns-1+(1/2)(\nrun) \ln(k/k_0)}, \label{PS1}
\end{equation}
with $n_{\rm s}$ and $\nrun$ taken to be constant.  For most of this paper we
shall assume no ``running'', i.e., a power-law spectrum with $d \ns / d\ln k = 0$.
The pivot scale, $k_0$, is chosen to be $k_0=0.05\,\Mpc^{-1}$, roughly in
the middle of the logarithmic range of scales probed by \planck.
With this choice, $\ns$ is not strongly degenerate with the amplitude
parameter $\As$.



The amplitude of the small-scale linear CMB power spectrum is proportional to
$e^{-2\tau}A_{\rm s}$.  Because \planck\ measures this amplitude very
accurately there is  a tight linear constraint between $\tau$ and
$\ln A_{\rm s}$ (see Sect.~\ref{subsec:tau}).
For this reason we usually use $\ln A_{\rm s}$ as a base parameter with
a flat prior, which has a significantly more Gaussian posterior than
$A_{\rm s}$. A linear parameter redefinition then also allows the degeneracy
between $\tau$ and $A_{\rm s}$ to be explored efficiently.
(The degeneracy between $\tau$ and $A_{\rm s}$ is broken by the relative
amplitudes of large-scale temperature and polarization CMB anisotropies and by
the non-linear effect of CMB lensing.)

%In the baseline model we assume purely adiabatic scalar perturbations.
We shall also consider  extended models with a significant amplitude of primordial
gravitational waves (tensor modes). Throughout this paper, the (dimensionless) tensor mode spectrum
is parameterized as a power-law with\footnote{For a transverse-traceless spatial tensor $H_{ij}$, the tensor
  part of the metric is $ds^2 = a^2[d\eta^2 - (\delta_{ij}+2H_{ij})
  dx^i dx^j]$, and $\clp_{\rm t}$ is defined so that $\clp_{\rm t}(k)
  = \partial_{\ln k} \langle 2 H_{ij} 2H^{ij}\rangle$.}
\be
 \clp_{\rm t}(k)=\At \left(\frac{k}{k_0}\right)^{\nt} .
% A_{\rm t}\left(\frac{k}{k_0}\right)^{n_{\rm t}+(1/2)n_{\rm t,run}\ln(k/k_0)+\cdots},
\ee
We define $\rpivot \equiv A_{\rm t}/A_{\rm s}$, the primordial tensor-to-scalar ratio at
$k=k_0$.
Our constraints are only weakly sensitive to the tensor spectral index,
$n_{\rm t}$ (which is assumed to be close to zero), and we adopt the theoretically motivated
single-field inflation consistency relation $n_{\rm t}=-\rpivot /8$,
rather than varying $n_{\rm t}$ independently.
We put a flat prior on $\rpivot$, but also report the constraint at
$k=0.002\,\Mpc^{-1}$ (denoted $r_{0.002}$),
which is closer to the scale at which there is some sensitivity to tensor modes in the large-angle temperature power spectrum. Most previous CMB experiments
have reported constraints on $r_{0.002}$.
For further discussion of the tensor-to-scalar ratio and its
implications for inflationary models see~\citet{planck2013-p17}.


\subsubsection{Dark energy}

In our baseline model we assume that the dark energy is a cosmological
constant with current density parameter $\Oml$. When considering a dynamical
dark energy component,  we parameterize the equation of state either as a constant $w$ or as a function of the cosmological scale factor, $a$, with
\be
  w(a) \equiv \frac{p}{\rho} = w_0 + (1-a)w_a,  \label{DE0}
\ee
and assume that the dark energy does not interact with other constituents
other than through gravity. Since this model allows the equation of
state to cross below $-1$, a single-fluid model cannot be used
self-consistently.  We therefore use the parameterized post-Friedmann (PPF)
model of~\citet{Fang:2008sn}. For models with
$w>-1$, the PPF model agrees with fluid
models to significantly better accuracy than required for the
results reported in this paper.

\subsubsection{Power spectra}

Over the last decades there has been significant progress in improving
the accuracy, speed and generality of the numerical calculation of the
CMB power spectra given an ionization history and set of cosmological
parameters \citep[see e.g.,][]{Bond:87,Sugiyama:95,Ma:1995ey,
Hu:1995fqa,Seljak:1996is,Hu:1997hp,Zaldarriaga:1997va,Lewis:1999bs,
Lesgourgues:2011rh}.
%  ,Hu:1997mn,Bucher:1999re,Hu:2000ee,Lewis:2002nc,Seljak:2003th,
%  Doran:2005ep,Challinor:2005jy,CyrRacine:2010bk,
%  Blas:2011rf,Lesgourgues:2011rh,Howlett:2012mh}. 
%\citep{Bond:87,Sugiyama:95,Ma:1995ey,Seljak:1996is,Seljak:1996ve,White:96,Hu:1997hp,
%  Zaldarriaga:1997va,Hu:1997mn,Bucher:1999re,Hu:2000ee,,Seljak:2003th,
%  Doran:2005ep,Challinor:2005jy,CyrRacine:2010bk,
%  Blas:2011rf,Lesgourgues:2011rh,Howlett:2012mh}. 
Our baseline
numerical Boltzmann code is \CAMB\footnote{\url{http://camb.info}}
\citep[March 2013;][]{Lewis:1999bs}, a parallelized line-of-sight code
developed from \CMBFAST\ \citep{Seljak:1996is} and \COSMICS\
\citep{Bertschinger:1995er,Ma:1995ey}, which calculates the lensed CMB
temperature and polarization power spectra.  The code has been
publicly available for over a decade and has been very well tested
(and improved) by the community.  Numerical stability and accuracy of
the calculation at the sensitivity of \planck\ has been explored in
detail \citep{Hamann:2009yy,Lesgourgues:2011rg,Howlett:2012mh},
demonstrating that the raw numerical precision is sufficient for
numerical errors on parameter constraints from \planck\ to be less
than $10\%$ of the statistical error around the assumed cosmological
model.  (For the high multipole CMB data at $\ell>2000$ introduced in
Sect. \ref{sec:highell}, the default \CAMB\ settings are adequate
because the power spectra of these experiments are dominated by
unresolved foregrounds and have large errors at high multipoles.)  To
test the potential impact of \CAMB\ errors, we importance-sample a
subset of samples from the posterior parameter space using higher
accuracy settings.  This confirms that differences purely due to
numerical error in the theory prediction are less than $10\%$ of the
statistical error
for all parameters, both with and without inclusion of CMB data at
high multipoles.
We also performed additional tests of the robustness and accuracy of our
 results by reproducing a fraction of them with the independent
 Boltzmann code \CLASS\ \citep{Lesgourgues:2011re,Blas:2011rf}.

In the parameter analysis, information from CMB lensing enters in two
ways. Firstly, all the CMB power spectra are modelled using the lensed
spectra, which includes the approximately 5\% smoothing effect on
the acoustic peaks due to lensing. Secondly, for some results we
include the \planck\ lensing likelihood, which encapsulates the
lensing information in the (mostly squeezed-shape) CMB trispectrum via
a lensing potential power spectrum \citep{planck2013-p12}.
The theoretical predictions for the lensing potential power spectrum are
calculated by \CAMB, optionally with corrections for the non-linear matter power spectrum, along with the (non-linear) lensed
CMB power spectra. For the \planck\ temperature power spectrum, corrections
to the lensing effect due to non-linear structure growth can be neglected,
however the impact on the lensing potential reconstruction is important.
We use the \HALOFIT\ model \citep{Smith:2002dz} as updated by
\cite{Takahashi:2012em} to model the impact of non-linear growth on the
theoretical prediction for the lensing potential power.

\subsection{Parameter choices}

\subsubsection{Base parameters}

The first section of Table~\ref{tab:params}
lists our base parameters that have flat priors
when they are varied, along with their default values in the baseline model.
When parameters are varied, unless otherwise stated, prior ranges are
chosen to be much larger than the posterior, and hence do not affect
the results of parameter estimation. In addition to these priors,
we impose a ``hard'' prior on the Hubble constant of $[20, 100]
\ {\rm km}\, {\rm s}^{-1}\, {\rm Mpc}^{-1}$.
%in \COSMOMC.

\subsubsection{Derived parameters}

Matter-radiation equality $\zeq$ is defined as the redshift at
which $\rho_\gamma+\rho_\nu = \rho_{\rm c}+\rho_{\rm b}$
(where $\rho_\nu$ approximates massive neutrinos as massless).

The redshift of last-scattering, $z_\ast$, is defined so that the optical
depth to Thomson scattering from $z=0$ (conformal time $\eta = \eta_0$)
to $z=z_\ast$ is unity, assuming no reionization.  The optical depth is
given by
\begin{equation}
  \tau(\eta) \equiv \int_{\eta_0}^\eta \dot\tau\ d\eta^\prime,
\end{equation}
where $\dot\tau = - an_{\rm e}\sigma_{\rm T}$ (and $n_{\rm e}$ is the density of free electrons and $\sigma_{\rm T}$ is the Thomson
cross section).  We define the angular scale of the sound horizon at last-scattering,
$\theta_\ast =r_{\rm s}(z_\ast)/D_{\rm A}(z_\ast)$, where $r_{\rm s}$ is
the sound horizon
\begin{equation}
  r_{\rm s}(z) = \int_0^{\eta(z)}  \frac{d\eta^\prime}{\sqrt{3(1+R)}},
\end{equation}
with $R \equiv 3 \rho_{\rm b}/(4\rho_\gamma)$. The parameter $\theta_{\mathrm{MC}}$
in Table 1 is an approximation to $\theta_\ast$ that is used in \COSMOMC\ and
is based on fitting formulae given in \citet{Hu:1995en}.

Baryon velocities decouple from the photon dipole when Compton
drag balances the gravitational force,  which happens at
$\tau_{\rm d} \sim 1$, where \citep{Hu:1995en}
\begin{equation}
  \tau_{\rm d}(\eta) \equiv \int^\eta_{\eta_0} \dot\tau\ d\eta^\prime/R.
\end{equation}
Here, again, $\tau$ is from recombination only, without reionization
contributions.
We define a drag redshift $z_{\rm drag}$, so that
$\tau_{\rm d}(\eta(z_{\rm drag})) = 1$.
The sound horizon at the drag epoch is an important scale that is often
used in studies of baryon acoustic oscillations; we denote this as
$r_{\rm drag}=r_{\rm s}(z_{\rm drag})$.  We compute $z_{\rm drag}$ and
$r_{\rm drag}$ numerically from \CAMB\ (see Sect.~\ref{sec:BAO} for details of application to BAO data).


The characteristic wavenumber for damping, $k_{\rm D}$, is given by
\begin{equation}
k_{\rm D}^{-2}(\eta) = -\frac{1}{6} \int_0^\eta d\eta^\prime
  \ \frac{1}{\dot\tau}\ \frac{R^2 + 16 (1+R)/15}{\left(1+R\right)^2}.
\end{equation}
We define the angular damping scale, $\theta_{\rm D} = \pi/(k_{\rm D} D_{\rm A})$, where $D_{\rm A}$
is the comoving angular diameter distance to $z_\ast$.

For our purposes, the normalization of the power spectrum is most conveniently
given by $A_{\rm s}$.  However, the alternative measure
$\sigma_8$ is often used in the literature, particularly in studies of large-scale
structure.  By definition, $\sigma_8$ is the rms fluctuation in total matter
(baryons + CDM + massive neutrinos) in $8\,h^{-1}$~Mpc spheres at
$z=0$, computed in linear theory.  It is related to the dimensionless matter power spectrum,
$\clp_{\rm m}$, by
\begin{equation}
  \sigma_R^2 = \int \frac{dk}{k}\ \clp_{\rm m}(k)
  \left[ \frac{3j_1(kR)}{kR} \right]^2,
\end{equation}
where $R=8\,h^{-1}$Mpc and $j_1$ is the spherical Bessel function of order 1.


In addition, we compute
$\Omega_{\rm m}h^3$ (a well-determined combination orthogonal to the acoustic
scale degeneracy in flat models; see e.g., \citealt{Percival:2002gq} and~\citealt{Howlett:2012mh}),
$10^9 A_{\rm s} e^{-2\tau}$ (which determines the small-scale linear
CMB anisotropy power),
$r_{0.002}$ (the ratio of the tensor to primordial curvature
power at $k = 0.002 \ {\rm Mpc}^{-1}$),
$\Omega_\nu h^2$ (the physical density in massive neutrinos), and the value of $\yhe$ from the BBN consistency condition.


\subsection{Likelihood}
\label{subsec:likelihood}

\cite{planck2013-p08} describes the \Planck\ temperature likelihood in
detail.  Briefly, at high multipoles ($\ell \ge 50$) we use the $100$,
$143$ and $217$ GHz temperature maps
\citep[constructed using {\tt HEALPix}][]{gorski2005} to form a high multipole
likelihood following the {\tt CamSpec} methodology described in
\cite{planck2013-p08}. Apodized Galactic masks, including an apodized
point source mask, are applied to individual detector/detector-set
maps at each frequency. The masks are carefully chosen to limit
contamination from diffuse Galactic emission to low levels (less than
$20\, \mu{\rm K}^2$ at all multipoles \referee{used in the likelihood) before
correction for Galactic dust emission}.\footnote{\referee{As described in
\citet{planck2013-p08}, we use spectra calculated
on different masks to isolate the contribution of Galactic dust at each
frequency, which we subtract from the $143\times143$,
$143\times 217$ and $217\times 217$ power spectra (i.e., the correction is
applied to the power spectra, not in the map domain). 
The Galactic dust templates are shown in Fig.~\ref{PlanckandHighL} and are
less than $5\,\muKsq$ at high multipoles
for the $217\times217$ spectrum and negligible at lower frequencies.
The residual contribution from Galactic dust after correction in the
$217\times217$ spectrum is smaller than $0.5\,\muKsq$ and smaller than the
errors from other sources such as beam uncertainties.}}
Thus we retain $57.8\%$ of
the sky at $100$ GHz and $37.3\%$ of the sky at $143$ and $217$
GHz. Mask-deconvolved and beam-corrected cross-spectra
\citep[following][]{Hietal02} are computed
for all detector/detector-set combinations and compressed to form
averaged $100\times100$, $143\times143$, $143\times217$ and
$217\times217$ pseudo-spectra (note that we do not retain the
$100\times143$ and $100\times 217$ cross-spectra in the likelihood).
Semi-analytic covariance matrices for these pseudo-spectra
\citep{Efstathiou:04} are used to form a high-multipole likelihood in
a fiducial Gaussian likelihood approximation
\citep{DABJK00,2008PhRvD..77j3013H}.

At low multipoles ($2 \le \ell \le 49$) the temperature likelihood is
based on a Blackwell-Rao estimator applied to Gibbs samples computed
by the {\tt Commander} algorithm \citep{Eriksen:08} from \planck\ maps
in the frequency range $30$--$353$ GHz over 91\% of the sky. The likelihood
at low multipoles therefore accounts for errors in foreground cleaning.

Detailed consistency tests of both the high- and low-multipole components
of the temperature likelihood are presented in \cite{planck2013-p08}.
The high-multipole \planck\ likelihood requires a number of additional
parameters to describe unresolved foreground components and other
``nuisance'' parameters (such as beam eigenmodes). The model adopted
for \planck\ is described in \cite{planck2013-p08}. A self-contained
account is given in Sect. \ref{sec:highell} which generalizes the model
to allow matching of the \planck\ likelihood to the likelihoods from
high-resolution CMB experiments. A complete list of the foreground
and  nuisance parameters is given in Table~\ref{tab:fgparams}.





\subsection{Sampling and confidence intervals}

We sample from the space of possible cosmological parameters with
Markov Chain Monte Carlo (MCMC) exploration using
\COSMOMC~\citep{Lewis:2002ah}.
This uses a Metropolis-Hastings algorithm to generate chains of samples
for a set of cosmological parameters, and also allows for importance
sampling of results to explore the impact of small changes in the analysis.
The set of parameters is internally orthogonalized to allow efficient
exploration of parameter degeneracies, and the baseline cosmological
parameters are chosen following \citet{Kosowsky:2002zt}, so that the
linear orthogonalisation allows efficient exploration of the main
geometric degeneracy \citep{Bond:97}.
The code \referee{has been thoroughly
tested by the community} and has recently been extended to sample efficiently
large numbers of ``fast'' parameters by use of a speed-ordered Cholesky
parameter rotation and a fast-parameter ``dragging'' scheme described
by~\cite{Neal04} and \cite{Lewis:2013hha}.

For our main cosmological parameter runs we execute eight chains until
they are converged, and the tails of the distribution are well enough
explored for the confidence intervals for each parameter to be evaluated
consistently in the last half of each chain.
We check that the spread in the means between chains is small
compared to the standard deviation, using the standard Gelman and Rubin
\citep{Gelman92} criterion $R-1 < 0.01$ in the least-converged
orthogonalized parameter. This is sufficient for reliable importance sampling
in most cases. We perform separate runs when the posterior volumes differ
enough that importance sampling is unreliable. Importance-sampled and extended
data-combination chains used for this paper satisfy $R-1 < 0.1$, and in almost
all cases are closer to 0.01.
We discard the first $30\%$ of each chain as burn in, where the
chains may be still converging and the sampling may be significantly
non-Markovian.  This is
due to the way \COSMOMC\ learns an accurate orthogonalisation
and proposal distribution for the parameters from the sample covariance
of previous samples.

From the samples, we generate estimates of the posterior mean of each
parameter of interest, along with a confidence interval.
We generally quote $68\%$ limits in the case of two-tail limits, so that
$32\%$ of samples are outside the limit range, and
there are $16\%$ of samples in each tail. For parameters where
the tails are significantly different shapes, we instead quote the interval
between extremal points with approximately equal marginalized probability density.
For parameters with prior bounds we either quote one-tail limits or
no constraint, depending on whether the posterior is significantly non-zero at the prior boundary. Our one-tail limits are always $95\%$ limits.
For parameters with nearly symmetric distribution we sometimes quote the mean and standard deviation ($\pm 1\,\sigma$).
The samples can also be used to estimate one,
two and three-dimensional marginalized parameter posteriors.
We use variable-width Gaussian kernel density estimates in all cases.

We have also performed an alternative analysis to the one described above, using an independent statistical method based on frequentist
profile likelihoods \citep{ProfileLik}.
This gives fits and error bars for the baseline cosmological parameters in excellent
agreement for both \Planck\ and \Planck\ combined with high-resolution CMB experiments,
consistent with the Gaussian form of the posteriors found from full parameter space sampling.



In addition to posterior means, we also quote maximum-likelihood
parameter values. These are generated using the {\tt BOBYQA} bounded
minimization
routine\footnote{\url{http://www.damtp.cam.ac.uk/user/na/NA_papers/NA2009_06.pdf}}.
Precision is limited by stability of the convergence, and values
quoted are typically reliable to within $\Delta \chi^2 \sim 0.6$,
which is the same order as differences arising from numerical errors
in the theory calculation.  For poorly constrained parameters the
actual value of the best-fit parameters is not very numerically stable
and should not be over-interpreted; in particular, highly degenerate
parameters in extended models and the foreground model can give many
apparently different solutions within this level of
accuracy.  The best-fit values should be interpreted as giving typical
theory and foreground power spectra that fit the data well, but are
generally non-unique at the numerical precision used; they are however
generally significantly better fits than any of the samples in the
parameter chains.  Best-fit values are useful for assessing residuals,
and differences between the best-fit and posterior means also help to
give an indication of the effect of asymmetries, parameter-volume and
prior-range effects on the posterior samples. We have cross-checked a
small subset of the best-fits with the widely used {\tt MINUIT}
software \citep{Minuit}, which can give somewhat more stable results.








\section{Cosmology 101}
{\bf From Henrot-Versille and ``Planck 2015 Results'': }
1.3 Statistical method and base $\Lambda$CDM results: 
The base $\Lambda$CDM model is described by 6 parameters: 
the baryon density of the Universe $\Omega_{b} h^{2}$; 
the Cold Dark Matter (CDM) density �cdmh2, the characteristic angular size of the CMB fluctu- ations (?MC), the optical depth to reionization (?) and the amplitude and spectral index of the primordial spectrum (ln(1010As) and ns)


\section{Shape of the CMB TT Power Spectrum:}
From e.g. Wayne Hu's thesis and webpage> 
Location of first peak due to: \\
Location of second peak due to: \\
Location of third peak due to: \\
Ratio of heigh of first and second peak:\\
Ratio of heigh of first and third peak:\\
Ratio of heigh of second and third peak:\\




\bibliographystyle{mn2e}
\bibliography{/cos_pc19a_npr/LaTeX/tester_mnras}



\end{document}