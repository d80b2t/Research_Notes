\documentclass[11pt]{article}
\setlength {\textwidth}{160mm} 
%\setlength {\textheight}{230mm}
%\topmargin=-35.00mm
\oddsidemargin= -10.00mm  
\pagestyle{empty}


\usepackage[toc,page]{appendix}
\usepackage{amsmath, amssymb}
\usepackage{bm}% bold math
\usepackage{cancel, caption}
\usepackage{dcolumn}% Align table columns on decimal point
\usepackage{epsfig, epsf}
\usepackage{graphicx,fancyhdr,natbib,subfigure}
\usepackage{lscape, longtable}
\usepackage{hyperref,ifthen}
\usepackage{verbatim}
\usepackage{color}
\usepackage[usenames,dvipsnames]{xcolor}
\usepackage{listings}
%% http://en.wikibooks.org/wiki/LaTeX/Colors



%%%%%%%%%%%%%%%%%%%%%%%%%%%%%%%%%%%%%%%%%%%
%       define Journal abbreviations      %
%%%%%%%%%%%%%%%%%%%%%%%%%%%%%%%%%%%%%%%%%%%
\def\nat{Nat} \def\apjl{ApJ~Lett.} \def\apj{ApJ}
\def\apjs{ApJS} \def\aj{AJ} \def\mnras{MNRAS}
\def\prd{Phys.~Rev.~D} \def\prl{Phys.~Rev.~Lett.}
\def\plb{Phys.~Lett.~B} \def\jhep{JHEP} \def\nar{NewAR}
\def\npbps{NUC.~Phys.~B~Proc.~Suppl.} \def\prep{Phys.~Rep.}
\def\pasp{PASP} \def\aap{Astron.~\&~Astrophys.} \def\araa{ARA\&A}
\def\jcap{\ref@jnl{J. Cosmology Astropart. Phys.}}%
\def\physrep{Phys.~Rep.}

\newcommand{\preep}[1]{{\tt #1} }

%%%%%%%%%%%%%%%%%%%%%%%%%%%%%%%%%%%%%%%%%%%%%%%%%%%%%
%              define symbols                       %
%%%%%%%%%%%%%%%%%%%%%%%%%%%%%%%%%%%%%%%%%%%%%%%%%%%%%
\def \Mpc {~{\rm Mpc} }
\def \Om {\Omega_0}
\def \Omb {\Omega_{\rm b}}
\def \Omcdm {\Omega_{\rm CDM}}
\def \Omlam {\Omega_{\Lambda}}
\def \Omm {\Omega_{\rm m}}
\def \ho {H_0}
\def \qo {q_0}
\def \lo {\lambda_0}
\def \kms {{\rm ~km~s}^{-1}}
\def \kmsmpc {{\rm ~km~s}^{-1}~{\rm Mpc}^{-1}}
\def \hmpc{~\;h^{-1}~{\rm Mpc}} 
\def \hkpc{\;h^{-1}{\rm kpc}} 
\def \hmpcb{h^{-1}{\rm Mpc}}
\def \dif {{\rm d}}
\def \mlim {m_{\rm l}}
\def \bj {b_{\rm J}}
\def \mb {M_{\rm b_{\rm J}}}
\def \mg {M_{\rm g}}
\def \qso {_{\rm QSO}}
\def \lrg {_{\rm LRG}}
\def \gal {_{\rm gal}}
\def \xibar {\bar{\xi}}
\def \xis{\xi(s)}
\def \xisp{\xi(\sigma, \pi)}
\def \Xisig{\Xi(\sigma)}
\def \xir{\xi(r)}
\def \max {_{\rm max}}
\def \gsim { \lower .75ex \hbox{$\sim$} \llap{\raise .27ex \hbox{$>$}} }
\def \lsim { \lower .75ex \hbox{$\sim$} \llap{\raise .27ex \hbox{$<$}} }
\def \deg {^{\circ}}
%\def \sqdeg {\rm deg^{-2}}
\def \deltac {\delta_{\rm c}}
\def \mmin {M_{\rm min}}
\def \mbh  {M_{\rm BH}}
\def \mdh  {M_{\rm DH}}
\def \msun {M_{\odot}}
\def \z {_{\rm z}}
\def \edd {_{\rm Edd}}
\def \lin {_{\rm lin}}
\def \nonlin {_{\rm non-lin}}
\def \wrms {\langle w_{\rm z}^2\rangle^{1/2}}
\def \dc {\delta_{\rm c}}
\def \wp {w_{p}(\sigma)}
\def \PwrSp {\mathcal{P}(k)}
\def \DelSq {$\Delta^{2}(k)$}
\def \WMAP {{\it WMAP \,}}
\def \cobe {{\it COBE }}
\def \COBE {{\it COBE \;}}
\def \HST  {{\it HST \,\,}}
\def \Spitzer  {{\it Spitzer \,}}
\def \ATLAS {VST-AA$\Omega$ {\it ATLAS} }
\def \BEST   {{\tt best} }
\def \TARGET {{\tt target} }
\def \TQSO   {{\tt TARGET\_QSO}}
\def \HIZ    {{\tt TARGET\_HIZ}}
\def \FIRST  {{\tt TARGET\_FIRST}}
\def \zc {z_{\rm c}}
\def \zcz {z_{\rm c,0}}

\newcommand{\ltsim}{\raisebox{-0.6ex}{$\,\stackrel
        {\raisebox{-.2ex}{$\textstyle <$}}{\sim}\,$}}
\newcommand{\gtsim}{\raisebox{-0.6ex}{$\,\stackrel
        {\raisebox{-.2ex}{$\textstyle >$}}{\sim}\,$}}
\newcommand{\simlt}{\raisebox{-0.6ex}{$\,\stackrel
        {\raisebox{-.2ex}{$\textstyle <$}}{\sim}\,$}}
\newcommand{\simgt}{\raisebox{-0.6ex}{$\,\stackrel
        {\raisebox{-.2ex}{$\textstyle >$}}{\sim}\,$}}

\newcommand{\Msun}{M_\odot}
\newcommand{\Lsun}{L_\odot}
\newcommand{\lsun}{L_\odot}
\newcommand{\Mdot}{\dot M}

\newcommand{\sqdeg}{deg$^{-2}$}
\newcommand{\lya}{Ly$\alpha$\ }
%\newcommand{\lya}{Ly\,$\alpha$\ }
\newcommand{\lyaf}{Ly\,$\alpha$\ forest}
%\newcommand{\eg}{e.g.~}
%\newcommand{\etal}{et~al.~}
\newcommand{\lyb}{Ly$\beta$\ }
\newcommand{\cii}{C\,{\sc ii}\ }
\newcommand{\ciii}{C\,{\sc iii}]\ }
\newcommand{\civ}{C\,{\sc iv}\ }
\newcommand{\SiIV}{Si\,{\sc iv}\ }
\newcommand{\mgii}{Mg\,{\sc ii}\ }
\newcommand{\feii}{Fe\,{\sc ii}\ }
\newcommand{\feiii}{Fe\,{\sc iii}\ }
\newcommand{\caii}{Ca\,{\sc ii}\ }
\newcommand{\halpha}{H\,$\alpha$\ }
\newcommand{\hbeta}{H\,$\beta$\ }
\newcommand{\hgamma}{H\,$\gamma$\ }
\newcommand{\hdelta}{H\,$\delta$\ }
\newcommand{\oi}{[O\,{\sc i}]\ }
\newcommand{\oii}{[O\,{\sc ii}]\ }
\newcommand{\oiii}{[O\,{\sc iii}]\ }
\newcommand{\heii}{[He\,{\sc ii}]\ }
\newcommand{\nv}{N\,{\sc v}\ }
\newcommand{\nev}{Ne\,{\sc v}\ }
\newcommand{\neiii}{[Ne\,{\sc iii}]\ }
\newcommand{\aliii}{Al\,{\sc iii}\ }
\newcommand{\siiii}{Si\,{\sc iii}]\ }


%%%%%%%%%%%%%%%%%%%%%%%%%%%%%%%%%%%%%%%%%%%%%%%%%%%%%
%              define Listings                       %
%%%%%%%%%%%%%%%%%%%%%%%%%%%%%%%%%%%%%%%%%%%%%%%%%%%%%
\definecolor{dkgreen}{rgb}{0,0.6,0}
\definecolor{gray}{rgb}{0.5,0.5,0.5}
\definecolor{mauve}{rgb}{0.58,0,0.82}

\lstset{frame=tb,
  language=Python,
  aboveskip=3mm,
  belowskip=3mm,
  showstringspaces=false,
  columns=flexible,
  basicstyle={\small\ttfamily},
  numbers=none,
  numberstyle=\tiny\color{gray},
  keywordstyle=\color{blue},
  commentstyle=\color{dkgreen},
  stringstyle=\color{mauve},
  breaklines=true,
  breakatwhitespace=true,
  tabsize=3
}

\begin{document}


\clearpage
\LARGE
\begin{table}
    \begin{center}
%       \begin{tabular}{l ccc c } 
       \begin{tabular}{l lll c } 

        \hline
        \hline 
                                               &                                         &                                        &                        &                   \\
                                               & QSO                                 & XRB                                  & TDE                &  References \\
                                               &                                         &                                        &                        &                    \\
         \hline 
                                               &                                         &                                         &                                &   \\
$M_{\rm BH}$                            & 10$^{6-9}$                        & 10$^{0-1.8}$                      &  $<10^{6-7}$(??)        &   \\ 
$\dot{M}$                               &  $\sim1$ $M_{\odot}$/yr   & $\sim1$ $M_{\circ}$         & $1-10 M_{\circ}$/yr  &   \\
$\ddot{M}$ ($\Rightarrow$LC shape?)   &  ?                      &  ?                                     & ?                              &  \\
$a$ (BH spin)                          & low, mode, high              & generally high                  &                                & \\
$\log L/ L_{\rm Edd}$                & -2 - 0                             & 0.01-1                              &                                & \\
preferential $L_{\rm Edd}$??       & {\it maybe} for CLQs       &                                         &                                & \\
ang. momen (accn disk)         &                                         &                                         &                               & \\           
$\frac{d}{dt}$ ang. momen     &    ?                                    &      ?                                  & ?                            &  \\           
fuel source                             & accn disk                         & accn disk                          &  star                      &  \\ 
opacity                                   &                                         &                                          &                             & \\
accn disk wind??                    &                                         &                                           &                             & \\
                                              &                                         &                                           &                              & \\
host galaxy                           & $\sim$whole population$^{*}$   & ---                           & post-starburst preference  & \\  
                                             & $^{*}$though not local AGN        & ---                           &                                            &  \\  
evolution with $z$                & peaks at $z$$\sim$ 2-3   & Yes                                    & ?                                          &  \\
binary BHs?                           & $\surd$                             &                                          & $\times$ (probably)             & \\
                                             &                                           &                                          &                                             & \\
BLR?                                      & $\surd$                             & No                                     &$\surd$ (but weird?)             & \\
CL-BLR?                                & $\surd$                             & No (but...)                          &                                             & \\
BLR in polarimatory?             & Yes                                     &  n/a                                   & ?                                          & \\  
\heii  ?                                  & rare                                    &                                          & $\surd$                                & \\
Coronal Lines                       & Sometimes                         & ?                                        & Sometimes                           & \\
Fe opacity important?           & $\surd$                             & ?                                        & ?                                            & \\
%\oiii/H$\beta$                   & defines Type \#                 & (no NLR/BLR)                  & ? \\
                                               &                                         &                                         &                                &   \\
         \hline
         \hline 
       \end{tabular}
      \caption{\href{https://github.com/d80b2t}{\tt github.com/d80b2t}}
    \end{center}
\end{table}
  

\clearpage
\LARGE
\begin{table}
    \begin{center}
      \begin{tabular}{lcccr} 
        \hline
        \hline 
                                            & & & \\
                                               & QSO                                & XRB                                  & TDE                \\
                                            & & & \\
        \hline 
                                            & & & \\
                               & & & \\
PSD in opt.     & changes with $\dot{M}$    &  & \\
PSD in X-ray   & no evolution    &  & \\
PSD  in IR         & ?   &  & n/a? \\
                                          & & & \\
X-rays                               & yes & By definiton & No (except when there are) \\
Hard state?                         &      & Yes  & \\
X-ray variability?  (soft)     & Yes  & Yes &  \\
X-ray variability?  (hard)    &         &       & \\
corona?                              &  Yes  & $\surd$/$\times$ (Big debate)        & ? \\
                                          & & & \\
Radio  variability                & $\surd$        &                                      & \\
Infrared variability            & $\surd$          &                                    & $\surd$ (probably)\\
                                         & & & \\
Is $x$ important?                     & & & \\
%$\;\;\;\;\;$ Viscous timescale         &  Incredibly & & \\
$\;\;\;\;$Viscous timescale         &  Incredibly & & \\
$\;\;\;\;$X-ray  Reprocessing            & Yes & & \\
$\;\;\;\;$IR       Reprocessing            & Yes & & \\
$\;\;\;\;$Atomic Physics                    & & & \\
% (value of $\alpha$??)        & & & \\
%$\;\;\;\;\;$ Light Travel    & & & \\
%$\;\;\;\;\;$ Thermal          & & & \\
%$\;\;\;\;\;$ Orbital            & & & \\
%$\;\;\;\;\;$ inflow            & & & \\
%                                         & & & \\
Challenges SS73?              & AGN disk (x4) too big      &     &     \\
%                                        &  AGN disk (x4) too big  &     & \\
        \hline
        \hline 
     \end{tabular}
%\caption{\href{https://github.com/d80b2t}{\tt github.com/d80b2t}}
  \end{center}
\end{table}



\clearpage
%% \medskip \medskip
\normalsize
From Nadia Blagorodnova:: \\ 
Using the last $M-\sigma$ relations for TDE
hosts, they have a figure showing that preferentially they are close
to $L_{\rm Edd}$, but the range is 0.01-1 of $L_{\rm Edd}$:
http://adsabs.harvard.edu/abs/2017arXiv170608965W \\

\medskip \medskip
From Ohad Shemmer:: \\
Going back to my (and others) "NLS1 philosophy", in a nutshell:
NLS1s have been identified back in 1986 as a "strange new class"
of broad-line Seyferts. Many things happened since then, and 1999
should have pretty much marked the end of the "NLS1" terminology.
Unfortunately, many folks are still having a hard time disengaging from
this exotic "NLS1 class".

These sources are simply understood as type 1 AGN lying at some
extreme corner of parameter space, driven mainly by high $L/L_{\rm
Edd}$.  So their BELR lines are relatively narrow with respect to
their luminosity, indicating high $L/L_{\rm Edd}$ and relatively low
$M_{\rm BH}$. This also dictates extremely low [O III]/Hb ratios, strong Fe II
lines, weak C IV lines, etc. etc.

So, for the Table, I'd simply remove (safely) the last two lines,
i.e., "[O III]/Hb" and "like NLS1", since these two lines are implicit
in the log $L/L_{\rm Edd}$ line above.

Also, I think you can safely change to a "Yes" the XRB evolution
with redshift; see, e.g., Lehmer+16, ApJ, 825, 7, and refs. therein.


\newpage
\section{Time Scales}
From Lawrence (2016) http://adsabs.harvard.edu/abs/2016ASPC..505..107L :: \\
All Type I AGN - those where we can see the strong blue continuum and
broad emission lines - are variable.\footnote{For simplicity, I am
only going to talk about the UVOIR spectral region, ignoring X-rays.}
This is important because variability can provide indirect information
on size scales that are otherwise unmeasurable. Suppose, for
illustration, we take an AGN at a distance of 100 Mpc, and we assume
that it contains a black hole of mass $10^8$ M$_\odot$. The Table
below shows the angular scale of well known AGN structures, in units
of the Schwarzschild radius $R_S=2GM/c^2$. The accretion disc, Broad
Line Region (BLR) and the geometrically thick obscuring region
sometimes known as the ``torus'' are all unresolvable by direct means,
although as we will describe later, may be mappable by microlensing
transits.

If the accretion disc is in a stable steady state, we might expect it
to evolve gradually on the inward drift timescale set by viscosity,
which is of the order 10,000 years (see
e.g. \citet{Netzer2013}). However, instabilities of various kinds
could give us much faster changes. The {\em light crossing timescale}
$t_{lt}=R/c$, is the shortest timescale that we could possibly see, if
for example one region has variations locked to those of another
region by radiation heating or reflection. This is of the order hours,
days, and years for disc, BLR, and torus respectively. The {\em
dynamical timescale}, $t_{dyn}=\sqrt{R^3/GM}$, is the shortest
timescale on which we are likely to see physical changes in a region,
and is of the order of days, years, and thousands of years for disc,
BLR, and torus respectively. (Free-fall time is roughly the same and
orbital timescale is $2\pi$ times longer.) More realistically,
perturbations may transmit across a region on the sound crossing
timescale $t_{snd}= R/ v_{snd}$. This is somewhat model dependent but
is of the order of years for the accretion disc. Note what I mean here
is the global time to cross the whole region. Local hot spots could
grow on the timescale it takes sound to cross the vertical height of
the disc, which could be 1--3 orders of magnitude faster. Somewhat
related is the ``thermal'' timescale $t_{therm}$ which is roughly the
time it takes for for energy to dissipate within the disc, i.e. it is
a kind of response timescale to a spike of energy input. This is model
dependent of course, but some standard formulae are given in
\citet{Collier2001a} and \citet{Kelly2009}. It is of the order of days
for the inner disc and years for the optical disc.  The analogous
``response'' timescale for the BLR and for the obscuring region is
actually the light-crossing time - the local response time to a change
in photo-ionisation or heating is very short, but what we see is
smeared out by the range of light travel delays.

\begin{table}[!ht]
\begin{center}
%\caption{AGN timescales}
\smallskip
{\small
\begin{tabular}{lllllll}  % l = left, c = centered

\hline
\noalign{\smallskip}
AGN & physical & angular & $t_{lt}$ & $t_{dyn}$ & $t_{snd}$ & $t_{therm}$ \\
\noalign{\smallskip}
Structure & size & size & &  &  & \\
\noalign{\smallskip}
\hline

\noalign{\smallskip}
Inner disc & 5 $R_S$ & 0.1$\mu$as & 1.4hrs & 4.3hrs & 1.3 yrs & 18.7days \\
\noalign{\smallskip}
Optical disc & 50 $R_S$ & 1$\mu$as & 14hrs & 5.7days & 23 yrs & 1.6yrs \\
\noalign{\smallskip}
Broad Line Region & 1000 $R_S$ & 20$\mu$as & 11days & 1.4yrs & 800 yrs & -- \\
\noalign{\smallskip}
Obscuring Region & $10^5 R_S$ & 2mas & 3.1yrs & 1.4kyrs & 350 kyrs & --  \\

\noalign{\smallskip}
\hline  
\end{tabular}
}
\end{center}
\end{table}

Are these timescales relevant to what we actually see? The UV continuum changes on timescales of weeks\footnote{Here I am assuming an Seyfert-like object appropriate to our $10^8 M_\odot$ example.}, with an RMS of around $\pm 30\%$, which means trough-to-peak changes of up to a factor of two are not unusual. The variations in the optical continuum, BLR, and IR seem to track these variations with roughly the light-travel time delays suggested in the Table, together with a similar amount of smearing (see recent examples in \citet{Edelson2015}, \cite{Grier2012}, and  \citet{Koshida2014}). This strongly suggests that almost all the changes we see on the relevant timescales represent reprocessed emission driven by changes in the very central regions. The conventional explanation for many years has been that the driving power is from the X-ray source (e.g \citet{McHardy2014}), but in many cases this does not work in either energy budget or correlation terms (see \citet{Lawrence2012} and references therein). A good alternative for the driving power is the (unseen) EUV peak of the very inner accretion disc.

The amplitude we see in the optical continuum on these $\sim$week timescales (around 3\%  RMS) is much smaller than that seen in the UV variations, which suggests that a very blue variable component mixes with an unchanging, or slower changing, redder component. \citet{Lawrence2012} argues that this variable reprocessor is a system of dense inner clouds surrounding the disc, rather than the disc itself.

The variations seen in the UV, which the optical and BLR emission track, seem to follow a red-noise or random-walk like pattern, increasing in amplitude to longer timescales, flattening at a characteristic timescale of the order tens of days. This timescale depends on the mass of the black hole (Collier and Peterson 2001). This characteristic timescale seems to match the thermal timescale of the inner disc, suggesting that variability is driven by some unknown stochastic process, filtered by the physical response of the disc \citep{Kelly2009, Kelly2011}.

Note that the changes we see in broad emission lines are also of the order weeks, tracking the changes in the UV photo-ionising source. This is much shorter than the dynamical timescale of the BLR, and means we are not seeing structural changes in this region. In the popular "local optimally emitting cloud (LOC)" models we will be lighting up different pre-existing clouds at different distances as the UV goes up and down \citep{Peterson2006,Goad2014}, which is why the amplitude of line variations (the ``responsivity'') varies with line species - Ly $\alpha$ has a large amplitude and Mg II hardly varies at all (e.g. \citet{Cackett2015}). However, it is possible that on longer timescales we {\em will} see BLR structural changes - a point we will return to in section 5.3.


\medskip \medskip
\noindent
From Aneta Siemiginowska's talk::\\
Light crossing time at 100 rs:
\begin{equation}
    t_{\rm lc} = 1.1 \,  M_{8} \, R_{100 r_{S}}  \; {\rm days}
\end{equation}

\noindent
Orbital::
\begin{equation}
  t_{\rm orb} = 104 \, M_{8} \,  (R_{100r_{S}})^{3/2} \;  {\rm days}
\end{equation}

\noindent
Thermal (note the viscosity dependence)
\begin{equation}
  t_{\rm th} = 4.6 \, (\alpha_{0.01})^{-1} \, M_{8}\, (R_{100r_{S}} )^{3/2}  \; {\rm years}
\end{equation}
$r_{s} = 2 \,  GM_{\rm bh}/c^{2}$ \\ 
$R_{100r_{S}} = \, R / 100 r_{S}$  \\
$M_{8} = \,  M_{\rm bh} /10^{8} M_{\odot}$.

\medskip
\noindent
Note:: 
\begin{equation}
  \Rightarrow t_{\rm th} \sim (h/r)^{2} t_{\rm visc}
\end{equation}


\newpage


\begin{table}
    \begin{center}
%       \begin{tabular}{l ccc c } 
       \begin{tabular}{l l l l l l} 
        \hline
        \hline 
         Timescale   &          & Equation                                                                                                         & Baseline& Range& Ref \\
        \hline  
        \hline 
%         Dynamical  & dyn    & $(R^3/GM)^{1/2} = P_{\rm orb}/2\pi = 1.4 \; R_{1000}^{3/2} \; M_{8} \; {\rm yr}$          &   & & \\
         Dynamical  & dyn    & $(R^3/GM)^{1/2}$                                            &   & & \\
                            &          & $P_{\rm orb}/2\pi$                                            &   & & \\
                            &          & $1.4 \; R_{1000}^{3/2} \; M_{8} \; {\rm yr}$         &   & & \\
%         Escape        &  esc &  $v_{\rm esc} / g = (v_{\rm esc} / v_{\rm Kep}) . \tau_{\rm dyn} = 1.4 \tau_{\rm dyn} \; s $  & & & \\
           Escape        &  esc &  $v_{\rm esc} / g$                                              & & & \\
                              &         & $(v_{\rm esc} / v_{\rm Kep}) . \tau_{\rm dyn} $       & & & \\
                            &           & $1.4 \tau_{\rm dyn} \; s $                                 & & & \\
         Apocenter   & apo   & $v_{\rm radial}/g$     & & & \\
                            &         &   $75 v_{1000}. R_{1000} 2 M8$ d & & & \\ 
         Cooling time      & cool  &    & & & \\
         Sound crossing  &  s      &    & & & \\
         Cloud crossing  &  ic      &    & & & \\
        Cloud crushing  & cc       &     & & & \\
         X-ray                 & X       &    & & & \\
         \hline
         \hline 
       \end{tabular}
      \caption{\href{https://github.com/d80b2t}{\tt github.com/d80b2t}}
    \end{center}
\end{table}
\noindent
Where:\\
$cs = (\gamma k_{\rm B} T/\mu m_{\rm H}^{1/2}) = 150 T_{6}^{1/2} km s^{-1}$ \\
$g$ is the local acceleration due to gravity, GM/R2; \\
$G$ is the gravitational constant; \\
$k_{\rm B}$ is Boltzmann’s constant = $1.38 \times10^{-16}$ erg K$^{-1}$;\\
$L/LEdd$ is the Eddington ratio; \\
$Lbol,44$ is the ultraviolet bolometric luminosity in units of 1044 erg s-1;\\
$\mathcal{M}$  is the Mach number; \\
$M$ is the mass of the black hole in solar masses;\\
$M_{8}$ is $M$ in units of 10$^{8}$ solar masses;\\
$m_{\rm H}$ is the mass of the hydrogen atom = $1.67\times10^{-24}$ g;\\
P$_{\rm orb}$ is the orbital period in s;\\
$R$ is the distance from the central black hole in cm;\\
$R_{\rm 1000}$ = is $R$ in units of 1000 Schwarzschild radii, $rg = 2GM/c2$; \\
$ri,13$ is ri the initial radius of a condensing cloud in units of 1013 cm;\\
$rc$ is the radius of the condensed cloud, = ri .χ-1/3, for a density ratio of χ; i.e. 0.22 ri for a density ratio of 100;\\
$T_{\rm i,6}$ is the initial temperature of the wind in units of 106 K; \\
$v_{\rm 1000}$ = initial radial WA velocity in units of 1000 km s-1; \\
$v_{\rm esc}$ = (2GM/R)1/2 is the escape velocity from radius R = 9500 R1000-1⁄2 km s-1 = √2×6700 R1000-1⁄2 km s-1;\\
$Z/Z$ is gas metallicity relative to solar (section 2.1); \\
$\Lambda(Τ)$ is the cooling coefficient (erg s-1 cm3) ; \\
$\Lambda_{\rm b}(T)$ is the cooling coefficient for bremsstrahlung; \\
$\Lambda (Τ)/\Lambda_{\rm b}(T)$ is the factor increase in the cooling
coefficient in a thermal plasma due to line cooling over
bremsstrahlung at solar metallicity, which has values of $\sim$35 for 
T =105 - 106 K, ~100forT=104.5 K, andpeaksat~500forT=105–5.5 K;\\
$\gamma$ is the ideal gas adiabatic index = 5/3; \\
$\mu$ is the mean molecular weight of the gas (~0.6); and\\
$\chi$ is the ratio of the cloud density to the ambient medium density.




%\bibliographystyle{mn2e}
%\bibliography{/cos_pc19a_npr/LaTeX/tester_mnras}

\end{document}

