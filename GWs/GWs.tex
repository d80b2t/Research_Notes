\documentclass[11pt]{article}
\setlength {\textwidth}{180mm} 
\setlength {\textheight}{260mm}
\topmargin=-35.00mm
\oddsidemargin=-10.00mm
\pagestyle{empty}


\usepackage[toc,page]{appendix}
\usepackage{amsmath, amssymb}
\usepackage{bm}% bold math
\usepackage{cancel, caption}
\usepackage{dcolumn}% Align table columns on decimal point
\usepackage{epsfig, epsf}
\usepackage{graphicx,fancyhdr,natbib,subfigure}
\usepackage{lscape, longtable}
\usepackage{hyperref,ifthen}
\usepackage{verbatim}
\usepackage{color}
\usepackage[usenames,dvipsnames]{xcolor}
\usepackage{listings}
%% http://en.wikibooks.org/wiki/LaTeX/Colors



%%%%%%%%%%%%%%%%%%%%%%%%%%%%%%%%%%%%%%%%%%%
%       define Journal abbreviations      %
%%%%%%%%%%%%%%%%%%%%%%%%%%%%%%%%%%%%%%%%%%%
\def\nat{Nat} \def\apjl{ApJ~Lett.} \def\apj{ApJ}
\def\apjs{ApJS} \def\aj{AJ} \def\mnras{MNRAS}
\def\prd{Phys.~Rev.~D} \def\prl{Phys.~Rev.~Lett.}
\def\plb{Phys.~Lett.~B} \def\jhep{JHEP} \def\nar{NewAR}
\def\npbps{NUC.~Phys.~B~Proc.~Suppl.} \def\prep{Phys.~Rep.}
\def\pasp{PASP} \def\aap{Astron.~\&~Astrophys.} \def\araa{ARA\&A}
\def\jcap{\ref@jnl{J. Cosmology Astropart. Phys.}}%
\def\physrep{Phys.~Rep.}

\newcommand{\preep}[1]{{\tt #1} }

%%%%%%%%%%%%%%%%%%%%%%%%%%%%%%%%%%%%%%%%%%%%%%%%%%%%%
%              define symbols                       %
%%%%%%%%%%%%%%%%%%%%%%%%%%%%%%%%%%%%%%%%%%%%%%%%%%%%%
\def \Mpc {~{\rm Mpc} }
\def \Om {\Omega_0}
\def \Omb {\Omega_{\rm b}}
\def \Omcdm {\Omega_{\rm CDM}}
\def \Omlam {\Omega_{\Lambda}}
\def \Omm {\Omega_{\rm m}}
\def \ho {H_0}
\def \qo {q_0}
\def \lo {\lambda_0}
\def \kms {{\rm ~km~s}^{-1}}
\def \kmsmpc {{\rm ~km~s}^{-1}~{\rm Mpc}^{-1}}
\def \hmpc{~\;h^{-1}~{\rm Mpc}} 
\def \hkpc{\;h^{-1}{\rm kpc}} 
\def \hmpcb{h^{-1}{\rm Mpc}}
\def \dif {{\rm d}}
\def \mlim {m_{\rm l}}
\def \bj {b_{\rm J}}
\def \mb {M_{\rm b_{\rm J}}}
\def \mg {M_{\rm g}}
\def \qso {_{\rm QSO}}
\def \lrg {_{\rm LRG}}
\def \gal {_{\rm gal}}
\def \xibar {\bar{\xi}}
\def \xis{\xi(s)}
\def \xisp{\xi(\sigma, \pi)}
\def \Xisig{\Xi(\sigma)}
\def \xir{\xi(r)}
\def \max {_{\rm max}}
\def \gsim { \lower .75ex \hbox{$\sim$} \llap{\raise .27ex \hbox{$>$}} }
\def \lsim { \lower .75ex \hbox{$\sim$} \llap{\raise .27ex \hbox{$<$}} }
\def \deg {^{\circ}}
%\def \sqdeg {\rm deg^{-2}}
\def \deltac {\delta_{\rm c}}
\def \mmin {M_{\rm min}}
\def \mbh  {M_{\rm BH}}
\def \mdh  {M_{\rm DH}}
\def \msun {M_{\odot}}
\def \z {_{\rm z}}
\def \edd {_{\rm Edd}}
\def \lin {_{\rm lin}}
\def \nonlin {_{\rm non-lin}}
\def \wrms {\langle w_{\rm z}^2\rangle^{1/2}}
\def \dc {\delta_{\rm c}}
\def \wp {w_{p}(\sigma)}
\def \PwrSp {\mathcal{P}(k)}
\def \DelSq {$\Delta^{2}(k)$}
\def \WMAP {{\it WMAP \,}}
\def \cobe {{\it COBE }}
\def \COBE {{\it COBE \;}}
\def \HST  {{\it HST \,\,}}
\def \Spitzer  {{\it Spitzer \,}}
\def \ATLAS {VST-AA$\Omega$ {\it ATLAS} }
\def \BEST   {{\tt best} }
\def \TARGET {{\tt target} }
\def \TQSO   {{\tt TARGET\_QSO}}
\def \HIZ    {{\tt TARGET\_HIZ}}
\def \FIRST  {{\tt TARGET\_FIRST}}
\def \zc {z_{\rm c}}
\def \zcz {z_{\rm c,0}}

\newcommand{\ltsim}{\raisebox{-0.6ex}{$\,\stackrel
        {\raisebox{-.2ex}{$\textstyle <$}}{\sim}\,$}}
\newcommand{\gtsim}{\raisebox{-0.6ex}{$\,\stackrel
        {\raisebox{-.2ex}{$\textstyle >$}}{\sim}\,$}}
\newcommand{\simlt}{\raisebox{-0.6ex}{$\,\stackrel
        {\raisebox{-.2ex}{$\textstyle <$}}{\sim}\,$}}
\newcommand{\simgt}{\raisebox{-0.6ex}{$\,\stackrel
        {\raisebox{-.2ex}{$\textstyle >$}}{\sim}\,$}}

\newcommand{\Msun}{M_\odot}
\newcommand{\Lsun}{L_\odot}
\newcommand{\lsun}{L_\odot}
\newcommand{\Mdot}{\dot M}

\newcommand{\sqdeg}{deg$^{-2}$}
\newcommand{\lya}{Ly$\alpha$\ }
%\newcommand{\lya}{Ly\,$\alpha$\ }
\newcommand{\lyaf}{Ly\,$\alpha$\ forest}
%\newcommand{\eg}{e.g.~}
%\newcommand{\etal}{et~al.~}
\newcommand{\lyb}{Ly$\beta$\ }
\newcommand{\cii}{C\,{\sc ii}\ }
\newcommand{\ciii}{C\,{\sc iii}]\ }
\newcommand{\civ}{C\,{\sc iv}\ }
\newcommand{\SiIV}{Si\,{\sc iv}\ }
\newcommand{\mgii}{Mg\,{\sc ii}\ }
\newcommand{\feii}{Fe\,{\sc ii}\ }
\newcommand{\feiii}{Fe\,{\sc iii}\ }
\newcommand{\caii}{Ca\,{\sc ii}\ }
\newcommand{\halpha}{H\,$\alpha$\ }
\newcommand{\hbeta}{H\,$\beta$\ }
\newcommand{\hgamma}{H\,$\gamma$\ }
\newcommand{\hdelta}{H\,$\delta$\ }
\newcommand{\oi}{[O\,{\sc i}]\ }
\newcommand{\oii}{[O\,{\sc ii}]\ }
\newcommand{\oiii}{[O\,{\sc iii}]\ }
\newcommand{\heii}{[He\,{\sc ii}]\ }
\newcommand{\nv}{N\,{\sc v}\ }
\newcommand{\nev}{Ne\,{\sc v}\ }
\newcommand{\neiii}{[Ne\,{\sc iii}]\ }
\newcommand{\aliii}{Al\,{\sc iii}\ }
\newcommand{\siiii}{Si\,{\sc iii}]\ }


%%%%%%%%%%%%%%%%%%%%%%%%%%%%%%%%%%%%%%%%%%%%%%%%%%%%%
%              define Listings                       %
%%%%%%%%%%%%%%%%%%%%%%%%%%%%%%%%%%%%%%%%%%%%%%%%%%%%%
\definecolor{dkgreen}{rgb}{0,0.6,0}
\definecolor{gray}{rgb}{0.5,0.5,0.5}
\definecolor{mauve}{rgb}{0.58,0,0.82}

\lstset{frame=tb,
  language=Python,
  aboveskip=3mm,
  belowskip=3mm,
  showstringspaces=false,
  columns=flexible,
  basicstyle={\small\ttfamily},
  numbers=none,
  numberstyle=\tiny\color{gray},
  keywordstyle=\color{blue},
  commentstyle=\color{dkgreen},
  stringstyle=\color{mauve},
  breaklines=true,
  breakatwhitespace=true,
  tabsize=3
}

\begin{document}

\title{Gravitational Waves}
\author{Nicholas P. Ross}
\date{\today}
\maketitle

%%

\begin{abstract}
This is a simple document which will... 
\end{abstract}


%Section heading
\section{PreAmble}


\section{Equations}
%% https://www.codecogs.com/latex/eqneditor.php


\section{Definitions}
strain $=$ fractional change in length/distance\\


\section{LIGO}




%\begin{landscape}
\begin{table}
  \caption{The Lines}
  \label{tab:the_lines}
  \begin{center}
    \begin{tabular}{lrllll} 
      \hline
      \hline
      Name & Wavelength / \AA & Transition & Rest Passband & 
      Interpreation & Reference \\
      \hline
      Lyman-$\alpha$ & 1215.67 & 2 to 1        & $\sim$FUV & Major QSO line       & 1 \\
      Lyman-$\beta$  & 1025.18 & 3 to 1        & $\sim$FUV &        & 1 \\
      Lyman-$\gamma$ &  972.02 & 4 to 1        & $\sim$FUV &        & 1 \\
      Lyman Limit    &  911.27 & $\infty$ to 1 & $\sim$FUV &        & 1 \\
      \hline
      H-$\alpha$     & 6563.   & 3 to 2        & R,r       & Recent major SF or AGN activity & 2 \\
      H-$\beta$      & 4861.   & 4 to 2        & B,V,g     &  & 2 \\
      H-$\gamma$     & 4341.   & 5 to 2        & U,B,u     &  & 2 \\
      H-$\delta$     & 4102.   & 6 to 2        & $\sim$FUV & Previous SF history  & 3 \\
      Balmer Limit   & 3646.   & $\infty$ to 2 & $\sim$FUV &  & 2 \\
      \hline
      HI              & 3646.   & $\infty$ to 2 & $\sim$FUV &  & 2 \\
      HII             & 3646.   & $\infty$ to 2 & $\sim$FUV &  & 2 \\
      \hline
      HeI              & 3646.   & $\infty$ to 2 & $\sim$FUV &  & 2 \\
      HeII             & 3646.   & $\infty$ to 2 & $\sim$FUV &  & 2 \\
      HeIII            & 3646.   & $\infty$ to 2 & $\sim$FUV &  & 2 \\
      \hline
      CIV              & 3646.   & $\infty$ to 2 & $\sim$FUV & Major QSO line & 2 \\
      \hline
      OII              & 3646.   & $\infty$ to 2 & $\sim$FUV & Major QSO line & 2 \\
      \hline
      OIII             & 3646.   & $\infty$ to 2 & $\sim$FUV & Recent major SF line & 2 \\ 
      OIII             & 5007.   & $\infty$ to 2 & $\sim$FUV & Recent major SF line & 2 \\
      \hline
      Ca II H          & 3999.   & $\infty$ to 2 & $\sim$FUV & Old stellar pop & 3 \\
      Ca II K          & 4001.   & $\infty$ to 2 & $\sim$FUV & Old stellar pop & 3 \\
      \hline
      NII              & 5007.   & $\infty$ to 2 & $\sim$FUV &  & 2 \\
      \hline
      NeV              & 3646.   & $\infty$ to 2 & $\sim$FUV & Major QSO line & 2 \\
      \hline
      $[$OIII $\lambda$ 5007/ H$\beta]$ &   &  &   & ``BPT'' diagram reliable tool for determining source & 2, 4, 5 \\
      $[$NII $\lambda$ 6583/ H$\alpha]$ & & & & of line emission from a galaxy visually differentiate & 2,4,5 \\
                                      & & & & between Seyferts, LINERs and SF gals. However, only at & \\
                                      & & & & ``low'' redshifts since need H$\alpha$, (not at $z\sim1$). & \\
                                      & & & &  Modified BPT with $(U-B)$ colour replacing & \\
                                      & & & & $[$NII $\lambda$ 6583/ H$\alpha]$ e.g. Montero-Dorta, 0801.2769. & \\
      \hline
      [SII $\lambda$ 6583/ H$\alpha$]   &    & $\infty$ to 2 & $\sim$FUV & Major QSO line & 2,4. 5  \\
      \hline
      [$\alpha$/Fe]             & 3646.   & $\infty$ to 2 & $\sim$FUV & Major QSO line & 2 \\
      \hline
      NV               & 1???.67 & 2 to 1        & $\sim$FUV & Major QSO line       & 1 \\
      SiIV             & 1???.67 & 2 to 1        & $\sim$FUV & Major QSO line       & 1 \\
      CIV              & 1???.67 & 2 to 1        & $\sim$FUV & Major QSO line       & 1 \\
      CIII]            & 1???.67 & 2 to 1        & $\sim$FUV & Major QSO line       & 1 \\
      MgII             & 1???.67 & 2 to 1        & $\sim$FUV & Major QSO line       & 1 \\
      \hline
      \hline
% GALEX FUV: 1350-1750, % GALEX NUV: 1750-2800
%Lyman alpha forest is the sum of absorption lines arising from the Lyman alpha transition of the neutral hydrogen in the spectra of distant galaxies and quasars.
    \end{tabular}
  \end{center}
\end{table}
%\end{landscape}




\section{References}
\citet{Ross07}

\bibliographystyle{mn2e}
\bibliography{/cos_pc19a_npr/LaTeX/tester_mnras}

\end{document}

