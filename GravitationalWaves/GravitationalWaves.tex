\documentclass[11pt,a4paper]{article}


\usepackage[toc,page]{appendix}
\usepackage{amsmath, amssymb}
\usepackage{bm}% bold math
\usepackage{cancel, caption}
\usepackage{dcolumn}% Align table columns on decimal point
\usepackage{epsfig, epsf}
\usepackage{graphicx,fancyhdr,natbib,subfigure}
\usepackage{lscape, longtable}
\usepackage{hyperref,ifthen}
\usepackage{verbatim}
\usepackage{color}
\usepackage[usenames,dvipsnames]{xcolor}
\usepackage{listings}
%% http://en.wikibooks.org/wiki/LaTeX/Colors



%%%%%%%%%%%%%%%%%%%%%%%%%%%%%%%%%%%%%%%%%%%
%       define Journal abbreviations      %
%%%%%%%%%%%%%%%%%%%%%%%%%%%%%%%%%%%%%%%%%%%
\def\nat{Nat} \def\apjl{ApJ~Lett.} \def\apj{ApJ}
\def\apjs{ApJS} \def\aj{AJ} \def\mnras{MNRAS}
\def\prd{Phys.~Rev.~D} \def\prl{Phys.~Rev.~Lett.}
\def\plb{Phys.~Lett.~B} \def\jhep{JHEP} \def\nar{NewAR}
\def\npbps{NUC.~Phys.~B~Proc.~Suppl.} \def\prep{Phys.~Rep.}
\def\pasp{PASP} \def\aap{Astron.~\&~Astrophys.} \def\araa{ARA\&A}
\def\jcap{\ref@jnl{J. Cosmology Astropart. Phys.}}%
\def\physrep{Phys.~Rep.}

\newcommand{\preep}[1]{{\tt #1} }

%%%%%%%%%%%%%%%%%%%%%%%%%%%%%%%%%%%%%%%%%%%%%%%%%%%%%
%              define symbols                       %
%%%%%%%%%%%%%%%%%%%%%%%%%%%%%%%%%%%%%%%%%%%%%%%%%%%%%
\def \Mpc {~{\rm Mpc} }
\def \Om {\Omega_0}
\def \Omb {\Omega_{\rm b}}
\def \Omcdm {\Omega_{\rm CDM}}
\def \Omlam {\Omega_{\Lambda}}
\def \Omm {\Omega_{\rm m}}
\def \ho {H_0}
\def \qo {q_0}
\def \lo {\lambda_0}
\def \kms {{\rm ~km~s}^{-1}}
\def \kmsmpc {{\rm ~km~s}^{-1}~{\rm Mpc}^{-1}}
\def \hmpc{~\;h^{-1}~{\rm Mpc}} 
\def \hkpc{\;h^{-1}{\rm kpc}} 
\def \hmpcb{h^{-1}{\rm Mpc}}
\def \dif {{\rm d}}
\def \mlim {m_{\rm l}}
\def \bj {b_{\rm J}}
\def \mb {M_{\rm b_{\rm J}}}
\def \mg {M_{\rm g}}
\def \qso {_{\rm QSO}}
\def \lrg {_{\rm LRG}}
\def \gal {_{\rm gal}}
\def \xibar {\bar{\xi}}
\def \xis{\xi(s)}
\def \xisp{\xi(\sigma, \pi)}
\def \Xisig{\Xi(\sigma)}
\def \xir{\xi(r)}
\def \max {_{\rm max}}
\def \gsim { \lower .75ex \hbox{$\sim$} \llap{\raise .27ex \hbox{$>$}} }
\def \lsim { \lower .75ex \hbox{$\sim$} \llap{\raise .27ex \hbox{$<$}} }
\def \deg {^{\circ}}
%\def \sqdeg {\rm deg^{-2}}
\def \deltac {\delta_{\rm c}}
\def \mmin {M_{\rm min}}
\def \mbh  {M_{\rm BH}}
\def \mdh  {M_{\rm DH}}
\def \msun {M_{\odot}}
\def \z {_{\rm z}}
\def \edd {_{\rm Edd}}
\def \lin {_{\rm lin}}
\def \nonlin {_{\rm non-lin}}
\def \wrms {\langle w_{\rm z}^2\rangle^{1/2}}
\def \dc {\delta_{\rm c}}
\def \wp {w_{p}(\sigma)}
\def \PwrSp {\mathcal{P}(k)}
\def \DelSq {$\Delta^{2}(k)$}
\def \WMAP {{\it WMAP \,}}
\def \cobe {{\it COBE }}
\def \COBE {{\it COBE \;}}
\def \HST  {{\it HST \,\,}}
\def \Spitzer  {{\it Spitzer \,}}
\def \ATLAS {VST-AA$\Omega$ {\it ATLAS} }
\def \BEST   {{\tt best} }
\def \TARGET {{\tt target} }
\def \TQSO   {{\tt TARGET\_QSO}}
\def \HIZ    {{\tt TARGET\_HIZ}}
\def \FIRST  {{\tt TARGET\_FIRST}}
\def \zc {z_{\rm c}}
\def \zcz {z_{\rm c,0}}

\newcommand{\ltsim}{\raisebox{-0.6ex}{$\,\stackrel
        {\raisebox{-.2ex}{$\textstyle <$}}{\sim}\,$}}
\newcommand{\gtsim}{\raisebox{-0.6ex}{$\,\stackrel
        {\raisebox{-.2ex}{$\textstyle >$}}{\sim}\,$}}
\newcommand{\simlt}{\raisebox{-0.6ex}{$\,\stackrel
        {\raisebox{-.2ex}{$\textstyle <$}}{\sim}\,$}}
\newcommand{\simgt}{\raisebox{-0.6ex}{$\,\stackrel
        {\raisebox{-.2ex}{$\textstyle >$}}{\sim}\,$}}

\newcommand{\Msun}{M_\odot}
\newcommand{\Lsun}{L_\odot}
\newcommand{\lsun}{L_\odot}
\newcommand{\Mdot}{\dot M}

\newcommand{\sqdeg}{deg$^{-2}$}
\newcommand{\lya}{Ly$\alpha$\ }
%\newcommand{\lya}{Ly\,$\alpha$\ }
\newcommand{\lyaf}{Ly\,$\alpha$\ forest}
%\newcommand{\eg}{e.g.~}
%\newcommand{\etal}{et~al.~}
\newcommand{\lyb}{Ly$\beta$\ }
\newcommand{\cii}{C\,{\sc ii}\ }
\newcommand{\ciii}{C\,{\sc iii}]\ }
\newcommand{\civ}{C\,{\sc iv}\ }
\newcommand{\SiIV}{Si\,{\sc iv}\ }
\newcommand{\mgii}{Mg\,{\sc ii}\ }
\newcommand{\feii}{Fe\,{\sc ii}\ }
\newcommand{\feiii}{Fe\,{\sc iii}\ }
\newcommand{\caii}{Ca\,{\sc ii}\ }
\newcommand{\halpha}{H\,$\alpha$\ }
\newcommand{\hbeta}{H\,$\beta$\ }
\newcommand{\hgamma}{H\,$\gamma$\ }
\newcommand{\hdelta}{H\,$\delta$\ }
\newcommand{\oi}{[O\,{\sc i}]\ }
\newcommand{\oii}{[O\,{\sc ii}]\ }
\newcommand{\oiii}{[O\,{\sc iii}]\ }
\newcommand{\heii}{[He\,{\sc ii}]\ }
\newcommand{\nv}{N\,{\sc v}\ }
\newcommand{\nev}{Ne\,{\sc v}\ }
\newcommand{\neiii}{[Ne\,{\sc iii}]\ }
\newcommand{\aliii}{Al\,{\sc iii}\ }
\newcommand{\siiii}{Si\,{\sc iii}]\ }


%%%%%%%%%%%%%%%%%%%%%%%%%%%%%%%%%%%%%%%%%%%%%%%%%%%%%
%              define Listings                       %
%%%%%%%%%%%%%%%%%%%%%%%%%%%%%%%%%%%%%%%%%%%%%%%%%%%%%
\definecolor{dkgreen}{rgb}{0,0.6,0}
\definecolor{gray}{rgb}{0.5,0.5,0.5}
\definecolor{mauve}{rgb}{0.58,0,0.82}

\lstset{frame=tb,
  language=Python,
  aboveskip=3mm,
  belowskip=3mm,
  showstringspaces=false,
  columns=flexible,
  basicstyle={\small\ttfamily},
  numbers=none,
  numberstyle=\tiny\color{gray},
  keywordstyle=\color{blue},
  commentstyle=\color{dkgreen},
  stringstyle=\color{mauve},
  breaklines=true,
  breakatwhitespace=true,
  tabsize=3
}

\begin{document}

\title{Gravitational Waves}
\maketitle



\section{Very General Equations}
From \href{https://en.wikipedia.org/wiki/Stress-Energy_tensor}{Wikipedia}:: \\

In general relativity, the stress tensor is studied in the context of the Einstein field equations which are often written as

\begin{equation}
  R_{\mu \nu }-{\tfrac {1}{2}}R\,g_{\mu \nu }+\Lambda g_{\mu \nu }={8\pi G \over c^{4}}T_{\mu \nu } 
\end{equation}
where $R_{\mu \nu}$ is the Ricci tensor, $R$ is the Ricci scalar (the
tensor contraction of the Ricci tensor), $g_{\mu \nu }$ is the metric
tensor, $\Lambda$ is the cosmological constant (negligible at the
scale of a galaxy or smaller), and $G$ is the universal gravitational
constant.

For starters, have $\Lambda = 0$ and $\kappa = \frac{8\pi G}{c^{4}}$:  
\begin{equation}
  R_{\mu \nu }-{\tfrac {1}{2}}R\,g_{\mu \nu } = \kappa T_{\mu \nu }.
\end{equation}
then (from [1]) and by assuming that the metric $g_{\mu \nu }$
representing the gravitational field has the form of a slightly
perturbed Minkowski metric $\eta_{\mu \nu }$
\begin{equation}
  g_{\mu \nu } = \eta_{\mu \nu } + \epsilon h_{\mu \nu }
\end{equation}
Here $0< \epsilon \ll 1$, and this linearization simply means that we
develope the left hand side of (0.1) in powers of $\epsilon$ and
neglected all terms involving $\epsilon^k$ with $k > 1$. As a result of this
linearization Einstein found the field equations of linearized general
relativity, which can conveniently be written for an unknown
\begin{equation}
    \bar{h}_{\mu \nu } = h_{\mu \nu } - \frac{1}{2} \, \eta_{\mu \nu } \, h_{\alpha \beta } \, \eta^{\alpha \beta } 
\end{equation}
as
\begin{equation}
    \Box \bar{h}_{\mu \nu } = 2 \kappa T_{\mu \nu },   \; \quad \quad \Box = \eta_{\mu \nu } \,  \partial^{\mu} \partial^{\nu} .
\end{equation}
These equations, outside the sources where 
\begin{equation}
  T_{\mu \nu } = 0 
\end{equation}
constitute a system of decoupled relativistic wave equations: 
\begin{equation}
 \Box {h}_{\mu \nu }  = 0
\end{equation}
for each component of $h_{\mu \nu }$. This enabled Einstein to conclude that
{\it linearized} general relativity theory admits solutions in which the
perturbations of Minkowski spacetime $h_{\mu \nu }$ are plane waves traveling
with the speed of light. Because of the {\it linearity}, by superposing
plane wave solutions with different propagation vectors $k_\mu$ one can
get waves having any desirable wave front. Einstein named these
gravitational waves. He also showed that within the linearized theory
these waves carry energy, and he found a formula for the energy loss
in terms of the third time derivative of the quadrupole moment of the
sources.  

Since far from the sources the gravitational field is very
weak, solutions from the linearized theory should coincide with
solutions from the full theory. Actually the wave detected by the
LIGO/Virgo team was so weak that it was treated as if it were a
gravitational plane wave from the linearized theory. We also mention
that essentially all visualizations of gravitational waves presented
during popular lectures or in the news are obtained using linearized
theory only.




\smallskip
\smallskip
e.g., notes from COTB 2014, Holz. \\

\begin{equation}
  g_{\alpha \beta}(x) = \eta_{\alpha \beta} + h_{\alpha \beta}(x)
\end{equation}

\begin{equation}
h_{\alpha \beta} = \begin{pmatrix}
                                0  &    0  & 0  & 0  \\ 
                                0  &  -1   & 0  & 0   \\ 
                                0  &    0  & 1  & 0  \\ 
                                0  &    0   & 0  & 0   \\ 
                         \end{pmatrix}
f(t-z)
\end{equation}

\noindent
Quadrupole formula gives the total power radiated in gravitational waves:
\begin{equation}
L_{\rm GW}  =  \frac{G}{5c^5} \left \langle  \dddot{I_{ij}} \dddot{I}^{ij} \right \rangle 
\end{equation}

\noindent
Luminosity of GW sources:
\begin{equation}
 L_{\rm GW} \sim \frac{c^5}{G} \sim \frac{2\times 10^{42}} {7\times 10^{-11}} \sim 10^{52} \;  {\rm Joules}
\end{equation}

\smallskip
\smallskip

Notes from THE MATHEMATICS OF GRAVITATIONAL WAVES
A Two-Part Feature;\\ 
\lbrack 1\rbrack: Part One: How the Green Light Was Given for Gravitational Wave Search by C. Denson Hill and Paweł Nurowski p.686 \\  
\lbrack 2\rbrack: Part Two: Gravitational Waves and Their Mathematics, by Lydia Bieri, David Garfinkle, and Nicolás Yunes, p 693 \\
{\tt DOI: http://dx.doi.org/10.1090/noti1551}. 



\href{https://www.quora.com/What-is-the-wave-equation-of-gravitational-waves}{Quora}\\
Final paragraph:: 
In other words, we solve the penultimate wave equation and the
solution represents plane gravitational waves traveling at the speed
of light: transverse and longitudinal but only the transverse carry
energy.

Also from 
\href{https://www.quora.com/Why-are-gravitational-waves-transverse}{Quora} 


\section{Gravitational waves demystified}
Straight from:: 
\href{http://www.tapir.caltech.edu/~teviet/Waves/index.html}{Teviet Creighton's Caltech page}.\\

\section{The strength of the (gravitational) quadrupole field}
The strength of the quadrupole field, i.e. the amplitude of gravitational radiation, scales as:
\begin{eqnarray}
g' & \sim & \frac{\ddddot{I}}{r} \\
g' & \sim & \frac{GM}{c^4}\frac{f^4 s^2}{r} \\
    & \sim & GM \frac{s^2}{\lambda^4 r} \\
\end{eqnarray}
where $s$ is the mass (charge in EM) separation, 

\section{Dimensionless amplitude}
\begin{equation}
  g'  = \frac{\Delta g}{d}  
\end{equation}
where $\Delta g$ is the change in gravity and $d$ is displacement.

The tidal field $g'$ is the physically measurable part of
gravitational phenomena: it represents an observable relative
acceleration or force between two displaced ``test masses''. However,
when discussing gravitational waves, the most common parameter
describing the amplitude is a dimensionless {\it strain},
\begin{equation}
  h = 2 \int\int g' dt^2 .
\end{equation}

What does this quantity mean? Remember that $g'$ is a gravity
gradient, so $g' \, d$ gives the difference in gravity, i.e. the
differential acceleration, between two objects separated by a small
displacement $d$. Two time integrals of acceleration give us the
instantaneous change in this displacement as a function of time. Thus
$h$ is twice the fractional change in displacement between two nearby
masses due to the gravitational wave. This change in displacement
occurs in the plane {\it transverse to the direction of radiation},
and causes a stretch along one axis and a squeeze along the orthogonal
axis.
%% this is illustrated below, showing how a ring of freely-floating masses would be disturbed by a passing gravitational wave. 
The net distortion is twice as much as a stretching or squeezing
alone, which is the reason for the factor of 2 in the equations for
$h$.

$h$ is not itself directly observeable. A constant $h$, or an $h$ that
varies linearly with time, is exactly equivalent to starting the
masses at slightly different positions, or with a slight relative
velocity. {\it Only the second and higher derivatives of $h$ produce
accelerations that would indicate the presence of gravitational
radiation.}

From the above scaling for $g'$ we get $h \sim GMs^2/ \lambda^{2} r$
or:
\begin{equation}
h \sim \frac{GM}{c^2}\frac{1}{r}(\frac{v}{c})^2 .
\end{equation}
The first term is roughly the size of a black hole of mass $M$, so the
distance $r$ to the system must clearly be much greater. Similarly, $v/c$ 
is the ratio of the speeds of masses in the system to the speed of
light, which must be less than (usually much less than) unity. Thus $h$ 
approaches unity when one is standing in the immediate vicinity of
black holes moving about at lightspeed, and is less for any other
circumstance.

In particular, the length scale of a ``typical'' black hole 10$\times$
as massive as our Sun is 14~km, and such objects achive speeds around $c$ 
only when they collide, which might occur on a yearly basis within a
volume of radius 6$\times10^{20}$km (20 Megaparsecs). So the strongest
waves we expect to observe passing the Earth will have $h\sim
10^{-20}$ or less. This is enough to distort the shape of the Earth by
10$^{-13}$ metres, or about 1\% of the size of an atom. By contrast,
the (nonradiative) tidal field of the Moon raises a tidal bulge of
about 1 metre on the Earth's oceans.
 


\newpage
\section{Detecting GWs}

\subsection{Pulsar timing arrays }
\begin{itemize}
\item  10$^{-6}$ - 10$^{-9}$ Hz. 
\item  Sensitive to supermassive binary black holes with orbital periods of months. 
\end{itemize}



\end{document}