\documentclass[11pt,a4paper]{article}

\usepackage{amsmath, amssymb}
\usepackage{bm, booktabs}
\usepackage{cancel, caption}
\usepackage{dcolumn}  % Align table columns on decimal point
\usepackage{epsfig, epsf, enumitem}
\usepackage{fancyhdr}
\usepackage[T1]{fontenc}
\usepackage{graphicx, geometry}
\usepackage{hyperref}
\usepackage{ifthen}
\usepackage[utf8]{inputenc}
\usepackage{lscape, longtable}
\usepackage{multirow}
\usepackage{natbib}
\usepackage{pifont}
\usepackage{ragged2e}
\usepackage{subfigure}
\usepackage{sectsty}
\usepackage{times, tabularx}
\usepackage{tcolorbox}
\usepackage{verbatim}
%\usepackage[usenames,dvipsnames,svgnames,table]{xcolor}



%%%%%%%%%%%%%%%%%%%%%%%%%%%%%%%%%%%%%%%%%%%
%       define Journal abbreviations      %
%%%%%%%%%%%%%%%%%%%%%%%%%%%%%%%%%%%%%%%%%%%
\def\nat{Nat} \def\apjl{ApJ~Lett.} \def\apj{ApJ}
\def\apjs{ApJS} \def\aj{AJ} \def\mnras{MNRAS}
\def\prd{Phys.~Rev.~D} \def\prl{Phys.~Rev.~Lett.}
\def\plb{Phys.~Lett.~B} \def\jhep{JHEP}
\def\npbps{NUC.~Phys.~B~Proc.~Suppl.} \def\prep{Phys.~Rep.}
\def\pasp{PASP} \def\aap{Astron.~\&~Astrophys.} \def\araa{ARA\&A}


%%%%%%%%%%%%%%%%%%%%%%%%%%%%%%%%%%%%%%%%%%%%%%%%%%%%%
%              define symbols                       %
%%%%%%%%%%%%%%%%%%%%%%%%%%%%%%%%%%%%%%%%%%%%%%%%%%%%%
\def \Mpc {~{\rm Mpc} }
\def \Om {\Omega_0}
\def \Omb {\Omega_{\rm b}}
\def \Omcdm {\Omega_{\rm CDM}}
\def \Omlam {\Omega_{\Lambda}}
\def \Omm {\Omega_{\rm m}}
\def \ho {H_0}
\def \qo {q_0}
\def \lo {\lambda_0}
\def \kms {{\rm ~km~s}^{-1}}
\def \kmsmpc {{\rm ~km~s}^{-1}~{\rm Mpc}^{-1}}
\def \hmpc{~\;h^{-1}~{\rm Mpc}} 
\def \hkpc{\;h^{-1}{\rm kpc}} 
\def \hmpcb{h^{-1}{\rm Mpc}}
\def \dif {{\rm d}}
\def \mlim {m_{\rm l}}
\def \bj {b_{\rm J}}
\def \mb {M_{\rm b_{\rm J}}}
\def \qso {_{\rm QSO}}
\def \lrg {_{\rm LRG}}
\def \gal {_{\rm gal}}
\def \xibar {\bar{\xi}}
\def \xis{\xi(s)}
\def \xisp{\xi(\sigma, \pi)}
\def \Xisig{\Xi(\sigma)}
\def \xir{\xi(r)}
\def \max {_{\rm max}}
\def \gsim { \lower .75ex \hbox{$\sim$} \llap{\raise .27ex \hbox{$>$}} }
\def \lsim { \lower .75ex \hbox{$\sim$} \llap{\raise .27ex \hbox{$<$}} }
\def \deg {^{\circ}}
\def \deltac {\delta_{\rm c}}
\def \mmin {M_{\rm min}}
\def \mbh  {M_{\rm BH}}
\def \mdh  {M_{\rm DH}}
\def \msun {M_{\odot}}
\def \z {_{\rm z}}
\def \edd {_{\rm Edd}}
\def \lin {_{\rm lin}}
\def \nonlin {_{\rm non-lin}}
\def \wrms {\langle w_{\rm z}^2\rangle^{1/2}}
\def \dc {\delta_{\rm c}}
\def \wp {w_{p}(\sigma)}
\def \PwrSp {\mathcal{P}(k)}
\def \DelSq {$\Delta^{2}(k)$}
\def \WMAP {{\it WMAP \,}}
\def \cobe {{\it COBE }}
\def \COBE {{\it COBE \;}}
\def \HST  {{\it HST \,\,}}
\def \Spitzer  {{\it Spitzer \,}}


\begin{document}

%   \title{TITLE HERE}
 % \author{AUTHOR HERE}
 %\date{\today}
%\maketitle

With the publicaiont of the ``GWTC-1: A Gravitational-Wave Transient
Catalog of Compact Binary Mergers Observed by LIGO and Virgo during
the First and Second Observing Runs'' (1811.12907v2), binary black
hole (BBH) mergers can be detected to a luminosity distance $d_{\rm L}
= 2750^{+1350}_{-1320}$~Mpc and a redshift of $z=0.48^{+0.19}_{-0.20}$
(for e.g. the event GW170729).

\citep{Ross2018} and \citet{McKernan2018} show that...

%%  Speaker: Brittany Kamai (CalTech)
%%
%% Title: Looking further back with LIGO
%%
%%Abstract:
%% Gravitational wave detectors require constant innovation in detector technology to meet the growing needs of the astrophysics community. With improved sensitivity, we can move beyond the measurements in the local universe and perform precision tests of cosmology. I will discuss our efforts to reduce the amount of low frequency noise within the LIGO detectors, which has a direct impact on our ability to measure heavier mass black hole mergers further out into the universe. I will focus on a proposed major upgrade to the LIGO detectors that will use silicon optics cooled to cryogenic temperatures. Additionally, I will discuss ideas for a passive seismic isolation system that uses advances in seismic cloaking technology. These improvements will extend distance measurements on gravitational wave mergers from the local universe out to much higher redshifts.

\clearpage
\bibliographystyle{mn2e}
\bibliography{/cos_pc19a_npr/LaTeX/tester_mnras}

\end{document}