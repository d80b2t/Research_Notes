\documentclass[11pt,a4paper]{article}


\usepackage{amsmath, amssymb}
\usepackage{bm, booktabs}
\usepackage{cancel, caption}
\usepackage{dcolumn}  % Align table columns on decimal point
\usepackage{epsfig, epsf, enumitem}
\usepackage{fancyhdr}
\usepackage[T1]{fontenc}
\usepackage{graphicx, geometry}
\usepackage{hyperref}
\usepackage{ifthen}
\usepackage[utf8]{inputenc}
\usepackage{lscape, longtable}
\usepackage{multirow}
\usepackage{natbib}
\usepackage{pifont}
\usepackage{ragged2e}
\usepackage{subfigure}
\usepackage{sectsty}
\usepackage{times, tabularx}
\usepackage{tcolorbox}
\usepackage{verbatim}
%\usepackage[usenames,dvipsnames,svgnames,table]{xcolor}



%%%%%%%%%%%%%%%%%%%%%%%%%%%%%%%%%%%%%%%%%%%
%       define Journal abbreviations      %
%%%%%%%%%%%%%%%%%%%%%%%%%%%%%%%%%%%%%%%%%%%
\def\nat{Nat} \def\apjl{ApJ~Lett.} \def\apj{ApJ}
\def\apjs{ApJS} \def\aj{AJ} \def\mnras{MNRAS}
\def\prd{Phys.~Rev.~D} \def\prl{Phys.~Rev.~Lett.}
\def\plb{Phys.~Lett.~B} \def\jhep{JHEP}
\def\npbps{NUC.~Phys.~B~Proc.~Suppl.} \def\prep{Phys.~Rep.}
\def\pasp{PASP} \def\aap{Astron.~\&~Astrophys.} \def\araa{ARA\&A}


%%%%%%%%%%%%%%%%%%%%%%%%%%%%%%%%%%%%%%%%%%%%%%%%%%%%%
%              define symbols                       %
%%%%%%%%%%%%%%%%%%%%%%%%%%%%%%%%%%%%%%%%%%%%%%%%%%%%%
\def \Mpc {~{\rm Mpc} }
\def \Om {\Omega_0}
\def \Omb {\Omega_{\rm b}}
\def \Omcdm {\Omega_{\rm CDM}}
\def \Omlam {\Omega_{\Lambda}}
\def \Omm {\Omega_{\rm m}}
\def \ho {H_0}
\def \qo {q_0}
\def \lo {\lambda_0}
\def \kms {{\rm ~km~s}^{-1}}
\def \kmsmpc {{\rm ~km~s}^{-1}~{\rm Mpc}^{-1}}
\def \hmpc{~\;h^{-1}~{\rm Mpc}} 
\def \hkpc{\;h^{-1}{\rm kpc}} 
\def \hmpcb{h^{-1}{\rm Mpc}}
\def \dif {{\rm d}}
\def \mlim {m_{\rm l}}
\def \bj {b_{\rm J}}
\def \mb {M_{\rm b_{\rm J}}}
\def \qso {_{\rm QSO}}
\def \lrg {_{\rm LRG}}
\def \gal {_{\rm gal}}
\def \xibar {\bar{\xi}}
\def \xis{\xi(s)}
\def \xisp{\xi(\sigma, \pi)}
\def \Xisig{\Xi(\sigma)}
\def \xir{\xi(r)}
\def \max {_{\rm max}}
\def \gsim { \lower .75ex \hbox{$\sim$} \llap{\raise .27ex \hbox{$>$}} }
\def \lsim { \lower .75ex \hbox{$\sim$} \llap{\raise .27ex \hbox{$<$}} }
\def \deg {^{\circ}}
\def \deltac {\delta_{\rm c}}
\def \mmin {M_{\rm min}}
\def \mbh  {M_{\rm BH}}
\def \mdh  {M_{\rm DH}}
\def \msun {M_{\odot}}
\def \z {_{\rm z}}
\def \edd {_{\rm Edd}}
\def \lin {_{\rm lin}}
\def \nonlin {_{\rm non-lin}}
\def \wrms {\langle w_{\rm z}^2\rangle^{1/2}}
\def \dc {\delta_{\rm c}}
\def \wp {w_{p}(\sigma)}
\def \PwrSp {\mathcal{P}(k)}
\def \DelSq {$\Delta^{2}(k)$}
\def \WMAP {{\it WMAP \,}}
\def \cobe {{\it COBE }}
\def \COBE {{\it COBE \;}}
\def \HST  {{\it HST \,\,}}
\def \Spitzer  {{\it Spitzer \,}}



\begin{document}

\title{Machine Learning}
\author{Coursera}
\date{\today}
\maketitle


\section{Introduction:}

Machine learning is the science of getting computers to learn, without being explicitly programmed. 

Machine Learning:\\
-- Grew out of work in AI\\
-- New capability for computers\\

Examples: 
-- Database mining: \\
Large datasets from growth of automaton/web\\
e.g., Web click data, medical records, biology, engineering\\
-- Applications can't program by hand.\\
e.g., Autonomous helicopeter, handwriting recognition, most of Natural 
Language Processing (NLP), Computer Vision.\\
-- Self-customizing programs\\
e.g., Amazon, Netflix product recommendations\\
-- Understanding human learning (brain, real AI). \\


\subsection{Supervised Learning}
{\bf e.g. Housing price prediction}

\underline{Definition:} Supervised Learning: ``right answers'' were given. 
{\bf e.g. house prices were {\it given} in the outsetthe }
``That is, we gave it a data set of houses in which for every example in this data set, we told it what is the right price so what is the actual price that, that house sold for and the toss of the algorithm was to just produce more of these right answers such as for this new house, you know, that your friend may be trying to sell.''\\

``To define with a bit more terminology this is also called a regression problem and by regression problem I mean we're trying to predict a continuous value output. Namely the price.'' \\

{\it Regression: Prediction is trying to predict a CONTINUOUS VALUED OUTPUT e.g. price}\\

{\bf e.g.} Is breast cancer malignant or benign??\\
Tumor size vs. Malignant, and being binary. 

This is a CLASSIFICATION problem. e.g. 0 or 1; malignant or benign;\\
DISCRETE VALUE OUTPUT....\\

``Support Vector Machine'', deals with an infinity long list of features (of data...). 

{\bf e.g.,} Problem 1.
``You have a large inventory of identical items. So imagine that you have thousands of copies of some identical items to sell and you want to predict how many of these items you sell within the next three months.(?)''\\

{\bf e.g.,} Problem 2.
``You have lots of users and you want to write software to examine each individual of your customer's accounts, so each one of your customer's accounts; and for each account, decide whether or not the account has been hacked or compromised.'' \\

Answer: Treat Problem 1 as a Regression Problem; Problem 2 as a Classification Problem. \\




\subsection{Unsupervised Learning}
{\it Supervised} Learning, we are told explicitly what is the so-called ``right'' answer, (``Is it benign or malignant'')\\

{\it UNSupervised} Learning, given data that doesn't have any labels or that all has the same label or really no labels. 
So we're given the data set and we're not told what to do with it and we're not told what each data point is. Instead we're just told, here is a data set. Can you find some structure in the data?\\

You don't know how the data should be ``spilt up''. \\

{\bf e.g.,} Google News (is Unsupervised Learning). \\

{\bf e.g.,} Understanding genomics\\
Unsupervised, here's a bunch of data; have ``no idea what the answer is'', but 
can you find structure in the data... :-) \\

{\bf e.g.'s} Organize computer clusters; Social network Analysis; Market Segmentation; Astro image data (!!!!) \\

{\bf Using Octave}. \\

Octave: single value decomposition; but that turns out to be a linear algebra routine, that is just built into Octave. \\

``What I've seen after having taught machine learning for almost a decade now, is that, you learn much faster if you use Octave as your programming environment, and if you use Octave as your learning tool and as your prototyping tool, it'll let you learn and prototype learning algorithms much more quickly. ''\\

Review Question:\\
Of the following examples, which would you address using an UNsupervised Learning Algorithm??
NOT: Spam fliter, NOT diabetes diagnois. 
YES: News articles on the web; YES: customer data for market segments. 

\newpage




\section{Linear Regression with One Variable}
``Linear regression predicts a real-valued output based on an input value. We discuss the application of linear regression to housing price prediction, present the notion of a cost function, and introduce the gradient descent method for learning.''

\subsection{Model and Cost Function: Model Representation}
{\bf e.g.} Housing Pricing; Supervised Learning, Regression problem. \\

$m=$ number of training examples\\
$x=$ ``input'' variables/features\\
$y=$ ``output'' variables/``target'' variable\\
$(x,y)$ -- one training example\\
$(x^{(i)},y^{(i)})$ -- $i^{\rm th}$ training example\\

``So here's how this supervised learning algorithm works. We saw that
with the training set like our training set of housing prices and we
feed that to our learning algorithm. Is the job of a learning
algorithm to then output a function which by convention is usually
denoted lowercase $h$ and $h$ stands for hypothesis. And what the job
of the hypothesis is, is, is a function that takes as input the size
of a house like maybe the size of the new house your friend's trying
to sell so it takes in the value of $x$ and it tries to output the
estimated value of $y$ for the corresponding house.'' \\

So $h$ is a function that maps from $x$'s to $y$'s. \\
($h$ is the ``hypothesis'', which isn't the best name!!) \\

How do we represent $h$??\\
$h_{\theta}(x) = \theta_{0} + \theta_{1} x$\\
just a simple linear function!!\\
Linear regression with one variable, $x$ 
(aka ``univariate linear regression'' !!). 


\subsection{Model and Cost Function: Cost Function}
$\theta_{0}$ and $\theta_{1}$ are the parameter of the model. \\
Linear regression. Have a training dataset. \\

Chose $\theta_{0}$ and $\theta_{1}$ so that $h_{\theta}(x)$ is close
to $y$ for our training examples $(x,y)$. \\

So, an e.g., of a cost function:\\
$J(\theta_{0}, \theta_{1}$)\\
this Cost Function, is the Squared Error Function. \\
And we want to MINIMIZE IT.\\



\subsection{Model and Cost Function: Cost Function -- Intuition 1}

\noindent
Hypothesis:\\

$h_{\theta}(x) = \theta_{0} + \theta_{1} x$\\

\noindent
Parameters:\\

$\theta_{0}$ and $\theta_{1}$\\

\noindent
Cost Function:\\

$J(\theta_{0}, \theta_{1} = \frac{1}{2m} \sum_{i=1}^{m} (h_{\theta} (x^{(i)}) - y^{(i)})^{2}$.\\

\noindent
Goal: Minimize $J(\theta_{0}, \theta_{1})$\\

Now just fit the best straight line... ;-) \\



\subsection{Model and Cost Function: Cost Function -- Intuition 2}
Same thing, now just with both parameters, $\theta_{0}$ and $\theta_{1}$.\\
Gives contour plots!! ;-) \\


\subsection{Parameter Learning: Gradient Descent}
Have some function $J(\theta_{0}, \theta_{1})$, and what to minimize
$J(\theta_{0}
, \theta_{1})$. \\

Outline: Start with some $\theta_{0}, \theta_{1}$ 
(say $\theta_{0}=0$ and $\theta_{1}=0$) and change until 
you (think you) find the minimum...

(The algorithm::) 
repeat until covergence \{ \\
\begin{equation}
\theta_{j} := \theta_{j} - \alpha \frac{\partial}{\partial \theta_{j}} J(\theta_{0}, \theta_{1})
\end{equation}
\}. \\
for $j=0$ and $j=1$. \\

where $:=$ is CS lingo for ``assignment''\\
$a := b$\\
(this means take the value in b and use it overwrite whatever value is a.)\\
$\alpha$ is the {\it learning rate}.\\
Now, need also to do {\it simultaneous update} (``this is a correct
implementation of gradient descent'').

\noindent
Intution:
$alpha$ is the learning rate (how big the step is for the e.g. derivative...).\\
Obvious, but good to note anyway: \\
if $\alpha$ too small, gradient descent can be slow; \\
if $\alpha$ too big, gradient descent can overshoot the minimum. It may fail to 
converge, or even diverge(!!) \\

Note, if you initilize G.D. at the local minimum, the derivate term {\bf is} zero, 
and hence leaves $\theta_{1}$ unchanged (which is what you want ;-) \\

GD can converge to a local minimum, even with the learning rate
$\alpha$ fixed. The derivate decreases as you approach the minimum. 
$\theta_{1} := \theta_{1} - \alpha \frac{\partial}{\partial \theta_{1}} J(\theta_{1})$.

\noindent
Linear Regression:
\begin{eqnarray}
\frac{\partial}{\partial \theta_{1}} J(\theta_{0}, \theta_{1}) & = & \frac{\partial}{\partial \theta_{1}}.\frac{1}{2m}\sum_{i=1}^{m}(h_{\theta} (x^{(i)} - y^{(i)})^2\\
 & = &  \frac{\partial}{\partial \theta_{1}}.\frac{1}{2m}\sum_{i=1}^{m}(\theta_{0} + \theta_{1} x^{(i)} - y^{(i)})^2
\end{eqnarray}

``But, it turns out that that the cost function for linear regression is always going to be a bow shaped function like this. The technical term for this is that this is called a convex function.''
Convex function has no local optima, other than the global optimum. 

``Batch'' Gradient Descent. ``Batch'': each step of GD uses all of the training examples (i.e. you are using the $\sum$).\\ 
G.D. is (obviosuly!!) an iterative method (cf. ``normal equation'' method...)\\
G.D. is a ML algorithm!!\\


\section{Linear Algebra Review}
    \subsection{Linear Algebra Review}
    \subsubsection{Matrices and Vectors}
    The dimension of the matrix is going to be written as the number of row times the number of columns in the matrix:
    $R \times C$
    $A_{i,j}$ is the $i,j$ entry in the $i^{\rm th}$ row, $j^{\rm th}$ column. 

    So it turns out that in most of math, the one index version is
    more common For a lot of machine learning applications, zero index.\\

    \subsubsection{Addition and Scalar Multiplication}
    Python pages for Linear Algebra:
    \href{http://docs.scipy.org/doc/numpy/reference/routines.linalg.html}{http://docs.scipy.org/doc/numpy/reference/routines.linalg.html}
   \subsubsection{Matrix Vector Multiplication}
Recall a $3\times4$ times a $4\times1$ gives a $3\times1$ matrix. 

$\begin{bmatrix}
 1 & 3 \\ 
4 & 0\\ 
2  & 1
\end{bmatrix} \begin{bmatrix}
1\\ 
5 
\end{bmatrix} = 
\begin{bmatrix}
16 \\ 
4 \\ 
7
\end{bmatrix}
$

Prediction = DataMatrix $\times$ ParameterVector. \\

      \subsubsection{Matrix Matrix Multiplication}
$\begin{bmatrix}
 1 & 3 & 2\\ 
4 & 0  & 1
\end{bmatrix} \begin{bmatrix}
1 & 3 \\ 
0 & 1 \\
5 & 2 \\
\end{bmatrix} = 
\begin{bmatrix}
11 & 10 \\ 
9 & 14\\ 
\end{bmatrix}
$\\

{\bf m x n matrix $\times$ n x o matrix gives you a m x o matrix}.
$ \begin{bmatrix}
 1 & 3 \\ 
2 & 5\\ 
\end{bmatrix} \begin{bmatrix}
0 & 1 \\ 
3 & 2
\end{bmatrix} = 
\begin{bmatrix}
9 & 4\\ 
15&12  \\ 
\end{bmatrix}
$
    \subsubsection{Matrix Multiplication Properties}
    In general, $A \times B \ne B \times A$.\\
    $A \times I =  I \times A = A$.\\

    \subsubsection{Inverse and Transpose}
    Only Square Matrices have inverses. 
    $A (A^{-1})  = A^{-1} A  = I$.\\
    $ 
    A = \begin{bmatrix}
      1 & 2 & 0 \\ 
      3 & 5 & 9 \\ 
    \end{bmatrix} 
    \Rightarrow
    A^T = A = \begin{bmatrix}
      1 & 3  \\ 
      2 & 5  \\ 
      0 & 9\\
    \end{bmatrix} 
    $

\newpage
%%%%%%%%%%%%%%%%%%%%%%%%%%%%%%%%%%%%%%%%%%%%%%%%%%%%%%%
%%%
%%%    W E E K     T W O
%%%
%%%%%%%%%%%%%%%%%%%%%%%%%%%%%%%%%%%%%%%%%%%%%%%%%%%%%%%
\section*{WEEK   TWO}

\section{Linear Regression with Multiple Variables}

    \subsection{Multivariate Linear Regression}
    \subsubsection{Multiple Features}
    $x_{j}^{(i)}$ = value of feature $j$ in $i^{th}za$ training example. 
    e.g., $x_{1}^{(4)}$ is the first `feature'/column for the fourth datapoint/row...\\
    
    $h_{\theta}(x) = \theta_{0} + \theta_{1}x_{1} + \theta_{2}x_{2} + ... + \theta_{n}x_{n}$\\
    
    $
    {\bf x} = \begin{bmatrix}
      x_{0}  \\
      x_{1}  \\
      x_{2}  \\
      \vdots \\
      x_{x}  \\
    \end{bmatrix}
    $
    \\
   $h_{\theta}(x)= \theta^{T}{\bf x}$\\

    \subsubsection{Gradient descent for Multiple Variables}
    $h_{\theta}(x)= \theta^{T}{\bf x}  =    \sum_{j=0}^{n} \theta_{j}x_{j}^{(i)}$. \\
    $\theta_{0} := \theta_{0}  - \alpha \frac{1}{m}\sum_{i=1}^{m}(h_{\theta} (x^{(i)}-y^{(i)})x_{0}^{(i)}.$\\

    \subsubsection{Gradient Descent in Practice I - Feature Scaling}
    Idea: make sure features are on a similar scale...
    $x_{1}=$size (0-2000 feet$^{2}$) and $x_{2}=$number of bedrooms (0-5)\\
    Just scale by e.g. /2000 and /5, $\Rightarrow$ circular contours, 
    and Gradient Desecent (GD) works a lot better...\\
    Basically want to get every feature into approximately a $-1 \leq x \leq +1$ range. 

    Mean normalization: replace $x_{i}$ with $x_{i} - \mu_{i}$. \\
    Also can e.g. replace $x_{i}$ with $(x_{i} - \mu_{i})/s_{i}$, where 
    $s_{i}$ is the range of the feature and/or the e.g. standard deviation.     
    
    \subsubsection{Gradient Descent in Practice II - Learning Rate}
    How to check GD is working correctly (!!) \\
    If GD is working properly, $J(\theta)$ should decrease after every iteration. \\
    (Recall $J(\theta)$ is the Cost Function.)\\
    Convergence test: e.g. Declare convergence if $J(\theta)$ by less than some 
    threshold $\epsilon$, (but this can be tricky...) Look at e.g. plots of $J(\theta)$ 
    vs. number of iterations....\\
    
    \subsubsection{Features and Polynomial Regression}
    Pretty self-explanatory... :-) \\


    \subsection{Computing Parameters Analytically}
    \subsubsection{Normal Equation}
    $\theta \in \mathbb{R}^{n+1}$. 
    Example of size(feet$^{2}$), no of bedrooms, no of floors, age of home ($x_{1}, x_{2}, x_{3}, x_{4})$
    and $x_{0}=1$ for all e.g. four $m=4$ examples. 
    The cost of home is $y$. \\
    Then 
    \begin{equation}
      \theta = (X^{T} X)^{-1} X^{T} y
    \end{equation}
    where $X$ is matrix with all the $m$ values of  $x_{i}$. 
    
    So, what is $(X^{T} X)^{-1}$? It is the inverse of $X^{T} X$. 
    Set $A = X^{T} X$, then $(X^{T} X)^{-1} =A$. \\

    In Octave: {\tt pinv(X' * X) * X' * y}. 
    
    \noindent
    $m$ training examples; $n$ features: \\
    Gradient Descent: $\bullet$Need to choose $\alpha$; $\bullet$needs many iterations. \\
    Normal Equation: $\bullet$No need to choose $\alpha$; $\bullet$Don't need to iterate. \\
    vs. \\
    Gradient Descent: $\bullet$Works well even when $n$ is large.\\
    Normal Equation: $\bullet$Need to  compute $(X^{T} X)^{-1}$ (which is $n \times n$ and 
    which scales as $\mathcal{0} n^3$.  (Slow if $n$ is very large (where $n\sim10^5$ is the tipping point...)

    \subsubsection{Normal Equation Noninvertibilty}
    \begin{equation}
      \theta = (X^{T} X)^{-1} X^{T} y
    \end{equation}
    What if  $X^{T} X$ is non-invertible??
    Basically use {\tt pinv} (pseudo-inv) (in e.g. Octave) to get round this..
    Two causes for $X^{T} X$ being non-invertible are: (1) Redundant
    features (linearly dependent) e.g. $x_{1}$ is size in feet and $x_{2}$ is
    size in m$^2$ and (2) Too many feature (e.g. $m\ leq n$). 
    If too many features, delete some features, or use regularization.
    

    \subsection{Submitting Programming Assignments}
    \subsubsection{Programming tips from mentors}
    The lectures and exercise PDF files are based on Prof. Ng's
    feeling that novice programmers will adapt to for-loop techniques more
    readily than vectorized methods. So the videos (and PDF files) are
    organized toward processing one training example at a time. The course
    uses column vectors (in most cases), so h (a scalar for one training
    example) is theta' * x.
    
    Lower-case x typically indicates a single training example.
    
    The more efficient vectorized techniques always use X as a matrix of
    all training examples, with each example as a row, and the features as
    columns. That makes X have dimensions of (m x n). where m is the
    number of training examples. This leaves us with h (a vector of all
    the hypothesis values for the entire training set) as X * theta, with
    dimensions of (m x 1).
    
    X (as a matrix of all training examples) is denoted as upper-case X.
    
    Throughout this course, dimensional analysis is your friend.
    

{\bf Linear Regression with Multiple Variables QUIZ}

1. Suppose $m=4$ students have taken some class, and the class had a midterm exam and a final exam. You have collected a dataset of their scores on the two exams, which is as follows:

\begin{table}
\begin{tabular}{ccc}
midterm exam & 	(midterm exam)$^2$	& final exam\\
89 & 7921 &	96\\
72 & 5184 &	74\\
94 & 8836 &	87\\
69 & 4761 &	78\\
\end{tabular}
\end{table}

You'd like to use polynomial regression to predict a student's final exam score from their midterm exam score. Concretely, suppose you want to fit a model of the form $h_{\theta}(x)={\theta}_{0} +{\theta}_{1} x 1+θ2x2$, where x1 is the midterm score and x2 is (midterm score)2. Further, you plan to use both feature scaling (dividing by the "max-min", or range, of a feature) and mean normalization.

What is the normalized feature x(1)1? (Hint: midterm = 89, final = 96 is training example 1.) Please round off your answer to two decimal places and enter in the text box below.
e.g. 
1. $x_2^{(4)} \rightarrow 4761.$ \\
2. Nomalized feature $\rightarrow \frac{x-u}{s}$ where $u$ is average of $X$ and $s=max - min$=8836-4761=4075. \\
3. Finally, 4761−6675.54075=−0.47

Thus, (89 - ((89+72+94+69)/4) ) / (94-69.) = 0.32. :-) \\ 


Normalised feature of $x_1^{(1)}$ is 0.32.

\smallskip \smallskip
\noindent
Q2. You run gradient descent for 15 iterations with $\alpha$=0.3 and
compute $J(\theta)$ after each iteration. You find that the value of
$J(\theta)$ decreases quickly then levels off. Based on this, which of
the following conclusions seems most plausible?

A2:
$\times$Rather than use the current value of $\alpha$, it'd be more promising to try a smaller value of $\alpha$ (say $\alpha$=0.1).\\

$\surd$ $\alpha$=0.3 is an effective choice of learning rate.\\

$\times$Rather than use the current value of $\alpha$, it'd be more promising to try a larger value of $\alpha$ (say $\alpha$=1.0).\\

\smallskip \smallskip
\noindent
Q3. Suppose you have $m=23$ training examples with $n=5$ features
(excluding the additional all-ones feature for the intercept term,
which you should add). The normal equation is $\theta = (X^{T} X)^{-1}
X^{T} y$. For the given values of $m$ and $n$, what are the dimensions
of $θ, X, and y$ in this equation?

A3:
$\surd:$ (I think!) $X$ is 23$\times$6, $y$ is $23 \times$1, $\theta$ is 6 $\times$1\\

$\times:$ $X$ is 23$\times$5, $y$ is 23$\times$1, $\theta$ is 5$\times$1\\

$\times:$ $X$ is 23$\times$5, $y$ is 23$\times$1, $\theta$ is 5$\times$5\\

$\times:$ $X$ is 23$\times$6, $y$ is 23$\times$6, $\theta$ is 6$\times$6\\


\smallskip \smallskip
\noindent
4. Suppose you have a dataset with m=50 examples and n=15 features for each example. You want to use multivariate linear regression to fit the parameters θ to our data. Should you prefer gradient descent or the normal equation?

$\times:$ The normal equation, since gradient descent might be unable to find the optimal θ.

$\surd:$ The normal equation, since it provides an efficient way to directly find the solution.

$\times:$ Gradient descent, since it will always converge to the optimal θ.

$\times:$  Gradient descent, since $(X^{t}X)^{-1}$ will be very slow to compute in the normal equation.

\smallskip \smallskip
\noindent
5. Which of the following are reasons for using feature scaling?

$\times:$ It prevents the matrix XTX (used in the normal equation) from being non-invertable (singular/degenerate).\\
$\times:$ It speeds up solving for θ using the normal equation.\\
$\surd:$ It speeds up gradient descent by making it require fewer iterations to get to a good solution.\\
$\times:$ It is necessary to prevent gradient descent from getting stuck in local optima.\\


\section{Octave/Matlab Tutorial}

    \subsection{Basic Operations}
    \begin{lstlisting}
      1 == 2  % false 
      ans = 0
      1 ~= 2     % not equals, ~ is NOT in Octave 
      ans =  1 
      1 && 0  % AND 
      ans = 0 
      1 || 0 % OR 
      ans =  1 
      xor(1,0)
      ans =  1 
      PS1('>> '); 
      >> 
      >> a = 3
      a =  3
      >> a = 3 ; % semicolon surpressing output...
      >> 
      >> disp(sprintf('2 decimals: \%0.2f', a))
      2 decimals: 3.14
      >> disp(sprintf('6 decimals: \%0.6f', a))
      6 decimals: 3.141593
      >> A = [1 2; 3 4; 5 6]
      A =
      
      1   2
      3   4
      5   6
      
      >> v = [1 2 3]
      v =
      
      1   2   3
      
      >> v = [1; 2; 3;]
      v =
      
      1
      2
      3

      >> v = 1:5
      v =
      
      1   2   3   4   5
      
      >> ones(2,3)
      ans =
      
      1   1   1
      1   1   1
      
      >> twos(2,3)
      error: 'twos' undefined near line 1 column 1
      >> C = 2*ones(2,3)
      C =
      
      2   2   2
      2   2   2
      
      >> 
      >> w = zeros(1,3)
      w =
      
      0   0   0
      
      >> w = rand(1,3)
      w =
      
      0.89676   0.65794   0.27405
      
      >> w = rand(3,3)
      w =
      
      0.44070   0.29262   0.99302
      0.75816   0.90571   0.96482
      0.52197   0.85563   0.79157

      >> w = randn(3,3) % Return a matrix with normally distributed random elements having zero mean and variance one.

      w =
      
      0.347484   0.957131  -1.135894
      0.848703   0.014639   0.207640
      1.569354  -1.489131   0.610893
      
      >> setenv("GNUTERM","qt")
      
      >> w = -6 + sqrt(10)*(randn(1,10000));
      >> hist(w)
      >> hist(w,50)
      >> I = eye(4)
      I =
      
      Diagonal Matrix
      
      1   0   0   0
      0   1   0   0
      0   0   1   0
      0   0   0   1

      >> help eye
    \end{lstlisting}


\href{http://stackoverflow.com/questions/22898609/octave-does-not-plot}{http://stackoverflow.com/questions/22898609/octave-does-not-plot}

\href{http://stackoverflow.com/questions/13786754/octave-gnuplot-aquaterm-error-set-terminal-aqua-enhanced-title-figure-1-unk}{http://stackoverflow.com/questions/13786754/octave-gnuplot-aquaterm-error-set-terminal-aqua-enhanced-title-figure-1-unk}
I had to add setenv("GNUTERM","X11") to OCTAVE\_HOME/share/octave/site/m/startup/octaverc (OCTAVE\_HOME usually is /usr/local) to make it work permanently.


    \subsection{Moving Data Arounds}

    \begin{lstlisting}
      >> load featuresX.dat
      >> load ('featuresX.dat')
      >> featuresX
      >> who
      >> whos
      >> clear featuresX  % gets ride of a variable
      >> v = priceY(1:10) % v is now a 10x1 vector
      >> save hello.mat v;  %saves v to hello.mat
      >> clear % clears all the variables
      >> load hello.mat
      >> save hello.txt v -ascii   % save as text (ASCII)
      >> A = [1 2; 3 4; 5 6;]
      A =
      
      1   2
      3   4
      5   6
      
      >> A(3,2)
      ans =  6
      >> A(2,:) 
      ans =
      
      3   4
      
      >> A(2,:) % : means every element in that row/column
      >> A([1 3], :)
      ans =
      
      1   2
      5   6

      >> 
      A(:, 2) = [10; 11; 12]
      A =
      
      1   10
      3   11
      5   12
      
      >>  A = [A, [100; 101; 102]];
      >> A
      A =
      
      1    10   100
      3    11   101
      5    12   102

      >>  A(:)  % put all elements of A into a single vector
      ans =
      
      1
      3
      5
      10
      11
      12
      100
      101
      102

      >> A = [1 2; 3 4; 5 6;]
      >> B = [11 12; 13 14; 15 16;]
      >> C = [A B]
      C =
      
      1    2   11   12
      3    4   13   14
      5    6   15   16
      
      >> C = [A; B]
      C =
      
      1    2
      3    4
      5    6
      11   12
      13   14
      15   16
      >> [A B]  % is the same as
      >> [A, B] 

    \end{lstlisting}


    
    \subsection{Computing on data}

    \begin{lstlisting}
      >> A = [1 2; 3 4; 5 6;];
      >> B = [11 12; 13 14; 15 16;];
      >> C = [1 1; 2 2];
      >> A* C 
      ans =
      
      5    5
      11   11
      17   17

      >>  A .* B    % N.B.  the dot . means element-wise multiplication
      ans =
      
      11   24
      39   56
      75   96
      
      >>v = [1; 2; 3;]
      v =
      
      1
      2
      3
      
      >> 1./v
      ans =
      
      1.00000
      0.50000
      0.33333
      
      >>  1./A
      ans =
      
      1.00000   0.50000
      0.33333   0.25000
      0.20000   0.16667
      
      >> v + ones(length(v),1)
      ans =
      
      2
      3
      4
      
      >> v +1
      ans =
      
      2
      3
      4
      
      >> A'   %  A transpose 
      ans =
      
      1   3   5
      2   4   6
      
      >> val = max(a)
      val =  15
      >> [val, ind] = max(a)
      val =  15
      ind =  2
      >> max(A)   % does **column wise** maximum
      ans =
      
      5   6
      
      >> a <3   % elementwise operation...
      ans =
      
      1   0   1   1
      
      >> find (a<3)
      ans =
      
      1   3   4
      
      >>A = magic(3)  %does a "magic" N-by-N matrix... :-) 
      A =
      
      8   1   6
      3   5   7
      4   9   2
      
      >> [r,c]  = find(A>=7)
      r =
      
      1
      3
      2
      
      c =
      
      1
      2
      3
      
      >> sum(A) 
      >> prod(A) 
      ans =
      
      96   45   84
      
      >> floor(A)
      ans =
      
      8   1   6
      3   5   7
      4   9   2
      
      >> ceil(A)
      ans =
      
      8   1   6
      3   5   7
      4   9   2
      >>max(A,[],1)
      
      ans =
      
      8   9   7
      >> max(A,[],2)
      ans =

      8
      7
      9

      >>max(max(A))
      ans =  9
      >> max(A(:))
      >> A = magic(9)
      A =
      
      47   58   69   80    1   12   23   34   45
      57   68   79    9   11   22   33   44   46
      67   78    8   10   21   32   43   54   56
      77    7   18   20   31   42   53   55   66
      6   17   19   30   41   52   63   65   76
      16   27   29   40   51   62   64   75    5
      26   28   39   50   61   72   74    4   15
      36   38   49   60   71   73    3   14   25
      37   48   59   70   81    2   13   24   35
      
      >> sum(A,1)  % per-column sum
      ans =

      369   369   369   369   369   369   369   369   369
      
      >> sum(A,2)  % row-wise sum!!
      >> A .* eye(9)
      ans =
      
      47    0    0    0    0    0    0    0    0
      0   68    0    0    0    0    0    0    0
      0    0    8    0    0    0    0    0    0
      0    0    0   20    0    0    0    0    0
      0    0    0    0   41    0    0    0    0
      0    0    0    0    0   62    0    0    0
      0    0    0    0    0    0   74    0    0
      0    0    0    0    0    0    0   14    0
      0    0    0    0    0    0    0    0   35
      
      >> sum(sum(A.*eye(9)))
      ans =  369
      >> sum(sum(A.*flipud(eye(9))))
      ans =  369
      >> flipud(eye(9)

      >> pinv(A)  % Inverse of A
      ans =
      
      0.147222  -0.144444   0.063889
      -0.061111   0.022222   0.105556
      -0.019444   0.188889  -0.102778
      
      >> temp = pinv(A)
      >> temp * A
      ans =
      
      1.00000   0.00000  -0.00000
      -0.00000   1.00000   0.00000
      0.00000   0.00000   1.00000

      >>
   \end{lstlisting}

   ``Now, let's sum the diagonal elements of A and make sure that also sums
   up to the same thing.  So what I'm gonna do is construct a nine by
   nine identity matrix, that's eye nine. And let me take A and
   construct, multiply A element wise, so here's my matrix A. I'm going
   to do A .\^ eye(9). What this will do is take the element wise product
   of these two matrices, and so this should Wipe out everything in A,
   except for the diagonal entries. And now, I'm gonna do sum sum of A of
   that and this gives me the sum of these diagonal elements, and indeed
   that is 369.''
   
    \subsection{Plotting Data}
     \begin{lstlisting}
       octave:1> setenv("GNUTERM","qt")
       octave:2> PS1('>> ')
       >> t=[0:0.01:0.98];
       >> y1 = sin(2*pi*4*t);
       >> plot(t,y1);
       >> y2 = cos(2*pi*4*t);
       >> plot(t,y2);
       >> plot(t,y1);
       >> hold on;    % doesn't clear plot
       >> plot(t,y2,'r');
       >> xlabel('time')
       >> ylabel('value')
       >> legend('sin', 'cos')
       >> title('my plot')
       >> cd /Users/npr1/Desktop
       >> print -dpng 'myPlot.png'
          warning: print.m: fig2dev binary is not available.
          Some output formats are not available.
      >> graphics_toolkit gnuplot  % just implemented after googling error...
      >> print -dpng 'myPlot.png'
      >> close % figure goes away
      >> figure(1); plot(t,y1);
      >> figure(2); plot(t,y2);
      >> subplot(1,2,1);   % e.g. divides plot into a 1x2 grid...
      >> plot(t,y1)
      >> axis([0.5 1 -1 1])  % resets axis values...
      >> clf;                       % clears figure...
      >> A= magic(5);
      >> imagesc(A);
      >> imagesc(A), colorbar, colormap gray;   % Cool; plots a grid of (gray) colors/shades, and a color bar
      >> A(1,2)   % row 1, 2nd column element of A 
      ans =  24
      >> imagesc(magic(15)), colorbar, colormap gray; 
      >> a=1, b=2, c=3   % Comma chaining of commands/function calls.
      a =  1
      b =  2
      c =  3
      >> a=1; b=2; c=3
      c =  3
      >> a=1; b=2; c=3;
    \end{lstlisting}
 

    \subsection{Control Statements: for, while, if statement}
    \begin{lstlisting}
      >> v = zeros(10,1)
      v =
      
      0
      0
      0
      0
      0
      0
      0
      0
      0
      0
      
      >> for i=1:10, 
      >     v(i)=2^i;   % whitespace actually doesn't matter
      >   end;
      >> v
      v =

      2
      4
      8
     16
     32
     64
    128
    256
    512
   1024
   
   >> indicies=1:10;
   >> for i=indicies, 
   > disp(i)
   > end;
   1
   2
   3
   4
   5
   6
   7
   8
   9
   10

   >> i = 1; 
   >> while i <=5, 
   > v(i) = 100; 
   > i=i+1;
   > end;
   >> v
   v =
   
   100
   100
   100
   100
   100
   64
   128
   256
   512
   1024

   >> i=1; 
   >> while true, 
   >     v(i) = 999; 
   >     i= i+1; 
   >     if i==6; 
   >       break;
   >     end;  %end's if
   >  end;  %end's while...
   >> v
   v =

    999
    999
    999
    999
    999
     64
    128
    256
    512
   1024

   >> v(1) = 2;
   >> if v(1) ==1, 
   >      disp('The value is one'); 
   >    elseif v(1) ==2, 
   >      disp('The value is two');
   >    else
   >      disp('The value is not one or two');
   >  end;
    The value is two

   >> cd /cos_pc19a_npr/programs/Octave 
   >> pwd
   ans = /cos_pc19a_npr/programs/Octave

   >> squareThisNumber(5)      %  squareThisNumber is a function...
   ans =  25
   >> addpath('/cos_pc19a_npr/programs/Octave')    % Octave search path (advanced/optical) 
   >> cd ~/Desktop
   >> pwd
   ans = /Users/npr1/Desktop
   >> squareThisNumber(52)  % still works fine...
   ans =  2704
   >>  [a,b] = squareAndCubeThisNumber(52)
   a =  2704
   b =  140608
   >>
   
   >> X = [1 1; 1 2; 1 3] 
   >> y  = [1; 2; 3]            % case-sensitive!!
   >> theta = [0; 1]
   >> j = costFunctionJ(X,y,theta)    % costFunctionJ  in the Octave directory...
   j = 0
   >> 
    \end{lstlisting}
    Can also use {\tt break} and {\tt continue} in Octave. 



    \subsection{Vectorization}

    Here's our usual hypothesis for linear regression, 
    \begin{eqnarray}
      h_{\theta}(x) & = & \sum_{j=0}^{n} \theta_{j} x_{j} \\
                           & = & \theta^{T} x
    \end{eqnarray}
    and if you want to compute h(x), notice that there's a sum on the
    right. And so one thing you could do is, compute the sum from $j = 0$ to
    $j = n$ yourself. 
    
    Unvectorized implementation:
    \begin{lstlisting}
      prediction = 0.0 
      for j = 1:n+1
           prediction = prediction + theta(j) * x(j)
      end; 
    \end{lstlisting}

    Vectorized implementation:\\
    {\tt prediction = theta' * x ; }

    
    Just to remind you, here's our update rule for a gradient descent
    of a linear regression:
    \begin{equation}
      \theta_{j} :=   \theta_{j} - \alpha \frac{1}{m}\sum_{i=1}^{m} (h_{\theta} (x^{(i)}) - y^{(i)}) \   x^{(i)}_{j}
    \end{equation}
    
    Vectorized implementation:
    $\theta :=  \theta - \alpha \delta$, 
    where $\theta$ is vector, $\alpha$ is a Real number, 
    $\delta$ is a vector and $\delta = \frac{1}{m}\sum_{i=1}^{m} (h_{\theta} (x^{(i)}) - y^{(i)}) \   x^{(i)}$ and $x^{(i)}$ is a vector.
    \begin{lstlisting}
     
    \end{lstlisting}


\subsection{Octave/Matlab Quiz}
1. Suppose I first execute the following in Octave/Matlab:
A = [1 2; 3 4; 5 6];\\
B = [1 2 3; 4 5 6];\\
Which of the following are then valid commands? Check all that apply. (Hint: A' denotes the transpose of A.)

$\surd$ C = A' + B;\\
$\surd$ C = B * A;\\
$\times$ C = A + B;\\
$\times$ C = B' * A;\\



3. Let A be a 10x10 matrix and x be a 10-element vector. Your friend wants to compute the product Ax and writes the following code:
v = zeros(10, 1);\\
for i = 1:10\\
  for j = 1:10\\
    v(i) = v(i) + A(i, j) * x(j);\\
  end\\
end\\
How would you vectorize this code to run without any FOR loops? Check all that apply.

$\surd$ v = A * x;\\

$\times$ v = Ax;\\

$\times$ v = A .* x;\\

$\times$ v = sum (A * x);\\



4. Say you have two column vectors v and w, each with 7 elements (i.e., they have dimensions 7x1). Consider the following code:

z = 0;\\
for i = 1:7\\
  z = z + v(i) * w(i)\\
end\\

Which of the following vectorizations correctly compute z? Check all that apply.

$\surd$ z = sum (v .* w);\\
$\surd$ z = w' * v;\\
$\times$ z = v * w';\\
$\times$ z = w * v';\\


5. In Octave/Matlab, many functions work on single numbers, vectors, and matrices. For example, the sin function when applied to a matrix will return a new matrix with the sin of each element. But you have to be careful, as certain functions have different behavior. Suppose you have an 7x7 matrix X. You want to compute the log of every element, the square of every element, add 1 to every element, and divide every element by 4. You will store the results in four matrices, A,B,C,D. One way to do so is the following code:

for i = 1:7
  for j = 1:7
    A(i, j) = log(X(i, j));
    B(i, j) = X(i, j) \^ 2;
    C(i, j) = X(i, j) + 1;
    D(i, j) = X(i, j) / 4;
  end
end

Which of the following correctly compute A,B,C, or D? Check all that apply.


C = X + 1;

D = X / 4;

A = log (X);

B = X \^ 2;

Still not sure on this last one ??!!!




\section{Programming Assignment: Linear Regression}

\subsection*{2.2 Gradient Descent}
In this part, you will fit the linear regression parameters θ to our
dataset using gradient descent.

\subsubsection*{2.2.1 Update Equations}
The objective of linear regression is to minimize the cost function
\begin{equation}
J(\theta) = \frac{1}{2m} \sum_{i=1}^{m} (h_{\theta} (x^{(i)}) - y^{(i)})^{2}
\end{equation}
where the hypothesis $h_{\theta}(x)$ is given by the linear model
\begin{equation}
h_{\theta}(x) = \theta^{T} x = \theta_{0} + \theta_{1} x_{1}
\end{equation}
Recall that the parameters of your model are the $\theta_{j}$
values. These are the values you will adjust to minimize cost
$J(\theta)$. One way to do this is to use the batch gradient descent
algorithm. In batch gradient descent, each iteration performs the
update
\begin{equation}
 \theta_{j} :=   \theta_{j} - \alpha \frac{1}{m}\sum_{i=1}^{m} (h_{\theta} (x^{(i)}) - y^{(i)}) \   x^{(i)}_{j}
\end{equation}
(simultaneously update $\theta_{j}$ for all $j$).


























\newpage
%%%%%%%%%%%%%%%%%%%%%%%%%%%%%%%%%%%%%%%%%%%%%%%%%%%%%%%
%%%
%%%    W E E K     T H R E E 
%%%
%%%%%%%%%%%%%%%%%%%%%%%%%%%%%%%%%%%%%%%%%%%%%%%%%%%%%%%
\section*{WEEK   THREE}

\section{Logistic Regression}
    \subsection{Classification and Representation}
    \subsection{Logistic Regression Model}
    \subsection{Multiclass Classification}




\section{Regularization}
    \subsection{Solving the Problem of Overfitting}
    \subsubsection{The Problem of Overfitting}



\newpage
%%%%%%%%%%%%%%%%%%%%%%%%%%%%%%%%%%%%%%%%%%%%%%%%%%%%%%%
%%%
%%%    W E E K     F O U R 
%%%
%%%%%%%%%%%%%%%%%%%%%%%%%%%%%%%%%%%%%%%%%%%%%%%%%%%%%%%
\section*{WEEK   FOUR}

\section{Neural Networks: Representation}
    \subsection{Motivations}


    \subsection{Neural Networks}


    \subsection{Applications}




\newpage
%%%%%%%%%%%%%%%%%%%%%%%%%%%%%%%%%%%%%%%%%%%%%%%%%%%%%%%
%%%
%%%    W E E K     F I V E
%%%
%%%%%%%%%%%%%%%%%%%%%%%%%%%%%%%%%%%%%%%%%%%%%%%%%%%%%%%
\section{Neural Networks: Learning}

    \subsection{Cost Function and Backpropagation}


    \subsection{Backpropagation in Practice}


    \subsection{Application of Neural Networks}
    \subsubsection{Autonomous Driving}









\newpage
%%%%%%%%%%%%%%%%%%%%%%%%%%%%%%%%%%%%%%%%%%%%%%%%%%%%%%%
%%%
%%%    W E E K    S I X
%%%
%%%%%%%%%%%%%%%%%%%%%%%%%%%%%%%%%%%%%%%%%%%%%%%%%%%%%%%
\section{Advice for Applying Machine Learning}
    \subsection{Evaluating a Learning Algorithm}
    \subsection{Bias vs. Variance}

\section{Machine Learning System Design}
    \subsection{Building a Spam Classifier}
    \subsection{Handling Skewed Data}
    \subsection{Using Large Data Sets}







\newpage
%%%%%%%%%%%%%%%%%%%%%%%%%%%%%%%%%%%%%%%%%%%%%%%%%%%%%%%
%%%
%%%    W E E K    S E V E N
%%%
%%%%%%%%%%%%%%%%%%%%%%%%%%%%%%%%%%%%%%%%%%%%%%%%%%%%%%%
\section{Support Vector Machines}











































\citet{Ross15}

\bibliographystyle{mn2e}
\bibliography{/cos_pc19a_npr/LaTeX/tester_mnras}


\end{document}
