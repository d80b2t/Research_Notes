\documentclass[11pt]{article}
\setlength {\textwidth}{180mm} 
\setlength {\textheight}{260mm}
\topmargin=-35.00mm
\oddsidemargin=-10.00mm
\pagestyle{empty}


\usepackage[toc,page]{appendix}
\usepackage{amsmath, amssymb}
\usepackage{bm}% bold math
\usepackage{cancel, caption}
\usepackage{dcolumn}% Align table columns on decimal point
\usepackage{epsfig, epsf}
\usepackage{graphicx,fancyhdr,natbib,subfigure}
\usepackage{lscape, longtable}
\usepackage{hyperref,ifthen}
\usepackage{verbatim}
\usepackage{color}
\usepackage[usenames,dvipsnames]{xcolor}
\usepackage{listings}
%% http://en.wikibooks.org/wiki/LaTeX/Colors



%%%%%%%%%%%%%%%%%%%%%%%%%%%%%%%%%%%%%%%%%%%
%       define Journal abbreviations      %
%%%%%%%%%%%%%%%%%%%%%%%%%%%%%%%%%%%%%%%%%%%
\def\nat{Nat} \def\apjl{ApJ~Lett.} \def\apj{ApJ}
\def\apjs{ApJS} \def\aj{AJ} \def\mnras{MNRAS}
\def\prd{Phys.~Rev.~D} \def\prl{Phys.~Rev.~Lett.}
\def\plb{Phys.~Lett.~B} \def\jhep{JHEP} \def\nar{NewAR}
\def\npbps{NUC.~Phys.~B~Proc.~Suppl.} \def\prep{Phys.~Rep.}
\def\pasp{PASP} \def\aap{Astron.~\&~Astrophys.} \def\araa{ARA\&A}
\def\jcap{\ref@jnl{J. Cosmology Astropart. Phys.}}%
\def\physrep{Phys.~Rep.}

\newcommand{\preep}[1]{{\tt #1} }

%%%%%%%%%%%%%%%%%%%%%%%%%%%%%%%%%%%%%%%%%%%%%%%%%%%%%
%              define symbols                       %
%%%%%%%%%%%%%%%%%%%%%%%%%%%%%%%%%%%%%%%%%%%%%%%%%%%%%
\def \Mpc {~{\rm Mpc} }
\def \Om {\Omega_0}
\def \Omb {\Omega_{\rm b}}
\def \Omcdm {\Omega_{\rm CDM}}
\def \Omlam {\Omega_{\Lambda}}
\def \Omm {\Omega_{\rm m}}
\def \ho {H_0}
\def \qo {q_0}
\def \lo {\lambda_0}
\def \kms {{\rm ~km~s}^{-1}}
\def \kmsmpc {{\rm ~km~s}^{-1}~{\rm Mpc}^{-1}}
\def \hmpc{~\;h^{-1}~{\rm Mpc}} 
\def \hkpc{\;h^{-1}{\rm kpc}} 
\def \hmpcb{h^{-1}{\rm Mpc}}
\def \dif {{\rm d}}
\def \mlim {m_{\rm l}}
\def \bj {b_{\rm J}}
\def \mb {M_{\rm b_{\rm J}}}
\def \mg {M_{\rm g}}
\def \qso {_{\rm QSO}}
\def \lrg {_{\rm LRG}}
\def \gal {_{\rm gal}}
\def \xibar {\bar{\xi}}
\def \xis{\xi(s)}
\def \xisp{\xi(\sigma, \pi)}
\def \Xisig{\Xi(\sigma)}
\def \xir{\xi(r)}
\def \max {_{\rm max}}
\def \gsim { \lower .75ex \hbox{$\sim$} \llap{\raise .27ex \hbox{$>$}} }
\def \lsim { \lower .75ex \hbox{$\sim$} \llap{\raise .27ex \hbox{$<$}} }
\def \deg {^{\circ}}
%\def \sqdeg {\rm deg^{-2}}
\def \deltac {\delta_{\rm c}}
\def \mmin {M_{\rm min}}
\def \mbh  {M_{\rm BH}}
\def \mdh  {M_{\rm DH}}
\def \msun {M_{\odot}}
\def \z {_{\rm z}}
\def \edd {_{\rm Edd}}
\def \lin {_{\rm lin}}
\def \nonlin {_{\rm non-lin}}
\def \wrms {\langle w_{\rm z}^2\rangle^{1/2}}
\def \dc {\delta_{\rm c}}
\def \wp {w_{p}(\sigma)}
\def \PwrSp {\mathcal{P}(k)}
\def \DelSq {$\Delta^{2}(k)$}
\def \WMAP {{\it WMAP \,}}
\def \cobe {{\it COBE }}
\def \COBE {{\it COBE \;}}
\def \HST  {{\it HST \,\,}}
\def \Spitzer  {{\it Spitzer \,}}
\def \ATLAS {VST-AA$\Omega$ {\it ATLAS} }
\def \BEST   {{\tt best} }
\def \TARGET {{\tt target} }
\def \TQSO   {{\tt TARGET\_QSO}}
\def \HIZ    {{\tt TARGET\_HIZ}}
\def \FIRST  {{\tt TARGET\_FIRST}}
\def \zc {z_{\rm c}}
\def \zcz {z_{\rm c,0}}

\newcommand{\ltsim}{\raisebox{-0.6ex}{$\,\stackrel
        {\raisebox{-.2ex}{$\textstyle <$}}{\sim}\,$}}
\newcommand{\gtsim}{\raisebox{-0.6ex}{$\,\stackrel
        {\raisebox{-.2ex}{$\textstyle >$}}{\sim}\,$}}
\newcommand{\simlt}{\raisebox{-0.6ex}{$\,\stackrel
        {\raisebox{-.2ex}{$\textstyle <$}}{\sim}\,$}}
\newcommand{\simgt}{\raisebox{-0.6ex}{$\,\stackrel
        {\raisebox{-.2ex}{$\textstyle >$}}{\sim}\,$}}

\newcommand{\Msun}{M_\odot}
\newcommand{\Lsun}{L_\odot}
\newcommand{\lsun}{L_\odot}
\newcommand{\Mdot}{\dot M}

\newcommand{\sqdeg}{deg$^{-2}$}
\newcommand{\lya}{Ly$\alpha$\ }
%\newcommand{\lya}{Ly\,$\alpha$\ }
\newcommand{\lyaf}{Ly\,$\alpha$\ forest}
%\newcommand{\eg}{e.g.~}
%\newcommand{\etal}{et~al.~}
\newcommand{\lyb}{Ly$\beta$\ }
\newcommand{\cii}{C\,{\sc ii}\ }
\newcommand{\ciii}{C\,{\sc iii}]\ }
\newcommand{\civ}{C\,{\sc iv}\ }
\newcommand{\SiIV}{Si\,{\sc iv}\ }
\newcommand{\mgii}{Mg\,{\sc ii}\ }
\newcommand{\feii}{Fe\,{\sc ii}\ }
\newcommand{\feiii}{Fe\,{\sc iii}\ }
\newcommand{\caii}{Ca\,{\sc ii}\ }
\newcommand{\halpha}{H\,$\alpha$\ }
\newcommand{\hbeta}{H\,$\beta$\ }
\newcommand{\hgamma}{H\,$\gamma$\ }
\newcommand{\hdelta}{H\,$\delta$\ }
\newcommand{\oi}{[O\,{\sc i}]\ }
\newcommand{\oii}{[O\,{\sc ii}]\ }
\newcommand{\oiii}{[O\,{\sc iii}]\ }
\newcommand{\heii}{[He\,{\sc ii}]\ }
\newcommand{\nv}{N\,{\sc v}\ }
\newcommand{\nev}{Ne\,{\sc v}\ }
\newcommand{\neiii}{[Ne\,{\sc iii}]\ }
\newcommand{\aliii}{Al\,{\sc iii}\ }
\newcommand{\siiii}{Si\,{\sc iii}]\ }


%%%%%%%%%%%%%%%%%%%%%%%%%%%%%%%%%%%%%%%%%%%%%%%%%%%%%
%              define Listings                       %
%%%%%%%%%%%%%%%%%%%%%%%%%%%%%%%%%%%%%%%%%%%%%%%%%%%%%
\definecolor{dkgreen}{rgb}{0,0.6,0}
\definecolor{gray}{rgb}{0.5,0.5,0.5}
\definecolor{mauve}{rgb}{0.58,0,0.82}

\lstset{frame=tb,
  language=Python,
  aboveskip=3mm,
  belowskip=3mm,
  showstringspaces=false,
  columns=flexible,
  basicstyle={\small\ttfamily},
  numbers=none,
  numberstyle=\tiny\color{gray},
  keywordstyle=\color{blue},
  commentstyle=\color{dkgreen},
  stringstyle=\color{mauve},
  breaklines=true,
  breakatwhitespace=true,
  tabsize=3
}

\begin{document}

\title{Machine Learning: A Very General Guide}
\author{Nicholas P. Ross}
\date{\today}
\maketitle


\begin{abstract}
This is a simple document which will make some v. general notes on 
things connected to ``Machine Learning''. This document can be found at:\\
\href{https://github.com/d80b2t/Research_Notes/tree/master/MachineLearning}{\tt https://github.com/d80b2t/Research\_Notes/tree/master/MachineLearning}. 
\end{abstract}




\section{Introduction}
From Wikipedia,  
\href{https://en.wikipedia.org/w/index.php?title=Machine_learning&oldid=753193210}{retrieved, Mon Dec  5 16:28:49 PST 2016}: \\

Machine learning is the subfield of computer science that "gives
computers the ability to learn without being explicitly programmed"
(Arthur Samuel, 1959). Evolved from the study of pattern recognition
and computational learning theory in artificial intelligence, machine
learning explores the study and construction of algorithms that can
learn from and make predictions on data – such algorithms overcome
following strictly static program instructions by making data driven
predictions or decisions, through building a model from sample
inputs. Machine learning is employed in a range of computing tasks
where designing and programming explicit algorithms is unfeasible;
example applications include spam filtering, detection of network
intruders or malicious insiders working towards a data breach, optical
character recognition (OCR), search engines and computer vision.\\


\smallskip \smallskip
\noindent
Broadly, there are 3 types of Machine Learning Algorithms:

\noindent
1. Supervised Learning\\
2. Unsupervised Learning\\
3. Reinforcement Learning\\

\footnote{\href{https://en.wikipedia.org/wiki/Supervised_learning}{\tt
https://en.wikipedia.org/wiki/Supervised\_learning}} {\it Supervised
learning} is the machine learning task of inferring a function from
labeled training data.[1] The training data consist of a set of
training examples. In supervised learning, each example is a pair
consisting of an input object (typically a vector) and a desired
output value (also called the supervisory signal). A supervised
learning algorithm analyzes the training data and produces an inferred
function, which can be used for mapping new examples. An optimal
scenario will allow for the algorithm to correctly determine the class
labels for unseen instances. This requires the learning algorithm to
generalize from the training data to unseen situations in a
"reasonable" way (see inductive bias).

The parallel task in human and animal psychology is often referred to
as concept learning.\\


\footnote{\href{https://en.wikipedia.org/wiki/Unsupervised_learning}{\tt
https://en.wikipedia.org/wiki/Unsupervised\_learning}}{\it
Unsupervised machine} learning is the machine learning task of
inferring a function to describe hidden structure from unlabeled
data. Since the examples given to the learner are unlabeled, there is
no error or reward signal to evaluate a potential solution – this
distinguishes unsupervised learning from supervised learning and
reinforcement learning.

Unsupervised learning is closely related to the problem of density
estimation in statistics.[1] However, unsupervised learning also
encompasses many other techniques that seek to summarize and explain
key features of the data.  Approaches to unsupervised learning
include: clustering (k-means; mixture models; hierarchical
clustering); anomaly detection; Neural Networks. \\


\footnote{\href{https://en.wikipedia.org/wiki/Reinforcement_learning}{\tt
https://en.wikipedia.org/wiki/Reinforcement\_learning}}{\it
Reinforcement learning} is an area of machine learning inspired by
behaviorist psychology, concerned with how software agents ought to
take actions in an environment so as to maximize some notion of
cumulative reward. The problem, due to its generality, is studied in
many other disciplines, such as game theory, control theory,
operations research, information theory, simulation-based
optimization, multi-agent systems, swarm intelligence, statistics, and
genetic algorithms. In the operations research and control literature,
the field where reinforcement learning methods are studied is called
approximate dynamic programming. The problem has been studied in the
theory of optimal control, though most studies are concerned with the
existence of optimal solutions and their characterization, and not
with the learning or approximation aspects. In economics and game
theory, reinforcement learning may be used to explain how equilibrium
may arise under bounded rationality.

In machine learning, the environment is typically formulated as a
Markov decision process (MDP) as many reinforcement learning
algorithms for this context utilize dynamic programming
techniques. The main difference between the classical techniques and
reinforcement learning algorithms is that the latter do not need
knowledge about the MDP and they target large MDPs where exact methods
become infeasible.


\subsection*{``The 10 Algorithms Machine Learning Engineers Need to Know''}

\smallskip \smallskip
\noindent
\href{http://www.kdnuggets.com/2016/08/10-algorithms-machine-learning-engineers.html}{{\tt http://www.kdnuggets.com/2016/08/10-algorithms-machine-learning-engineers.html}}

Supervised Learning:\\
\begin{itemize}
\item{Decision Trees}
\item{Naïve Bayes Classification}
\item{}
\item{}
\item{}
\item{}
\end{itemize}

Unsupervised Learning:\\
\begin{itemize}
\item{Clustering Algorithms:}
\item{Principal Component Analysis:}
\item{Singular Value Decomposition:}
\item{Independent Component Analysis:}
\end{itemize}



And techniques/methods of ML:
\begin{itemize}
\item{Linear Regression}
\item{K-means} 
\item{Decision Trees}
\item{Random Forest}
\item{PCA}
\item{SVM: 
\href{http://www.kdnuggets.com/2016/09/support-vector-machines-concise-technical-overview.html}{\tt Concise technical overview}. } 
\item{Artificial Neural Networks (ANN)}
\end{itemize}

With links from
\href{https://www.linkedin.com/pulse/machine-learning-algorithms-concise-technical-overview-matthew-mayo}{Machine
Learning Algorithms: A Concise Technical Overview, Matthew Mayo.}


\section{Deep Learning}


































\section{Useful URLs and References}
\href{http://www.zdnet.com/topic/how-to-implement-ai-and-machine-learning/}{http://www.zdnet.com/topic/how-to-implement-ai-and-machine-learning/}\\















\bibliographystyle{mn2e}
\bibliography{/cos_pc19a_npr/LaTeX/tester_mnras}

\end{document}

