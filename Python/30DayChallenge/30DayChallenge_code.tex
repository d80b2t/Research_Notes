\documentclass[11pt,a4paper]{article}


\usepackage[toc,page]{appendix}
\usepackage{amsmath, amssymb}
\usepackage{bm}% bold math
\usepackage{cancel, caption}
\usepackage{dcolumn}% Align table columns on decimal point
\usepackage{epsfig, epsf}
\usepackage{graphicx,fancyhdr,natbib,subfigure}
\usepackage{lscape, longtable}
\usepackage{hyperref,ifthen}
\usepackage{verbatim}
\usepackage{color}
\usepackage[usenames,dvipsnames]{xcolor}
\usepackage{listings}
%% http://en.wikibooks.org/wiki/LaTeX/Colors



%%%%%%%%%%%%%%%%%%%%%%%%%%%%%%%%%%%%%%%%%%%
%       define Journal abbreviations      %
%%%%%%%%%%%%%%%%%%%%%%%%%%%%%%%%%%%%%%%%%%%
\def\nat{Nat} \def\apjl{ApJ~Lett.} \def\apj{ApJ}
\def\apjs{ApJS} \def\aj{AJ} \def\mnras{MNRAS}
\def\prd{Phys.~Rev.~D} \def\prl{Phys.~Rev.~Lett.}
\def\plb{Phys.~Lett.~B} \def\jhep{JHEP} \def\nar{NewAR}
\def\npbps{NUC.~Phys.~B~Proc.~Suppl.} \def\prep{Phys.~Rep.}
\def\pasp{PASP} \def\aap{Astron.~\&~Astrophys.} \def\araa{ARA\&A}
\def\jcap{\ref@jnl{J. Cosmology Astropart. Phys.}}%
\def\physrep{Phys.~Rep.}

\newcommand{\preep}[1]{{\tt #1} }

%%%%%%%%%%%%%%%%%%%%%%%%%%%%%%%%%%%%%%%%%%%%%%%%%%%%%
%              define symbols                       %
%%%%%%%%%%%%%%%%%%%%%%%%%%%%%%%%%%%%%%%%%%%%%%%%%%%%%
\def \Mpc {~{\rm Mpc} }
\def \Om {\Omega_0}
\def \Omb {\Omega_{\rm b}}
\def \Omcdm {\Omega_{\rm CDM}}
\def \Omlam {\Omega_{\Lambda}}
\def \Omm {\Omega_{\rm m}}
\def \ho {H_0}
\def \qo {q_0}
\def \lo {\lambda_0}
\def \kms {{\rm ~km~s}^{-1}}
\def \kmsmpc {{\rm ~km~s}^{-1}~{\rm Mpc}^{-1}}
\def \hmpc{~\;h^{-1}~{\rm Mpc}} 
\def \hkpc{\;h^{-1}{\rm kpc}} 
\def \hmpcb{h^{-1}{\rm Mpc}}
\def \dif {{\rm d}}
\def \mlim {m_{\rm l}}
\def \bj {b_{\rm J}}
\def \mb {M_{\rm b_{\rm J}}}
\def \mg {M_{\rm g}}
\def \qso {_{\rm QSO}}
\def \lrg {_{\rm LRG}}
\def \gal {_{\rm gal}}
\def \xibar {\bar{\xi}}
\def \xis{\xi(s)}
\def \xisp{\xi(\sigma, \pi)}
\def \Xisig{\Xi(\sigma)}
\def \xir{\xi(r)}
\def \max {_{\rm max}}
\def \gsim { \lower .75ex \hbox{$\sim$} \llap{\raise .27ex \hbox{$>$}} }
\def \lsim { \lower .75ex \hbox{$\sim$} \llap{\raise .27ex \hbox{$<$}} }
\def \deg {^{\circ}}
%\def \sqdeg {\rm deg^{-2}}
\def \deltac {\delta_{\rm c}}
\def \mmin {M_{\rm min}}
\def \mbh  {M_{\rm BH}}
\def \mdh  {M_{\rm DH}}
\def \msun {M_{\odot}}
\def \z {_{\rm z}}
\def \edd {_{\rm Edd}}
\def \lin {_{\rm lin}}
\def \nonlin {_{\rm non-lin}}
\def \wrms {\langle w_{\rm z}^2\rangle^{1/2}}
\def \dc {\delta_{\rm c}}
\def \wp {w_{p}(\sigma)}
\def \PwrSp {\mathcal{P}(k)}
\def \DelSq {$\Delta^{2}(k)$}
\def \WMAP {{\it WMAP \,}}
\def \cobe {{\it COBE }}
\def \COBE {{\it COBE \;}}
\def \HST  {{\it HST \,\,}}
\def \Spitzer  {{\it Spitzer \,}}
\def \ATLAS {VST-AA$\Omega$ {\it ATLAS} }
\def \BEST   {{\tt best} }
\def \TARGET {{\tt target} }
\def \TQSO   {{\tt TARGET\_QSO}}
\def \HIZ    {{\tt TARGET\_HIZ}}
\def \FIRST  {{\tt TARGET\_FIRST}}
\def \zc {z_{\rm c}}
\def \zcz {z_{\rm c,0}}

\newcommand{\ltsim}{\raisebox{-0.6ex}{$\,\stackrel
        {\raisebox{-.2ex}{$\textstyle <$}}{\sim}\,$}}
\newcommand{\gtsim}{\raisebox{-0.6ex}{$\,\stackrel
        {\raisebox{-.2ex}{$\textstyle >$}}{\sim}\,$}}
\newcommand{\simlt}{\raisebox{-0.6ex}{$\,\stackrel
        {\raisebox{-.2ex}{$\textstyle <$}}{\sim}\,$}}
\newcommand{\simgt}{\raisebox{-0.6ex}{$\,\stackrel
        {\raisebox{-.2ex}{$\textstyle >$}}{\sim}\,$}}

\newcommand{\Msun}{M_\odot}
\newcommand{\Lsun}{L_\odot}
\newcommand{\lsun}{L_\odot}
\newcommand{\Mdot}{\dot M}

\newcommand{\sqdeg}{deg$^{-2}$}
\newcommand{\lya}{Ly$\alpha$\ }
%\newcommand{\lya}{Ly\,$\alpha$\ }
\newcommand{\lyaf}{Ly\,$\alpha$\ forest}
%\newcommand{\eg}{e.g.~}
%\newcommand{\etal}{et~al.~}
\newcommand{\lyb}{Ly$\beta$\ }
\newcommand{\cii}{C\,{\sc ii}\ }
\newcommand{\ciii}{C\,{\sc iii}]\ }
\newcommand{\civ}{C\,{\sc iv}\ }
\newcommand{\SiIV}{Si\,{\sc iv}\ }
\newcommand{\mgii}{Mg\,{\sc ii}\ }
\newcommand{\feii}{Fe\,{\sc ii}\ }
\newcommand{\feiii}{Fe\,{\sc iii}\ }
\newcommand{\caii}{Ca\,{\sc ii}\ }
\newcommand{\halpha}{H\,$\alpha$\ }
\newcommand{\hbeta}{H\,$\beta$\ }
\newcommand{\hgamma}{H\,$\gamma$\ }
\newcommand{\hdelta}{H\,$\delta$\ }
\newcommand{\oi}{[O\,{\sc i}]\ }
\newcommand{\oii}{[O\,{\sc ii}]\ }
\newcommand{\oiii}{[O\,{\sc iii}]\ }
\newcommand{\heii}{[He\,{\sc ii}]\ }
\newcommand{\nv}{N\,{\sc v}\ }
\newcommand{\nev}{Ne\,{\sc v}\ }
\newcommand{\neiii}{[Ne\,{\sc iii}]\ }
\newcommand{\aliii}{Al\,{\sc iii}\ }
\newcommand{\siiii}{Si\,{\sc iii}]\ }


%%%%%%%%%%%%%%%%%%%%%%%%%%%%%%%%%%%%%%%%%%%%%%%%%%%%%
%              define Listings                       %
%%%%%%%%%%%%%%%%%%%%%%%%%%%%%%%%%%%%%%%%%%%%%%%%%%%%%
\definecolor{dkgreen}{rgb}{0,0.6,0}
\definecolor{gray}{rgb}{0.5,0.5,0.5}
\definecolor{mauve}{rgb}{0.58,0,0.82}

\lstset{frame=tb,
  language=Python,
  aboveskip=3mm,
  belowskip=3mm,
  showstringspaces=false,
  columns=flexible,
  basicstyle={\small\ttfamily},
  numbers=none,
  numberstyle=\tiny\color{gray},
  keywordstyle=\color{blue},
  commentstyle=\color{dkgreen},
  stringstyle=\color{mauve},
  breaklines=true,
  breakatwhitespace=true,
  tabsize=3
}
\setcounter{section}{-1} 

\begin{document}


\title{HackerRank: 30 Days of Code Challenge}
\author{Nicholas P. Ross}
\date{\today}
\maketitle


%% Usually omit these for ApJ or MNRAS style files:


\begin{abstract}
The is my (NPR's) set of notes from the \href{www.hackerrank.com}{\tt
HackerRank}
\href{https://www.hackerrank.com/domains/tutorials/30-days-of-code}{
``30 Days of Code Challenge''}.  The general idea here is to help my
learn/brush up on my Python skillz.  The .tex and .pdf of thes notes
can be found at: \\
 \href{https://github.com/d80b2t/Research\_Notes}{\tt https://github.com/d80b2t/Research\_Notes}.
\end{abstract}


\newpage
\tableofcontents


\newpage
\section*{General Notes}
All in {\tt Python3}. 


\newpage
\section{Day 0: Hello, World.}
\begin{lstlisting}
  # Read a full line of input from stdin and save it to our dynamically typed variable, input_string.
  inputString = input()
  print (inputString)
\end{lstlisting}


\newpage
\section{Day 1: Data Types}
\begin{lstlisting}
i = 4
d = 4.0
s = 'HackerRank '

# Declare second integer, double, and String variables.
i2 = int(input())       # read int
d2 = float(input())   # read double 
s2 = input()            # read string

# print summed and concatenated values
print(i + i2)
print(d + d2)
print(s + s2)
\end{lstlisting}


\newpage
\section{Day 2: Operators}
\begin{lstlisting}
mealCost   = float(input()) 
tipPercent = int(input())   
taxPercent = int(input()) 

tip = mealCost * (tipPercent/100.)
tax = mealCost * (taxPercent/100.)

totalCost = mealCost + tip + tax
total = round(totalCost)

print ('The total meal cost is', int(total), 'dollars.')
\end{lstlisting}


\newpage
\section{Day 3: Intro to Conditional Statements}
\begin{lstlisting}
import sys

N = int(input().strip())
condition = 'Not Weird' 

if N % 2 !=  0:
    condition = 'Weird'
elif N % 2 ==  0 and (N >= 6 and N <= 20):
    condition = 'Weird'
else:
    condition = 'Not Weird' 

print(condition)        

\end{lstlisting}


\newpage
\section{Day 4: Class vs. Instance}
\begin{lstlisting}
'''
Objective:

In this challenge, we are going to learn about the difference between a class and an instance; because this is an Object Oriented concept, it is only enabled in certain languages. Check out the Tutorial tab for learning materials and an instructional video!

Task: 
Write a Person class with an instance variable, age, and a constructor that takes an integer, initialAge, as a parameter. The constructor must assign initialAge to age after confirming the argument passed as initialAge is not negative; if a negative argument is passed as initialAge, the constructor should set age to 0 and print Age is not valid, setting age to 0. In addition, you must write the following instance methods:

yearPasses() should increase the age  instance variable by 1.

amIOld() should perform the following conditional actions:
   If age< 13 , print You are young.
   If >= 13 and age < 18, print You are a teenager.
   Otherwise, print You are old.

To help you learn by example and complete this challenge, much of the code is provided for you, but you''ll be writing everything in the future. The code that creates each instance of your Person class is in the main method. Dont worry if you dont understand it all quite yet!
'''

class Person:
    def __init__(self,initialAge):
        # Add some more code to run some checks on initialAge
        self.age = 0
        if initialAge < 0:
            print ("Age is not valid, setting age to 0.")
        else:
            self.age = initialAge
            
    def amIOld(self):
        # Do some computations in here and print out the correct statement to the console
        if age < 13:
            print("You are young.")
        elif 13 <= age < 18:
            print("You are a teenager.")
        elif age >= 18:
            print("You are old.")
        
    def yearPasses(self):
        # Increment the age of the person in here
        global age   #NPR: don't quite undesrstand what global does here...
        age += 1

t = int(input())
for i in range(0, t):
    age = int(input())         
    p = Person(age)  
    p.amIOld()
    for j in range(0, 3):
        p.yearPasses()       
    p.amIOld()
    print("")
\end{lstlisting}


\newpage
\section{Day 5: Loops}
\begin{lstlisting}
'''
Objective: 
In this challenge, we are going to use loops to help us do some simple math. Check out the Tutorial tab to learn more.

Task 
Given an integer, , print its first  multiples. Each multiple  (where ) should be printed on a new line in the form: N x i = result.
'''
import sys

N = int(input().strip())

for ii in range(1, 11):
    print (N,'x', ii ,'=', N*ii)

\end{lstlisting}


\newpage
\section{Day 6: Let's Review}
\begin{lstlisting}
Task:
Given a string, S, of length N that is indexed from 0 to N-1, print its even-indexed and odd-indexed characters as  space-separated strings on a single line (see the Sample below for more detail).

Note: 0 is considered to be an even index.

Sample Input:
2
Hacker
Rank

Sample Output:
Hce akr
Rn ak
'''

for i in range(int(eval(input()))):
    s=eval(input())
    print((*["".join(s[::2]),"".join(s[1::2])]))

\end{lstlisting}


\newpage
\section{Day 7: Arrays}
\begin{lstlisting}
'''
Task: Given an array, A, of N integers, print A''s elements in reverse order as a single line of space-separated numbers. 

http://docs.scipy.org/doc/numpy/reference/routines.array-manipulation.html
http://www.scipy-lectures.org/intro/numpy/numpy.html
'''

import sys

n = int(input().strip())
arr = [int(arr_temp) for arr_temp in input().strip().split(' ')]

# print(arr[::-1])
print(" ".join(map(str, arr[::-1])))
\end{lstlisting}


\newpage
\section{Day 8: Dictionaries and Maps}
\begin{lstlisting}
'''
Objective:: Today, we are learning about Key-Value pair mappings using a Map or Dictionary data structure. Check out the Tutorial tab for learning materials and an instructional video!

Task:: Given N names and phone numbers, assemble a phone book that maps friends names to their respective phone numbers. You will then be given an unknown number of names to query your phone book for; for each name queried, print the associated entry from your phone book (in the form ) or  if there is no entry for .

Note: Your phone book should be a Dictionary/Map/HashMap data structure.

Sample Input:
3
sam 99912222
tom 11122222
harry 12299933
sam
edward
harry
'''

import sys 

# Read input and assemble phoneBook
n = int(input())
phoneBook = {}
for i in range(n):
    contact = input().split(' ')
    phoneBook[contact[0]] = contact[1]

# Process Queries
lines = sys.stdin.readlines()
for i in lines:
    name = i.strip()
    if name in phoneBook:
        print(name + '=' + str( phoneBook[name] ))
    else:
        print('Not found')
\end{lstlisting}


\newpage
\section{Day 9: Recursion}
\begin{lstlisting}
'''
Objective:: Today, we are learning and practicing an algorithmic concept called Recursion. Check out the Tutorial tab for learning materials and an instructional video!

Task:: Write a factorial function that takes a positive integer, N, as a parameter and prints the result of N! 

Note: If you fail to use recursion or fail to name your recursive function factorial or Factorial, you will get a score of 0.

Input Format: A single integer, N (the argument to pass to factorial).

'''

def factorial(n):
    if n == 0 or n == 1:
        return 1
    else:
        return n * factorial(n-1)

print((factorial(int(eval(input())))))

\end{lstlisting}


\newpage
\section{Day 10: Binary Numbers}
\begin{lstlisting}
'''
Task:: Given a base-10 integer, n, convert it to binary (base-2). Then find and print the base-1- integer denoting the maximum number of consecutive 1''s in n''s binary representation.

'''

bin(11111113)
# '0b101010011000101011001001'

bin(11111113)[2:]
# '101010011000101011001001'

bin(11111113)[2:].split()
# ['101010011000101011001001']

bin(11111113)[2:].split('0')
# ['1', '1', '1', '', '11', '', '', '1', '1', '11', '', '1', '', '1']

max(bin(11111113)[2:].split('0'))
# '11'

#len(max(bin(11111113)[2:].split('0')))

print(len(max(bin(int(input().strip()))[2:].split('0'))))
\end{lstlisting}


\newpage
\section{Day 11: 2D Arrays}
\begin{lstlisting}
'''
Context: Given a 6x6 2D Array, A:

1 1 1 0 0 0
0 1 0 0 0 0
1 1 1 0 0 0
0 0 0 0 0 0
0 0 0 0 0 0
0 0 0 0 0 0

We define an hourglass in A to be a subset of values with indices falling in this pattern in As graphical representation:

a b c
  d
e f g

There are 16 hourglasses in A, and an hourglass sum is the sum of an hourglass'' values.

Task::  Calculate the hourglass sum for every hourglass in A, then print the maximum hourglass sum.

Sample Input::
1 1 1 0 0 0
0 1 0 0 0 0
1 1 1 0 0 0
0 0 2 4 4 0
0 0 0 2 0 0
0 0 1 2 4 0

Sample Output::
19
'''

import sys

arr = []
for arr_i in range(6):
   arr_t = [int(arr_temp) for arr_temp in input().strip().split(' ')]
   arr.append(arr_t)

res = []
for x in range(0, 4):
    for y in range(0, 4):
        print(x,y)
        print('arr[x][y:y+3]',   arr[x][y:y+3])
        print('arr[x+1][y+1]',   arr[x+1][y+1])
        print('arr[x+2][y:y+3]', arr[x+2][y:y+3])
        s = sum(arr[x][y:y+3]) + arr[x+1][y+1] + sum(arr[x+2][y:y+3])
        res.append(s)

print(max(res))
\end{lstlisting}


\newpage
\section{Day 12: Inheritance}
\begin{lstlisting}
'''
Objective:: Today, we are delving into Inheritance. Check out the Tutorial tab for learning materials and an instructional video!

Task:: You are given two classes, Person and Student, where Person is the base class and Student is the derived class. Completed code for Person and a declaration for Student are provided for you in the editor. Observe that Student inherits all the properties of Person.

Complete the Student class by writing the following:
A Student class constructor, which has  parameters:
A string, firstName.
A string, lastName.
An integer, id.
An integer array (or vector) of test scores, scores.

A char calculate() method that calculates a Student object''s average and returns the grade character representative of their calculated average. 


Sample Input:: 
Heraldo Memelli 8135627
2
100 80

Sample Output::

 Name: Memelli, Heraldo
 ID: 8135627
 Grade: O

'''

class Person:
    def __init__(self, firstName, lastName, idNumber):
        self.firstName = firstName
	self.lastName = lastName
	self.idNumber = idNumber
    def printPerson(self):
        print("Name:", self.lastName + ",", self.firstName)
        print("ID:", self.idNumber)

class Student(Person):
    def __init__(self, firstName, lastName, idNumber, scores):
        Person.__init__(self, firstName, lastName, idNumber)
        self.testScores = scores
		
    def calculate(self):
        average = 0
        for i in self.testScores:
            average += i

        average = average / len(self.testScores)
		
        if(average >= 90):
            return 'O' # Outstanding
        elif(average >= 80):
            return 'E' # Exceeds Expectations
        elif(average >= 70):
            return 'A' # Acceptable
        elif(average >= 55):
            return 'P' # Poor
        elif(average >= 40):
            return 'D' # Dreadful
        else:
            return 'T' # Troll


line = input().split()
firstName = line[0]
lastName = line[1]
idNum = line[2]
numScores = int(input()) # not needed for Python
scores = list( map(int, input().split()) )
s = Student(firstName, lastName, idNum, scores)
s.printPerson()
print("Grade:", s.calculate())
\end{lstlisting}


\newpage
\section{Day 13: Abstract Classes}

\begin{lstlisting}
'''
Objective:: Today, we are taking what we learned yesterday about Inheritance and extending it to Abstract Classes. Because this is a very specific Object-Oriented concept, submissions are limited to the few languages that use this construct. Check out the Tutorial tab for learning materials and an instructional video!

Task:: Given a Book class and a Solution class, write a MyBook class that does the following:
-- Inherits from Book

-- Has a parameterized constructor taking these  parameters:
   -- string title
   -- string author
   -- int price
   
Implements the Book class abstract display() method so it prints these 3 lines:
 1. Title,  a space, and then the current instances title. 
 2. Author, a space, and then the current instances author. 
 3. Price,  a space, and then the current instances price.

Note: Because these classes are being written in the same file, you
must not use an access modifier (e.g.: ) when declaring MyBook or your
code will not execute.

Input Format:: You are not responsible for reading any input from stdin. The Solution class creates a Book object and calls the MyBook class constructor (passing it the necessary arguments). It then calls the display method on the Book object.

Output Format:: The void display() method should print and label the respective, title, author and price of the MyBook objects instance (with each value on its own line) like so:
Title:     $title
Author: $author
Price:     $price
Note: The $ is prepended to variable names to indicate they are placeholders for variables.

Sample Input::
The following input from stdin is handled by the locked stub code in your editor:
The Alchemist
Paulo Coelho
248
'''

from abc import ABCMeta, abstractmethod

class Book(object, metaclass=ABCMeta):
    def __init__(self,title,author):
        self.title=title
        self.author=author   
    @abstractmethod
    def display(): pass

class MyBook(Book):
    price = 0
    def __init__(self, title, author, price):
        super(Book, self).__init__()
        self.price = price 

    def display(self):
        print("Title: "+ title)
        print("Author: "+ author)
        print("Price: "+ str(price))

title    =input()
author   =input()
price    =int(input())
new_novel=MyBook(title,author,price)
new_novel.display()
\end{lstlisting}


\newpage
\section{Day 14: Scope}
\begin{lstlisting}
"""
Objective:: Today we're discussing scope. Check out the Tutorial tab for learning materials and an instructional video!

The absolute difference between two integers, a and b, is written as |a-b|. The maximum absolute difference between two integers in a set of positive integers, elements, is the largest absolute difference between any two integers in elements.

The Difference class is started for you in the editor. It has a private integer array (elements) for storing N non-negative integers, and a public integer (maximumDifference) for storing the maximum absolute difference.

Task:: Complete the Difference class by writing the following:
-- A class constructor that takes an array of integers as a parameter and saves it to the  instance variable.

-- A computeDifference method that finds the maximum absolute difference between any  numbers in  and stores it in the  instance variable.

Input Format:: You are not responsible for reading any input from stdin. The locked Solution class in your editor reads in 2 lines of input; the first line contains N, and the second line describes the elements array.

Sample Input::
3
1 2 5

Sample Output::
4
"""
class Difference:
    def __init__(self, a):
        self.__elements = a
# Add your code here
    def computeDifference(self):
        self.maximumDifference = abs(max(self.__elements) - min(self.__elements)) 
# End of Difference class

##
##  NOTE TO NPR: The indentation really had to be correct here!!
##  aka, you kinda really need to understand these Classes a bit better!!
##

"""
Your constructor must save the argument passed as its integer array parameter to the integer array instance variable (). 

The computeDifference method must then access the the integer array instance variable () and find its maximum and minimum elements. Once these are found, the method must save their absolute difference to the  instance variable. 

Note: The use of Math.abs is not really necessary. Because the problem constraints stipulate that we are only dealing with positive numbers,  will always be positive. 
"""


_ = input()
a = [int(e) for e in input().split(' ')]

d = Difference(a)
d.computeDifference()

print(d.maximumDifference)
\end{lstlisting}


\newpage
\section{Day 15: Linked List}
\begin{lstlisting}
"""
Objective:: Today we're working with Linked Lists. Check out the Tutorial tab for learning materials and an instructional video!

A Node class is provided for you in the editor. A Node object has an integer data field, data, and a Node instance pointer, next, pointing to another node (i.e.: the next node in a list).

A Node insert function is also declared in your editor. It has two parameters: a pointer, head, pointing to the first node of a linked list, and an integer data value that must be added to the end of the list as a new Node object.

Task:: Complete the insert function in your editor so that it creates a new Node (pass data as the Node constructor argument) and inserts it at the tail of the linked list referenced by the head parameter. Once the new node is added, return the reference to the head node.

Note: If the head  argument passed to the insert function is null, then the initial list is empty.

Input Format:: The insert function has 2 parameters: a pointer to a Node named head, and an integer value, data. The constructor for Node has 1 parameter: an integer value for the data field.

You do not need to read anything from stdin.

Output Format:: 
Your insert function should return a reference to the head node of the linked list.

Sample Input:: 
The following input is handled for you by the locked code in the editor: 
The first line contains T, the number of test cases. 
The T subsequent lines of test cases each contain an integer to be inserted at the list's tail.
4
2
3
4
1

Sample Output::
The locked code in your editor prints the ordered data values for each element in your list as a single line of space-separated integers:

2 3 4 1
"""

class Node:
    def __init__(self,data):
        self.data = data
        self.next = None 
class Solution: 
    def display(self,head):
        current = head
        while current:
            print(current.data,end=' ')
            current = current.next
            
    def insert(self,head,data):
        #Complete this method
        current = head
        if(current):
            while(current.next):
                current = current.next
            current.next = Node(data)
            return head
        else:
            return Node(data)
            
            
mylist= Solution()
T=int(input())
head=None
for i in range(T):
    data=int(input())
    head=mylist.insert(head,data)    
mylist.display(head); 	  
\end{lstlisting}


\newpage
\section{Day 16: Exceptions - String to Integer}
\begin{lstlisting}
#!/bin/python3
"""
Objective: Today, we're getting started with Exceptions by learning how to parse an integer from a string and print a custom error message. Check out the Tutorial tab for learning materials and an instructional video!

Task: Read a string, S, and print its integer value; if S cannot be converted to an integer, print Bad String.

Note: You must use the String-to-Integer and exception handling constructs built into your submission language. If you attempt to use loops/conditional statements, you will get a 0 score.

Input Format: A single string, S.
Output Format: Print the parsed integer value of , or Bad String if  cannot be converted to an integer.

Sample Input 0
3
Sample Output 0
3

Sample Input 1
za
Sample Output 1
Bad String
"""

import sys

S = input().strip()

try:
    print(int(S))
except:
    print("Bad String")
\end{lstlisting}


\newpage
\section{Day 17: More Exceptions}
\begin{lstlisting}
"""
Objective: Yesterday's challenge taught you to manage exceptional situations by using try and catch blocks. In today's challenge, you're going to practice throwing and propagating an exception. Check out the Tutorial tab for learning materials and an instructional video!

Task: Write a Calculator class with a single method: int power(int,int). The power method takes two integers, n and p, as parameters and returns the integer result of n^p. If either n or p is negative, then the method must throw an exception with the message: n and p should be non-negative.

Note: Do not use an access modifier (e.g.: public) in the declaration for your Calculator class.

Input Format: Input from stdin is handled for you by the locked stub code in your editor. The first line contains an integer, , the number of test cases. Each of the  subsequent lines describes a test case in  space-separated integers denoting  and , respectively.

Constraints: No Test Case will result in overflow for correctly written code.

Output Format: Output to stdout is handled for you by the locked stub code in your editor. There are  lines of output, where each line contains the result of  as calculated by your Calculator class' power method.

Sample Input:
4
3 5
2 4
-1 -2
-1 3

Sample Output
243
16
n and p should be non-negative
n and p should be non-negative
"""


class Calculator:
    def power(self,n,p):
        self.n = n
        self.p = p
        if n < 0 or p < 0:
            raise Exception("n and p should be non-negative")
        else:
            return n**p


myCalculator=Calculator()
T=int(input())
for i in range(T):
    n,p = map(int, input().split())
    try:
        ans=myCalculator.power(n,p)
        print(ans)
    except Exception as e:
        print(e) 
\end{lstlisting}


\newpage
\section{Day 18: Queues and Stacks}
\begin{lstlisting}
"""
Welcome to Day 18! Today we're learning about Stacks and Queues. Check out the Tutorial tab for learning materials and an instructional video!

A palindrome is a word, phrase, number, or other sequence of characters which reads the same backwards and forwards. Can you determine if a given string, s, is a palindrome?

To solve this challenge, we must first take each character in , enqueue it in a queue, and also push that same character onto a stack. Once that's done, we must dequeue the first character from the queue and pop the top character off the stack, then compare the two characters to see if they are the same; as long as the characters match, we continue dequeueing, popping, and comparing each character until our containers are empty (a non-match means  isn't a palindrome).

Write the following declarations and implementations:

1. Two instance variables: one for your stack, and one for your queue.
2. A void pushCharacter(char ch) method that pushes a character onto a stack.
3. A void enqueueCharacter(char ch) method that enqueues a character in the  instance variable.
4. A char popCharacter() method that pops and returns the character at the top of the  instance variable.
5. A char dequeueCharacter() method that dequeues and returns the first character in the  instance variable.

Input Format:: You do not need to read anything from stdin. The locked stub code in your editor reads a single line containing string . It then calls the methods specified above to pass each character to your instance variables.

Constraints: s is composed of lowercase English letters.

Output Format: You are not responsible for printing any output to stdout. 
If your code is correctly written and  is a palindrome, the locked stub code will print ; otherwise, it will print 

Sample Input:
racecar

Sample Output:
The word, racecar, is a palindrome
"""

import sys

class Solution:
    # Write your code here
    def __init__(self):
        self.mystack = list()
        self.myqueue = list()
        return(None)

    def pushCharacter(self, char):
        self.mystack.append(char)

    def popCharacter(self):
        return(self.mystack.pop(-1))
        ## pop(0) removes and returns the first entry from the list, pop(-1) removes and returns the last entry in the list.

    def enqueueCharacter(self, char):
        self.myqueue.append(char)

    def dequeueCharacter(self):
        return(self.myqueue.pop(0))

    
# read the string s
s=input()

#Create the Solution class object
obj=Solution()   

l=len(s)
# push/enqueue all the characters of string s to stack
for i in range(l):
    obj.pushCharacter(s[i])
    obj.enqueueCharacter(s[i])
    
isPalindrome=True
'''
pop the top character from stack
dequeue the first character from queue
compare both the characters
''' 
for i in range(l // 2):
    if obj.popCharacter()!=obj.dequeueCharacter():
        isPalindrome=False
        break
#finally print whether string s is palindrome or not.
if isPalindrome:
    print("The word, "+s+", is a palindrome.")
else:
    print("The word, "+s+", is not a palindrome.") 
\end{lstlisting}


\newpage
\section{Day 19: Interfaces}
\begin{lstlisting}
"""

Nota Bene: Day 19 was about Interfaces.

This was acutally not a Python solveable challenge (only C++, C#, Java 7, Java 8 and PHP) were offered, so no code was written here or progress made. 

"""
\end{lstlisting}


\newpage
\section{Day 20: Sorting}
\begin{lstlisting}
\end{lstlisting}

\newpage
\section{Day 21: Generics}
\begin{lstlisting}
\end{lstlisting}

\newpage
\section{Day 22: Binary Search Trees}
\begin{lstlisting}
\end{lstlisting}

\newpage
\section{Day 23: BST Level-Order Traversal}
\begin{lstlisting}
\end{lstlisting}

\newpage
\section{Day 24: More Linked Lists}
\begin{lstlisting}
\end{lstlisting}

\newpage
\section{Day 25: Running Time and Complexity}
\begin{lstlisting}
\end{lstlisting}

\newpage
\section{Day 26: Nested Logic}
\begin{lstlisting}
\end{lstlisting}

\newpage
\section{Day 27: Testing}
\begin{lstlisting}
\end{lstlisting}

\newpage
\section{Day 28: RegEx, Patterns, and Intro to Databases}
\begin{lstlisting}
\end{lstlisting}

\newpage
\section{Day 29: Bitwise AND}
\begin{lstlisting}
\end{lstlisting}





\end{document}