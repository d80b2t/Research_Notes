\documentclass[11pt,a4paper]{article}


\usepackage[toc,page]{appendix}
\usepackage{amsmath, amssymb}
\usepackage{bm}% bold math
\usepackage{cancel, caption}
\usepackage{dcolumn}% Align table columns on decimal point
\usepackage{epsfig, epsf}
\usepackage{graphicx,fancyhdr,natbib,subfigure}
\usepackage{lscape, longtable}
\usepackage{hyperref,ifthen}
\usepackage{verbatim}
\usepackage{color}
\usepackage[usenames,dvipsnames]{xcolor}
\usepackage{listings}
%% http://en.wikibooks.org/wiki/LaTeX/Colors



%%%%%%%%%%%%%%%%%%%%%%%%%%%%%%%%%%%%%%%%%%%
%       define Journal abbreviations      %
%%%%%%%%%%%%%%%%%%%%%%%%%%%%%%%%%%%%%%%%%%%
\def\nat{Nat} \def\apjl{ApJ~Lett.} \def\apj{ApJ}
\def\apjs{ApJS} \def\aj{AJ} \def\mnras{MNRAS}
\def\prd{Phys.~Rev.~D} \def\prl{Phys.~Rev.~Lett.}
\def\plb{Phys.~Lett.~B} \def\jhep{JHEP} \def\nar{NewAR}
\def\npbps{NUC.~Phys.~B~Proc.~Suppl.} \def\prep{Phys.~Rep.}
\def\pasp{PASP} \def\aap{Astron.~\&~Astrophys.} \def\araa{ARA\&A}
\def\jcap{\ref@jnl{J. Cosmology Astropart. Phys.}}%
\def\physrep{Phys.~Rep.}

\newcommand{\preep}[1]{{\tt #1} }

%%%%%%%%%%%%%%%%%%%%%%%%%%%%%%%%%%%%%%%%%%%%%%%%%%%%%
%              define symbols                       %
%%%%%%%%%%%%%%%%%%%%%%%%%%%%%%%%%%%%%%%%%%%%%%%%%%%%%
\def \Mpc {~{\rm Mpc} }
\def \Om {\Omega_0}
\def \Omb {\Omega_{\rm b}}
\def \Omcdm {\Omega_{\rm CDM}}
\def \Omlam {\Omega_{\Lambda}}
\def \Omm {\Omega_{\rm m}}
\def \ho {H_0}
\def \qo {q_0}
\def \lo {\lambda_0}
\def \kms {{\rm ~km~s}^{-1}}
\def \kmsmpc {{\rm ~km~s}^{-1}~{\rm Mpc}^{-1}}
\def \hmpc{~\;h^{-1}~{\rm Mpc}} 
\def \hkpc{\;h^{-1}{\rm kpc}} 
\def \hmpcb{h^{-1}{\rm Mpc}}
\def \dif {{\rm d}}
\def \mlim {m_{\rm l}}
\def \bj {b_{\rm J}}
\def \mb {M_{\rm b_{\rm J}}}
\def \mg {M_{\rm g}}
\def \qso {_{\rm QSO}}
\def \lrg {_{\rm LRG}}
\def \gal {_{\rm gal}}
\def \xibar {\bar{\xi}}
\def \xis{\xi(s)}
\def \xisp{\xi(\sigma, \pi)}
\def \Xisig{\Xi(\sigma)}
\def \xir{\xi(r)}
\def \max {_{\rm max}}
\def \gsim { \lower .75ex \hbox{$\sim$} \llap{\raise .27ex \hbox{$>$}} }
\def \lsim { \lower .75ex \hbox{$\sim$} \llap{\raise .27ex \hbox{$<$}} }
\def \deg {^{\circ}}
%\def \sqdeg {\rm deg^{-2}}
\def \deltac {\delta_{\rm c}}
\def \mmin {M_{\rm min}}
\def \mbh  {M_{\rm BH}}
\def \mdh  {M_{\rm DH}}
\def \msun {M_{\odot}}
\def \z {_{\rm z}}
\def \edd {_{\rm Edd}}
\def \lin {_{\rm lin}}
\def \nonlin {_{\rm non-lin}}
\def \wrms {\langle w_{\rm z}^2\rangle^{1/2}}
\def \dc {\delta_{\rm c}}
\def \wp {w_{p}(\sigma)}
\def \PwrSp {\mathcal{P}(k)}
\def \DelSq {$\Delta^{2}(k)$}
\def \WMAP {{\it WMAP \,}}
\def \cobe {{\it COBE }}
\def \COBE {{\it COBE \;}}
\def \HST  {{\it HST \,\,}}
\def \Spitzer  {{\it Spitzer \,}}
\def \ATLAS {VST-AA$\Omega$ {\it ATLAS} }
\def \BEST   {{\tt best} }
\def \TARGET {{\tt target} }
\def \TQSO   {{\tt TARGET\_QSO}}
\def \HIZ    {{\tt TARGET\_HIZ}}
\def \FIRST  {{\tt TARGET\_FIRST}}
\def \zc {z_{\rm c}}
\def \zcz {z_{\rm c,0}}

\newcommand{\ltsim}{\raisebox{-0.6ex}{$\,\stackrel
        {\raisebox{-.2ex}{$\textstyle <$}}{\sim}\,$}}
\newcommand{\gtsim}{\raisebox{-0.6ex}{$\,\stackrel
        {\raisebox{-.2ex}{$\textstyle >$}}{\sim}\,$}}
\newcommand{\simlt}{\raisebox{-0.6ex}{$\,\stackrel
        {\raisebox{-.2ex}{$\textstyle <$}}{\sim}\,$}}
\newcommand{\simgt}{\raisebox{-0.6ex}{$\,\stackrel
        {\raisebox{-.2ex}{$\textstyle >$}}{\sim}\,$}}

\newcommand{\Msun}{M_\odot}
\newcommand{\Lsun}{L_\odot}
\newcommand{\lsun}{L_\odot}
\newcommand{\Mdot}{\dot M}

\newcommand{\sqdeg}{deg$^{-2}$}
\newcommand{\lya}{Ly$\alpha$\ }
%\newcommand{\lya}{Ly\,$\alpha$\ }
\newcommand{\lyaf}{Ly\,$\alpha$\ forest}
%\newcommand{\eg}{e.g.~}
%\newcommand{\etal}{et~al.~}
\newcommand{\lyb}{Ly$\beta$\ }
\newcommand{\cii}{C\,{\sc ii}\ }
\newcommand{\ciii}{C\,{\sc iii}]\ }
\newcommand{\civ}{C\,{\sc iv}\ }
\newcommand{\SiIV}{Si\,{\sc iv}\ }
\newcommand{\mgii}{Mg\,{\sc ii}\ }
\newcommand{\feii}{Fe\,{\sc ii}\ }
\newcommand{\feiii}{Fe\,{\sc iii}\ }
\newcommand{\caii}{Ca\,{\sc ii}\ }
\newcommand{\halpha}{H\,$\alpha$\ }
\newcommand{\hbeta}{H\,$\beta$\ }
\newcommand{\hgamma}{H\,$\gamma$\ }
\newcommand{\hdelta}{H\,$\delta$\ }
\newcommand{\oi}{[O\,{\sc i}]\ }
\newcommand{\oii}{[O\,{\sc ii}]\ }
\newcommand{\oiii}{[O\,{\sc iii}]\ }
\newcommand{\heii}{[He\,{\sc ii}]\ }
\newcommand{\nv}{N\,{\sc v}\ }
\newcommand{\nev}{Ne\,{\sc v}\ }
\newcommand{\neiii}{[Ne\,{\sc iii}]\ }
\newcommand{\aliii}{Al\,{\sc iii}\ }
\newcommand{\siiii}{Si\,{\sc iii}]\ }


%%%%%%%%%%%%%%%%%%%%%%%%%%%%%%%%%%%%%%%%%%%%%%%%%%%%%
%              define Listings                       %
%%%%%%%%%%%%%%%%%%%%%%%%%%%%%%%%%%%%%%%%%%%%%%%%%%%%%
\definecolor{dkgreen}{rgb}{0,0.6,0}
\definecolor{gray}{rgb}{0.5,0.5,0.5}
\definecolor{mauve}{rgb}{0.58,0,0.82}

\lstset{frame=tb,
  language=Python,
  aboveskip=3mm,
  belowskip=3mm,
  showstringspaces=false,
  columns=flexible,
  basicstyle={\small\ttfamily},
  numbers=none,
  numberstyle=\tiny\color{gray},
  keywordstyle=\color{blue},
  commentstyle=\color{dkgreen},
  stringstyle=\color{mauve},
  breaklines=true,
  breakatwhitespace=true,
  tabsize=3
}
\setcounter{section}{-1} 


\begin{document}
\section*{General Notes}
All in {\tt Python3}. 

\newpage
\section{Day 0: Hello, World.}
\begin{lstlisting}
  # Read a full line of input from stdin and save it to our dynamically typed variable, input_string.
  inputString = input()
  print (inputString)
\end{lstlisting}

\newpage
\section{Day 1: Data Types}
\begin{lstlisting}
i = 4
d = 4.0
s = 'HackerRank '

# Declare second integer, double, and String variables.
i2 = int(input())       # read int
d2 = float(input())   # read double 
s2 = input()            # read string

# print summed and concatenated values
print(i + i2)
print(d + d2)
print(s + s2)
\end{lstlisting}

\newpage
\section{Day 2: Operators}
\begin{lstlisting}
mealCost   = float(input()) 
tipPercent = int(input())   
taxPercent = int(input()) 

tip = mealCost * (tipPercent/100.)
tax = mealCost * (taxPercent/100.)

totalCost = mealCost + tip + tax
total = round(totalCost)

print ('The total meal cost is', int(total), 'dollars.')
\end{lstlisting}


\newpage
\section{Day 3: Intro to Conditional Statements}
\begin{lstlisting}
import sys

N = int(input().strip())
condition = 'Not Weird' 

if N % 2 !=  0:
    condition = 'Weird'
elif N % 2 ==  0 and (N >= 6 and N <= 20):
    condition = 'Weird'
else:
    condition = 'Not Weird' 

print(condition)        

\end{lstlisting}


\newpage
\section{Day 4: Class vs. Instance}
\begin{lstlisting}
'''
Objective:

In this challenge, we are going to learn about the difference between a class and an instance; because this is an Object Oriented concept, it is only enabled in certain languages. Check out the Tutorial tab for learning materials and an instructional video!

Task: 
Write a Person class with an instance variable, age, and a constructor that takes an integer, initialAge, as a parameter. The constructor must assign initialAge to age after confirming the argument passed as initialAge is not negative; if a negative argument is passed as initialAge, the constructor should set age to 0 and print Age is not valid, setting age to 0. In addition, you must write the following instance methods:

yearPasses() should increase the age  instance variable by 1.

amIOld() should perform the following conditional actions:
   If age< 13 , print You are young.
   If >= 13 and age < 18, print You are a teenager.
   Otherwise, print You are old.

To help you learn by example and complete this challenge, much of the code is provided for you, but you''ll be writing everything in the future. The code that creates each instance of your Person class is in the main method. Dont worry if you dont understand it all quite yet!
'''

class Person:
    def __init__(self,initialAge):
        # Add some more code to run some checks on initialAge
        self.age = 0
        if initialAge < 0:
            print ("Age is not valid, setting age to 0.")
        else:
            self.age = initialAge
            
    def amIOld(self):
        # Do some computations in here and print out the correct statement to the console
        if age < 13:
            print("You are young.")
        elif 13 <= age < 18:
            print("You are a teenager.")
        elif age >= 18:
            print("You are old.")
        
    def yearPasses(self):
        # Increment the age of the person in here
        global age   #NPR: don't quite undesrstand what global does here...
        age += 1

t = int(input())
for i in range(0, t):
    age = int(input())         
    p = Person(age)  
    p.amIOld()
    for j in range(0, 3):
        p.yearPasses()       
    p.amIOld()
    print("")
\end{lstlisting}


\newpage
\section{Day 5: Loops}
\begin{lstlisting}
'''
Objective: 
In this challenge, we are going to use loops to help us do some simple math. Check out the Tutorial tab to learn more.

Task 
Given an integer, , print its first  multiples. Each multiple  (where ) should be printed on a new line in the form: N x i = result.
'''
import sys

N = int(input().strip())

for ii in range(1, 11):
    print (N,'x', ii ,'=', N*ii)

\end{lstlisting}


\newpage
\section{Day 6: Let's Review}
\begin{lstlisting}
Task:
Given a string, S, of length N that is indexed from 0 to N-1, print its even-indexed and odd-indexed characters as  space-separated strings on a single line (see the Sample below for more detail).

Note: 0 is considered to be an even index.

Sample Input:
2
Hacker
Rank

Sample Output:
Hce akr
Rn ak
'''

for i in range(int(eval(input()))):
    s=eval(input())
    print((*["".join(s[::2]),"".join(s[1::2])]))

\end{lstlisting}


\newpage
\section{Day 7: Arrays}
\begin{lstlisting}
'''
Task: Given an array, A, of N integers, print A''s elements in reverse order as a single line of space-separated numbers. 

http://docs.scipy.org/doc/numpy/reference/routines.array-manipulation.html
http://www.scipy-lectures.org/intro/numpy/numpy.html
'''

import sys

n = int(input().strip())
arr = [int(arr_temp) for arr_temp in input().strip().split(' ')]

# print(arr[::-1])
print(" ".join(map(str, arr[::-1])))

\end{lstlisting}




\newpage
\section{Day 8: Dictionaries and Maps}
\begin{lstlisting}
'''
Objective:: Today, we are learning about Key-Value pair mappings using a Map or Dictionary data structure. Check out the Tutorial tab for learning materials and an instructional video!

Task:: Given N names and phone numbers, assemble a phone book that maps friends names to their respective phone numbers. You will then be given an unknown number of names to query your phone book for; for each name queried, print the associated entry from your phone book (in the form ) or  if there is no entry for .

Note: Your phone book should be a Dictionary/Map/HashMap data structure.

Sample Input:
3
sam 99912222
tom 11122222
harry 12299933
sam
edward
harry
'''

import sys 

# Read input and assemble phoneBook
n = int(input())
phoneBook = {}
for i in range(n):
    contact = input().split(' ')
    phoneBook[contact[0]] = contact[1]

# Process Queries
lines = sys.stdin.readlines()
for i in lines:
    name = i.strip()
    if name in phoneBook:
        print(name + '=' + str( phoneBook[name] ))
    else:
        print('Not found')
\end{lstlisting}



\newpage
\section{Day 9: Recursion}
\begin{lstlisting}
'''
Objective:: Today, we are learning and practicing an algorithmic concept called Recursion. Check out the Tutorial tab for learning materials and an instructional video!

Task:: Write a factorial function that takes a positive integer, N, as a parameter and prints the result of N! 

Note: If you fail to use recursion or fail to name your recursive function factorial or Factorial, you will get a score of 0.

Input Format: A single integer, N (the argument to pass to factorial).

'''

def factorial(n):
    if n == 0 or n == 1:
        return 1
    else:
        return n * factorial(n-1)

print((factorial(int(eval(input())))))

\end{lstlisting}



























\newpage
\section{Day 10: Binary Numbers}
\begin{lstlisting}
\end{lstlisting}

\newpage
\section{Day 11: 2D Arrays}
\begin{lstlisting}
\end{lstlisting}

\newpage
\section{Day 12: Inheritance}
\begin{lstlisting}
\end{lstlisting}

\newpage
\section{Day 13: Abstract Classes}
\begin{lstlisting}
\end{lstlisting}

\newpage
\section{Day 14: Scope}
\begin{lstlisting}
\end{lstlisting}

\newpage
\section{Day 15: Linked List}
\begin{lstlisting}
\end{lstlisting}

\newpage
\section{Day 16: Exceptions - String to Integer}
\begin{lstlisting}
\end{lstlisting}

\newpage
\section{Day 17: More Exceptions}
\begin{lstlisting}
\end{lstlisting}

\newpage
\section{Day 18: Queues and Stacks}
\begin{lstlisting}
\end{lstlisting}

\newpage
\section{Day 19: Interfaces}
\begin{lstlisting}
\end{lstlisting}

\newpage
\section{Day 20: Sorting}
\begin{lstlisting}
\end{lstlisting}

\newpage
\section{Day 21: Generics}
\begin{lstlisting}
\end{lstlisting}

\newpage
\section{Day 22: Binary Search Trees}
\begin{lstlisting}
\end{lstlisting}

\newpage
\section{Day 23: BST Level-Order Traversal}
\begin{lstlisting}
\end{lstlisting}

\newpage
\section{Day 24: More Linked Lists}
\begin{lstlisting}
\end{lstlisting}

\newpage
\section{Day 25: Running Time and Complexity}
\begin{lstlisting}
\end{lstlisting}

\newpage
\section{Day 26: Nested Logic}
\begin{lstlisting}
\end{lstlisting}

\newpage
\section{Day 27: Testing}
\begin{lstlisting}
\end{lstlisting}

\newpage
\section{Day 28: RegEx, Patterns, and Intro to Databases}
\begin{lstlisting}
\end{lstlisting}

\newpage
\section{Day 29: Bitwise AND}
\begin{lstlisting}
\end{lstlisting}





\end{document}