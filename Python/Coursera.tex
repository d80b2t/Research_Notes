\documentclass[11pt]{article}
%\setlength {\textwidth}{180mm} 
%\setlength {\textheight}{260mm}
%\topmargin=-35.00mm
%\oddsidemargin=-10.00mm
%\pagestyle{empty}



\usepackage{amsmath, amssymb}
\usepackage{bm, booktabs}
\usepackage{cancel, caption}
\usepackage{dcolumn}  % Align table columns on decimal point
\usepackage{epsfig, epsf, enumitem}
\usepackage{fancyhdr}
\usepackage[T1]{fontenc}
\usepackage{graphicx, geometry}
\usepackage{hyperref}
\usepackage{ifthen}
\usepackage[utf8]{inputenc}
\usepackage{lscape, longtable}
\usepackage{multirow}
\usepackage{natbib}
\usepackage{pifont}
\usepackage{ragged2e}
\usepackage{subfigure}
\usepackage{sectsty}
\usepackage{times, tabularx}
\usepackage{tcolorbox}
\usepackage{verbatim}
%\usepackage[usenames,dvipsnames,svgnames,table]{xcolor}



%%%%%%%%%%%%%%%%%%%%%%%%%%%%%%%%%%%%%%%%%%%
%       define Journal abbreviations      %
%%%%%%%%%%%%%%%%%%%%%%%%%%%%%%%%%%%%%%%%%%%
\def\nat{Nat} \def\apjl{ApJ~Lett.} \def\apj{ApJ}
\def\apjs{ApJS} \def\aj{AJ} \def\mnras{MNRAS}
\def\prd{Phys.~Rev.~D} \def\prl{Phys.~Rev.~Lett.}
\def\plb{Phys.~Lett.~B} \def\jhep{JHEP}
\def\npbps{NUC.~Phys.~B~Proc.~Suppl.} \def\prep{Phys.~Rep.}
\def\pasp{PASP} \def\aap{Astron.~\&~Astrophys.} \def\araa{ARA\&A}


%%%%%%%%%%%%%%%%%%%%%%%%%%%%%%%%%%%%%%%%%%%%%%%%%%%%%
%              define symbols                       %
%%%%%%%%%%%%%%%%%%%%%%%%%%%%%%%%%%%%%%%%%%%%%%%%%%%%%
\def \Mpc {~{\rm Mpc} }
\def \Om {\Omega_0}
\def \Omb {\Omega_{\rm b}}
\def \Omcdm {\Omega_{\rm CDM}}
\def \Omlam {\Omega_{\Lambda}}
\def \Omm {\Omega_{\rm m}}
\def \ho {H_0}
\def \qo {q_0}
\def \lo {\lambda_0}
\def \kms {{\rm ~km~s}^{-1}}
\def \kmsmpc {{\rm ~km~s}^{-1}~{\rm Mpc}^{-1}}
\def \hmpc{~\;h^{-1}~{\rm Mpc}} 
\def \hkpc{\;h^{-1}{\rm kpc}} 
\def \hmpcb{h^{-1}{\rm Mpc}}
\def \dif {{\rm d}}
\def \mlim {m_{\rm l}}
\def \bj {b_{\rm J}}
\def \mb {M_{\rm b_{\rm J}}}
\def \qso {_{\rm QSO}}
\def \lrg {_{\rm LRG}}
\def \gal {_{\rm gal}}
\def \xibar {\bar{\xi}}
\def \xis{\xi(s)}
\def \xisp{\xi(\sigma, \pi)}
\def \Xisig{\Xi(\sigma)}
\def \xir{\xi(r)}
\def \max {_{\rm max}}
\def \gsim { \lower .75ex \hbox{$\sim$} \llap{\raise .27ex \hbox{$>$}} }
\def \lsim { \lower .75ex \hbox{$\sim$} \llap{\raise .27ex \hbox{$<$}} }
\def \deg {^{\circ}}
\def \deltac {\delta_{\rm c}}
\def \mmin {M_{\rm min}}
\def \mbh  {M_{\rm BH}}
\def \mdh  {M_{\rm DH}}
\def \msun {M_{\odot}}
\def \z {_{\rm z}}
\def \edd {_{\rm Edd}}
\def \lin {_{\rm lin}}
\def \nonlin {_{\rm non-lin}}
\def \wrms {\langle w_{\rm z}^2\rangle^{1/2}}
\def \dc {\delta_{\rm c}}
\def \wp {w_{p}(\sigma)}
\def \PwrSp {\mathcal{P}(k)}
\def \DelSq {$\Delta^{2}(k)$}
\def \WMAP {{\it WMAP \,}}
\def \cobe {{\it COBE }}
\def \COBE {{\it COBE \;}}
\def \HST  {{\it HST \,\,}}
\def \Spitzer  {{\it Spitzer \,}}


\begin{document}

\title{Introduction to Data Science in Python}
\author{Nicholas P. Ross}
\date{\today}
\maketitle


\begin{abstract}
Here are my (NPR's) notes on the ``Introduction to Data Science in Python''
Coursera course from the University of Michigan that I'm taking in
November 2016.  The URL for that course is\\
\href{https://www.coursera.org/learn/python-data-analysis/home/welcome}{{\tt
https://www.coursera.org/learn/python-data-analysis/home/welcome}}.
The URL for these notes is:\\
\href{https://github.com/d80b2t/Research_Notes/tree/master/Python}{\tt
https://github.com/d80b2t/Research\_Notes/tree/master/Python}
\end{abstract}


\tableofcontents


\newpage
\section{Week 1: Python Fudamentals}

\subsection{Introduction to Specialization}
Kinda a preamble!\\
General Course Outline (4 modules) \\
1. General Python Basics \\
2. The {\it pandas} Toolkit \\
3. Advanced Querying and Manipulation in {\it pandas}\\
4. Basic Statistical Analysis with {\it numpy} and {\it scipy}, and project.\\

\subsection{Syllabus}
\href{https://www.coursera.org/learn/python-data-analysis/supplement/68grE/syllabus}{\tt https://www.coursera.org/learn/python-data-analysis/supplement/68grE/syllabus}. 

If you're having problems, here are a couple of great places to go for help:
\begin{itemize}
\item{1. If the problem is with the Coursera platform such as
verification on assignments, in video quiz problems, or the Jupyter
Notebooks, please check out the Coursera Learner Support Forums.}
\item{2. If the problem deals with understanding the assignment or how
to use the Jupyter Notebooks, please read our Jupyter Notebook FAQ
page in the course resources.}
\item{3. If you have questions with the content of the course, or
questions about programming in python or with the toolkits described,
you can contact your peers and the course instructors in the
discussion forums, or go to Stack Overflow.}
\end{itemize}

\subsection{Data Science}
\href{http://drewconway.com/zia/2013/3/26/the-data-science-venn-diagram}{\tt http://drewconway.com/zia/2013/3/26/the-data-science-venn-diagram}

\begin{figure}[p]
    \includegraphics[width=0.8\textwidth]{Data_Science_VD.png} 
 \caption{Drew Conway's Venn Diagram.}
    \label{fig:DS_Venn}
\end{figure}

David Donoho, Professor of Statistics in Stanford., ``50 Years of Data Science''. 
1. Data Exploration and Preparation.\\
2. Data Representation and Transformation. \\
3. Computing with Data. \\
4. Data Modeling.\\
5. Data Visualization and Presentation. \\
6. Science about Data Science. \\



\subsection{The Coursera Jupyter Notebook System}
All pretty standard, straighforward. 


\subsection{Python Functions}
Of course, Python has traditional software structures like
functions. Here's an example, refactoring that previous code into a
function. You'll see the def statement indicates that we're writing a
function. Then each line that is part of the function needs to be
indented with a tab character or a couple of spaces.

\begin{lstlisting}
def add_numbers(x, y):
    return x + y

add_numbers(1, 2)
\end{lstlisting}

Okay, functions are great but they're a bit different than you might
find in other languages and here are some of subtleties
involved. First, since there's no typing, you don't have to set your
return type. Second, you don't have to use a return statement at all
actually. There's a special value called None that's returned. None is
similar to null in Java and represents the absence of value.  Third,
in Python, you can have default values for parameters.
Here's an example.
\begin{lstlisting}
def add_numbers(x,y,z=None):
    if (z==None):
        return x+y
    else:
        return x+y+z

print(add_numbers(1, 2))
print(add_numbers(1, 2, 3))
\end{lstlisting}
 In this example, we can rewrite the add numbers function to take
three parameters, but we could set the last parameter to be None by
default. This means that you can call add numbers with just two values
or with three, and you don't have to rewrite the function signature to
overload it.

\begin{lstlisting}
def do_math(a, b, kind='add'):
  if (kind=='add'):
    return a+b
  else:
    return a-b

do_math(1, 2)
\end{lstlisting}


    \subsection{Python Types and Sequences}
    The absence of static typing in Python doesn't mean that there
    aren't types. The Python language has a built in function called type
    which will show you what type of given reference is. Some of the
    common types includes strings, the type is discussed. Integers and
    floating point variables. As we've seen you can have reference as to
    function as well as a function type also exist.
    
    Typed objects have properties associated with them, and these
    properties can be data or functions. A lot of Python's built around
    different kinds of sequences or collection types. And there's three
    native kinds of collections that we're going to talk about, {\it
      tuples}, {\it lists}, and {\it dictionaries}.
    
        \subsubsection{Tuples}
        {\it A tuple is a sequence of variables which itself is immutable.}
        That means that a tuple has items in an ordering, but that it cannot
        be changed once created. We write tuples using parentheses, and we can
        mix types for the contents for the tuples. Here's a tuple which has
        four items. Two are numbers, and two are strings.
        \begin{lstlisting}
          x = (1, 'a', 2, 'b')
          type(x)
        \end{lstlisting}
        
        \subsubsection{Lists}
        Lists are very similar, but they can be mutable, so you can
        change their length, number of elements, and the element values. A
        list is declared using the square brackets.
        \begin{lstlisting}
          x = [1, 'a', 2, 'b']
          type(x)
        \end{lstlisting}
        There are a couple of different ways to change the contents of a
        list. One is through the append function which allows you to append
        new items to the end of the list.  
        \begin{lstlisting}
          x.append(3.3)
          print(x)
          \end{lstlisting}
          
        Both lists in tuple are iterable
        types, so you can write loops to go through every value they hold. The
        norm, if you want to look each item in the list is to use a {\tt for} 
        statement. This is similar to the for each loop in languages like Java
        and C\# but note that there's no typing required.
        \begin{lstlisting}
          for item in x:
          print(item)
        \end{lstlisting}

        List and tuples can also be accessed as arrays might in other
        languages, by using the square brackets operator, which is called the
        indexing operator. The first item of the list starts at position zero
        and to get the length of the list, we use the built in lan
        function. There are some other common functions that you might expect
        like min and max which will find the minimum or maximum values in a
        given list or tuple.
        \begin{lstlisting}
          i=0
          while( i != len(x) ):
              print(x[i])
              i = i + 1
          \end{lstlisting}

          

       
        
        
        
    


































\subsection{Python More on Strings}

\subsection{Python Demonstration: Reading and Writing CSV files}

\subsection{Python Dates and Times}

\subsection{Advanced Python Objects map()}

\subsection{Advanced Python Lambda and List Comprehensions}

\subsection{Advanced Python Demonstration: The Numerical Python Library (Numpy)}
























\section{References and Bibliography}




\bibliographystyle{mn2e}
\bibliography{/cos_pc19a_npr/LaTeX/tester_mnras}

\end{document}

