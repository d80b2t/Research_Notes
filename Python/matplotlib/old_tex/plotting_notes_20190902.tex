\documentclass[11pt,a4paper]{article}


\usepackage[toc,page]{appendix}
\usepackage{amsmath, amssymb}
\usepackage{bm}% bold math
\usepackage{cancel, caption}
\usepackage{dcolumn}% Align table columns on decimal point
\usepackage{epsfig, epsf}
\usepackage{graphicx,fancyhdr,natbib,subfigure}
\usepackage{lscape, longtable}
\usepackage{hyperref,ifthen}
\usepackage{verbatim}
\usepackage{color}
\usepackage[usenames,dvipsnames]{xcolor}
\usepackage{listings}
%% http://en.wikibooks.org/wiki/LaTeX/Colors



%%%%%%%%%%%%%%%%%%%%%%%%%%%%%%%%%%%%%%%%%%%
%       define Journal abbreviations      %
%%%%%%%%%%%%%%%%%%%%%%%%%%%%%%%%%%%%%%%%%%%
\def\nat{Nat} \def\apjl{ApJ~Lett.} \def\apj{ApJ}
\def\apjs{ApJS} \def\aj{AJ} \def\mnras{MNRAS}
\def\prd{Phys.~Rev.~D} \def\prl{Phys.~Rev.~Lett.}
\def\plb{Phys.~Lett.~B} \def\jhep{JHEP} \def\nar{NewAR}
\def\npbps{NUC.~Phys.~B~Proc.~Suppl.} \def\prep{Phys.~Rep.}
\def\pasp{PASP} \def\aap{Astron.~\&~Astrophys.} \def\araa{ARA\&A}
\def\jcap{\ref@jnl{J. Cosmology Astropart. Phys.}}%
\def\physrep{Phys.~Rep.}

\newcommand{\preep}[1]{{\tt #1} }

%%%%%%%%%%%%%%%%%%%%%%%%%%%%%%%%%%%%%%%%%%%%%%%%%%%%%
%              define symbols                       %
%%%%%%%%%%%%%%%%%%%%%%%%%%%%%%%%%%%%%%%%%%%%%%%%%%%%%
\def \Mpc {~{\rm Mpc} }
\def \Om {\Omega_0}
\def \Omb {\Omega_{\rm b}}
\def \Omcdm {\Omega_{\rm CDM}}
\def \Omlam {\Omega_{\Lambda}}
\def \Omm {\Omega_{\rm m}}
\def \ho {H_0}
\def \qo {q_0}
\def \lo {\lambda_0}
\def \kms {{\rm ~km~s}^{-1}}
\def \kmsmpc {{\rm ~km~s}^{-1}~{\rm Mpc}^{-1}}
\def \hmpc{~\;h^{-1}~{\rm Mpc}} 
\def \hkpc{\;h^{-1}{\rm kpc}} 
\def \hmpcb{h^{-1}{\rm Mpc}}
\def \dif {{\rm d}}
\def \mlim {m_{\rm l}}
\def \bj {b_{\rm J}}
\def \mb {M_{\rm b_{\rm J}}}
\def \mg {M_{\rm g}}
\def \qso {_{\rm QSO}}
\def \lrg {_{\rm LRG}}
\def \gal {_{\rm gal}}
\def \xibar {\bar{\xi}}
\def \xis{\xi(s)}
\def \xisp{\xi(\sigma, \pi)}
\def \Xisig{\Xi(\sigma)}
\def \xir{\xi(r)}
\def \max {_{\rm max}}
\def \gsim { \lower .75ex \hbox{$\sim$} \llap{\raise .27ex \hbox{$>$}} }
\def \lsim { \lower .75ex \hbox{$\sim$} \llap{\raise .27ex \hbox{$<$}} }
\def \deg {^{\circ}}
%\def \sqdeg {\rm deg^{-2}}
\def \deltac {\delta_{\rm c}}
\def \mmin {M_{\rm min}}
\def \mbh  {M_{\rm BH}}
\def \mdh  {M_{\rm DH}}
\def \msun {M_{\odot}}
\def \z {_{\rm z}}
\def \edd {_{\rm Edd}}
\def \lin {_{\rm lin}}
\def \nonlin {_{\rm non-lin}}
\def \wrms {\langle w_{\rm z}^2\rangle^{1/2}}
\def \dc {\delta_{\rm c}}
\def \wp {w_{p}(\sigma)}
\def \PwrSp {\mathcal{P}(k)}
\def \DelSq {$\Delta^{2}(k)$}
\def \WMAP {{\it WMAP \,}}
\def \cobe {{\it COBE }}
\def \COBE {{\it COBE \;}}
\def \HST  {{\it HST \,\,}}
\def \Spitzer  {{\it Spitzer \,}}
\def \ATLAS {VST-AA$\Omega$ {\it ATLAS} }
\def \BEST   {{\tt best} }
\def \TARGET {{\tt target} }
\def \TQSO   {{\tt TARGET\_QSO}}
\def \HIZ    {{\tt TARGET\_HIZ}}
\def \FIRST  {{\tt TARGET\_FIRST}}
\def \zc {z_{\rm c}}
\def \zcz {z_{\rm c,0}}

\newcommand{\ltsim}{\raisebox{-0.6ex}{$\,\stackrel
        {\raisebox{-.2ex}{$\textstyle <$}}{\sim}\,$}}
\newcommand{\gtsim}{\raisebox{-0.6ex}{$\,\stackrel
        {\raisebox{-.2ex}{$\textstyle >$}}{\sim}\,$}}
\newcommand{\simlt}{\raisebox{-0.6ex}{$\,\stackrel
        {\raisebox{-.2ex}{$\textstyle <$}}{\sim}\,$}}
\newcommand{\simgt}{\raisebox{-0.6ex}{$\,\stackrel
        {\raisebox{-.2ex}{$\textstyle >$}}{\sim}\,$}}

\newcommand{\Msun}{M_\odot}
\newcommand{\Lsun}{L_\odot}
\newcommand{\lsun}{L_\odot}
\newcommand{\Mdot}{\dot M}

\newcommand{\sqdeg}{deg$^{-2}$}
\newcommand{\lya}{Ly$\alpha$\ }
%\newcommand{\lya}{Ly\,$\alpha$\ }
\newcommand{\lyaf}{Ly\,$\alpha$\ forest}
%\newcommand{\eg}{e.g.~}
%\newcommand{\etal}{et~al.~}
\newcommand{\lyb}{Ly$\beta$\ }
\newcommand{\cii}{C\,{\sc ii}\ }
\newcommand{\ciii}{C\,{\sc iii}]\ }
\newcommand{\civ}{C\,{\sc iv}\ }
\newcommand{\SiIV}{Si\,{\sc iv}\ }
\newcommand{\mgii}{Mg\,{\sc ii}\ }
\newcommand{\feii}{Fe\,{\sc ii}\ }
\newcommand{\feiii}{Fe\,{\sc iii}\ }
\newcommand{\caii}{Ca\,{\sc ii}\ }
\newcommand{\halpha}{H\,$\alpha$\ }
\newcommand{\hbeta}{H\,$\beta$\ }
\newcommand{\hgamma}{H\,$\gamma$\ }
\newcommand{\hdelta}{H\,$\delta$\ }
\newcommand{\oi}{[O\,{\sc i}]\ }
\newcommand{\oii}{[O\,{\sc ii}]\ }
\newcommand{\oiii}{[O\,{\sc iii}]\ }
\newcommand{\heii}{[He\,{\sc ii}]\ }
\newcommand{\nv}{N\,{\sc v}\ }
\newcommand{\nev}{Ne\,{\sc v}\ }
\newcommand{\neiii}{[Ne\,{\sc iii}]\ }
\newcommand{\aliii}{Al\,{\sc iii}\ }
\newcommand{\siiii}{Si\,{\sc iii}]\ }


%%%%%%%%%%%%%%%%%%%%%%%%%%%%%%%%%%%%%%%%%%%%%%%%%%%%%
%              define Listings                       %
%%%%%%%%%%%%%%%%%%%%%%%%%%%%%%%%%%%%%%%%%%%%%%%%%%%%%
\definecolor{dkgreen}{rgb}{0,0.6,0}
\definecolor{gray}{rgb}{0.5,0.5,0.5}
\definecolor{mauve}{rgb}{0.58,0,0.82}

\lstset{frame=tb,
  language=Python,
  aboveskip=3mm,
  belowskip=3mm,
  showstringspaces=false,
  columns=flexible,
  basicstyle={\small\ttfamily},
  numbers=none,
  numberstyle=\tiny\color{gray},
  keywordstyle=\color{blue},
  commentstyle=\color{dkgreen},
  stringstyle=\color{mauve},
  breaklines=true,
  breakatwhitespace=true,
  tabsize=3
}

\begin{document}

\title{NPRs Python Notes}
\author{Nicholas P. Ross}
\date{\today}
\maketitle


%% Usually omit these for ApJ or MNRAS style files:


\begin{abstract}
The is my (NPR's) set of {\tt matplotlib} notes.  
%% Things started as wanting to just be an ``IDL to Python CheatSheet'', and have naturally and organically snowballed from there. Suffice to say, when this document reached 47 pages long (and I started to  take notes on how to do pytests ;-), this was no long a Cheat Sheet, and became something else; a general Python resource. \\

You will be able to find the latest version of these notes
and indeed the .tex file at:\\
\href{https://github.com/d80b2t/Research\_Notes}{\tt
https://github.com/d80b2t/Research\_Notes}.
\end{abstract}


\newpage
\section{matplotib}
From \href{http://matplotlib.org/}{{\tt http://matplotlib.org/}} \\

\noindent
matplotlib is a python 2D plotting library which produces publication
quality figures in a variety of hardcopy formats and interactive
environments across platforms. matplotlib can be used in python
scripts, the python and ipython shell (ala MATLAB* or Mathematica†),
web application servers, and six graphical user interface toolkits.

matplotlib tries to make easy things easy and hard things
possible. You can generate plots, histograms, power spectra, bar
charts, errorcharts, scatterplots, etc, with just a few lines of
code. For a sampling, see the screenshots, thumbnail gallery, and
examples directory

For simple plotting the pyplot interface provides a MATLAB-like
interface, particularly when combined with IPython. For the power
user, you have full control of line styles, font properties, axes
properties, etc, via an object oriented interface or via a set of
functions familiar to MATLAB users.\\

The real introduction is here::\\
\href{https://www.labri.fr/perso/nrougier/teaching/matplotlib/}{\tt https://www.labri.fr/perso/nrougier/teaching/matplotlib/}

and here:: \\
\href{https://www.oreilly.com/learning/simple-line-plots-with-matplotlib}{\tt https://www.oreilly.com/learning/simple-line-plots-with-matplotlib} 


\smallskip
\smallskip
\noindent
From \href{http://stackoverflow.com/questions/12987624/confusion-between-numpy-scipy-matplotlib-and-pylab}{http://stackoverflow.com/questions/12987624/confusion-between-numpy-scipy-matplotlib-and-pylab}:\\

- pylab is part of matplotlib (in matplotlib.pylab) and tries to give
you a MatLab like environment. matplotlib has a number of
dependencies, among them numpy which it imports under the common alias
np. scipy is not a dependency of matplotlib.

- If you run ipython --pylab an automatic import will put all symbols
from matplotlib.pylab into global scope. Like you wrote numpy gets
imported under the npalias. Symbols from matplotlib are available
under the mpl alias.

{\tt > ipython --pylab}\\
\begin{lstlisting}
import matplotlib.pyplot as plt

\end{lstlisting}

From \href{http://stackoverflow.com/questions/16522380/python-pandas-plot-is-a-no-show}{http://stackoverflow.com/questions/16522380/python-pandas-plot-is-a-no-show}: 
Put
\begin{lstlisting}
import matplotlib.pyplot as plt
\end{lstlisting}
at the top, and
\begin{lstlisting}
plt.show()
\end{lstlisting}
at the end.

\subsection{matplotlibrc}
\href{https://matplotlib.org/users/customizing.html\#matplotlibrc-sample}{\href https://matplotlib.org/users/customizing.html\#matplotlibrc-sample}\\
{\href https://matplotlib.org/users/dflt\_style\_changes.html}{https://matplotlib.org/users/dflt\_style\_changes.html}\\
{\href https://github.com/matplotlib/matplotlib/blob/master/matplotlibrc.template}{https://github.com/matplotlib/matplotlib/blob/master/matplotlibrc.template}\\


\subsection{ColorMaps}
\href{A Better Default Colormap for Matplotlib (SciPy 2015; Nathaniel Smith and Stéfan van der Walt)}{https://www.youtube.com/watch?v=xAoljeRJ3lU} (a.k.a. ``introducing viridis''). 




\section{Avoid-overlapping-in-scatterplot-with-2d-density}
From::
\href{https://python-graph-gallery.com/86-avoid-overlapping-in-scatterplot-with-2d-density/}{Python
Graph Gallery}.

\begin{lstlisting}
# Libraries
import numpy as np
import matplotlib.pyplot as plt
from scipy.stats import kde
 
# Create data: 200 points
data = np.random.multivariate_normal([0, 0], [[1, 0.5], [0.5, 3]], 200)
x, y = data.T
 
# Create a figure with 6 plot areas
fig, axes = plt.subplots(ncols=6, nrows=1, figsize=(21, 5))
 
# Everything sarts with a Scatterplot
axes[0].set_title('Scatterplot')
axes[0].plot(x, y, 'ko')
# As you can see there is a lot of overplottin here!
 
# Thus we can cut the plotting window in several hexbins
nbins = 20
axes[1].set_title('Hexbin')
axes[1].hexbin(x, y, gridsize=nbins, cmap=plt.cm.BuGn_r)
 
# 2D Histogram
axes[2].set_title('2D Histogram')
axes[2].hist2d(x, y, bins=nbins, cmap=plt.cm.BuGn_r)
 
# Evaluate a gaussian kde on a regular grid of nbins x nbins over data extents
k = kde.gaussian_kde(data.T)
xi, yi = np.mgrid[x.min():x.max():nbins*1j, y.min():y.max():nbins*1j]
zi = k(np.vstack([xi.flatten(), yi.flatten()]))
 
# plot a density
axes[3].set_title('Calculate Gaussian KDE')
axes[3].pcolormesh(xi, yi, zi.reshape(xi.shape), cmap=plt.cm.BuGn_r)
 
# add shading
axes[4].set_title('2D Density with shading')
axes[4].pcolormesh(xi, yi, zi.reshape(xi.shape), shading='gouraud', cmap=plt.cm.BuGn_r)
 
# contour
axes[5].set_title('Contour')
axes[5].pcolormesh(xi, yi, zi.reshape(xi.shape), shading='gouraud', cmap=plt.cm.BuGn_r)
axes[5].contour(xi, yi, zi.reshape(xi.shape) )

\end{lstlisting}



\newpage
\section{Good links}
https://www.machinelearningplus.com/plots/top-50-matplotlib-visualizations-the-master-plots-python/

\end{document}
