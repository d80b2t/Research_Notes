\documentclass[11pt,a4paper]{article}

\usepackage{amsmath, amssymb}
\usepackage{bm, booktabs}
\usepackage{cancel, caption}
\usepackage{dcolumn}  % Align table columns on decimal point
\usepackage{epsfig, epsf, enumitem}
\usepackage{fancyhdr}
\usepackage[T1]{fontenc}
\usepackage{graphicx, geometry}
\usepackage{hyperref}
\usepackage{ifthen}
\usepackage[utf8]{inputenc}
\usepackage{lscape, longtable}
\usepackage{multirow}
\usepackage{natbib}
\usepackage{pifont}
\usepackage{ragged2e}
\usepackage{subfigure}
\usepackage{sectsty}
\usepackage{times, tabularx}
\usepackage{tcolorbox}
\usepackage{verbatim}
%\usepackage[usenames,dvipsnames,svgnames,table]{xcolor}



%%%%%%%%%%%%%%%%%%%%%%%%%%%%%%%%%%%%%%%%%%%
%       define Journal abbreviations      %
%%%%%%%%%%%%%%%%%%%%%%%%%%%%%%%%%%%%%%%%%%%
\def\nat{Nat} \def\apjl{ApJ~Lett.} \def\apj{ApJ}
\def\apjs{ApJS} \def\aj{AJ} \def\mnras{MNRAS}
\def\prd{Phys.~Rev.~D} \def\prl{Phys.~Rev.~Lett.}
\def\plb{Phys.~Lett.~B} \def\jhep{JHEP}
\def\npbps{NUC.~Phys.~B~Proc.~Suppl.} \def\prep{Phys.~Rep.}
\def\pasp{PASP} \def\aap{Astron.~\&~Astrophys.} \def\araa{ARA\&A}


%%%%%%%%%%%%%%%%%%%%%%%%%%%%%%%%%%%%%%%%%%%%%%%%%%%%%
%              define symbols                       %
%%%%%%%%%%%%%%%%%%%%%%%%%%%%%%%%%%%%%%%%%%%%%%%%%%%%%
\def \Mpc {~{\rm Mpc} }
\def \Om {\Omega_0}
\def \Omb {\Omega_{\rm b}}
\def \Omcdm {\Omega_{\rm CDM}}
\def \Omlam {\Omega_{\Lambda}}
\def \Omm {\Omega_{\rm m}}
\def \ho {H_0}
\def \qo {q_0}
\def \lo {\lambda_0}
\def \kms {{\rm ~km~s}^{-1}}
\def \kmsmpc {{\rm ~km~s}^{-1}~{\rm Mpc}^{-1}}
\def \hmpc{~\;h^{-1}~{\rm Mpc}} 
\def \hkpc{\;h^{-1}{\rm kpc}} 
\def \hmpcb{h^{-1}{\rm Mpc}}
\def \dif {{\rm d}}
\def \mlim {m_{\rm l}}
\def \bj {b_{\rm J}}
\def \mb {M_{\rm b_{\rm J}}}
\def \qso {_{\rm QSO}}
\def \lrg {_{\rm LRG}}
\def \gal {_{\rm gal}}
\def \xibar {\bar{\xi}}
\def \xis{\xi(s)}
\def \xisp{\xi(\sigma, \pi)}
\def \Xisig{\Xi(\sigma)}
\def \xir{\xi(r)}
\def \max {_{\rm max}}
\def \gsim { \lower .75ex \hbox{$\sim$} \llap{\raise .27ex \hbox{$>$}} }
\def \lsim { \lower .75ex \hbox{$\sim$} \llap{\raise .27ex \hbox{$<$}} }
\def \deg {^{\circ}}
\def \deltac {\delta_{\rm c}}
\def \mmin {M_{\rm min}}
\def \mbh  {M_{\rm BH}}
\def \mdh  {M_{\rm DH}}
\def \msun {M_{\odot}}
\def \z {_{\rm z}}
\def \edd {_{\rm Edd}}
\def \lin {_{\rm lin}}
\def \nonlin {_{\rm non-lin}}
\def \wrms {\langle w_{\rm z}^2\rangle^{1/2}}
\def \dc {\delta_{\rm c}}
\def \wp {w_{p}(\sigma)}
\def \PwrSp {\mathcal{P}(k)}
\def \DelSq {$\Delta^{2}(k)$}
\def \WMAP {{\it WMAP \,}}
\def \cobe {{\it COBE }}
\def \COBE {{\it COBE \;}}
\def \HST  {{\it HST \,\,}}
\def \Spitzer  {{\it Spitzer \,}}


\begin{document}

   \title{Baldwin Luminosites}
 \date{\today}
\maketitle


\section{Baldwin plots}
\citet{Hamann2017} provide:: \\

\noindent
$f_{1450}$, the
flux in the uncorrected BOSS spectrum at 1450\AA\ rest ($10^{−17}$
ergs s$^{−1}$ cm$^{-2}$ \AA$^{-1}$ ) used to anchor the power-law
continuum fits beneath \civ and \nv, e.g. $f_{\lambda}$ = $f_{1450} (\lambda/1450$\AA\ $)^{\alpha}$. \\
The quasar redshift and cosmology are also given. 

\begin{lstlisting}
from astropy.cosmology import FlatLambdaCDM

## Setting up the cosmology
cosmo = FlatLambdaCDM(H0=71, Om0=0.27, Tcmb0=2.725)                                                    

##  Luminosity  = 4 * pi * (D_L^2) * F      
##   where  L  is in W   and  F is in W/m^2
##   D_L is Luminosity Distance 

## Set-up the Luminosity Distance 
DL       = cosmo.luminosity_distance(redshift)             
DL_incm = DL.value * 3.08567758128e24

## DL in cm;   f in 10-17 ergs s-1 cm-2 Å-1
Lum     = (4 * np.pi * (DL_incm*DL_incm) * 1e-17 * f_lam) * 1450.
log_Lum = np.log10(Lum)
\end{lstlisting}



\section{Dyer et al. (2019)}

\subsection*{2.2. Quasar Properties}
Our analysis uses luminosity, spectral index, and \textsc{C~iv}
$W_\lambda$ to explore the intrinsic spectral variability of quasars
and to assess the impact of quasar variability on Lyman-$\alpha$
forest clustering studies. We compute $L_{\mathrm{bol}}$,
$\alpha_{\lambda}$, and \textsc{C~iv} $W_\lambda$ from the spectrum of
every epoch of every quasar. The median $\log{L_{\mathrm{bol}}}$,
$\alpha_\lambda$, and \textsc{C~iv} $W_\lambda$ for each quasar are
considered to represent the quasar's steady state properties.

The bolometric luminosity of each spectrum is estimated from the
monochromatic luminosity at 1740~\AA\ (i.e., the median flux in the
range 1680~-- 1800~\AA) corrected for luminosity distance under a
$\Lambda \mathrm{CDM}$ cosmology with $H_0 = 70\ \mathrm{km \, s^{-1}
\, Mpc}$, $\Omega_{M} = 0.3$, and $\Omega_{\Lambda} = 0.7$. We
approximate $L_{\mathrm{bol}}=A*L_{1740}$, where we compute the
bolometric luminosity at 1450 \AA\ with the correction factor
suggested by \citet{Runnoe2012, Runnoe2012Erratum} and determine the scaling factor
$A=4.28$ that minimizes the residual between the bolometric
luminosities computed at 1740~\AA\ and 1450~\AA.  We note that the
analysis that follows uses the fractional change in luminosity, so the
choice of scaling of monochromatic flux to bolometric luminosity does
not impact any of the results.

The spectral index of each spectrum is determined by fitting a power
law to the quasar continuum in the wavelength ranges 1680~-- 1800~\AA\
and 2000~-- 2050~\AA. These ranges were chosen for their lack of
emission lines (except broadband iron which we assume is negligible)
according to a high S/N composite spectrum \citep{Harris2016}. Every
spectrum in our sample is observed over this wavelength range. Only
unmasked pixels with flux within three standard deviations of the
median are used for this fit to mitigate the influence of intervening
absorption from the intergalactic medium.

To calculate \textsc{C~iv} $W_\lambda$ for each spectrum, we first
estimate the continuum around \textsc{C~iv} emission with a linear fit
to the spectrum in the wavelength ranges 1450~-- 1465~\AA\ and 1685~--
1700~\AA. We model the \textsc{C~iv} emission using a double Gaussian
fit to the continuum-subtracted spectrum over the wavelength range
1500~-- 1580~\AA. Only unmasked flux measurements within three
standard deviations of the best fit are used. \textsc{C~iv}
$W_\lambda$ is finally computed by taking the integral of the
estimated emission flux relative to the estimated continuum flux over
1520~-- 1580~\AA.




\bibliographystyle{mn2e}
\bibliography{/cos_pc19a_npr/LaTeX/tester_mnras}

\end{document}