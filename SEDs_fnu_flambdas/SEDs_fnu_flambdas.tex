\documentclass[11pt,a4paper]{article}

\usepackage{epsfig}
\usepackage{amsmath}
\usepackage{natbib}


\usepackage[toc,page]{appendix}
\usepackage{amsmath, amssymb}
\usepackage{bm}% bold math
\usepackage{cancel, caption}
\usepackage{dcolumn}% Align table columns on decimal point
\usepackage{epsfig, epsf}
\usepackage{graphicx,fancyhdr,natbib,subfigure}
\usepackage{lscape, longtable}
\usepackage{hyperref,ifthen}
\usepackage{verbatim}
\usepackage{color}
\usepackage[usenames,dvipsnames]{xcolor}
\usepackage{listings}
%% http://en.wikibooks.org/wiki/LaTeX/Colors



%%%%%%%%%%%%%%%%%%%%%%%%%%%%%%%%%%%%%%%%%%%
%       define Journal abbreviations      %
%%%%%%%%%%%%%%%%%%%%%%%%%%%%%%%%%%%%%%%%%%%
\def\nat{Nat} \def\apjl{ApJ~Lett.} \def\apj{ApJ}
\def\apjs{ApJS} \def\aj{AJ} \def\mnras{MNRAS}
\def\prd{Phys.~Rev.~D} \def\prl{Phys.~Rev.~Lett.}
\def\plb{Phys.~Lett.~B} \def\jhep{JHEP} \def\nar{NewAR}
\def\npbps{NUC.~Phys.~B~Proc.~Suppl.} \def\prep{Phys.~Rep.}
\def\pasp{PASP} \def\aap{Astron.~\&~Astrophys.} \def\araa{ARA\&A}
\def\jcap{\ref@jnl{J. Cosmology Astropart. Phys.}}%
\def\physrep{Phys.~Rep.}

\newcommand{\preep}[1]{{\tt #1} }

%%%%%%%%%%%%%%%%%%%%%%%%%%%%%%%%%%%%%%%%%%%%%%%%%%%%%
%              define symbols                       %
%%%%%%%%%%%%%%%%%%%%%%%%%%%%%%%%%%%%%%%%%%%%%%%%%%%%%
\def \Mpc {~{\rm Mpc} }
\def \Om {\Omega_0}
\def \Omb {\Omega_{\rm b}}
\def \Omcdm {\Omega_{\rm CDM}}
\def \Omlam {\Omega_{\Lambda}}
\def \Omm {\Omega_{\rm m}}
\def \ho {H_0}
\def \qo {q_0}
\def \lo {\lambda_0}
\def \kms {{\rm ~km~s}^{-1}}
\def \kmsmpc {{\rm ~km~s}^{-1}~{\rm Mpc}^{-1}}
\def \hmpc{~\;h^{-1}~{\rm Mpc}} 
\def \hkpc{\;h^{-1}{\rm kpc}} 
\def \hmpcb{h^{-1}{\rm Mpc}}
\def \dif {{\rm d}}
\def \mlim {m_{\rm l}}
\def \bj {b_{\rm J}}
\def \mb {M_{\rm b_{\rm J}}}
\def \mg {M_{\rm g}}
\def \qso {_{\rm QSO}}
\def \lrg {_{\rm LRG}}
\def \gal {_{\rm gal}}
\def \xibar {\bar{\xi}}
\def \xis{\xi(s)}
\def \xisp{\xi(\sigma, \pi)}
\def \Xisig{\Xi(\sigma)}
\def \xir{\xi(r)}
\def \max {_{\rm max}}
\def \gsim { \lower .75ex \hbox{$\sim$} \llap{\raise .27ex \hbox{$>$}} }
\def \lsim { \lower .75ex \hbox{$\sim$} \llap{\raise .27ex \hbox{$<$}} }
\def \deg {^{\circ}}
%\def \sqdeg {\rm deg^{-2}}
\def \deltac {\delta_{\rm c}}
\def \mmin {M_{\rm min}}
\def \mbh  {M_{\rm BH}}
\def \mdh  {M_{\rm DH}}
\def \msun {M_{\odot}}
\def \z {_{\rm z}}
\def \edd {_{\rm Edd}}
\def \lin {_{\rm lin}}
\def \nonlin {_{\rm non-lin}}
\def \wrms {\langle w_{\rm z}^2\rangle^{1/2}}
\def \dc {\delta_{\rm c}}
\def \wp {w_{p}(\sigma)}
\def \PwrSp {\mathcal{P}(k)}
\def \DelSq {$\Delta^{2}(k)$}
\def \WMAP {{\it WMAP \,}}
\def \cobe {{\it COBE }}
\def \COBE {{\it COBE \;}}
\def \HST  {{\it HST \,\,}}
\def \Spitzer  {{\it Spitzer \,}}
\def \ATLAS {VST-AA$\Omega$ {\it ATLAS} }
\def \BEST   {{\tt best} }
\def \TARGET {{\tt target} }
\def \TQSO   {{\tt TARGET\_QSO}}
\def \HIZ    {{\tt TARGET\_HIZ}}
\def \FIRST  {{\tt TARGET\_FIRST}}
\def \zc {z_{\rm c}}
\def \zcz {z_{\rm c,0}}

\newcommand{\ltsim}{\raisebox{-0.6ex}{$\,\stackrel
        {\raisebox{-.2ex}{$\textstyle <$}}{\sim}\,$}}
\newcommand{\gtsim}{\raisebox{-0.6ex}{$\,\stackrel
        {\raisebox{-.2ex}{$\textstyle >$}}{\sim}\,$}}
\newcommand{\simlt}{\raisebox{-0.6ex}{$\,\stackrel
        {\raisebox{-.2ex}{$\textstyle <$}}{\sim}\,$}}
\newcommand{\simgt}{\raisebox{-0.6ex}{$\,\stackrel
        {\raisebox{-.2ex}{$\textstyle >$}}{\sim}\,$}}

\newcommand{\Msun}{M_\odot}
\newcommand{\Lsun}{L_\odot}
\newcommand{\lsun}{L_\odot}
\newcommand{\Mdot}{\dot M}

\newcommand{\sqdeg}{deg$^{-2}$}
\newcommand{\lya}{Ly$\alpha$\ }
%\newcommand{\lya}{Ly\,$\alpha$\ }
\newcommand{\lyaf}{Ly\,$\alpha$\ forest}
%\newcommand{\eg}{e.g.~}
%\newcommand{\etal}{et~al.~}
\newcommand{\lyb}{Ly$\beta$\ }
\newcommand{\cii}{C\,{\sc ii}\ }
\newcommand{\ciii}{C\,{\sc iii}]\ }
\newcommand{\civ}{C\,{\sc iv}\ }
\newcommand{\SiIV}{Si\,{\sc iv}\ }
\newcommand{\mgii}{Mg\,{\sc ii}\ }
\newcommand{\feii}{Fe\,{\sc ii}\ }
\newcommand{\feiii}{Fe\,{\sc iii}\ }
\newcommand{\caii}{Ca\,{\sc ii}\ }
\newcommand{\halpha}{H\,$\alpha$\ }
\newcommand{\hbeta}{H\,$\beta$\ }
\newcommand{\hgamma}{H\,$\gamma$\ }
\newcommand{\hdelta}{H\,$\delta$\ }
\newcommand{\oi}{[O\,{\sc i}]\ }
\newcommand{\oii}{[O\,{\sc ii}]\ }
\newcommand{\oiii}{[O\,{\sc iii}]\ }
\newcommand{\heii}{[He\,{\sc ii}]\ }
\newcommand{\nv}{N\,{\sc v}\ }
\newcommand{\nev}{Ne\,{\sc v}\ }
\newcommand{\neiii}{[Ne\,{\sc iii}]\ }
\newcommand{\aliii}{Al\,{\sc iii}\ }
\newcommand{\siiii}{Si\,{\sc iii}]\ }


%%%%%%%%%%%%%%%%%%%%%%%%%%%%%%%%%%%%%%%%%%%%%%%%%%%%%
%              define Listings                       %
%%%%%%%%%%%%%%%%%%%%%%%%%%%%%%%%%%%%%%%%%%%%%%%%%%%%%
\definecolor{dkgreen}{rgb}{0,0.6,0}
\definecolor{gray}{rgb}{0.5,0.5,0.5}
\definecolor{mauve}{rgb}{0.58,0,0.82}

\lstset{frame=tb,
  language=Python,
  aboveskip=3mm,
  belowskip=3mm,
  showstringspaces=false,
  columns=flexible,
  basicstyle={\small\ttfamily},
  numbers=none,
  numberstyle=\tiny\color{gray},
  keywordstyle=\color{blue},
  commentstyle=\color{dkgreen},
  stringstyle=\color{mauve},
  breaklines=true,
  breakatwhitespace=true,
  tabsize=3
}

\begin{document}

\title{SEDs, $f_{\nu}$ and $f_{\lambda}$'s}
\author{Nic ``What Have I got wrong here'' Ross}
\date{\today}
\maketitle


% Usually omit these for ApJ or MNRAS style files:
%\tableofcontents
%\listoffigures
%\listoftables

\begin{abstract}
%{\bf The} definitive document, 
``README'' and ``Cheat Sheet'' to SEDs, $f_{\nu}$ and $f_{\lambda}$ and all that carry-on...
\end{abstract}


%Section heading
\section{Definitions, terms and Units}

%%\cite{Richards06b}, \citet{Richards06b}, \citep{Richards06b}

\begin{table*}
  \begin{center}
    \setlength{\tabcolsep}{4pt}
    \begin{tabular}{llllr}
      \hline
      \hline
      Physical Quantity & symbol & Unit name & Units & e.g. $\log$(OoM)   \\
      \hline
      spectral flux density$^{a}$ &  $f_{\nu}$        &             & W                  m$^{-2}$ Hz$^{-1}$                  & -27 -- -35 \\
      spectral flux density          &  $f_{\nu}$        &             & erg s$^{-1}$ cm$^{-2}$ Hz$^{-1}$                  & -24 -- -32 \\
      spectral flux density          &  $f_{\nu}$        & Janksy  & $10^{-26}$ W     m$^{-2}$ Hz$^{-1}$              & $\mu$ to 10's of milli \\
      spectral flux density          &  $f_{\nu}$        & Jansky  & $10^{-23}$ erg s$^{-1}$ cm$^{-2}$ Hz$^{-1}$ & $\mu$ to 10's of milli \\
      spectral flux density          &  $f_{\lambda}$  &             &  W                  m$^{-2}$ m$^{-1}$                  &  \\
      spectral flux density          &  $f_{\lambda}$  &             &  W                  m$^{-2}$ $\mu$m$^{-1}$                  &  \\
                                                &                       &             &                        & \\
      $^{a}$energy density         & $\nu f_{\nu}$   &  erg s$^{-1}$ cm$^{-2}$       &        & -12 -- -16 \\
     $^{b}$ --                            & $\nu F_{\nu}$ &                                            &        &      \\
                                                &                       &             &                        & \\
     $^{c}$ --                            &        $L_{\nu}$ &  erg s$^{-1}$ Hz$^{-1}$       &        &    26 -- 34  \\
                                                &                       &             &                        & \\

     $^{a}$Luminosity              &  $\nu L_{\nu}$ &  erg s$^{-1}$                        &       &   43 -- 47 \\
     $^{a}$Luminosity              &  $L$ &  erg s$^{-1}$                        &       &   43 -- 47 \\
      \hline
      \hline
   \end{tabular}
    \caption{$^{a}$see e.g. Fig. 10 of \citet{Richards06b}.\\
      $^{b}$e.g. URL [1]\\
      $^{c}$e.g. Bourne et al. (2011)
}
     \label{tab:units_overview}
  \end{center}
\end{table*}



\clearpage
\begin{landscape} 
\begin{table*}
  \begin{center}
    \setlength{\tabcolsep}{4pt}
%    \begin{tabular}{l l l l l l}
    \begin{tabular}{  p{45mm}   p{20mm}  p{40mm}  p{20mm}   p{20mm}  p{90mm}} 
      \hline
      \hline
      Name	                                      & Symbol                       & Unit name                     & Unit symbol  & Dimension                 & Notes   \\
      \hline
      Radiant energy	                      &  $Q_{\rm e}$                 & Joule                              &	J             & M L$^{2}$    T$^{-2}$     & Energy of electromagnetic radiation. \\
      Radiant energy density            & w$_{\rm e}$	             & Joule per cubic metre    & J/m$^3$	      & M L$^{-1}$  T$^{-2}$     & Radiant energy per unit volume. \\
      Radiant flux                             & $\phi_{\rm e}$              & Watt	                              & W = J/s         & M L$^{2}$    T$^{-3}$    & Radiant energy emitted per unit time$^{a}$. \\
      \multirow{3}{*}{Spectral flux}  &  $\phi_{{\rm e}, \nu}$       & Watt per hertz                & W/Hz            & M L$^{2}$    T$^{−2}$    & Radiant flux per unit frequency or wavelength.  \\
                                                     &  or                                & or                                   & or                 & or                                & \\
                                                     & $\phi_{{\rm e}, \lambda}$  & Watt per metre                 & W/m            & M L             T$^{-3}$   & \\
      \hline
      \hline
    \end{tabular}
    \caption{{\bf STRAIGHT FROM::} 
      \href{https://en.wikipedia.org/wiki/Optical\_depth}{https://en.wikipedia.org/wiki/Optical\_depth}\\
      $^{a}$Also sometimes called ``radiant power''.
    }
     \label{tab:units_overview}
  \end{center}
\end{table*}
\end{landscape} 




From Table~\ref{tab:units_overview}, to get from $L_{\nu}$ to $\nu L_{\nu}$, at say 1.0$\mu$m, you just have to multiply by
3$\times10^{14}$ (Hz), and this gives e.g. $\approx3\times10^{44}$ erg s$^{-1}$ :-) 

\begin{table}
  \begin{center}
    \setlength{\tabcolsep}{4pt}
    \begin{tabular}{lrrrl}
      \hline       \hline
      Unit Name   & Physical Quantity & symbos & Units     \\
      \hline
      Janksy & spectral flux density &  $f_{\nu}$ & $10^{-26}$ W                  m$^{-2}$ Hz$^{-1}$  \\
     Jansky  & spectral flux density &  $f_{\nu}$  & $10^{-23}$ erg s$^{-1}$ cm$^{-2}$ Hz$^{-1}$ \\
      \hline       \hline
    \label{tab:The_LRG_numbers}
    \end{tabular}
    \caption{e.g. Fig. 10 of Richards et al. (2006b).}
  \end{center}
\end{table}


\begin{eqnarray}
  m_{\rm AB} & =  & -2.5 \log_{10} \left(  \frac{f_{\nu}}{3631 \ {\rm Jy}} \right) \\
                 & =  & -2.5 \log_{10} \left(  \frac{f_{\nu}}{  {\rm Jy}} \right) + 8.90 \\
                 & =  &-2.5 \log_{10} f_{\nu} - 48.60 
\end{eqnarray}
with the -48.6 thing coming in if in $cgs$ units of erg s$^{-1}$ cm$^{-2}$ Hz$^{-1}$.\\

\noindent
So, with $\nu  f_{\nu}  = \lambda f_{\lambda}$, we can have: 
\begin{eqnarray}
  f_{\nu}              & = &  \frac{\lambda^{2}}{c} \ f_{\lambda}\\
\end{eqnarray}
Now, if $f_{\nu}$ is in Jansky's and $\lambda$ is in \AA\, then, 
$c$ must be in \AA\ s$^{-1}$, i.e.,  $c = 3\times10^{18}$ \AA\ s$^{-1}$.  
Then you just replace:
\begin{eqnarray}
%f_{\nu} & = & 1e23 \times \frac{\lambda^2}{c} f_{\lambda}\\
  f_{\nu}                      & = &  \frac{\lambda^{2}}{c} f_{\lambda}\\
  f_{\nu} {\rm (in\ Jy)} & = & 1  \times 10^{23}/(3e18) \ \frac{\lambda^{2}}{\AA} \ f_{\lambda}\\
  f_{\nu} {\rm (in\ Jy)} & = & 0.3 \times10^{5} \times \lambda^{2} \times f_{\lambda} \\ 
  f_{\nu} {\rm (in\ Jy)} & = & 3.34e4 \times \lambda^{2} \times f_{\lambda} \\
  \frac{f_{\nu}}{ [ {\rm Jansky} ] }  & = & 33,356 \left( \frac{\lambda}{ [ {\rm \AA} ] } \right)^{2} \frac{f_{\lambda}} {\rm erg s^{-1} cm^{-2}  \AA^{-1}} 
\end{eqnarray}
with 1 cm being 1$\times10^{8}$\AA\ and $c=2.99792\times10^{10}$ cm s$^{-1}$. \\

\noindent
Following e.g. URL [5], 
\begin{eqnarray}
f_{\nu}                      & = & A \times \lambda f_{\gamma}\\
f_{\nu} {\rm (in\ Jy)} & = & 6.626\times10^{-8} \frac{\lambda}{[\mu m]} f_{\gamma}\\
\end{eqnarray}
where $f_{\nu}$ is the `energy flux', aka the spectral flux density,
measured in Janskys (10$^{-26}$ W m$^{-2}$ Hz$^{-1}$), $f_{\gamma}$ is
the `photon flux' measured in s$^{-1}$ m$^{-2}$ $\mu$m$^{-1}$,
$\lambda$ is the wavelength measured in $\mu$m and Planck's constant
is 6.626$\times10^{-34}$ m$^{2}$ kg s$^{-1}$ (i.e. 6.626e-34*1e26 =
6.626e-8).\\

\noindent
Take your $f_{\nu}$ measurements that are in Jy. (Ensure they are in
Jy! If they're in magnitudes, convert them to Jy first; see
'magnitude' discussion above.) Multiply by $1\times10^{-23}$ to get
them into {\it cgs} units. Multiply these $f_{\nu}$ values by
$\frac{c}{\lambda^{2}}$ to get them into $f_{\lambda}$. Multiply them
by $\lambda$ to get them into $\lambda f_{\lambda}$. WATCH YOUR
UNITS. NB: $c = 2.997924\times10^{10}$ cm sec$^{-1}$. \\

\noindent
Also note/recall, 
\begin{equation}
  \nu  f_{\nu}    =  \lambda  f_{\lambda}
\end{equation}
with units of ergs s$^{-1}$ cm$^{-2}$. 

\section{Unit Conversions}
    \subsection{Flux density to AB}
    The flux density in Jy can be converted to a magnitude basis:\\
    (straight from Wiki!! ;-):\\
    \begin{eqnarray}
      S_{\nu} [\mu {\rm Jy} ]  & = & 10^{6} \cdot 10^{23} \cdot 10^{-{{\rm AB} +48.6/2.5 }} \\
                                        & = & 10^{(23.9 - {\rm AB})/2.5} \\
    \end{eqnarray}

%% https://en.wikipedia.org/wiki/Luminosity

\section{WISE}
The source flux density, in Jansky [Jy] units, is computed from the calibrated WISE (Vega) magnitudes, $m_{\rm Vega}$ using:
N.B the WISE webpage (given in the URL notes below) uses $F_{\nu}$ for source flux density. I'm going to stick with my convention and use little $f$, $f_{\nu}$, 
\begin{equation}
  f_{\nu} [{\rm Jy}]  =  f_{\nu, 0} \times 10^{(-m_{\rm Vega}/2.5)} 
\end{equation}
where $f_{\nu, 0}$ is the zero magnitude flux density corresponding to the constant that gives the same response as that of Alpha Lyrae (Vega). For most sources, the zero magnitude flux density, derived using a constant power-law spectra, is appropriate and may be used to convert WISE magnitudes to flux density [Jy] units. Table 1 lists the zero magnitude flux density (column 2) for each WISE band.\\

\noindent
For sources with steeply rising MIR spectra or with spectra that deviate significantly from $f_{\nu}=$constant, including cool asteroids and dusty star-forming galaxies, a color correction is required, especially for W3 due to its wide bandpass. With a given flux correction, $f_{c}$, the flux density conversion is given by:
\begin{equation}
  f_{\nu} [{\rm Jy}]  = (f^{*}_{\nu, 0}/f_{c}) \times 10^{(-m_{\rm Vega}/2.5)} 
\end{equation}
where $f^{*}_{\nu, 0}$ is the zero magnitude flux density derived for sources with power-law spectra: $f_{\nu} \propto \nu^{2}$, listed in Table 1 (column 3) and the flux correction, $f_{c}$, listed in Table 2 for $f_{\nu} \propto \nu^{-\alpha}$, where the index $\alpha$  ranges from: -3, -2, -1, 0, 1, 2, 3, and 4, and for blackbody spectra, B$_{\nu}$(T) for a variety of temperatures, and for stars of two main-sequence spectral types (K2V and G2V). 


\section{Links}
%% Using \\ and square brackets seems to crash with [2] !!
[1] \href{http://www.iasf-milano.inaf.it/$\sim$polletta/templates/images/new\_Arp220\_template.jpg}{http://www.iasf-milano.inaf.it/$\sim$polletta/templates/images/new\_Arp220\_template.jpg}

\noindent
[2] \href{http://www.astro.soton.ac.uk/$\sim$td/flux\_convert.html}{http://www.astro.soton.ac.uk/$\sim$td/flux\_convert.html}

\noindent
[3] \href{http://coolwiki.ipac.caltech.edu/index.php/Units}{http://coolwiki.ipac.caltech.edu/index.php/Units}

\noindent
[4] \href{http://wise2.ipac.caltech.edu/docs/release/allsky/expsup/sec4\_4h.html}{http://wise2.ipac.caltech.edu/docs/release/allsky/expsup/sec4\_4h.html}

\noindent
[5] \href{http://www.astro.ljmu.ac.uk/~ikb/convert-units/node1.html}{http://www.astro.ljmu.ac.uk/~ikb/convert-units/node1.html}

\noindent
[6] \href{http://xingxinghuang.blogspot.co.uk/2013/06/hello-everybody-if-you-still-get.html}{http://xingxinghuang.blogspot.co.uk/2013/06/hello-everybody-if-you-still-get.html}


\bibliographystyle{mn2e}
%\bibliography{tester_mnras}
\bibliography{/cos_pc19a_npr/LaTeX/tester_mnras}



\end{document}

