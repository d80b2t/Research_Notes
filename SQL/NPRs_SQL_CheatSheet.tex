\documentclass[11pt,a4paper]{article}


\usepackage[toc,page]{appendix}
\usepackage{amsmath, amssymb}
\usepackage{bm}% bold math
\usepackage{cancel, caption}
\usepackage{dcolumn}% Align table columns on decimal point
\usepackage{epsfig, epsf}
\usepackage{graphicx,fancyhdr,natbib,subfigure}
\usepackage{lscape, longtable}
\usepackage{hyperref,ifthen}
\usepackage{verbatim}
\usepackage{color}
\usepackage[usenames,dvipsnames]{xcolor}
\usepackage{listings}
%% http://en.wikibooks.org/wiki/LaTeX/Colors



%%%%%%%%%%%%%%%%%%%%%%%%%%%%%%%%%%%%%%%%%%%
%       define Journal abbreviations      %
%%%%%%%%%%%%%%%%%%%%%%%%%%%%%%%%%%%%%%%%%%%
\def\nat{Nat} \def\apjl{ApJ~Lett.} \def\apj{ApJ}
\def\apjs{ApJS} \def\aj{AJ} \def\mnras{MNRAS}
\def\prd{Phys.~Rev.~D} \def\prl{Phys.~Rev.~Lett.}
\def\plb{Phys.~Lett.~B} \def\jhep{JHEP} \def\nar{NewAR}
\def\npbps{NUC.~Phys.~B~Proc.~Suppl.} \def\prep{Phys.~Rep.}
\def\pasp{PASP} \def\aap{Astron.~\&~Astrophys.} \def\araa{ARA\&A}
\def\jcap{\ref@jnl{J. Cosmology Astropart. Phys.}}%
\def\physrep{Phys.~Rep.}

\newcommand{\preep}[1]{{\tt #1} }

%%%%%%%%%%%%%%%%%%%%%%%%%%%%%%%%%%%%%%%%%%%%%%%%%%%%%
%              define symbols                       %
%%%%%%%%%%%%%%%%%%%%%%%%%%%%%%%%%%%%%%%%%%%%%%%%%%%%%
\def \Mpc {~{\rm Mpc} }
\def \Om {\Omega_0}
\def \Omb {\Omega_{\rm b}}
\def \Omcdm {\Omega_{\rm CDM}}
\def \Omlam {\Omega_{\Lambda}}
\def \Omm {\Omega_{\rm m}}
\def \ho {H_0}
\def \qo {q_0}
\def \lo {\lambda_0}
\def \kms {{\rm ~km~s}^{-1}}
\def \kmsmpc {{\rm ~km~s}^{-1}~{\rm Mpc}^{-1}}
\def \hmpc{~\;h^{-1}~{\rm Mpc}} 
\def \hkpc{\;h^{-1}{\rm kpc}} 
\def \hmpcb{h^{-1}{\rm Mpc}}
\def \dif {{\rm d}}
\def \mlim {m_{\rm l}}
\def \bj {b_{\rm J}}
\def \mb {M_{\rm b_{\rm J}}}
\def \mg {M_{\rm g}}
\def \qso {_{\rm QSO}}
\def \lrg {_{\rm LRG}}
\def \gal {_{\rm gal}}
\def \xibar {\bar{\xi}}
\def \xis{\xi(s)}
\def \xisp{\xi(\sigma, \pi)}
\def \Xisig{\Xi(\sigma)}
\def \xir{\xi(r)}
\def \max {_{\rm max}}
\def \gsim { \lower .75ex \hbox{$\sim$} \llap{\raise .27ex \hbox{$>$}} }
\def \lsim { \lower .75ex \hbox{$\sim$} \llap{\raise .27ex \hbox{$<$}} }
\def \deg {^{\circ}}
%\def \sqdeg {\rm deg^{-2}}
\def \deltac {\delta_{\rm c}}
\def \mmin {M_{\rm min}}
\def \mbh  {M_{\rm BH}}
\def \mdh  {M_{\rm DH}}
\def \msun {M_{\odot}}
\def \z {_{\rm z}}
\def \edd {_{\rm Edd}}
\def \lin {_{\rm lin}}
\def \nonlin {_{\rm non-lin}}
\def \wrms {\langle w_{\rm z}^2\rangle^{1/2}}
\def \dc {\delta_{\rm c}}
\def \wp {w_{p}(\sigma)}
\def \PwrSp {\mathcal{P}(k)}
\def \DelSq {$\Delta^{2}(k)$}
\def \WMAP {{\it WMAP \,}}
\def \cobe {{\it COBE }}
\def \COBE {{\it COBE \;}}
\def \HST  {{\it HST \,\,}}
\def \Spitzer  {{\it Spitzer \,}}
\def \ATLAS {VST-AA$\Omega$ {\it ATLAS} }
\def \BEST   {{\tt best} }
\def \TARGET {{\tt target} }
\def \TQSO   {{\tt TARGET\_QSO}}
\def \HIZ    {{\tt TARGET\_HIZ}}
\def \FIRST  {{\tt TARGET\_FIRST}}
\def \zc {z_{\rm c}}
\def \zcz {z_{\rm c,0}}

\newcommand{\ltsim}{\raisebox{-0.6ex}{$\,\stackrel
        {\raisebox{-.2ex}{$\textstyle <$}}{\sim}\,$}}
\newcommand{\gtsim}{\raisebox{-0.6ex}{$\,\stackrel
        {\raisebox{-.2ex}{$\textstyle >$}}{\sim}\,$}}
\newcommand{\simlt}{\raisebox{-0.6ex}{$\,\stackrel
        {\raisebox{-.2ex}{$\textstyle <$}}{\sim}\,$}}
\newcommand{\simgt}{\raisebox{-0.6ex}{$\,\stackrel
        {\raisebox{-.2ex}{$\textstyle >$}}{\sim}\,$}}

\newcommand{\Msun}{M_\odot}
\newcommand{\Lsun}{L_\odot}
\newcommand{\lsun}{L_\odot}
\newcommand{\Mdot}{\dot M}

\newcommand{\sqdeg}{deg$^{-2}$}
\newcommand{\lya}{Ly$\alpha$\ }
%\newcommand{\lya}{Ly\,$\alpha$\ }
\newcommand{\lyaf}{Ly\,$\alpha$\ forest}
%\newcommand{\eg}{e.g.~}
%\newcommand{\etal}{et~al.~}
\newcommand{\lyb}{Ly$\beta$\ }
\newcommand{\cii}{C\,{\sc ii}\ }
\newcommand{\ciii}{C\,{\sc iii}]\ }
\newcommand{\civ}{C\,{\sc iv}\ }
\newcommand{\SiIV}{Si\,{\sc iv}\ }
\newcommand{\mgii}{Mg\,{\sc ii}\ }
\newcommand{\feii}{Fe\,{\sc ii}\ }
\newcommand{\feiii}{Fe\,{\sc iii}\ }
\newcommand{\caii}{Ca\,{\sc ii}\ }
\newcommand{\halpha}{H\,$\alpha$\ }
\newcommand{\hbeta}{H\,$\beta$\ }
\newcommand{\hgamma}{H\,$\gamma$\ }
\newcommand{\hdelta}{H\,$\delta$\ }
\newcommand{\oi}{[O\,{\sc i}]\ }
\newcommand{\oii}{[O\,{\sc ii}]\ }
\newcommand{\oiii}{[O\,{\sc iii}]\ }
\newcommand{\heii}{[He\,{\sc ii}]\ }
\newcommand{\nv}{N\,{\sc v}\ }
\newcommand{\nev}{Ne\,{\sc v}\ }
\newcommand{\neiii}{[Ne\,{\sc iii}]\ }
\newcommand{\aliii}{Al\,{\sc iii}\ }
\newcommand{\siiii}{Si\,{\sc iii}]\ }


%%%%%%%%%%%%%%%%%%%%%%%%%%%%%%%%%%%%%%%%%%%%%%%%%%%%%
%              define Listings                       %
%%%%%%%%%%%%%%%%%%%%%%%%%%%%%%%%%%%%%%%%%%%%%%%%%%%%%
\definecolor{dkgreen}{rgb}{0,0.6,0}
\definecolor{gray}{rgb}{0.5,0.5,0.5}
\definecolor{mauve}{rgb}{0.58,0,0.82}

\lstset{frame=tb,
  language=Python,
  aboveskip=3mm,
  belowskip=3mm,
  showstringspaces=false,
  columns=flexible,
  basicstyle={\small\ttfamily},
  numbers=none,
  numberstyle=\tiny\color{gray},
  keywordstyle=\color{blue},
  commentstyle=\color{dkgreen},
  stringstyle=\color{mauve},
  breaklines=true,
  breakatwhitespace=true,
  tabsize=3
}

\begin{document}

\title{NPRs SQL Notes}
\author{Nicholas P. Ross}
\date{\today}
\maketitle


%% Usually omit these for ApJ or MNRAS style files:


\begin{abstract}
The is my (NPR's) set of SQL notes.  
%% 
You will be able to find the latest version of these notes
and indeed the .tex file at:\\
\href{https://github.com/d80b2t/Research\_Notes}{\tt
https://github.com/d80b2t/Research\_Notes}.
\end{abstract}


\newpage
\tableofcontents
%\listoffigures
%\listoftables


\newpage
\section{The VERY basics}
SQL stands for Structured Query Language.\\
SQL is pronounced ``sequel''.\\
SQL is declarative language.\\
SQL is used to access \& manipulate data in databases.\\
Top SQL DBs are MS SQL Server, Oracle, DB2, and MySQL. \\


\newpage
\section{SQL Queries}
\begin{lstlisting}
SELECT x   --  <Column List>
FROM y     -- <Table Name>
WHERE z   -- <Search Condition>
\end{lstlisting}

\newpage
\section{JOINS}
{\bf READ
\href{https://blog.codinghorror.com/a-visual-explanation-of-sql-joins/}{{\tt THIS PAGE}} NOW}. 

    \subsection{LEFT JOIN}
    The LEFT JOIN keyword returns all rows from the left table
    (table1), with the matching rows in the right table (table2). The
    result is NULL in the right side when there is no match.
    
    \subsection{INNER vs. OUTER JOIN}
    An INNER JOIN will only select records where the joined keys are in
    both specified tables. A LEFT OUTER JOIN will select all records from
    the first table, and any records in the second table that match the
    joined keys.
    
    Example from \href{http://dba.stackexchange.com/questions/153/what-is-the-difference-between-an-inner-join-and-an-outer-join}{\tt here}: \\
    I am new to SQL and wanted to know what is the difference between those two JOIN types?
    \begin{lstlisting}
      SELECT * 
      FROM user u
      INNER JOIN telephone t ON t.user_id = u.id

      SELECT * 
      FROM user u
      LEFT OUTER JOIN telephone t ON t.user_id = u.id    
   \end{lstlisting}
   An inner join will only select records where the joined keys are in both specified tables.\\
   A left outer join will select all records from the first table, and any records in the second table that match the joined keys.\\
   A right outer join will select all records from the second table, and any records in the first table that match the joined keys.\\
   
   In the first example, you will only return a list of users and telephone numbers if at least one telephone record exists for the user.
   
   In the second example, you will return a list of all users, plus any telephone records if they are available (if they aren't available, you'll get NULL for the telephone values).
   

    
    \subsection{NATURAL JOIN}


\newpage
\section{Keywords and Clauses}

    \subsection*{FROM}
    \subsection*{WHERE}
    \subsection*{GROUP BY}
    \subsection*{HAVING}
    \subsection*{ORDER BY}
    \subsection*{DISTINCT}
    


\newpage
\section{SDSS SQL Queries}
Straight from: \\
\href{http://skyserver.sdss.org/dr8/en/help/docs/realquery.asp}{\tt http://skyserver.sdss.org/dr8/en/help/docs/realquery.asp}\\
\href{http://cas.sdss.org/dr5/en/proj/advanced/quasars/query.asp}{http://cas.sdss.org/dr5/en/proj/advanced/quasars/query.asp}


\begin{lstlisting}
-- Determine area of sky targeted by v3_1_0 or later of the target selection algorithm 
-- Note that the "min" just happens to work, it is not robust to changes in the min value.
-- (from Gordon Richards). This query also contains examples of setting the output 
-- precision of columns with STR. 

select sum(area) 
FROM Region 
where regionid in (
select b.boxid
FROM region2box b JOIN tilinggeometry g on b.id = g.tilinggeometryid
where b.boxtype = 'SECTOR'
and b.regiontype = 'TIPRIMARY'
group by b.boxid
having min(g.targetversion) >= 'v3_1_0' 
) 

-- Extract all quasars and quasar candidates from that area 

SELECT
dbo.fSDSS(p.objId) as oid,
p.objId,
str(p.ra,10,6) as ra,
str(p.dec,10,6) as dec,
dbo.fHMS(p.ra) as raHMS,
dbo.fDMS(p.dec) as decDMS,
str(s.z,6,4) as z,
r.area as regarea,
r.regionid,
ti.targetID,
seg.segmentid,
seg.photoVersion
-- Start with list of all DR7 targeted objects (BESTDR7..Target) 
FROM Target AS t
-- Need the region info to restrict to >=v3_1_0
inner JOIN Region as r on t.regionid = r.regionid
-- Need targetobjid to get photometry. Must get this from targetInfo table
inner JOIN TargetInfo as ti on t.targetid = ti.targetid
inner join TARGDR7..PhotoTag as p on ti.targetobjid = p.objid
-- Pull out spectral information
left outer join SpecObj as s on s.targetid = t.targetid
-- I want to know the photoVersion, so we need to join the next two tables
inner join TARGDR7..field as f on p.fieldid = f.fieldid
inner join TARGDR7..segment as seg on f.segmentid = seg.segmentid 
WHERE 
(
-- restrict objects to be in regions where target selection was >=v3_1_0
r.regionid in (
select b.boxid
FROM region2box b JOIN tilinggeometry g on b.id = g.tilinggeometryid
where b.boxtype = 'SECTOR'
and b.regiontype = 'TIPRIMARY'
group by b.boxid
having min(g.targetversion) >= 'v3_1_0'
)
AND
-- primary objects only, since we are selecting from PhotoTag
( (p.mode = 1) AND ((p.status & 0x10) > 0) AND ((p.status & 0x2000) > 0) )
AND
-- quasar candidates only
((p.primTarget & 0x0000001f) > 0) 
) 
ORDER by p.ra 
\end{lstlisting}

\end{document}