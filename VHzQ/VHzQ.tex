\documentclass[11pt,a4paper]{article}


\usepackage{amsmath, amssymb}
\usepackage{bm, booktabs}
\usepackage{cancel, caption}
\usepackage{dcolumn}  % Align table columns on decimal point
\usepackage{epsfig, epsf, enumitem}
\usepackage{fancyhdr}
\usepackage[T1]{fontenc}
\usepackage{graphicx, geometry}
\usepackage{hyperref}
\usepackage{ifthen}
\usepackage[utf8]{inputenc}
\usepackage{lscape, longtable}
\usepackage{multirow}
\usepackage{natbib}
\usepackage{pifont}
\usepackage{ragged2e}
\usepackage{subfigure}
\usepackage{sectsty}
\usepackage{times, tabularx}
\usepackage{tcolorbox}
\usepackage{verbatim}
%\usepackage[usenames,dvipsnames,svgnames,table]{xcolor}



%%%%%%%%%%%%%%%%%%%%%%%%%%%%%%%%%%%%%%%%%%%
%       define Journal abbreviations      %
%%%%%%%%%%%%%%%%%%%%%%%%%%%%%%%%%%%%%%%%%%%
\def\nat{Nat} \def\apjl{ApJ~Lett.} \def\apj{ApJ}
\def\apjs{ApJS} \def\aj{AJ} \def\mnras{MNRAS}
\def\prd{Phys.~Rev.~D} \def\prl{Phys.~Rev.~Lett.}
\def\plb{Phys.~Lett.~B} \def\jhep{JHEP}
\def\npbps{NUC.~Phys.~B~Proc.~Suppl.} \def\prep{Phys.~Rep.}
\def\pasp{PASP} \def\aap{Astron.~\&~Astrophys.} \def\araa{ARA\&A}


%%%%%%%%%%%%%%%%%%%%%%%%%%%%%%%%%%%%%%%%%%%%%%%%%%%%%
%              define symbols                       %
%%%%%%%%%%%%%%%%%%%%%%%%%%%%%%%%%%%%%%%%%%%%%%%%%%%%%
\def \Mpc {~{\rm Mpc} }
\def \Om {\Omega_0}
\def \Omb {\Omega_{\rm b}}
\def \Omcdm {\Omega_{\rm CDM}}
\def \Omlam {\Omega_{\Lambda}}
\def \Omm {\Omega_{\rm m}}
\def \ho {H_0}
\def \qo {q_0}
\def \lo {\lambda_0}
\def \kms {{\rm ~km~s}^{-1}}
\def \kmsmpc {{\rm ~km~s}^{-1}~{\rm Mpc}^{-1}}
\def \hmpc{~\;h^{-1}~{\rm Mpc}} 
\def \hkpc{\;h^{-1}{\rm kpc}} 
\def \hmpcb{h^{-1}{\rm Mpc}}
\def \dif {{\rm d}}
\def \mlim {m_{\rm l}}
\def \bj {b_{\rm J}}
\def \mb {M_{\rm b_{\rm J}}}
\def \qso {_{\rm QSO}}
\def \lrg {_{\rm LRG}}
\def \gal {_{\rm gal}}
\def \xibar {\bar{\xi}}
\def \xis{\xi(s)}
\def \xisp{\xi(\sigma, \pi)}
\def \Xisig{\Xi(\sigma)}
\def \xir{\xi(r)}
\def \max {_{\rm max}}
\def \gsim { \lower .75ex \hbox{$\sim$} \llap{\raise .27ex \hbox{$>$}} }
\def \lsim { \lower .75ex \hbox{$\sim$} \llap{\raise .27ex \hbox{$<$}} }
\def \deg {^{\circ}}
\def \deltac {\delta_{\rm c}}
\def \mmin {M_{\rm min}}
\def \mbh  {M_{\rm BH}}
\def \mdh  {M_{\rm DH}}
\def \msun {M_{\odot}}
\def \z {_{\rm z}}
\def \edd {_{\rm Edd}}
\def \lin {_{\rm lin}}
\def \nonlin {_{\rm non-lin}}
\def \wrms {\langle w_{\rm z}^2\rangle^{1/2}}
\def \dc {\delta_{\rm c}}
\def \wp {w_{p}(\sigma)}
\def \PwrSp {\mathcal{P}(k)}
\def \DelSq {$\Delta^{2}(k)$}
\def \WMAP {{\it WMAP \,}}
\def \cobe {{\it COBE }}
\def \COBE {{\it COBE \;}}
\def \HST  {{\it HST \,\,}}
\def \Spitzer  {{\it Spitzer \,}}



\begin{document}

\title{On the Number Density of Very High Redshift Quasars (VH$z$Qs), or, \\
``What do I point JWST at??''}
\author{NPR}
\date{\today}
\maketitle


% Usually omit these for ApJ or MNRAS style files:
%\tableofcontents
%\listoffigures
%\listoftables

\begin{abstract}
This is a sample document which demonstrates some of the basic features
of \LaTeX.  You can easily reformat it for different document 
or bibliography styles.
\end{abstract}




%Section heading
\section{Section Heading}




\begin{landscape}
\begin{table*}
  \begin{center}
   \begin{tabular}{lrrccl}
      \hline
      \hline
      Survey                      & Area (deg$^{2}$) & N$_{\rm Q}$ & Magnitude Range & $z$-range & Reference  \\
      \hline 
      GOODS(+SDSS)        &   0.1+(4200)  &   13(+656)    &  $22.25 < z_{850} < 25.25$  & $3.5<z<5.2$  & \citet{Fontanot07} \\
      VVDS                       &        0.62    & 130      &  $17.5 < I_{\rm AB} < 24.0$ &  $0<z<5$    &  \citet{Bongiorno07} \\
      COMBO-17              &       0.8     & \hbox{  192} & $R<24$                       &  $1.2 < z < 4.8$  & \citet{Wolf03} \\
      COSMOS$^{a}$         &       1.64   & 8     & $22 < i' < 24$              & $3.7 \lesssim z \lesssim 4.7$   & \citet{Ikeda11} \\      
      COSMOS                  &       1.64    & $^{b}$0         & $22 < i' < 24$     & $4.5 \lesssim z \lesssim 5.5$   & \citet{Ikeda12} \\   
      COSMOS                  &       1.64    & 155              & $16 \leq  I_{\rm AB} \leq 25$      & $3<z<5$   & \citet{Masters12} \\   
      NDWFS+DFS$^{c}$   &       4        &                24  & $R \leq 24$                    & $3.7<z<5.1$       & \citet{Glikman11} \\
      SFQS$^{d}$               &       4        & \hbox{  414} & $g<22.5$                          & $z<5$                 & \citet{Jiang06} \\
      BOSS$^{e}$+MMT     & 14.5+3.92 & \hbox{1 877} & $g\lesssim 23$  & $0.7<z<4.0$ & \citet{Palanque-Delabrouille12} \\ 
      2SLAQ$^{f}$            &     105     & \hbox{ 5 645} & $18.00 < g < 21.85$         & $z\leq2.1$               & \citet{Richards05}     \\
      SDSS$^{g}$               &    182     &                  39  &  $i\leq 20$                        &  $3.6<z<5.0$    & \citet{Fan01b} \\
      SDSS+2SLAQ           &    192     & \hbox{10 637} & $18.00 < g < 21.85$        & $0.4<z<2.6 $    & \citet{Croom09b}     \\
      SDSS Main+Deep     &    195     &                     6 & $z_{\rm AB} < 21.80$        & $z\sim6 $         & \citet{Jiang09}     \\
      {\bf BOSS Stripe 82} & {\bf 220}  &{\bf \hbox{5 476}} & $i>${\bf 18.0 and} $g< ${\bf 22.3}  & {\bf 2.2}$<z<${\bf 3.5}  &\citet{Palanque-Delabrouille11} \\ 
      CFHQS$^{h}$             & 500      &  \hbox{ 19}                  & $z'<22.63$           &  $5.74 < z < 6.42. $ & \citet{Willott10} \\
      2QZ$^{i}$                 &   700     & \hbox{23 338} & $18.25<b_{\rm J}<20.85$   & $0.4<z<2.1$     & \citet{Boyle00, Croom04}     \\
      SDSS DR3                 & 1622     & \hbox{15 343} &  $i\leq 19.1$ and $i\leq 20.2$ &  $0.3<z<5.0$ & \citet{Richards06} \\
     {\bf BOSS DR9}  &  {\bf 2236}  & $^{j}${\bf \hbox{23 201}} & $g<${\bf 22.00 or }$r<${\bf 21.85}  & {\bf 2.2}$<z<${\bf 3.5}  & {\bf this paper}           \\
     SDSS DR7                 &  6248 & \hbox{57 959}  &  $i\leq 19.1$ and $i\leq 20.2$ &  $0.3<z<5.0$ & \citet{Shen_Kelly12} \\
      SDSS Type 2            &  6293 & \hbox{   887}  & $L_{\rm O III} \geq 10^{8.3} L_\odot$ &  $z<0.83$ & \citet{Reyes08} \\
      SDSS DR6$^{k}$        &  8417 & $\gtrsim\hbox{850,000}$ &  $i<21.3 $        &  $z\sim2$ and $z\sim4.25$ & \citet{Richards09}
  \\
      \hline
      \hline
   \end{tabular}
    \caption{Selected optical quasar luminosity function measurements.\\ 
      $^{a}$Cosmic Evolution Survey \citep{Scoville07}. \\
      $^{b}$No Type-1 quasars were identified, though a low-luminosity $z\sim5.07$ Type-2 quasar was discovered. \\
      $^{c}$NOAO Deep Wide-Field Survey \citep{JD99} and the Deep Lens Survey \citep{Wittman02}. \\
      $^{d}$SDSS Faint Quasar Survey. \\
      $^{e}$The ``boss21'' area on the SDSS Stripe 82 field. \\
      $^{f}$2dF-SDSS LRG And QSO Survey \citep{Croom09a}. \\
      $^{g}$Photometric sample from SDSS; spectroscopic confirmation from SDSS and other telescopes.\\
      $^{h}$Canada-France High-$z$ Quasar Survey \citep{Willott09}\\
      $^{i}$2dF Quasar Redshift Survey \citep{Croom04}. \\
      $^{j}$From our ``uniform'' sample defined in Section~\ref{sec:data_uniform} \\
      $^{k}$From a catalog of $>$1,000,000 photometrically classified quasar candidates. \\}
      \label{tab:previous_surveys}
  \end{center}
\end{table*}
\end{landscape}


\subsection{Subsection heading}

The QLF is defined as the number density of quasars per unit
luminosity. It is often described by a double power-law
\citep[][hereafter, R06]{Boyle00,Croom04,Richards06} of the form
\begin{equation}
  \Phi(L, z) = \frac{ \phi_{*}^{(L)} }
                            {  (L/L^{*})^{\alpha}    +  (L/L^{*})^{\beta}  }
 \ \,
 \label{eq:double_powerlaw}
\end{equation}
with a characteristic, or break, luminosity $L_{*}$.  An alternative
definition of this form of the QLF gives the number density of quasars
per unit magnitude,
\begin{equation}
  \Phi(M, z) = \frac{ \phi_{*}^{(M)} }
       { 10^{0.4{(\alpha +1)[M-M^{*}(z)]}}+10^{0.4{(\beta +1)[M-M^{*}(z) ]}} }
 \label{eq:double_powerlaw_mag}
\end{equation}
The dimensions of $\Phi$ differ in the two conventions.  We have
followed R06 such that $\alpha$ describes the faint end QLF slope, and
$\beta$ the bright end slope.  The $\alpha$/$\beta$ convention in some
other works \citep[e.g.,][]{Croom09b} is in the opposite sense from our 
definition. Evolution of the QLF can be encoded in the redshift dependence 
of the break luminosity, $\phi_{*}$, and also potentially in the evolution of
the power-law slopes.


%\bibliographystyle{apj}
\bibliographystyle{mn2e}
\bibliography{/cos_pc19a_npr/LaTeX/tester_mnras}

\end{document}

