 \documentclass[11pt]{article}
\setlength {\textwidth}{180mm} 
\setlength {\textheight}{260mm}
\topmargin=-35.00mm
\oddsidemargin=-10.00mm
\pagestyle{empty}


\usepackage{amsmath, amssymb}
\usepackage{bm, booktabs}
\usepackage{cancel, caption}
\usepackage{dcolumn}  % Align table columns on decimal point
\usepackage{epsfig, epsf, enumitem}
\usepackage{fancyhdr}
\usepackage[T1]{fontenc}
\usepackage{graphicx, geometry}
\usepackage{hyperref}
\usepackage{ifthen}
\usepackage[utf8]{inputenc}
\usepackage{lscape, longtable}
\usepackage{multirow}
\usepackage{natbib}
\usepackage{pifont}
\usepackage{ragged2e}
\usepackage{subfigure}
\usepackage{sectsty}
\usepackage{times, tabularx}
\usepackage{tcolorbox}
\usepackage{verbatim}
%\usepackage[usenames,dvipsnames,svgnames,table]{xcolor}



%%%%%%%%%%%%%%%%%%%%%%%%%%%%%%%%%%%%%%%%%%%
%       define Journal abbreviations      %
%%%%%%%%%%%%%%%%%%%%%%%%%%%%%%%%%%%%%%%%%%%
\def\nat{Nat} \def\apjl{ApJ~Lett.} \def\apj{ApJ}
\def\apjs{ApJS} \def\aj{AJ} \def\mnras{MNRAS}
\def\prd{Phys.~Rev.~D} \def\prl{Phys.~Rev.~Lett.}
\def\plb{Phys.~Lett.~B} \def\jhep{JHEP}
\def\npbps{NUC.~Phys.~B~Proc.~Suppl.} \def\prep{Phys.~Rep.}
\def\pasp{PASP} \def\aap{Astron.~\&~Astrophys.} \def\araa{ARA\&A}


%%%%%%%%%%%%%%%%%%%%%%%%%%%%%%%%%%%%%%%%%%%%%%%%%%%%%
%              define symbols                       %
%%%%%%%%%%%%%%%%%%%%%%%%%%%%%%%%%%%%%%%%%%%%%%%%%%%%%
\def \Mpc {~{\rm Mpc} }
\def \Om {\Omega_0}
\def \Omb {\Omega_{\rm b}}
\def \Omcdm {\Omega_{\rm CDM}}
\def \Omlam {\Omega_{\Lambda}}
\def \Omm {\Omega_{\rm m}}
\def \ho {H_0}
\def \qo {q_0}
\def \lo {\lambda_0}
\def \kms {{\rm ~km~s}^{-1}}
\def \kmsmpc {{\rm ~km~s}^{-1}~{\rm Mpc}^{-1}}
\def \hmpc{~\;h^{-1}~{\rm Mpc}} 
\def \hkpc{\;h^{-1}{\rm kpc}} 
\def \hmpcb{h^{-1}{\rm Mpc}}
\def \dif {{\rm d}}
\def \mlim {m_{\rm l}}
\def \bj {b_{\rm J}}
\def \mb {M_{\rm b_{\rm J}}}
\def \qso {_{\rm QSO}}
\def \lrg {_{\rm LRG}}
\def \gal {_{\rm gal}}
\def \xibar {\bar{\xi}}
\def \xis{\xi(s)}
\def \xisp{\xi(\sigma, \pi)}
\def \Xisig{\Xi(\sigma)}
\def \xir{\xi(r)}
\def \max {_{\rm max}}
\def \gsim { \lower .75ex \hbox{$\sim$} \llap{\raise .27ex \hbox{$>$}} }
\def \lsim { \lower .75ex \hbox{$\sim$} \llap{\raise .27ex \hbox{$<$}} }
\def \deg {^{\circ}}
\def \deltac {\delta_{\rm c}}
\def \mmin {M_{\rm min}}
\def \mbh  {M_{\rm BH}}
\def \mdh  {M_{\rm DH}}
\def \msun {M_{\odot}}
\def \z {_{\rm z}}
\def \edd {_{\rm Edd}}
\def \lin {_{\rm lin}}
\def \nonlin {_{\rm non-lin}}
\def \wrms {\langle w_{\rm z}^2\rangle^{1/2}}
\def \dc {\delta_{\rm c}}
\def \wp {w_{p}(\sigma)}
\def \PwrSp {\mathcal{P}(k)}
\def \DelSq {$\Delta^{2}(k)$}
\def \WMAP {{\it WMAP \,}}
\def \cobe {{\it COBE }}
\def \COBE {{\it COBE \;}}
\def \HST  {{\it HST \,\,}}
\def \Spitzer  {{\it Spitzer \,}}


\begin{document}

\title{Astrophysics Authorlist Psychology}
%\author{Nicholas P. Ross}
\date{\today}
\maketitle


%\begin{abstract}
%This is a simple document which will... 
%\end{abstract}

\medskip
\medskip
\noindent
\underline{Single author paper:} (Slightly) crazy theorist, OR an observer where literally NO ONE else believes your observations. 


\medskip
\medskip
\noindent
\underline{Double author papers:} Usually very powerful, for either theory or observations (with the latter basically meaning it's a Review). Folks think work split was 50:50 regardless of what it actually was. 

\medskip
\medskip
\noindent
\underline{Triple author paper:}. Very nice and powerful too. Has the massive benefit of potentially becoming a TLA in the field. 

\medskip
\medskip
\noindent
\underline{Four author paper:} Potentially a 4-of-kind type of  thing (think DEFW). Rarely do theory papers have more authors than this (usually since that would result in 2$n$+1 papers instead). 


\medskip
\medskip
\noindent
\underline{$\sim$5-15 author paper:} Good observational team size with likely a strong balance of postgrads and postdocs (who did the actual work and analysis) and profs (who can actually write proper prose). Some examples include the 2QZ and the WMAP team. 


\medskip
\medskip
\noindent
\underline{15 - $\sim$30 author paper:} Pretty good observational team size. However, the real debate is how many names you recognise and how much credit you give to authors 3$^{\rm rd}$ and later...

\medskip
\medskip
\noindent
\underline{30 - $\sim$100 author paper:} Hmmm, either a key collaboration paper, or worse yet, a joining of two collaborations where the authorlist was most likely decided by a completely illogical computer code/Publication Policy. 


\medskip
\medskip
\noindent
\underline{$\gtrsim$100 author paper:} Well done. You're on a {\it Planck} paper (probably actually written by Efstathiou or M. White). 


\medskip
\medskip
\noindent
\underline{$\gtrsim$200 author paper:} Well done. You're on an SDSS Data Release paper. 

\medskip
\medskip
\noindent
\underline{$\gtrsim$1000 author paper:} Well done. You discovered Gravitational Waves.   

\medskip
\medskip
\noindent
\underline{$\gtrsim$3000 author paper:} Well done. You discovered the Higgs.  


\medskip
\medskip 
\noindent
\underline{$n$ author paper in Nature:} (where $n$ is a member of $\mathbb{R}$.) 
Would {\it hate} to imply *anything* about the validity of Nature papers...

\iffalse
\newpage
\topmargin=0.00mm

\medskip
\medskip 
\noindent
\underline{Single author on a humorous document tenuously related to astronomy:} Congratulations: You've got an ERF and you've clearly got too much time on your hands!


\bibliographystyle{mn2e}
\bibliography{/cos_pc19a_npr/LaTeX/tester_mnras}
\fi

\end{document}

