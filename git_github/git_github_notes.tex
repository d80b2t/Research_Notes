\documentclass[11pt,a4paper]{article}

\bibliographystyle{apalike}


\usepackage[toc,page]{appendix}
\usepackage{amsmath, amssymb}
\usepackage{bm}% bold math
\usepackage{cancel, caption}
\usepackage{dcolumn}% Align table columns on decimal point
\usepackage{epsfig, epsf}
\usepackage{graphicx,fancyhdr,natbib,subfigure}
\usepackage{lscape, longtable}
\usepackage{hyperref,ifthen}
\usepackage{verbatim}
\usepackage{color}
\usepackage[usenames,dvipsnames]{xcolor}
\usepackage{listings}
%% http://en.wikibooks.org/wiki/LaTeX/Colors



%%%%%%%%%%%%%%%%%%%%%%%%%%%%%%%%%%%%%%%%%%%
%       define Journal abbreviations      %
%%%%%%%%%%%%%%%%%%%%%%%%%%%%%%%%%%%%%%%%%%%
\def\nat{Nat} \def\apjl{ApJ~Lett.} \def\apj{ApJ}
\def\apjs{ApJS} \def\aj{AJ} \def\mnras{MNRAS}
\def\prd{Phys.~Rev.~D} \def\prl{Phys.~Rev.~Lett.}
\def\plb{Phys.~Lett.~B} \def\jhep{JHEP} \def\nar{NewAR}
\def\npbps{NUC.~Phys.~B~Proc.~Suppl.} \def\prep{Phys.~Rep.}
\def\pasp{PASP} \def\aap{Astron.~\&~Astrophys.} \def\araa{ARA\&A}
\def\jcap{\ref@jnl{J. Cosmology Astropart. Phys.}}%
\def\physrep{Phys.~Rep.}

\newcommand{\preep}[1]{{\tt #1} }

%%%%%%%%%%%%%%%%%%%%%%%%%%%%%%%%%%%%%%%%%%%%%%%%%%%%%
%              define symbols                       %
%%%%%%%%%%%%%%%%%%%%%%%%%%%%%%%%%%%%%%%%%%%%%%%%%%%%%
\def \Mpc {~{\rm Mpc} }
\def \Om {\Omega_0}
\def \Omb {\Omega_{\rm b}}
\def \Omcdm {\Omega_{\rm CDM}}
\def \Omlam {\Omega_{\Lambda}}
\def \Omm {\Omega_{\rm m}}
\def \ho {H_0}
\def \qo {q_0}
\def \lo {\lambda_0}
\def \kms {{\rm ~km~s}^{-1}}
\def \kmsmpc {{\rm ~km~s}^{-1}~{\rm Mpc}^{-1}}
\def \hmpc{~\;h^{-1}~{\rm Mpc}} 
\def \hkpc{\;h^{-1}{\rm kpc}} 
\def \hmpcb{h^{-1}{\rm Mpc}}
\def \dif {{\rm d}}
\def \mlim {m_{\rm l}}
\def \bj {b_{\rm J}}
\def \mb {M_{\rm b_{\rm J}}}
\def \mg {M_{\rm g}}
\def \qso {_{\rm QSO}}
\def \lrg {_{\rm LRG}}
\def \gal {_{\rm gal}}
\def \xibar {\bar{\xi}}
\def \xis{\xi(s)}
\def \xisp{\xi(\sigma, \pi)}
\def \Xisig{\Xi(\sigma)}
\def \xir{\xi(r)}
\def \max {_{\rm max}}
\def \gsim { \lower .75ex \hbox{$\sim$} \llap{\raise .27ex \hbox{$>$}} }
\def \lsim { \lower .75ex \hbox{$\sim$} \llap{\raise .27ex \hbox{$<$}} }
\def \deg {^{\circ}}
%\def \sqdeg {\rm deg^{-2}}
\def \deltac {\delta_{\rm c}}
\def \mmin {M_{\rm min}}
\def \mbh  {M_{\rm BH}}
\def \mdh  {M_{\rm DH}}
\def \msun {M_{\odot}}
\def \z {_{\rm z}}
\def \edd {_{\rm Edd}}
\def \lin {_{\rm lin}}
\def \nonlin {_{\rm non-lin}}
\def \wrms {\langle w_{\rm z}^2\rangle^{1/2}}
\def \dc {\delta_{\rm c}}
\def \wp {w_{p}(\sigma)}
\def \PwrSp {\mathcal{P}(k)}
\def \DelSq {$\Delta^{2}(k)$}
\def \WMAP {{\it WMAP \,}}
\def \cobe {{\it COBE }}
\def \COBE {{\it COBE \;}}
\def \HST  {{\it HST \,\,}}
\def \Spitzer  {{\it Spitzer \,}}
\def \ATLAS {VST-AA$\Omega$ {\it ATLAS} }
\def \BEST   {{\tt best} }
\def \TARGET {{\tt target} }
\def \TQSO   {{\tt TARGET\_QSO}}
\def \HIZ    {{\tt TARGET\_HIZ}}
\def \FIRST  {{\tt TARGET\_FIRST}}
\def \zc {z_{\rm c}}
\def \zcz {z_{\rm c,0}}

\newcommand{\ltsim}{\raisebox{-0.6ex}{$\,\stackrel
        {\raisebox{-.2ex}{$\textstyle <$}}{\sim}\,$}}
\newcommand{\gtsim}{\raisebox{-0.6ex}{$\,\stackrel
        {\raisebox{-.2ex}{$\textstyle >$}}{\sim}\,$}}
\newcommand{\simlt}{\raisebox{-0.6ex}{$\,\stackrel
        {\raisebox{-.2ex}{$\textstyle <$}}{\sim}\,$}}
\newcommand{\simgt}{\raisebox{-0.6ex}{$\,\stackrel
        {\raisebox{-.2ex}{$\textstyle >$}}{\sim}\,$}}

\newcommand{\Msun}{M_\odot}
\newcommand{\Lsun}{L_\odot}
\newcommand{\lsun}{L_\odot}
\newcommand{\Mdot}{\dot M}

\newcommand{\sqdeg}{deg$^{-2}$}
\newcommand{\lya}{Ly$\alpha$\ }
%\newcommand{\lya}{Ly\,$\alpha$\ }
\newcommand{\lyaf}{Ly\,$\alpha$\ forest}
%\newcommand{\eg}{e.g.~}
%\newcommand{\etal}{et~al.~}
\newcommand{\lyb}{Ly$\beta$\ }
\newcommand{\cii}{C\,{\sc ii}\ }
\newcommand{\ciii}{C\,{\sc iii}]\ }
\newcommand{\civ}{C\,{\sc iv}\ }
\newcommand{\SiIV}{Si\,{\sc iv}\ }
\newcommand{\mgii}{Mg\,{\sc ii}\ }
\newcommand{\feii}{Fe\,{\sc ii}\ }
\newcommand{\feiii}{Fe\,{\sc iii}\ }
\newcommand{\caii}{Ca\,{\sc ii}\ }
\newcommand{\halpha}{H\,$\alpha$\ }
\newcommand{\hbeta}{H\,$\beta$\ }
\newcommand{\hgamma}{H\,$\gamma$\ }
\newcommand{\hdelta}{H\,$\delta$\ }
\newcommand{\oi}{[O\,{\sc i}]\ }
\newcommand{\oii}{[O\,{\sc ii}]\ }
\newcommand{\oiii}{[O\,{\sc iii}]\ }
\newcommand{\heii}{[He\,{\sc ii}]\ }
\newcommand{\nv}{N\,{\sc v}\ }
\newcommand{\nev}{Ne\,{\sc v}\ }
\newcommand{\neiii}{[Ne\,{\sc iii}]\ }
\newcommand{\aliii}{Al\,{\sc iii}\ }
\newcommand{\siiii}{Si\,{\sc iii}]\ }


%%%%%%%%%%%%%%%%%%%%%%%%%%%%%%%%%%%%%%%%%%%%%%%%%%%%%
%              define Listings                       %
%%%%%%%%%%%%%%%%%%%%%%%%%%%%%%%%%%%%%%%%%%%%%%%%%%%%%
\definecolor{dkgreen}{rgb}{0,0.6,0}
\definecolor{gray}{rgb}{0.5,0.5,0.5}
\definecolor{mauve}{rgb}{0.58,0,0.82}

\lstset{frame=tb,
  language=Python,
  aboveskip=3mm,
  belowskip=3mm,
  showstringspaces=false,
  columns=flexible,
  basicstyle={\small\ttfamily},
  numbers=none,
  numberstyle=\tiny\color{gray},
  keywordstyle=\color{blue},
  commentstyle=\color{dkgreen},
  stringstyle=\color{mauve},
  breaklines=true,
  breakatwhitespace=true,
  tabsize=3
}

\begin{document}

\title{{\tt git} ``Cheat Sheet''}
\author{Nicholas P. Ross et al. (your name here....)}
\date{\today}
\maketitle


% Usually omit these for ApJ or MNRAS style files:
%\tableofcontents
%\listoffigures
%\listoftables

\begin{abstract}
This is a simple few notes that will hopefully form something like a ``cheat sheet'' for 
folks using git, and e.g. github. Slightly ``meta''-ly, I've uploaded this file to github:\\
\href{https://github.com/d80b2t/git\_cheatsheet}{{\tt https://github.com/d80b2t/git\_cheatsheet/}}.
\end{abstract}


%\section{General introductions}

\newpage
\section{Learn Git branching}

\footnote{from Loki, thank you!! :-)}This link:: \\
\href{https://learngitbranching.js.org/}{https://learngitbranching.js.org/} \\
seems to be the best thing I've seen so far.

\weeskip
Welcome to Learn Git Branching
Interested in learning Git? Well you've come to the right place! "Learn Git Branching" is the most visual and interactive way to learn Git on the web; you'll be challenged with exciting levels, given step-by-step demonstrations of powerful features, and maybe even have a bit of fun along the way.

\weeskip
After this dialog you'll see the variety of levels we have to offer. If you're a beginner, just go ahead and start with the first. If you already know some Git basics, try some of our later more challenging levels.

\weeskip
You can see all the commands available with show commands at the terminal.

\weeskip
PS: Want to go straight to a sandbox next time? Try out this \href{https://learngitbranching.js.org/?NODEMO}{special link}.



\newpage
\section{Some Youtube links}
 \href{https://www.youtube.com/watch?v=HkdAHXoRtos}{Git It? How to use Git and Github}.





 

\section{My github accounts}
d80b2t\\
npr247\\




\section{git Notes}

    \subsection{To Check-out:}
   {\tt  $>$ git init\\
    $>$ git clone [url]} \\
     e.g.\\ 
    {\tt $>$ git clone https://github.com/crazygirl9991/SeniorResearch\\
   }

    \subsection{Status check:}
   {\tt $>$ git status}\\

 
    \subsection{Update your local branch:}
    {\tt $>$ git pull}\\


    \subsection{To Add new files/material:}
    {\tt $>$ git add\\
     $>$ git commit -m "[descriptive message]" } \\

   \subsection{To then place on repository:}
   {\tt $>$ git push }\\


   \subsection{Version Control Basics (1)}
   %% https://git-scm.com/book/en/v2/Git-Basics-Viewing-the-Commit-History
   {\tt  $>$ git log}\\
   will give the log of commits, where the beginning of the hash
   (which will be something a bit crazy like 80dd432df626ebf67d75adbd75bf0fad2a01f024E)
   is also the `revision number', in this case: \\
   Latest commit 80dd432 6 minutes ago \\

   \noindent
   {\tt $>$ git log --pretty=oneline}\\ 
   looks better ;-), or even: \\

   \noindent
   {\tt $>$ git log --pretty=format:"\%h - \%an, \%ar : \%s"} \\


\newpage
\section{Git-Specific Commands}
{\bf THESE NOTES ARE STRAIGHT FROM\\
\href{http://readwrite.com/2013/09/30/understanding-github-a-journey-for-beginners-part-1/}{``Understanding Github a-journey for beginners-part-1/}}.
Since Git was designed with a big project like Linux in mind, there
are a lot of Git commands. However, to use the basics of Git, you’ll
only need to know a few terms. They all begin the same way, with the
word ``git''.

\weeskip
{\tt git init}: Initializes a new Git repository. Until you run this
command inside a repository or directory, it’s just a regular
folder. Only after you input this does it accept further Git commands.

\weeskip
{\tt git config}: Short for ``configure'' this is most useful when
you’re setting up Git for the first time.

\weeskip
{\tt git status}: Check the status of your repository. See which files
are inside it, which changes still need to be committed, and which
branch of the repository you’re currently working on.

\weeskip
{\tt git add}: This does not add new files to your
repository. Instead, it brings new files to Git’s attention. After you
add files, they’re included in Git's ``snapshots'' of the repository.

\weeskip
{\tt git commit}: Git’s most important command. After you make any
sort of change, you input this in order to take a ``snapshot'' of the
repository. Usually it goes {\tt git commit -m ``Message here''}.  The
{\tt -m} indicates that the following section of the command should be
read as a message.

\weeskip
{\tt git branch}: Working with multiple collaborators and want to make
changes on your own? This command will let you build a new branch, or
timeline of commits, of changes and file additions that are completely
your own. Your title goes after the command. If you wanted a new
branch called ``cats'', you'd type {\tt git branch cats}.

\weeskip
{\tt git checkout:} Literally allows you to ``check out'' a repository
that you are not currently inside. This is a navigational command that
lets you move to the repository you want to check. You can use this
command as {\tt git checkout master} to look at the master branch, or
{\tt git checkout cats} to look at another branch.

\weeskip
{\tt git merge}: When you’re done working on a branch, you can merge
your changes back to the master branch, which is visible to all
collaborators. {\tt git merge cats} would take all the changes you
made to the ``cats'' branch and add them to the master.

\weeskip
{\tt git push}: If you’re working on your local computer, and want
your commits to be visible online on GitHub as well, you ``push'' the
changes up to GitHub with this command.

\weeskip
{\tt git pull}: If you’re working on your local computer and want the
most up-to-date version of your repository to work with, you ``pull''
the changes down from GitHub with this command.

\weeskip
{\tt git help}: Forgot a command? Type this into the command line to
bring up the 21 most common git commands. You can also be more
specific and type ``git help init'' or another term to figure out how
to use and configure a specific git command.

{\bf 
CONTINUE READING 
http://readwrite.com/2013/09/30/understanding-github-a-journey-for-beginners-part-1/
!!!
}
































\section{Things to learn...}
git remote add $<$name$>$ $<$ur$l>$\\
%% http://stackoverflow.com/questions/5617211/what-is-git-remote-add-and-git-push-origin-master

\noindent
https://help.github.com/articles/removing-a-remote/ \\

\noindent
\href{http://gitref.org/remotes/}{http://gitref.org/remotes/}\\




\section{Useful URLs}
\noindent
\href{http://stackoverflow.com/questions/5617211/what-is-git-remote-add-and-git-push-origin-master}{http://stackoverflow.com/questions/5617211/what-is-git-remote-add-and-git-push-origin-master}

\noindent
\href{https://help.github.com/articles/git-cheatsheet}{https://help.github.com/articles/git-cheatsheet/}
for the e.g. github-git-cheat-sheet.pdf \\

\noindent
\href{https://github.com/AlexZeitler/gitcheatsheet}{https://github.com/AlexZeitler/gitcheatsheet}






\section*{Resources}

\noindent
\href{http://readwrite.com/2013/09/30/understanding-github-a-journey-for-beginners-part-1/}{http://readwrite.com/2013/09/30/understanding-github-a-journey-for-beginners-part-1/}\\

\noindent
\href{http://readwrite.com/2013/10/02/github-for-beginners-part-2/}{http://readwrite.com/2013/10/02/github-for-beginners-part-2/}\\


    \subsection*{YouTube}
    \noindent
    \href{https://www.youtube.com/watch?v=OqmSzXDrJBk}{What is Git - A Quick Introduction to the Git Version Control System}\\
    
    \noindent
    \href{https://www.youtube.com/watch?v=Y9XZQO1n_7c}{Learn Git in 20 Minutes}\\
    
    \noindent
    \href{https://www.youtube.com/watch?v=O72FWNeO-xY}{Copying a GitHub Repository to Your Local Computer (Youtube)}\\
    




\end{document}

