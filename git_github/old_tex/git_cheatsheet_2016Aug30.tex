\documentclass[11pt,a4paper]{article}

\bibliographystyle{apalike}

\usepackage{epsfig, amsmath, natbib}
\usepackage{hyperref}

\begin{document}

\title{{\tt git} ``Cheat Sheet''}
\author{Nicholas P. Ross et al. (your name here....)}
\date{\today}
\maketitle


% Usually omit these for ApJ or MNRAS style files:
%\tableofcontents
%\listoffigures
%\listoftables

\begin{abstract}
This is a simple few notes that will hopefully form something like a ``cheat sheet'' for 
folks using git, and e.g. github. Slightly ``meta''-ly, I've uploaded this file to github:\\
\href{https://github.com/d80b2t/git\_cheatsheet}{{\tt https://github.com/d80b2t/git\_cheatsheet/}}.
\end{abstract}


\section{My github accounts}
d80b2t\\
npr247\\



\subsection{notes}
On Jul 23, 2014, at 7:37 PM, Nic Ross $<$npross@lbl.gov$>$ wrote:\\

Gang, \\

In an effort to get our code repository going, here's my github details: \\
	https://github.com/d80b2t \\
	https://github.com/d80b2t/spies\\
	https://github.com/d80b2t/group\_code\\

So, after creating a repository, you can go to that repository, click 
settings, and click "collaborators" to add collaborators. I have 
done this for John and Rob for the https://github.com/d80b2t/group\_code
repository. 

Thanks, 
Nic



\section{git Notes}

    \subsection{To Check-out:}
   {\tt  $>$ git init\\
    $>$ git clone [url]} \\
     e.g.\\ 
    {\tt $>$ git clone https://github.com/crazygirl9991/SeniorResearch\\
   }

    \subsection{Status check:}
   {\tt $>$ git status}\\

 
    \subsection{Update your local branch:}
    {\tt $>$ git pull}\\


    \subsection{To Add new files/material:}
    {\tt $>$ git add\\
     $>$ git commit -m "[descriptive message]" } \\

   \subsection{To then place on repository:}
   {\tt $>$ git push }\\


   \subsection{Version Control Basics (1)}
   %% https://git-scm.com/book/en/v2/Git-Basics-Viewing-the-Commit-History
   {\tt  $>$ git log}\\
   will give the log of commits, where the beginning of the hash
   (which will be something a bit crazy like 80dd432df626ebf67d75adbd75bf0fad2a01f024E)
   is also the `revision number', in this case: \\
   Latest commit 80dd432 6 minutes ago \\

   \noindent
   {\tt $>$ git log --pretty=oneline}\\ 
   looks better ;-), or even: \\

   \noindent
   {\tt $>$ git log --pretty=format:"\%h - \%an, \%ar : \%s"} \\


\newpage
\section{Git-Specific Commands}
{\bf THESE NOTES ARE STRAIGHT FROM\\
\href{http://readwrite.com/2013/09/30/understanding-github-a-journey-for-beginners-part-1/}{http://readwrite.com/2013/09/30/understanding-github-a-journey-for-beginners-part-1/}}.
Since Git was designed with a big project like Linux in mind, there
are a lot of Git commands. However, to use the basics of Git, you’ll
only need to know a few terms. They all begin the same way, with the
word ``git''.

{\tt git init}: Initializes a new Git repository. Until you run this
command inside a repository or directory, it’s just a regular
folder. Only after you input this does it accept further Git commands.

{\tt git config}: Short for ``configure'' this is most useful when
you’re setting up Git for the first time.

{\tt git status}: Check the status of your repository. See which files
are inside it, which changes still need to be committed, and which
branch of the repository you’re currently working on.

{\tt git add}: This does not add new files to your
repository. Instead, it brings new files to Git’s attention. After you
add files, they’re included in Git's ``snapshots'' of the repository.

{\tt git commit}: Git’s most important command. After you make any
sort of change, you input this in order to take a ``snapshot'' of the
repository. Usually it goes {\tt git commit -m ``Message here''}.  The
{\tt -m} indicates that the following section of the command should be
read as a message.

{\tt git branch}: Working with multiple collaborators and want to make
changes on your own? This command will let you build a new branch, or
timeline of commits, of changes and file additions that are completely
your own. Your title goes after the command. If you wanted a new
branch called ``cats'', you'd type {\tt git branch cats}.

{\tt git checkout:} Literally allows you to ``check out'' a repository
that you are not currently inside. This is a navigational command that
lets you move to the repository you want to check. You can use this
command as {\tt git checkout master} to look at the master branch, or
{\tt git checkout cats} to look at another branch.

{\tt git merge}: When you’re done working on a branch, you can merge
your changes back to the master branch, which is visible to all
collaborators. {\tt git merge cats} would take all the changes you
made to the ``cats'' branch and add them to the master.

{\tt git push}: If you’re working on your local computer, and want
your commits to be visible online on GitHub as well, you ``push'' the
changes up to GitHub with this command.

{\tt git pull}: If you’re working on your local computer and want the
most up-to-date version of your repository to work with, you ``pull''
the changes down from GitHub with this command.

{\tt git help}: Forgot a command? Type this into the command line to
bring up the 21 most common git commands. You can also be more
specific and type ``git help init'' or another term to figure out how
to use and configure a specific git command.

{\bf 
CONTINUE READING 
http://readwrite.com/2013/09/30/understanding-github-a-journey-for-beginners-part-1/
!!!
}
































\section{Things to learn...}
git remote add $<$name$>$ $<$ur$l>$\\
%% http://stackoverflow.com/questions/5617211/what-is-git-remote-add-and-git-push-origin-master

\noindent
https://help.github.com/articles/removing-a-remote/ \\

\noindent
\href{http://gitref.org/remotes/}{http://gitref.org/remotes/}\\




\section{Useful URLs}
\noindent
\href{http://stackoverflow.com/questions/5617211/what-is-git-remote-add-and-git-push-origin-master}{http://stackoverflow.com/questions/5617211/what-is-git-remote-add-and-git-push-origin-master}

\noindent
\href{https://help.github.com/articles/git-cheatsheet}{https://help.github.com/articles/git-cheatsheet/}
for the e.g. github-git-cheat-sheet.pdf \\

\noindent
\href{https://github.com/AlexZeitler/gitcheatsheet}{https://github.com/AlexZeitler/gitcheatsheet}






\section*{Resources}

\noindent
\href{http://readwrite.com/2013/09/30/understanding-github-a-journey-for-beginners-part-1/}{http://readwrite.com/2013/09/30/understanding-github-a-journey-for-beginners-part-1/}\\

\noindent
\href{http://readwrite.com/2013/10/02/github-for-beginners-part-2/}{http://readwrite.com/2013/10/02/github-for-beginners-part-2/}\\


    \subsection*{YouTube}
    \noindent
    \href{https://www.youtube.com/watch?v=OqmSzXDrJBk}{What is Git - A Quick Introduction to the Git Version Control System}\\
    
    \noindent
    \href{https://www.youtube.com/watch?v=Y9XZQO1n_7c}{Learn Git in 20 Minutes}\\
    
    \noindent
    \href{https://www.youtube.com/watch?v=O72FWNeO-xY}{Copying a GitHub Repository to Your Local Computer (Youtube)}\\
    




\end{document}

